\input Calculemus2_preambulo

\chapter{Introducción}
\label{sec:org79e9174}

Este libro es una recopilación de los ejercicios de demostración con
Lean4 que se han ido publicando, desde el 10 de julio de 20023, en el
blog \href{https://www.glc.us.es}{Calculemus}.

La ordenación de los ejercicios es simplemente temporal según su fecha
de publicación en Calculemus y el orden de los ejercicios en Calculemus
responde a los que me voy encontrando en mis \href{https://github.com/jaalonso/Lecturas\_GLC}{lecturas}.

En cada ejercicio, se comienza proponiendo soluciones en lenguaje
natural y, a continuación, se exponen distintas demostraciones con Lean4
ordenadas desde las más detalladas a las más automáticas. Al final de
cada ejercicio hay un enlace para interactuar con sus soluciones en
\href{https://lean.math.hhu.de/}{Lean4 Web}.

Las soluciones del libro están en \href{https://github.com/jaalonso/Calculemus2}{este repositorio de GitHub}.

El libro se irá actualizando periódicamente con los nuevos ejercicios que se
proponen diariamente en \href{https://www.glc.us.es}{Calculemus}.

Este libro es una continuación de
\begin{itemize}
\item \href{https://raw.githubusercontent.com/jaalonso/DAO\_con\_Lean/master/DAO\_con\_Lean.pdf}{DAO (Demostración Asistida por Ordenador) con Lean} que es una
introducción a la demostración con Lean3 y
\item \href{https://raw.githubusercontent.com/jaalonso/Calculemus/master/Calculemus.pdf}{Calculemus (Vol. 1: Demostraciones con Isabelle/HOL y Lean3)} que es la
recopilación de la primera parte de los ejercicios del blog con
demostraciones en Isabelle/HOL y Lean3.
\end{itemize}

\chapter{Demostraciones de una propiedad de los números enteros}
\label{sec:orgfca12e4}

\section{∀ m n ∈ ℕ, Even n → Even (m * n)}
\label{sec:org9387a9b}
\begin{verbatim}
-- ---------------------------------------------------------------------
-- Demostrar que los productos de los números naturales por números
-- pares son pares.
-- ---------------------------------------------------------------------

-- Demostración en lenguaje natural
-- ================================

-- Si n es par, entonces (por la definición de `Even`) existe un k tal que
--    n = k + k         (1)
-- Por tanto,
--    mn = m(k + k)     (por (1))
--       = mk + mk      (por la propiedad distributiva)
-- Por consiguiente, mn es par.

-- Demostraciones en Lean4
-- =======================

import Mathlib.Data.Nat.Basic
import Mathlib.Data.Nat.Parity
import Mathlib.Tactic

open Nat

-- 1ª demostración
-- ===============

example : ∀ m n : ℕ, Even n → Even (m * n) := by
  rintro m n ⟨k, hk⟩
  use m * k
  rw [hk]
  ring

-- 2ª demostración
-- ===============

example : ∀ m n : ℕ, Even n → Even (m * n) := by
  rintro m n ⟨k, hk⟩
  use m * k
  rw [hk]
  rw [mul_add]

-- 3ª demostración
-- ===============

example : ∀ m n : ℕ, Even n → Even (m * n) := by
  rintro m n ⟨k, hk⟩
  use m * k
  rw [hk, mul_add]

-- 4ª demostración
-- ===============

example : ∀ m n : Nat, Even n → Even (m * n) := by
  rintro m n ⟨k, hk⟩; use m * k; rw [hk, mul_add]

-- 5ª demostración
-- ===============

example : ∀ m n : ℕ, Even n → Even (m * n) := by
  rintro m n ⟨k, hk⟩
  exact ⟨m * k, by rw [hk, mul_add]⟩

-- 6ª demostración
-- ===============

example : ∀ m n : Nat, Even n → Even (m * n) :=
fun m n ⟨k, hk⟩ ↦ ⟨m * k, by rw [hk, mul_add]⟩

-- 7ª demostración
-- ===============

example : ∀ m n : ℕ, Even n → Even (m * n) := by
  rintro m n ⟨k, hk⟩
  use m * k
  rw [hk]
  exact mul_add m k k

-- 8ª demostración
-- ===============

example : ∀ m n : ℕ, Even n → Even (m * n) := by
  intros m n hn
  unfold Even at *
  cases hn with
  | intro k hk =>
    use m * k
    rw [hk, mul_add]

-- 9ª demostración
-- ===============

example : ∀ m n : ℕ, Even n → Even (m * n) := by
  intros m n hn
  unfold Even at *
  cases hn with
  | intro k hk =>
    use m * k
    calc m * n
       = m * (k + k)   := by exact congrArg (HMul.hMul m) hk
     _ = m * k + m * k := by exact mul_add m k k

-- 10ª demostración
-- ================

example : ∀ m n : Nat, Even n → Even (m * n) := by
  intros; simp [*, parity_simps]
\end{verbatim}
Se puede interactuar con las pruebas anteriores en \href{https://lean.math.hhu.de/\#url=https://raw.githubusercontent.com/jaalonso/Calculemus2/main/src/El\_producto\_por\_un\_par\_es\_par.lean}{Lean 4 Web}.

\chapter{Propiedades elementales de los números reales}
\label{sec:org77143f1}

\section{En ℝ, (ab)c = b(ac)}
\label{sec:org074cb82}
\begin{verbatim}
-- ---------------------------------------------------------------------
-- Demostrar que los números reales tienen la siguiente propiedad
--    (a * b) * c = b * (a * c)
-- ---------------------------------------------------------------------

-- Demostración en lenguaje natural
-- ================================

-- Por la siguiente cadena de igualdades
--    (ab)c = (ba)c   [por la conmutativa]
--          = b(ac)   [por la asociativa]

-- Demostraciones con Lean4
-- ========================

import Mathlib.Tactic
import Mathlib.Data.Real.Basic

-- 1ª demostración
example
  (a b c : ℝ)
  : (a * b) * c = b * (a * c) :=
calc
  (a * b) * c = (b * a) * c := by rw [mul_comm a b]
            _ = b * (a * c) := by rw [mul_assoc b a c]

-- 2ª demostración
example (a b c : ℝ) : (a * b) * c = b * (a * c) :=
by
  rw [mul_comm a b]
  rw [mul_assoc b a c]

-- 3ª demostración
example (a b c : ℝ) : (a * b) * c = b * (a * c) :=
by ring
\end{verbatim}
Se puede interactuar con las pruebas anteriores en \href{https://lean.math.hhu.de/\#url=https://raw.githubusercontent.com/jaalonso/Calculemus2/main/src/Asociativa\_conmutativa\_de\_los\_reales.lean}{Lean 4 Web}.

\section{En ℝ, (cb)a = b(ac)}
\label{sec:orgd922ebb}
\begin{verbatim}
-- ---------------------------------------------------------------------
-- Demostrar que los números reales tienen la siguiente propiedad
--    (c * b) * a = b * (a * c)
-- ---------------------------------------------------------------------

-- Demostración en lenguaje natural
-- ================================

-- Por la siguiente cadena de igualdades:
--    (c * b) * a
--    = (b * c) * a    [por la conmutativa]
--    = b * (c * a)    [por la asociativa]
--    = b * (a * c)    [por la conmutativa]

-- Demostraciones con Lean4
-- ========================

import Mathlib.Tactic
import Mathlib.Data.Real.Basic

-- 1ª demostración
example
  (a b c : ℝ)
  : (c * b) * a = b * (a * c) :=
calc
  (c * b) * a
    = (b * c) * a := by rw [mul_comm c b]
  _ = b * (c * a) := by rw [mul_assoc]
  _ = b * (a * c) := by rw [mul_comm c a]

-- 2ª demostración
example
  (a b c : ℝ)
  : (c * b) * a = b * (a * c) :=
by
  rw [mul_comm c b]
  rw [mul_assoc]
  rw [mul_comm c a]

-- 3ª demostración
example
  (a b c : ℝ)
  : (c * b) * a = b * (a * c) :=
by ring
\end{verbatim}
Se puede interactuar con las pruebas anteriores en \href{https://lean.math.hhu.de/\#url=https://raw.githubusercontent.com/jaalonso/Calculemus2/main/src/(cb)a\_eq\_b(ac).lean}{Lean 4 Web}.

\section{En ℝ, a(bc) = b(ac)}
\label{sec:orgb4f58e1}
\begin{verbatim}
-- ---------------------------------------------------------------------
-- Demostrar que los números reales tienen la siguiente propiedad
--    a * (b * c) = b * (a * c)
-- ---------------------------------------------------------------------

-- Demostración en lenguaje natural
-- ================================

-- Por la siguiente cadena de igualdades:
--    a(bc)
--    = (ab)c    [por la asociativa]
--    = (ba)c    [por la conmutativa]
--    = b(ac)    [por la asociativa]

-- Demostraciones en Lean4
-- =======================

import Mathlib.Tactic
import Mathlib.Data.Real.Basic

-- 1ª demostración
example
  (a b c : ℝ) : a * (b * c) = b * (a * c) :=
calc
  a * (b * c)
    = (a * b) * c := by rw [←mul_assoc]
  _ = (b * a) * c := by rw [mul_comm a b]
  _ = b * (a * c) := by rw [mul_assoc]

-- 2ª demostración
example
  (a b c : ℝ) : a * (b * c) = b * (a * c) :=
by
  rw [←mul_assoc]
  rw [mul_comm a b]
  rw [mul_assoc]

-- 3ª demostración
example
  (a b c : ℝ) : a * (b * c) = b * (a * c) :=
by ring
\end{verbatim}
Se puede interactuar con las pruebas anteriores en \href{https://lean.math.hhu.de/\#url=https://raw.githubusercontent.com/jaalonso/Calculemus2/main/src/a(bc)\_eq\_b(ac).lean}{Lean 4 Web}.

\section{En ℝ, si ab = cd y e = f, entonces a(be) = c(df)}
\label{sec:org45a8cae}
\begin{verbatim}
-- ---------------------------------------------------------------------
-- Demostrar que si a, b, c, d, e y f son números reales tales que
--    a * b = c * d y
--    e = f,
-- entonces
--    a * (b * e) = c * (d * f)
-- ---------------------------------------------------------------------

-- Demostración en leguaje natural
-- ===============================

-- Por la siguiente cadena de igualdades
--    a(be)
--    = a(bf)    [por la segunda hipótesis]
--    = (ab)f    [por la asociativa]
--    = (cd)f    [por la primera hipótesis]
--    = c(df)    [por la asociativa]

-- Demostraciones en Lean4
-- =======================

import Mathlib.Tactic
import Mathlib.Data.Real.Basic

-- 1ª demostración
example
  (a b c d e f : ℝ)
  (h1 : a * b = c * d)
  (h2 : e = f)
  : a * (b * e) = c * (d * f) :=
calc
  a * (b * e)
    = a * (b * f) := by rw [h2]
  _ = (a * b) * f := by rw [←mul_assoc]
  _ = (c * d) * f := by rw [h1]
  _ = c * (d * f) := by rw [mul_assoc]

-- 2ª demostración
example
  (a b c d e f : ℝ)
  (h1 : a * b = c * d)
  (h2 : e = f)
  : a * (b * e) = c * (d * f) :=
by
  rw [h2]
  rw [←mul_assoc]
  rw [h1]
  rw [mul_assoc]

-- 3ª demostración
example
  (a b c d e f : ℝ)
  (h1 : a * b = c * d)
  (h2 : e = f)
  : a * (b * e) = c * (d * f) :=
by
  simp [*, ←mul_assoc]
\end{verbatim}
Se puede interactuar con las pruebas anteriores en \href{https://lean.math.hhu.de/\#url=https://raw.githubusercontent.com/jaalonso/Calculemus2/main/src/a(be)\_eq\_c(df).lean}{Lean 4 Web}.

\section{En ℝ, si bc = ef, entonces ((ab)c)d = ((ae)f)d}
\label{sec:org9ff6f4f}
\begin{verbatim}
-- ---------------------------------------------------------------------
-- Demostrar que si a, b, c, d, e y f son números reales tales que
--    b * c = e * f
-- entonces
--    ((a * b) * c) * d = ((a * e) * f) * d
-- ---------------------------------------------------------------------

-- Demostración en lenguaje natural
-- ================================

-- Por la siguiente cadena de igualdades
--    ((ab)c)d
--    = (a(bc))d    [por la asociativa]
--    = (a(ef))d    [por la hipótesis]
--    = ((ae)f)d    [por la asociativa]

-- Demostraciones con Lean4
-- ========================

import Mathlib.Data.Real.Basic
import Mathlib.Tactic

-- 1ª demostración
example
  (a b c d e f : ℝ)
  (h : b * c = e * f)
  : ((a * b) * c) * d = ((a * e) * f) * d :=
calc
  ((a * b) * c) * d
    = (a * (b * c)) * d := by rw [mul_assoc a]
  _ = (a * (e * f)) * d := by rw [h]
  _ = ((a * e) * f) * d := by rw [←mul_assoc a]

-- 2ª demostración
example
  (a b c d e f : ℝ)
  (h : b * c = e * f)
  : ((a * b) * c) * d = ((a * e) * f) * d :=
by
  rw [mul_assoc a]
  rw [h]
  rw [←mul_assoc a]

-- 3ª demostración
example
  (a b c d e f : ℝ)
  (h : b * c = e * f)
  : ((a * b) * c) * d = ((a * e) * f) * d :=
by
  rw [mul_assoc a, h, ←mul_assoc a]
\end{verbatim}
Se puede interactuar con las pruebas anteriores en \href{https://lean.math.hhu.de/\#url=https://raw.githubusercontent.com/jaalonso/Calculemus2/main/src/Si\_bc\_eq\_ef\_entonces\_((ab)c)d\_eq\_((ae)f)d.lean}{Lean 4 Web}.

\section{En ℝ, si c = ba-d y d = ab, entonces c = 0}
\label{sec:org31c1419}
\begin{verbatim}
-- ---------------------------------------------------------------------
-- Demostrar que si a, b, c y d son números reales tales que
--    c = b * a - d
--    d = a * b
-- entonces
--    c = 0
-- ---------------------------------------------------------------------

-- Demostración en lenguaje natural
-- ================================

-- Por la siguiente cadena de igualdades
--    c = ba - d     [por la primera hipótesis]
--      = ab - d     [por la conmutativa]
--      = ab - ab    [por la segunda hipótesis]
--      = 0

-- Demostraciones en Lean4
-- =======================

import Mathlib.Data.Real.Basic
import Mathlib.Tactic

-- 1ª demostración
example
  (a b c d : ℝ)
  (h1 : c = b * a - d)
  (h2 : d = a * b)
  : c = 0 :=
calc
  c = b * a - d     := by rw [h1]
  _ = a * b - d     := by rw [mul_comm]
  _ = a * b - a * b := by rw [h2]
  _ = 0             := by rw [sub_self]

-- 2ª demostración
example
  (a b c d : ℝ)
  (h1 : c = b * a - d)
  (h2 : d = a * b)
  : c = 0 :=
by
  rw [h1]
  rw [mul_comm]
  rw [h2]
  rw [sub_self]

-- 3ª demostración
example
  (a b c d : ℝ)
  (h1 : c = b * a - d)
  (h2 : d = a * b)
  : c = 0 :=
by
  rw [h1, mul_comm, h2, sub_self]
\end{verbatim}
Se puede interactuar con las pruebas anteriores en \href{https://lean.math.hhu.de/\#url=https://raw.githubusercontent.com/jaalonso/Calculemus2/main/src/Si\_c\_eq\_ba-d\_y\_d\_eq\_ab\_entonces\_c\_eq\_0.lean}{Lean 4 Web}.

\section{En ℝ, (a+b)(a+b) = aa+2ab+bb}
\label{sec:org7137cde}
\begin{verbatim}
-- ---------------------------------------------------------------------
-- Demostrar que si a y b son números reales, entonces
--    (a + b) * (a + b) = a * a + 2 * (a * b) + b * b
-- ---------------------------------------------------------------------

-- Demostración en lenguaje natural
-- ================================

-- Por la siguiente cadena de igualdades
--    (a + b)(a + b)
--    = (a + b)a + (a + b)b    [por la distributiva]
--    = aa + ba + (a + b)b     [por la distributiva]
--    = aa + ba + (ab + bb)    [por la distributiva]
--    = aa + ba + ab + bb      [por la asociativa]
--    = aa + (ba + ab) + bb    [por la asociativa]
--    = aa + (ab + ab) + bb    [por la conmutativa]
--    = aa + 2(ab) + bb        [por def. de doble]

-- Demostraciones con Lean4
-- ========================

import Mathlib.Data.Real.Basic
import Mathlib.Tactic

variable (a b c : ℝ)

-- 1ª demostración
example :
  (a + b) * (a + b) = a * a + 2 * (a * b) + b * b :=
calc
  (a + b) * (a + b)
    = (a + b) * a + (a + b) * b       := by rw [mul_add]
  _ = a * a + b * a + (a + b) * b     := by rw [add_mul]
  _ = a * a + b * a + (a * b + b * b) := by rw [add_mul]
  _ = a * a + b * a + a * b + b * b   := by rw [←add_assoc]
  _ = a * a + (b * a + a * b) + b * b := by rw [add_assoc (a * a)]
  _ = a * a + (a * b + a * b) + b * b := by rw [mul_comm b a]
  _ = a * a + 2 * (a * b) + b * b     := by rw [←two_mul]

-- 2ª demostración
example :
  (a + b) * (a + b) = a * a + 2 * (a * b) + b * b :=
calc
  (a + b) * (a + b)
    = a * a + b * a + (a * b + b * b) := by rw [mul_add, add_mul, add_mul]
  _ = a * a + (b * a + a * b) + b * b := by rw [←add_assoc, add_assoc (a * a)]
  _ = a * a + 2 * (a * b) + b * b     := by rw [mul_comm b a, ←two_mul]

-- 3ª demostración
example :
  (a + b) * (a + b) = a * a + 2 * (a * b) + b * b :=
calc
  (a + b) * (a + b)
    = a * a + b * a + (a * b + b * b) := by ring
  _ = a * a + (b * a + a * b) + b * b := by ring
  _ = a * a + 2 * (a * b) + b * b     := by ring

-- 4ª demostración
example :
  (a + b) * (a + b) = a * a + 2 * (a * b) + b * b :=
by ring

-- 5ª demostración
example :
  (a + b) * (a + b) = a * a + 2 * (a * b) + b * b :=
by
  rw [mul_add]
  rw [add_mul]
  rw [add_mul]
  rw [←add_assoc]
  rw [add_assoc (a * a)]
  rw [mul_comm b a]
  rw [←two_mul]

-- 6ª demostración
example :
  (a + b) * (a + b) = a * a + 2 * (a * b) + b * b :=
by
  rw [mul_add, add_mul, add_mul]
  rw [←add_assoc, add_assoc (a * a)]
  rw [mul_comm b a, ←two_mul]

-- 7ª demostración
example :
  (a + b) * (a + b) = a * a + 2 * (a * b) + b * b :=
by linarith
\end{verbatim}
Se puede interactuar con las pruebas anteriores en \href{https://lean.math.hhu.de/\#url=https://raw.githubusercontent.com/jaalonso/Calculemus2/main/src/(a+b)(a+b)\_eq\_aa+2ab+bb.lean}{Lean 4 Web}.

\section{En ℝ, (a+b)(c+d) = ac+ad+bc+bd}
\label{sec:orge26f481}
\begin{verbatim}
-- ---------------------------------------------------------------------
-- Demostrar que si a, b, c y d son números reales, entonces
--    (a + b) * (c + d) = a * c + a * d + b * c + b * d
-- ---------------------------------------------------------------------

-- Demostración en lenguaje natural
-- ================================

-- Por la siguiente cadena de igualdades
--    (a + b)(c + d)
--    = a(c + d) + b(c + d)    [por la distributiva]
--    = ac + ad + b(c + d)     [por la distributiva]
--    = ac + ad + (bc + bd)    [por la distributiva]
--    = ac + ad + bc + bd      [por la asociativa]

-- Demostraciones con Lean4
-- ========================

import Mathlib.Data.Real.Basic
import Mathlib.Tactic

variable (a b c d : ℝ)

-- 1ª demostración
example
  : (a + b) * (c + d) = a * c + a * d + b * c + b * d :=
calc
  (a + b) * (c + d)
    = a * (c + d) + b * (c + d)       := by rw [add_mul]
  _ = a * c + a * d + b * (c + d)     := by rw [mul_add]
  _ = a * c + a * d + (b * c + b * d) := by rw [mul_add]
  _ = a * c + a * d + b * c + b * d   := by rw [←add_assoc]

-- 2ª demostración
example
  : (a + b) * (c + d) = a * c + a * d + b * c + b * d :=
calc
  (a + b) * (c + d)
    = a * (c + d) + b * (c + d)       := by ring
  _ = a * c + a * d + b * (c + d)     := by ring
  _ = a * c + a * d + (b * c + b * d) := by ring
  _ = a * c + a * d + b * c + b * d   := by ring

-- 3ª demostración
example : (a + b) * (c + d) = a * c + a * d + b * c + b * d :=
by ring

-- 4ª demostración
example
  : (a + b) * (c + d) = a * c + a * d + b * c + b * d :=
by
   rw [add_mul]
   rw [mul_add]
   rw [mul_add]
   rw [← add_assoc]

-- 5ª demostración
example : (a + b) * (c + d) = a * c + a * d + b * c + b * d :=
by rw [add_mul, mul_add, mul_add, ←add_assoc]
\end{verbatim}
Se puede interactuar con las pruebas anteriores en \href{https://lean.math.hhu.de/\#url=https://raw.githubusercontent.com/jaalonso/Calculemus2/main/src/(a+b)(c+d)\_eq\_ac+ad+bc+bd.lean}{Lean 4 Web}.

\section{En ℝ, (a+b)(a-b) = a\^{}2-b\^{}2}
\label{sec:orgc67b42b}
\begin{verbatim}
-- ---------------------------------------------------------------------
-- Demostrar que si a y b son números reales, entonces
--    (a + b) * (a - b) = a^2 - b^2
-- ---------------------------------------------------------------------

-- Demostración en lenguaje natural
-- ================================

-- Por la siguiente cadena de igualdades:
--    (a + b)(a - b)
--    = a(a - b) + b(a - b)            [por la distributiva]
--    = (aa - ab) + b(a - b)           [por la distributiva]
--    = (a^2 - ab) + b(a - b)          [por def. de cuadrado]
--    = (a^2 - ab) + (ba - bb)         [por la distributiva]
--    = (a^2 - ab) + (ba - b^2)        [por def. de cuadrado]
--    = (a^2 + -(ab)) + (ba - b^2)     [por def. de resta]
--    = a^2 + (-(ab) + (ba - b^2))     [por la asociativa]
--    = a^2 + (-(ab) + (ba + -b^2))    [por def. de resta]
--    = a^2 + ((-(ab) + ba) + -b^2)    [por la asociativa]
--    = a^2 + ((-(ab) + ab) + -b^2)    [por la conmutativa]
--    = a^2 + (0 + -b^2)               [por def. de opuesto]
--    = (a^2 + 0) + -b^2               [por asociativa]
--    = a^2 + -b^2                     [por def. de cero]
--    = a^2 - b^2                      [por def. de resta]

-- Demostraciones con Lean4
-- ========================

import Mathlib.Data.Real.Basic
import Mathlib.Tactic

variable (a b : ℝ)

-- 1ª demostración
-- ===============

example : (a + b) * (a - b) = a^2 - b^2 :=
calc
  (a + b) * (a - b)
    = a * (a - b) + b * (a - b)         := by rw [add_mul]
  _ = (a * a - a * b) + b * (a - b)     := by rw [mul_sub]
  _ = (a^2 - a * b) + b * (a - b)       := by rw [← pow_two]
  _ = (a^2 - a * b) + (b * a - b * b)   := by rw [mul_sub]
  _ = (a^2 - a * b) + (b * a - b^2)     := by rw [← pow_two]
  _ = (a^2 + -(a * b)) + (b * a - b^2)  := by ring
  _ = a^2 + (-(a * b) + (b * a - b^2))  := by rw [add_assoc]
  _ = a^2 + (-(a * b) + (b * a + -b^2)) := by ring
  _ = a^2 + ((-(a * b) + b * a) + -b^2) := by rw [← add_assoc
                                              (-(a * b)) (b * a) (-b^2)]
  _ = a^2 + ((-(a * b) + a * b) + -b^2) := by rw [mul_comm]
  _ = a^2 + (0 + -b^2)                  := by rw [neg_add_self (a * b)]
  _ = (a^2 + 0) + -b^2                  := by rw [← add_assoc]
  _ = a^2 + -b^2                        := by rw [add_zero]
  _ = a^2 - b^2                         := by linarith

-- 2ª demostración
-- ===============

example : (a + b) * (a - b) = a^2 - b^2 :=
calc
  (a + b) * (a - b)
    = a * (a - b) + b * (a - b)         := by ring
  _ = (a * a - a * b) + b * (a - b)     := by ring
  _ = (a^2 - a * b) + b * (a - b)       := by ring
  _ = (a^2 - a * b) + (b * a - b * b)   := by ring
  _ = (a^2 - a * b) + (b * a - b^2)     := by ring
  _ = (a^2 + -(a * b)) + (b * a - b^2)  := by ring
  _ = a^2 + (-(a * b) + (b * a - b^2))  := by ring
  _ = a^2 + (-(a * b) + (b * a + -b^2)) := by ring
  _ = a^2 + ((-(a * b) + b * a) + -b^2) := by ring
  _ = a^2 + ((-(a * b) + a * b) + -b^2) := by ring
  _ = a^2 + (0 + -b^2)                  := by ring
  _ = (a^2 + 0) + -b^2                  := by ring
  _ = a^2 + -b^2                        := by ring
  _ = a^2 - b^2                         := by ring

-- 3ª demostración
-- ===============

example : (a + b) * (a - b) = a^2 - b^2 :=
by ring

-- 4ª demostración
-- ===============

-- El lema anterior es
lemma aux : (a + b) * (c + d) = a * c + a * d + b * c + b * d :=
by ring

-- La demostración es
example : (a + b) * (a - b) = a^2 - b^2 :=
by
  rw [sub_eq_add_neg]
  rw [aux]
  rw [mul_neg]
  rw [add_assoc (a * a)]
  rw [mul_comm b a]
  rw [neg_add_self]
  rw [add_zero]
  rw [← pow_two]
  rw [mul_neg]
  rw [← pow_two]
  rw [← sub_eq_add_neg]
\end{verbatim}
Se puede interactuar con las pruebas anteriores en \href{https://lean.math.hhu.de/\#url=https://raw.githubusercontent.com/jaalonso/Calculemus2/main/src/(a+b)(a-b)\_eq\_aa-bb.lean}{Lean 4 Web}.

\section{En ℝ, si c = da+b y b = ad, entonces c = 2ad}
\label{sec:orgf2d5dd3}
\begin{verbatim}
-- ---------------------------------------------------------------------
-- Demostrar que si a, b, c y d son números reales tales que
--    c = d * a + b
--    b = a * d
-- entonces
--    c = 2 * a * d
-- ---------------------------------------------------------------------

-- Demostración en lenguaje natural
-- ================================

-- Por la siguiente cadena de igualdades
--    c = da + b     [por la primera hipótesis]
--      = da + ad    [por la segunda hipótesis]
--      = ad + ad    [por la conmutativa]
--      = 2(ad)      [por la def. de doble]
--      = 2ad        [por la asociativa]

-- Demostraciones con Lean4
-- ========================

import Mathlib.Data.Real.Basic
import Mathlib.Tactic

variable (a b c d : ℝ)

-- 1ª demostración
example
  (h1 : c = d * a + b)
  (h2 : b = a * d)
  : c = 2 * a * d :=
calc
  c = d * a + b     := by rw [h1]
  _ = d * a + a * d := by rw [h2]
  _ = a * d + a * d := by rw [mul_comm d a]
  _ = 2 * (a * d)   := by rw [← two_mul (a * d)]
  _ = 2 * a * d     := by rw [mul_assoc]

-- 2ª demostración
example
  (h1 : c = d * a + b)
  (h2 : b = a * d)
  : c = 2 * a * d :=
by
  rw [h2] at h1
  clear h2
  rw [mul_comm d a] at h1
  rw [← two_mul (a*d)] at h1
  rw [← mul_assoc 2 a d] at h1
  exact h1

-- 3ª demostración
example
  (h1 : c = d * a + b)
  (h2 : b = a * d)
  : c = 2 * a * d :=
by rw [h1, h2, mul_comm d a, ← two_mul (a * d), mul_assoc]

-- 4ª demostración
example
  (h1 : c = d * a + b)
  (h2 : b = a * d)
  : c = 2 * a * d :=
by
  rw [h1]
  rw [h2]
  ring

-- 5ª demostración
example
  (h1 : c = d * a + b)
  (h2 : b = a * d)
  : c = 2 * a * d :=
by
  rw [h1, h2]
  ring

-- 6ª demostración
example
  (h1 : c = d * a + b)
  (h2 : b = a * d)
  : c = 2 * a * d :=
by rw [h1, h2] ; ring

-- 7ª demostración
example
  (h1 : c = d * a + b)
  (h2 : b = a * d)
  : c = 2 * a * d :=
by linarith
\end{verbatim}
Se puede interactuar con las pruebas anteriores en \href{https://lean.math.hhu.de/\#url=https://raw.githubusercontent.com/jaalonso/Calculemus2/main/src/Si\_c\_eq\_da+b\_y\_b\_eq\_ad\_entonces\_c\_eq\_2ad.lean}{Lean 4 Web}.

\section{En ℝ, si a+b = c, entonces (a+b)(a+b) = ac+bc}
\label{sec:orgcf6ca04}
\begin{verbatim}
-- ---------------------------------------------------------------------
-- Demostrar que si a, b y c son números reales tales que
--    a + b = c,
-- entonces
--    (a + b) * (a + b) = a * c + b * c
-- ---------------------------------------------------------------------

-- Demostración en lenguaje natural
-- ================================

-- Por la siguiente cadena de igualdades
--    (a + b)(a + b)
--    = (a + b)c        [por la hipótesis]
--    = ac + bc         [por la distributiva]

-- Demostraciones con Lean4
-- ========================

import Mathlib.Data.Real.Basic
import Mathlib.Tactic

variable (a b c : ℝ)

-- 1ª demostración
example
  (h : a + b = c)
  : (a + b) * (a + b) = a * c + b * c :=
calc
  (a + b) * (a + b)
    = (a + b) * c   := by exact congrArg (HMul.hMul (a + b)) h
  _ = a * c + b * c := by rw [add_mul]

-- 2ª demostración
example
  (h : a + b = c)
  : (a + b) * (a + b) = a * c + b * c :=
by
  nth_rewrite 2 [h]
  rw [add_mul]
\end{verbatim}
Se puede interactuar con las pruebas anteriores en \href{https://lean.math.hhu.de/\#url=https://raw.githubusercontent.com/jaalonso/Calculemus2/main/src/Sia+b\_eq\_c\_entonces\_(a+b)(a+b)\_eq\_ac+bc.lean}{Lean 4 Web}.

\chapter{Propiedades elementales de los anillos}
\label{sec:org0189707}

\section{Si R es un anillo y a ∈ R, entonces a + 0 = a}
\label{sec:org8dbdd90}
\begin{verbatim}
-- ---------------------------------------------------------------------
-- Demostrar en Lean4 que si R es un anillo, entonces
--    ∀ a : R, a + 0 = a
-- ---------------------------------------------------------------------

-- Demostración en lenguaje natural
-- ================================

-- Por la siguiente cadena de igualdades
--    a + 0 = 0 + a    [por la conmutativa de la suma]
--          = a        [por el axioma del cero por la izquierda]

import Mathlib.Algebra.Ring.Defs

variable {R : Type _} [Ring R]
variable (a : R)

-- Demostraciones con Lean4
-- ========================

-- 1ª demostración
example : a + 0 = a :=
calc a + 0
     = 0 + a := by rw [add_comm]
   _ = a     := by rw [zero_add]

-- 2ª demostración
example : a + 0 = a :=
by
  rw [add_comm]
  rw [zero_add]

-- 3ª demostración
example : a + 0 = a :=
by rw [add_comm, zero_add]

-- 4ª demostración
example : a + 0 = a :=
by exact add_zero a

-- 5ª demostración
example : a + 0 = a :=
  add_zero a

-- 5ª demostración
example : a + 0 = a :=
by simp
\end{verbatim}
Se puede interactuar con las pruebas anteriores en \href{https://lean.math.hhu.de/\#url=https://raw.githubusercontent.com/jaalonso/Calculemus2/main/src/Suma\_con\_cero.lean}{Lean 4 Web}.

\section{Si R es un anillo y a ∈ R, entonces a + -a = 0}
\label{sec:orgae8a789}
\begin{verbatim}
-- ---------------------------------------------------------------------
-- Demostrar en Lean4 que si R es un anillo, entonces
--    ∀ a : R, a + -a = 0
-- ---------------------------------------------------------------------

-- Demostración en lenguaje natural
-- ================================

-- Por la siguiente cadena de igualdades
--    a + -a = -a + a    [por la conmutativa de la suma]
--           = 0         [por el axioma de inverso por la izquierda]

import Mathlib.Algebra.Ring.Defs

variable {R : Type _} [Ring R]
variable (a : R)

-- 1ª demostración
-- ===============

example : a + -a = 0 :=
calc a + -a = -a + a := by rw [add_comm]
          _ = 0      := by rw [add_left_neg]

-- 2ª demostración
-- ===============

example : a + -a = 0 :=
by
  rw [add_comm]
  rw [add_left_neg]

-- 3ª demostración
-- ===============

example : a + -a = 0 :=
by rw [add_comm, add_left_neg]

-- 4ª demostración
-- ===============

example : a + -a = 0 :=
by exact add_neg_self a

-- 5ª demostración
-- ===============

example : a + -a = 0 :=
  add_neg_self a

-- 6ª demostración
-- ===============

example : a + -a = 0 :=
by simp
\end{verbatim}
Se puede interactuar con las pruebas anteriores en \href{https://lean.math.hhu.de/\#url=https://raw.githubusercontent.com/jaalonso/Calculemus2/main/src/Suma\_con\_opuesto.lean}{Lean 4 Web}.

\section{Si R es un anillo y a, b ∈ R, entonces -a + (a + b) = b}
\label{sec:org551b6f6}
\begin{verbatim}
-- ---------------------------------------------------------------------
-- Demostrar en Lean4 que si R es un anillo, entonces
--    ∀ a, b : R, -a + (a + b) = b
-- ---------------------------------------------------------------------

-- Demostración en lenguaje natural
-- ================================

-- Por la siguiente cadena de igualdades
--    -a + (a + b) = (-a + a) + b [por la asociativa]
--                 = 0 + b        [por inverso por la izquierda]
--                 = b            [por cero por la izquierda]

import Mathlib.Algebra.Ring.Defs

variable {R : Type _} [Ring R]
variable (a b : R)

-- Demostraciones con Lean4
-- ========================

-- 1ª demostración
example : -a + (a + b) = b :=
calc -a + (a + b) = (-a + a) + b := by rw [← add_assoc]
                _ = 0 + b        := by rw [add_left_neg]
                _ = b            := by rw [zero_add]

-- 2ª demostración
example : -a + (a + b) = b :=
by
  rw [←add_assoc]
  rw [add_left_neg]
  rw [zero_add]

-- 3ª demostración
example : -a + (a + b) = b :=
by rw [←add_assoc, add_left_neg, zero_add]

-- 4ª demostración
example : -a + (a + b) = b :=
by exact neg_add_cancel_left a b

-- 5ª demostración
example : -a + (a + b) = b :=
  neg_add_cancel_left a b

-- 6ª demostración
example : -a + (a + b) = b :=
by simp
\end{verbatim}
Se puede interactuar con las pruebas anteriores en \href{https://lean.math.hhu.de/\#url=https://raw.githubusercontent.com/jaalonso/Calculemus2/main/src/Opuesto\_se\_cancela\_con\_la\_suma\_por\_la\_izquierda.lean}{Lean 4 Web}.

\section{Si R es un anillo y a, b ∈ R, entonces (a + b) + -b = a}
\label{sec:orgc1e6779}
\begin{verbatim}
-- ---------------------------------------------------------------------
-- Demostrar en Lean4 que si R es un anillo, entonces
--    ∀ a, b : R, (a + b) + -b = a
-- ---------------------------------------------------------------------

-- Demostración en lenguaje natural
-- ================================

-- Por la siguiente cadena de igualdades
--    (a + b) + -b = a + (b + -b)    [por la asociativa]
--               _ = a + 0           [por suma con opuesto]
--               _ = a               [por suma con cero]

-- Demostraciones con Lean4
-- ========================

import Mathlib.Algebra.Ring.Defs

variable {R : Type _} [Ring R]
variable (a b : R)

-- 1ª demostración
example : (a + b) + -b = a :=
calc
  (a + b) + -b = a + (b + -b) := by rw [add_assoc]
             _ = a + 0        := by rw [add_right_neg]
             _ = a            := by rw [add_zero]

-- 2ª demostración
example : (a + b) + -b = a :=
by
  rw [add_assoc]
  rw [add_right_neg]
  rw [add_zero]

-- 3ª demostración
example : (a + b) + -b = a :=
by rw [add_assoc, add_right_neg, add_zero]

-- 4ª demostración
example : (a + b) + -b = a :=
  add_neg_cancel_right a b

-- 5ª demostración
example : (a + b) + -b = a :=
  add_neg_cancel_right _ _

-- 6ª demostración
example : (a + b) + -b = a :=
by simp
\end{verbatim}
Se puede interactuar con las pruebas anteriores en \href{https://lean.math.hhu.de/\#url=https://raw.githubusercontent.com/jaalonso/Calculemus2/main/src/Opuesto\_se\_cancela\_con\_la\_suma\_por\_la\_derecha.lean}{Lean 4 Web}.

\section{Si R es un anillo y a, b, c ∈ R tales que a+b=a+c, entonces b=c}
\label{sec:org9d23ce7}
\begin{verbatim}
-- ---------------------------------------------------------------------
-- Demostrar que si R es un anillo y a, b, c ∈ R tales que
--    a + b = a + c
-- entonces
--    b = c
-- ----------------------------------------------------------------------

-- Demostraciones en lenguaje natural (LN)
-- ======================================

-- 1ª demostración en LN
-- =====================

-- Por la siguiente cadena de igualdades
--    b = 0 + b           [por suma con cero]
--      = (-a + a) + b    [por suma con opuesto]
--      = -a + (a + b)    [por asociativa]
--      = -a + (a + c)    [por hipótesis]
--      = (-a + a) + c    [por asociativa]
--      = 0 + c           [por suma con opuesto]
--      = c               [por suma con cero]

-- 2ª demostración en LN
-- =====================

-- Por la siguiente cadena de implicaciones
--    a + b = a + c
--    ==> -a + (a + b) = -a + (a + c)     [sumando -a]
--    ==>  (-a + a) + b = (-a + a) + c    [por la asociativa]
--    ==>  0 + b = 0 + b                  [suma con opuesto]
--    ==>  b = c                          [suma con cero]

-- 3ª demostración en LN
-- =====================

-- Por la siguiente cadena de igualdades
--    b = -a + (a + b)
--      = -a + (a + c)   [por la hipótesis]
--      = c

-- Demostraciones con Lean4
-- ========================

import Mathlib.Algebra.Ring.Defs
import Mathlib.Tactic

variable {R : Type _} [Ring R]
variable {a b c : R}

-- 1ª demostración
example
  (h : a + b = a + c)
  : b = c :=
calc
  b = 0 + b        := by rw [zero_add]
  _ = (-a + a) + b := by rw [add_left_neg]
  _ = -a + (a + b) := by rw [add_assoc]
  _ = -a + (a + c) := by rw [h]
  _ = (-a + a) + c := by rw [←add_assoc]
  _ = 0 + c        := by rw [add_left_neg]
  _ = c            := by rw [zero_add]

-- 2ª demostración
example
  (h : a + b = a + c)
  : b = c :=
by
  have h1 : -a + (a + b) = -a + (a + c) :=
    congrArg (HAdd.hAdd (-a)) h
  clear h
  rw [← add_assoc] at h1
  rw [add_left_neg] at h1
  rw [zero_add] at h1
  rw [← add_assoc] at h1
  rw [add_left_neg] at h1
  rw [zero_add] at h1
  exact h1

-- 3ª demostración
example
  (h : a + b = a + c)
  : b = c :=
calc
  b = -a + (a + b) := by rw [neg_add_cancel_left a b]
  _ = -a + (a + c) := by rw [h]
  _ = c            := by rw [neg_add_cancel_left]

-- 4ª demostración
example
  (h : a + b = a + c)
  : b = c :=
by
  rw [← neg_add_cancel_left a b]
  rw [h]
  rw [neg_add_cancel_left]

-- 5ª demostración
example
  (h : a + b = a + c)
  : b = c :=
by
  rw [← neg_add_cancel_left a b, h, neg_add_cancel_left]

-- 6ª demostración
example
  (h : a + b = a + c)
  : b = c :=
add_left_cancel h
\end{verbatim}
Se puede interactuar con las pruebas anteriores en \href{https://lean.math.hhu.de/\#url=https://raw.githubusercontent.com/jaalonso/Calculemus2/main/src/Cancelativa\_izquierda.lean}{Lean 4 Web}.

\section{Si R es un anillo y a, b, c ∈ R tales que a+b=c+b, entonces a=c}
\label{sec:org05fe953}
\begin{verbatim}
-- ---------------------------------------------------------------------
-- Demostrar que si R es un anillo y a, b, c ∈ R tales que
--    a + b = c + b
-- entonces
--    a = c
-- ----------------------------------------------------------------------

-- Demostraciones en lenguaje natural (LN)
-- =======================================

-- 1ª demostración en LN
-- =====================

-- Por la siguiente cadena de igualdades
--    a = a + 0           [por suma con cero]
--      = a + (b + -b)    [por suma con opuesto]
--      = (a + b) + -b    [por asociativa]
--      = (c + b) + -b    [por hipótesis]
--      = c + (b + -b)    [por asociativa]
--      = c + 0           [por suma con opuesto]
--      = c               [por suma con cero]

-- 2ª demostración en LN
-- =====================

-- Por la siguiente cadena de igualdades
--    a = (a + b) + -b
--      = (c + b) + -b    [por hipótesis]
--      = c

-- Demostraciones con Lean4
-- ========================

import Mathlib.Algebra.Ring.Defs
import Mathlib.Tactic

variable {R : Type _} [Ring R]
variable {a b c : R}

-- 1ª demostración con Lean4
-- =========================

example
  (h : a + b = c + b)
  : a = c :=
calc
  a = a + 0        := by rw [add_zero]
  _ = a + (b + -b) := by rw [add_right_neg]
  _ = (a + b) + -b := by rw [add_assoc]
  _ = (c + b) + -b := by rw [h]
  _ = c + (b + -b) := by rw [← add_assoc]
  _ = c + 0        := by rw [← add_right_neg]
  _ = c            := by rw [add_zero]

-- 2ª demostración con Lean4
-- =========================

example
  (h : a + b = c + b)
  : a = c :=
calc
  a = (a + b) + -b := (add_neg_cancel_right a b).symm
  _ = (c + b) + -b := by rw [h]
  _ = c            := add_neg_cancel_right c b

-- 3ª demostración con Lean4
-- =========================

example
  (h : a + b = c + b)
  : a = c :=
by
  rw [← add_neg_cancel_right a b]
  rw [h]
  rw [add_neg_cancel_right]

-- 4ª demostración con Lean4
-- =========================

example
  (h : a + b = c + b)
  : a = c :=
by
  rw [← add_neg_cancel_right a b, h, add_neg_cancel_right]

-- 5ª demostración con Lean4
-- =========================

example
  (h : a + b = c + b)
  : a = c :=
add_right_cancel h
\end{verbatim}
Se puede interactuar con las pruebas anteriores en \href{https://lean.math.hhu.de/\#url=https://raw.githubusercontent.com/jaalonso/Calculemus2/main/src/Cancelativa\_derecha.lean}{Lean 4 Web}.

\section{Si R es un anillo y a ∈ R, entonces a.0 = 0}
\label{sec:org7b9fdb6}
\begin{verbatim}
-- ---------------------------------------------------------------------
-- Demostrar que si R es un anillo y a ∈ R, entonces
--    a * 0 = 0
-- ----------------------------------------------------------------------

-- Demostración en lenguaje natural
-- ================================

-- Basta aplicar la propiedad cancelativa a
--    a.0 + a.0 = a.0 + 0
-- que se demuestra mediante la siguiente cadena de igualdades
--    a.0 + a.0 = a.(0 + 0)    [por la distributiva]
--              = a.0          [por suma con cero]
--              = a.0 + 0      [por suma con cero]

-- Demostraciones con Lean4
-- ========================

import Mathlib.Algebra.Ring.Defs
import Mathlib.Tactic

variable {R : Type _} [Ring R]
variable (a : R)

-- 1ª demostración
-- ===============

example : a * 0 = 0 :=
by
  have h : a * 0 + a * 0 = a * 0 + 0 :=
    calc a * 0 + a * 0 = a * (0 + 0) := by rw [mul_add a 0 0]
                     _ = a * 0       := by rw [add_zero 0]
                     _ = a * 0 + 0   := by rw [add_zero (a * 0)]
  rw [add_left_cancel h]

-- 2ª demostración
-- ===============

example : a * 0 = 0 :=
by
  have h : a * 0 + a * 0 = a * 0 + 0 :=
    calc a * 0 + a * 0 = a * (0 + 0) := by rw [← mul_add]
                     _ = a * 0       := by rw [add_zero]
                     _ = a * 0 + 0   := by rw [add_zero]
  rw [add_left_cancel h]

-- 3ª demostración
-- ===============

example : a * 0 = 0 :=
by
  have h : a * 0 + a * 0 = a * 0 + 0 :=
    by rw [← mul_add, add_zero, add_zero]
  rw [add_left_cancel h]

-- 4ª demostración
-- ===============

example : a * 0 = 0 :=
by
  have : a * 0 + a * 0 = a * 0 + 0 :=
    calc a * 0 + a * 0 = a * (0 + 0) := by simp
                     _ = a * 0       := by simp
                     _ = a * 0 + 0   := by simp
  simp

-- 5ª demostración
-- ===============

example : a * 0 = 0 :=
  mul_zero a

-- 6ª demostración
-- ===============

example : a * 0 = 0 :=
by simp
\end{verbatim}
Se puede interactuar con las pruebas anteriores en \href{https://lean.math.hhu.de/\#url=https://raw.githubusercontent.com/jaalonso/Calculemus2/main/src/Multiplicacion\_por\_cero.lean}{Lean 4 Web}.

\section{Si R es un anillo y a ∈ R, entonces 0.a = 0}
\label{sec:org6932844}
\begin{verbatim}
-- ---------------------------------------------------------------------
-- Demostrar que si R es un anillo y a ∈ R, entonces
--    0 * a = 0
-- ----------------------------------------------------------------------

-- Demostración en lenguaje natural
-- ================================

-- Basta aplicar la propiedad cancelativa a
--    0.a + 0.a = 0.a + 0
-- que se demuestra mediante la siguiente cadena de igualdades
--    0.a + 0.a = (0 + 0).a    [por la distributiva]
--              = 0.a          [por suma con cero]
--              = 0.a + 0      [por suma con cero]

-- Demostraciones con Lean4
-- ========================

import Mathlib.Algebra.Ring.Defs
import Mathlib.Tactic

variable {R : Type _} [Ring R]
variable (a : R)

-- 1ª demostración
example : 0 * a = 0 :=
by
  have h : 0 * a + 0 * a = 0 * a + 0 :=
    calc 0 * a + 0 * a = (0 + 0) * a := by rw [add_mul]
                     _ = 0 * a       := by rw [add_zero]
                     _ = 0 * a + 0   := by rw [add_zero]
  rw [add_left_cancel h]

-- 2ª demostración
example : 0 * a = 0 :=
by
  have h : 0 * a + 0 * a = 0 * a + 0 :=
    by rw [←add_mul, add_zero, add_zero]
  rw [add_left_cancel h]

-- 3ª demostración
example : 0 * a = 0 :=
by
  have : 0 * a + 0 * a = 0 * a + 0 :=
    calc 0 * a + 0 * a = (0 + 0) * a := by simp
                     _ = 0 * a       := by simp
                     _ = 0 * a + 0   := by simp
  simp

-- 4ª demostración
example : 0 * a = 0 :=
by
  have : 0 * a + 0 * a = 0 * a + 0 := by simp
  simp

-- 5ª demostración
example : 0 * a = 0 :=
by simp

-- 6ª demostración
example : 0 * a = 0 :=
zero_mul a
\end{verbatim}
Se puede interactuar con las pruebas anteriores en \href{https://lean.math.hhu.de/\#url=https://raw.githubusercontent.com/jaalonso/Calculemus2/main/src/Multiplicacion\_por\_cero\_izquierda.lean}{Lean 4 Web}.

\section{Si R es un anillo y a, b ∈ R tales que a+b=0, entonces -a=b}
\label{sec:org334f637}
\begin{verbatim}
-- ---------------------------------------------------------------------
-- Demostrar que si es un anillo y a, b ∈ R tales que
--    a + b = 0
-- entonces
--    -a = b
-- ----------------------------------------------------------------------

-- Demostraciones en lenguaje natural (LN)
-- =======================================

-- 1ª demostración en LN
-- ---------------------

-- Por la siguiente cadena de igualdades
--    -a = -a + 0          [por suma cero]
--       = -a + (a + b)    [por hipótesis]
--       = b               [por cancelativa]

-- 2ª demostración en LN
-- ---------------------

-- Sumando -a a ambos lados de la hipótesis, se tiene
--    -a + (a + b) = -a + 0
-- El término de la izquierda se reduce a b (por la cancelativa) y el de
-- la derecha a -a (por la suma con cero). Por tanto, se tiene
--     b = -a
-- Por la simetría de la igualdad, se tiene
--     -a = b

-- Demostraciones con Lean 4
-- =========================

import Mathlib.Algebra.Ring.Defs
import Mathlib.Tactic

variable {R : Type _} [Ring R]
variable {a b : R}

-- 1ª demostración (basada en la 1º en LN)
example
  (h : a + b = 0)
  : -a = b :=
calc
  -a = -a + 0       := by rw [add_zero]
   _ = -a + (a + b) := by rw [h]
   _ = b            := by rw [neg_add_cancel_left]

-- 2ª demostración (basada en la 1º en LN)
example
  (h : a + b = 0)
  : -a = b :=
calc
  -a = -a + 0       := by simp
   _ = -a + (a + b) := by rw [h]
   _ = b            := by simp

-- 3ª demostración (basada en la 2º en LN)
example
  (h : a + b = 0)
  : -a = b :=
by
  have h1 : -a + (a + b) = -a + 0 := congrArg (HAdd.hAdd (-a)) h
  have h2 : -a + (a + b) = b := neg_add_cancel_left a b
  have h3 : -a + 0 = -a := add_zero (-a)
  rw [h2, h3] at h1
  exact h1.symm

-- 4ª demostración
example
  (h : a + b = 0)
  : -a = b :=
neg_eq_iff_add_eq_zero.mpr h
\end{verbatim}
Se puede interactuar con las pruebas anteriores en \href{https://lean.math.hhu.de/\#url=https://raw.githubusercontent.com/jaalonso/Calculemus2/main/src/Opuesto\_ig\_si\_suma\_ig\_cero.lean}{Lean 4 Web}.

\section{Si R es un anillo y a, b ∈ R tales que a+b=0, entonces a=-b}
\label{sec:org52279c7}
\begin{verbatim}
-- ---------------------------------------------------------------------
-- Demostrar que si R es un anillo y a, b ∈ R tales que
--    a + b = 0
-- entonces
--    a = -b
-- ----------------------------------------------------------------------

-- Demostraciones en lenguaje natural (LN)
-- =======================================

-- 1ª demostración en LN
-- ---------------------

-- Por la siguiente cadena de igualdades
--    a = (a + b) + -b    [por la concelativa]
--      = 0 + -b          [por la hipótesis]
--      = -b              [por la suma con cero]

-- 2ª demostración en LN
-- ---------------------

-- Sumando -a a ambos lados de la hipótesis, se tiene
--    (a + b) + -b = 0 + -b
-- El término de la izquierda se reduce a a (por la cancelativa) y el de
-- la derecha a -b (por la suma con cero). Por tanto, se tiene
--     a = -b

-- Demostraciones con Lean4
-- ========================

import Mathlib.Algebra.Ring.Defs
import Mathlib.Tactic

variable {R : Type _} [Ring R]
variable {a b : R}

-- 1ª demostración (basada en la 1ª en LN)
example
  (h : a + b = 0)
  : a = -b :=
calc
  a = (a + b) + -b := by rw [add_neg_cancel_right]
  _ = 0 + -b       := by rw [h]
  _ = -b           := by rw [zero_add]

-- 2ª demostración (basada en la 1ª en LN)
example
  (h : a + b = 0)
  : a = -b :=
calc
  a = (a + b) + -b := by simp
  _ = 0 + -b       := by rw [h]
  _ = -b           := by simp

-- 3ª demostración (basada en la 1ª en LN)
example
  (h : a + b = 0)
  : a = -b :=
by
  have h1 : (a + b) + -b = 0 + -b := by rw [h]
  have h2 : (a + b) + -b = a := add_neg_cancel_right a b
  have h3 : 0 + -b = -b := zero_add (-b)
  rwa [h2, h3] at h1

-- 4ª demostración
example
  (h : a + b = 0)
  : a = -b :=
add_eq_zero_iff_eq_neg.mp h
\end{verbatim}
Se puede interactuar con las pruebas anteriores en \href{https://lean.math.hhu.de/\#url=https://raw.githubusercontent.com/jaalonso/Calculemus2/main/src/Ig\_opuesto\_si\_suma\_ig\_cero.lean}{Lean 4 Web}.

\section{Si R es un anillo, entonces -0 = 0}
\label{sec:orgb98a613}
\begin{verbatim}
-- ---------------------------------------------------------------------
-- Demostrar que si R es un anillo, entonces
--    -0 = 0
-- ----------------------------------------------------------------------

-- Demostraciones en lenguaje natural (LN)
-- =======================================

-- 1ª demostración en LN
-- =====================

-- Por la suma con cero se tiene
--    0 + 0 = 0
-- Aplicándole la propiedad
--    ∀ a b ∈ R, a + b = 0 → -a = b
-- se obtiene
--    -0 = 0

-- 2ª demostración en LN
-- =====================

-- Puesto que
--    ∀ a b ∈ R, a + b = 0 → -a = b
-- basta demostrar que
--    0 + 0 = 0
-- que es cierta por la suma con cero.

-- Demostraciones con Lean4
-- ========================

import Mathlib.Algebra.Ring.Defs
import Mathlib.Tactic

variable {R : Type _} [Ring R]

-- 1ª demostración (basada en la 1ª en LN)
example : (-0 : R) = 0 :=
by
  have h1 : (0 : R) + 0 = 0 := add_zero 0
  show (-0 : R) = 0
  exact neg_eq_of_add_eq_zero_left h1

-- 2ª demostración (basada en la 2ª en LN)
example : (-0 : R) = 0 :=
by
  apply neg_eq_of_add_eq_zero_left
  rw [add_zero]

-- 3ª demostración
example : (-0 : R) = 0 :=
  neg_zero

-- 4ª demostración
example : (-0 : R) = 0 :=
by simp
\end{verbatim}
Se puede interactuar con las pruebas anteriores en \href{https://lean.math.hhu.de/\#url=https://raw.githubusercontent.com/jaalonso/Calculemus2/main/src/Opuesto\_del\_cero.lean}{Lean 4 Web}.

\section{Si R es un anillo y a ∈ R, entonces -(-a) = a}
\label{sec:org416c33f}
\begin{verbatim}
-- ---------------------------------------------------------------------
-- Demostrar que si R es un anillo y a ∈ R, entonces
--     -(-a) = a
-- ----------------------------------------------------------------------

-- Demostración en lenguaje natural
-- ================================

-- Es consecuencia de las siguiente propiedades demostradas en
-- ejercicios anteriores:
--    ∀ a b ∈ R, a + b = 0 → -a = b
--    ∀ a ∈ R, -a + a = 0

-- Demostraciones con Lean4
-- ========================

import Mathlib.Algebra.Ring.Defs
import Mathlib.Tactic

variable {R : Type _} [Ring R]
variable {a : R}

-- 1ª demostración
example : -(-a) = a :=
by
  have h1 : -a + a = 0 := add_left_neg a
  show -(-a) = a
  exact neg_eq_of_add_eq_zero_right h1

-- 2ª demostración
example : -(-a) = a :=
by
  apply neg_eq_of_add_eq_zero_right
  rw [add_left_neg]

-- 3ª demostración
example : -(-a) = a :=
neg_neg a

-- 4ª demostración
example : -(-a) = a :=
by simp
\end{verbatim}
Se puede interactuar con las pruebas anteriores en \href{https://lean.math.hhu.de/\#url=https://raw.githubusercontent.com/jaalonso/Calculemus2/main/src/Opuesto\_del\_opuesto.lean}{Lean 4 Web}.

\input Calculemus2_bibliografia

\end{document}
