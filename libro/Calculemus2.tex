\input Calculemus2_preambulo

\chapter{Introducción}
\label{sec:orgc5889ca}

Este libro es una recopilación de los ejercicios de demostración con
Lean4 que se han ido publicando, desde el 10 de julio de 20023, en el
blog \href{https://www.glc.us.es}{Calculemus}.

La ordenación de los ejercicios es simplemente temporal según su fecha
de publicación en Calculemus y el orden de los ejercicios en Calculemus
responde a los que me voy encontrando en mis \href{https://github.com/jaalonso/Lecturas\_GLC}{lecturas}.

En cada ejercicio, se comienza proponiendo soluciones en lenguaje
natural y, a continuación, se exponen distintas demostraciones con Lean4
ordenadas desde las más detalladas a las más automáticas. Al final de
cada ejercicio hay un enlace para interactuar con sus soluciones en
\href{https://lean.math.hhu.de/}{Lean4 Web}.

Las soluciones del libro están en \href{https://github.com/jaalonso/Calculemus2}{este repositorio de GitHub}.

El libro se irá actualizando periódicamente con los nuevos ejercicios que se
proponen diariamente en \href{https://www.glc.us.es}{Calculemus}.

Este libro es una continuación de
\begin{itemize}
\item \href{https://raw.githubusercontent.com/jaalonso/DAO\_con\_Lean/master/DAO\_con\_Lean.pdf}{DAO (Demostración Asistida por Ordenador) con Lean} que es una
introducción a la demostración con Lean3 y
\item \href{https://raw.githubusercontent.com/jaalonso/Calculemus/master/Calculemus.pdf}{Calculemus (Vol. 1: Demostraciones con Isabelle/HOL y Lean3)} que es la
recopilación de la primera parte de los ejercicios del blog con
demostraciones en Isabelle/HOL y Lean3.
\end{itemize}

\chapter{Demostraciones de una propiedad de los números enteros}
\label{sec:org8ad91a0}

\section{∀ m n ∈ ℕ, Even n → Even (m * n)}
\label{sec:orge0d6994}
\begin{verbatim}
-- ---------------------------------------------------------------------
-- Demostrar que los productos de los números naturales por números
-- pares son pares.
-- ---------------------------------------------------------------------

-- Demostración en lenguaje natural
-- ================================

-- Si n es par, entonces (por la definición de `Even`) existe un k tal que
--    n = k + k         (1)
-- Por tanto,
--    mn = m(k + k)     (por (1))
--       = mk + mk      (por la propiedad distributiva)
-- Por consiguiente, mn es par.

-- Demostraciones en Lean4
-- =======================

import Mathlib.Data.Nat.Basic
import Mathlib.Data.Nat.Parity
import Mathlib.Tactic

open Nat

-- 1ª demostración
-- ===============

example : ∀ m n : ℕ, Even n → Even (m * n) := by
  rintro m n ⟨k, hk⟩
  use m * k
  rw [hk]
  ring

-- 2ª demostración
-- ===============

example : ∀ m n : ℕ, Even n → Even (m * n) := by
  rintro m n ⟨k, hk⟩
  use m * k
  rw [hk]
  rw [mul_add]

-- 3ª demostración
-- ===============

example : ∀ m n : ℕ, Even n → Even (m * n) := by
  rintro m n ⟨k, hk⟩
  use m * k
  rw [hk, mul_add]

-- 4ª demostración
-- ===============

example : ∀ m n : Nat, Even n → Even (m * n) := by
  rintro m n ⟨k, hk⟩; use m * k; rw [hk, mul_add]

-- 5ª demostración
-- ===============

example : ∀ m n : ℕ, Even n → Even (m * n) := by
  rintro m n ⟨k, hk⟩
  exact ⟨m * k, by rw [hk, mul_add]⟩

-- 6ª demostración
-- ===============

example : ∀ m n : Nat, Even n → Even (m * n) :=
fun m n ⟨k, hk⟩ ↦ ⟨m * k, by rw [hk, mul_add]⟩

-- 7ª demostración
-- ===============

example : ∀ m n : ℕ, Even n → Even (m * n) := by
  rintro m n ⟨k, hk⟩
  use m * k
  rw [hk]
  exact mul_add m k k

-- 8ª demostración
-- ===============

example : ∀ m n : ℕ, Even n → Even (m * n) := by
  intros m n hn
  unfold Even at *
  cases hn with
  | intro k hk =>
    use m * k
    rw [hk, mul_add]

-- 9ª demostración
-- ===============

example : ∀ m n : ℕ, Even n → Even (m * n) := by
  intros m n hn
  unfold Even at *
  cases hn with
  | intro k hk =>
    use m * k
    calc m * n
       = m * (k + k)   := by exact congrArg (HMul.hMul m) hk
     _ = m * k + m * k := by exact mul_add m k k

-- 10ª demostración
-- ================

example : ∀ m n : Nat, Even n → Even (m * n) := by
  intros; simp [*, parity_simps]

-- Lemas usados
-- ============

-- #check (mul_add : ∀ a b c : ℕ, a * (b + c) = a * b + a * c)
\end{verbatim}
Se puede interactuar con las pruebas anteriores en \href{https://lean.math.hhu.de/\#url=https://raw.githubusercontent.com/jaalonso/Calculemus2/main/src/El\_producto\_por\_un\_par\_es\_par.lean}{Lean 4 Web}.

\chapter{Propiedades elementales de los números reales}
\label{sec:orge532e7d}

\section{En ℝ, (ab)c = b(ac)}
\label{sec:org3d061b0}
\begin{verbatim}
-- ---------------------------------------------------------------------
-- Demostrar que los números reales tienen la siguiente propiedad
--    (a * b) * c = b * (a * c)
-- ---------------------------------------------------------------------

-- Demostración en lenguaje natural
-- ================================

-- Por la siguiente cadena de igualdades
--    (ab)c = (ba)c   [por la conmutativa]
--          = b(ac)   [por la asociativa]

-- Demostraciones con Lean4
-- ========================

import Mathlib.Tactic
import Mathlib.Data.Real.Basic

-- 1ª demostración
example
  (a b c : ℝ)
  : (a * b) * c = b * (a * c) :=
calc
  (a * b) * c = (b * a) * c := by rw [mul_comm a b]
            _ = b * (a * c) := by rw [mul_assoc b a c]

-- 2ª demostración
example (a b c : ℝ) : (a * b) * c = b * (a * c) :=
by
  rw [mul_comm a b]
  rw [mul_assoc b a c]

-- 3ª demostración
example (a b c : ℝ) : (a * b) * c = b * (a * c) :=
by ring

-- Lemas usados
-- ============

-- #check (mul_comm : ∀ (a b : ℝ), a * b = b * a)
-- #check (mul_assoc : ∀ (a b c : ℝ), (a * b) * c = a * (b * c))
\end{verbatim}
Se puede interactuar con las pruebas anteriores en \href{https://lean.math.hhu.de/\#url=https://raw.githubusercontent.com/jaalonso/Calculemus2/main/src/Asociativa\_conmutativa\_de\_los\_reales.lean}{Lean 4 Web}.

\section{En ℝ, (cb)a = b(ac)}
\label{sec:orgaac3245}
\begin{verbatim}
-- ---------------------------------------------------------------------
-- Demostrar que los números reales tienen la siguiente propiedad
--    (c * b) * a = b * (a * c)
-- ---------------------------------------------------------------------

-- Demostración en lenguaje natural
-- ================================

-- Por la siguiente cadena de igualdades:
--    (c * b) * a
--    = (b * c) * a    [por la conmutativa]
--    = b * (c * a)    [por la asociativa]
--    = b * (a * c)    [por la conmutativa]

-- Demostraciones con Lean4
-- ========================

import Mathlib.Tactic
import Mathlib.Data.Real.Basic

-- 1ª demostración
example
  (a b c : ℝ)
  : (c * b) * a = b * (a * c) :=
calc
  (c * b) * a
    = (b * c) * a := by rw [mul_comm c b]
  _ = b * (c * a) := by rw [mul_assoc]
  _ = b * (a * c) := by rw [mul_comm c a]

-- 2ª demostración
example
  (a b c : ℝ)
  : (c * b) * a = b * (a * c) :=
by
  rw [mul_comm c b]
  rw [mul_assoc]
  rw [mul_comm c a]

-- 3ª demostración
example
  (a b c : ℝ)
  : (c * b) * a = b * (a * c) :=
by ring

-- Lemas usados
-- ============

-- #check (mul_comm : ∀ (a b : ℝ), a * b = b * a)
-- #check (mul_assoc : ∀ (a b c : ℝ), (a * b) * c = a * (b * c))
\end{verbatim}
Se puede interactuar con las pruebas anteriores en \href{https://lean.math.hhu.de/\#url=https://raw.githubusercontent.com/jaalonso/Calculemus2/main/src/(cb)a\_eq\_b(ac).lean}{Lean 4 Web}.

\section{En ℝ, a(bc) = b(ac)}
\label{sec:org670e825}
\begin{verbatim}
-- ---------------------------------------------------------------------
-- Demostrar que los números reales tienen la siguiente propiedad
--    a * (b * c) = b * (a * c)
-- ---------------------------------------------------------------------

-- Demostración en lenguaje natural
-- ================================

-- Por la siguiente cadena de igualdades:
--    a(bc)
--    = (ab)c    [por la asociativa]
--    = (ba)c    [por la conmutativa]
--    = b(ac)    [por la asociativa]

-- Demostraciones en Lean4
-- =======================

import Mathlib.Tactic
import Mathlib.Data.Real.Basic

-- 1ª demostración
example
  (a b c : ℝ) : a * (b * c) = b * (a * c) :=
calc
  a * (b * c)
    = (a * b) * c := by rw [←mul_assoc]
  _ = (b * a) * c := by rw [mul_comm a b]
  _ = b * (a * c) := by rw [mul_assoc]

-- 2ª demostración
example
  (a b c : ℝ) : a * (b * c) = b * (a * c) :=
by
  rw [←mul_assoc]
  rw [mul_comm a b]
  rw [mul_assoc]

-- 3ª demostración
example
  (a b c : ℝ) : a * (b * c) = b * (a * c) :=
by ring

-- Lemas usados
-- ============

-- #check (mul_comm : ∀ (a b : ℝ), a * b = b * a)
-- #check (mul_assoc : ∀ (a b c : ℝ), (a * b) * c = a * (b * c))
\end{verbatim}
Se puede interactuar con las pruebas anteriores en \href{https://lean.math.hhu.de/\#url=https://raw.githubusercontent.com/jaalonso/Calculemus2/main/src/a(bc)\_eq\_b(ac).lean}{Lean 4 Web}.

\section{En ℝ, si ab = cd y e = f, entonces a(be) = c(df)}
\label{sec:org8d963b9}
\begin{verbatim}
-- ---------------------------------------------------------------------
-- Demostrar que si a, b, c, d, e y f son números reales tales que
--    a * b = c * d y
--    e = f,
-- entonces
--    a * (b * e) = c * (d * f)
-- ---------------------------------------------------------------------

-- Demostración en leguaje natural
-- ===============================

-- Por la siguiente cadena de igualdades
--    a(be)
--    = a(bf)    [por la segunda hipótesis]
--    = (ab)f    [por la asociativa]
--    = (cd)f    [por la primera hipótesis]
--    = c(df)    [por la asociativa]

-- Demostraciones en Lean4
-- =======================

import Mathlib.Tactic
import Mathlib.Data.Real.Basic

-- 1ª demostración
example
  (a b c d e f : ℝ)
  (h1 : a * b = c * d)
  (h2 : e = f)
  : a * (b * e) = c * (d * f) :=
calc
  a * (b * e)
    = a * (b * f) := by rw [h2]
  _ = (a * b) * f := by rw [←mul_assoc]
  _ = (c * d) * f := by rw [h1]
  _ = c * (d * f) := by rw [mul_assoc]

-- 2ª demostración
example
  (a b c d e f : ℝ)
  (h1 : a * b = c * d)
  (h2 : e = f)
  : a * (b * e) = c * (d * f) :=
by
  rw [h2]
  rw [←mul_assoc]
  rw [h1]
  rw [mul_assoc]

-- 3ª demostración
example
  (a b c d e f : ℝ)
  (h1 : a * b = c * d)
  (h2 : e = f)
  : a * (b * e) = c * (d * f) :=
by
  simp [*, ←mul_assoc]

-- Lemas usados
-- ============

-- #check (mul_assoc : ∀ (a b c : ℝ), (a * b) * c = a * (b * c))
\end{verbatim}
Se puede interactuar con las pruebas anteriores en \href{https://lean.math.hhu.de/\#url=https://raw.githubusercontent.com/jaalonso/Calculemus2/main/src/a(be)\_eq\_c(df).lean}{Lean 4 Web}.

\section{En ℝ, si bc = ef, entonces ((ab)c)d = ((ae)f)d}
\label{sec:org72f49cb}
\begin{verbatim}
-- ---------------------------------------------------------------------
-- Demostrar que si a, b, c, d, e y f son números reales tales que
--    b * c = e * f
-- entonces
--    ((a * b) * c) * d = ((a * e) * f) * d
-- ---------------------------------------------------------------------

-- Demostración en lenguaje natural
-- ================================

-- Por la siguiente cadena de igualdades
--    ((ab)c)d
--    = (a(bc))d    [por la asociativa]
--    = (a(ef))d    [por la hipótesis]
--    = ((ae)f)d    [por la asociativa]

-- Demostraciones con Lean4
-- ========================

import Mathlib.Data.Real.Basic
import Mathlib.Tactic

-- 1ª demostración
example
  (a b c d e f : ℝ)
  (h : b * c = e * f)
  : ((a * b) * c) * d = ((a * e) * f) * d :=
calc
  ((a * b) * c) * d
    = (a * (b * c)) * d := by rw [mul_assoc a]
  _ = (a * (e * f)) * d := by rw [h]
  _ = ((a * e) * f) * d := by rw [←mul_assoc a]

-- 2ª demostración
example
  (a b c d e f : ℝ)
  (h : b * c = e * f)
  : ((a * b) * c) * d = ((a * e) * f) * d :=
by
  rw [mul_assoc a]
  rw [h]
  rw [←mul_assoc a]

-- 3ª demostración
example
  (a b c d e f : ℝ)
  (h : b * c = e * f)
  : ((a * b) * c) * d = ((a * e) * f) * d :=
by
  rw [mul_assoc a, h, ←mul_assoc a]

-- Lemas usados
-- ============

-- #check (mul_assoc : ∀ (a b c : ℝ), (a * b) * c = a * (b * c))
\end{verbatim}
Se puede interactuar con las pruebas anteriores en \href{https://lean.math.hhu.de/\#url=https://raw.githubusercontent.com/jaalonso/Calculemus2/main/src/Si\_bc\_eq\_ef\_entonces\_((ab)c)d\_eq\_((ae)f)d.lean}{Lean 4 Web}.

\section{En ℝ, si c = ba-d y d = ab, entonces c = 0}
\label{sec:org9808f8d}
\begin{verbatim}
-- ---------------------------------------------------------------------
-- Demostrar que si a, b, c y d son números reales tales que
--    c = b * a - d
--    d = a * b
-- entonces
--    c = 0
-- ---------------------------------------------------------------------

-- Demostración en lenguaje natural
-- ================================

-- Por la siguiente cadena de igualdades
--    c = ba - d     [por la primera hipótesis]
--      = ab - d     [por la conmutativa]
--      = ab - ab    [por la segunda hipótesis]
--      = 0

-- Demostraciones en Lean4
-- =======================

import Mathlib.Data.Real.Basic
import Mathlib.Tactic

-- 1ª demostración
example
  (a b c d : ℝ)
  (h1 : c = b * a - d)
  (h2 : d = a * b)
  : c = 0 :=
calc
  c = b * a - d     := by rw [h1]
  _ = a * b - d     := by rw [mul_comm]
  _ = a * b - a * b := by rw [h2]
  _ = 0             := by rw [sub_self]

-- 2ª demostración
example
  (a b c d : ℝ)
  (h1 : c = b * a - d)
  (h2 : d = a * b)
  : c = 0 :=
by
  rw [h1]
  rw [mul_comm]
  rw [h2]
  rw [sub_self]

-- 3ª demostración
example
  (a b c d : ℝ)
  (h1 : c = b * a - d)
  (h2 : d = a * b)
  : c = 0 :=
by
  rw [h1, mul_comm, h2, sub_self]

-- Lemas usados
-- ============

-- #check (mul_comm : ∀ (a b : ℝ), a * b = b * a)
-- #check (sub_self : ∀ (a : ℝ), a - a = 0)
\end{verbatim}
Se puede interactuar con las pruebas anteriores en \href{https://lean.math.hhu.de/\#url=https://raw.githubusercontent.com/jaalonso/Calculemus2/main/src/Si\_c\_eq\_ba-d\_y\_d\_eq\_ab\_entonces\_c\_eq\_0.lean}{Lean 4 Web}.

\section{En ℝ, (a+b)(a+b) = aa+2ab+bb}
\label{sec:org82dae00}
\begin{verbatim}
-- ---------------------------------------------------------------------
-- Demostrar que si a y b son números reales, entonces
--    (a + b) * (a + b) = a * a + 2 * (a * b) + b * b
-- ---------------------------------------------------------------------

-- Demostración en lenguaje natural
-- ================================

-- Por la siguiente cadena de igualdades
--    (a + b)(a + b)
--    = (a + b)a + (a + b)b    [por la distributiva]
--    = aa + ba + (a + b)b     [por la distributiva]
--    = aa + ba + (ab + bb)    [por la distributiva]
--    = aa + ba + ab + bb      [por la asociativa]
--    = aa + (ba + ab) + bb    [por la asociativa]
--    = aa + (ab + ab) + bb    [por la conmutativa]
--    = aa + 2(ab) + bb        [por def. de doble]

-- Demostraciones con Lean4
-- ========================

import Mathlib.Data.Real.Basic
import Mathlib.Tactic

variable (a b c : ℝ)

-- 1ª demostración
example :
  (a + b) * (a + b) = a * a + 2 * (a * b) + b * b :=
calc
  (a + b) * (a + b)
    = (a + b) * a + (a + b) * b       := by rw [mul_add]
  _ = a * a + b * a + (a + b) * b     := by rw [add_mul]
  _ = a * a + b * a + (a * b + b * b) := by rw [add_mul]
  _ = a * a + b * a + a * b + b * b   := by rw [←add_assoc]
  _ = a * a + (b * a + a * b) + b * b := by rw [add_assoc (a * a)]
  _ = a * a + (a * b + a * b) + b * b := by rw [mul_comm b a]
  _ = a * a + 2 * (a * b) + b * b     := by rw [←two_mul]

-- 2ª demostración
example :
  (a + b) * (a + b) = a * a + 2 * (a * b) + b * b :=
calc
  (a + b) * (a + b)
    = a * a + b * a + (a * b + b * b) := by rw [mul_add, add_mul, add_mul]
  _ = a * a + (b * a + a * b) + b * b := by rw [←add_assoc, add_assoc (a * a)]
  _ = a * a + 2 * (a * b) + b * b     := by rw [mul_comm b a, ←two_mul]

-- 3ª demostración
example :
  (a + b) * (a + b) = a * a + 2 * (a * b) + b * b :=
calc
  (a + b) * (a + b)
    = a * a + b * a + (a * b + b * b) := by ring
  _ = a * a + (b * a + a * b) + b * b := by ring
  _ = a * a + 2 * (a * b) + b * b     := by ring

-- 4ª demostración
example :
  (a + b) * (a + b) = a * a + 2 * (a * b) + b * b :=
by ring

-- 5ª demostración
example :
  (a + b) * (a + b) = a * a + 2 * (a * b) + b * b :=
by
  rw [mul_add]
  rw [add_mul]
  rw [add_mul]
  rw [←add_assoc]
  rw [add_assoc (a * a)]
  rw [mul_comm b a]
  rw [←two_mul]

-- 6ª demostración
example :
  (a + b) * (a + b) = a * a + 2 * (a * b) + b * b :=
by
  rw [mul_add, add_mul, add_mul]
  rw [←add_assoc, add_assoc (a * a)]
  rw [mul_comm b a, ←two_mul]

-- 7ª demostración
example :
  (a + b) * (a + b) = a * a + 2 * (a * b) + b * b :=
by linarith

-- Lemas usados
-- ============

-- #check (add_assoc : ∀ a b c : ℝ, (a + b) + c = a + (b + c))
-- #check (add_mul : ∀ a b c : ℝ, (a + b) * c = a * c + b * c)
-- #check (mul_add : ∀ a b c : ℝ, a * (b + c) = a * b + a * c)
-- #check (mul_comm : ∀ (a b : ℝ), a * b = b * a)
-- #check (two_mul : ∀ (a : ℝ), 2 * a = a + a)
\end{verbatim}
Se puede interactuar con las pruebas anteriores en \href{https://lean.math.hhu.de/\#url=https://raw.githubusercontent.com/jaalonso/Calculemus2/main/src/(a+b)(a+b)\_eq\_aa+2ab+bb.lean}{Lean 4 Web}.

\section{En ℝ, (a+b)(c+d) = ac+ad+bc+bd}
\label{sec:orgacacfb4}
\begin{verbatim}
-- ---------------------------------------------------------------------
-- Demostrar que si a, b, c y d son números reales, entonces
--    (a + b) * (c + d) = a * c + a * d + b * c + b * d
-- ---------------------------------------------------------------------

-- Demostración en lenguaje natural
-- ================================

-- Por la siguiente cadena de igualdades
--    (a + b)(c + d)
--    = a(c + d) + b(c + d)    [por la distributiva]
--    = ac + ad + b(c + d)     [por la distributiva]
--    = ac + ad + (bc + bd)    [por la distributiva]
--    = ac + ad + bc + bd      [por la asociativa]

-- Demostraciones con Lean4
-- ========================

import Mathlib.Data.Real.Basic
import Mathlib.Tactic

variable (a b c d : ℝ)

-- 1ª demostración
example
  : (a + b) * (c + d) = a * c + a * d + b * c + b * d :=
calc
  (a + b) * (c + d)
    = a * (c + d) + b * (c + d)       := by rw [add_mul]
  _ = a * c + a * d + b * (c + d)     := by rw [mul_add]
  _ = a * c + a * d + (b * c + b * d) := by rw [mul_add]
  _ = a * c + a * d + b * c + b * d   := by rw [←add_assoc]

-- 2ª demostración
example
  : (a + b) * (c + d) = a * c + a * d + b * c + b * d :=
calc
  (a + b) * (c + d)
    = a * (c + d) + b * (c + d)       := by ring
  _ = a * c + a * d + b * (c + d)     := by ring
  _ = a * c + a * d + (b * c + b * d) := by ring
  _ = a * c + a * d + b * c + b * d   := by ring

-- 3ª demostración
example : (a + b) * (c + d) = a * c + a * d + b * c + b * d :=
by ring

-- 4ª demostración
example
  : (a + b) * (c + d) = a * c + a * d + b * c + b * d :=
by
   rw [add_mul]
   rw [mul_add]
   rw [mul_add]
   rw [← add_assoc]

-- 5ª demostración
example : (a + b) * (c + d) = a * c + a * d + b * c + b * d :=
by rw [add_mul, mul_add, mul_add, ←add_assoc]

-- Lemas usados
-- ============

-- #check (add_mul : ∀ (a b c : ℝ), (a + b) * c = a * c + b * c)
-- #check (mul_add : ∀ (a b c : ℝ), a * (b + c) = a * b + a * c)
-- #check (add_assoc : ∀ (a b c : ℝ), (a + b) + c = a + (b + c))
\end{verbatim}
Se puede interactuar con las pruebas anteriores en \href{https://lean.math.hhu.de/\#url=https://raw.githubusercontent.com/jaalonso/Calculemus2/main/src/(a+b)(c+d)\_eq\_ac+ad+bc+bd.lean}{Lean 4 Web}.

\section{En ℝ, (a+b)(a-b) = a\^{}2-b\^{}2}
\label{sec:orgade55b2}
\begin{verbatim}
-- ---------------------------------------------------------------------
-- Demostrar que si a y b son números reales, entonces
--    (a + b) * (a - b) = a^2 - b^2
-- ---------------------------------------------------------------------

-- Demostración en lenguaje natural
-- ================================

-- Por la siguiente cadena de igualdades:
--    (a + b)(a - b)
--    = a(a - b) + b(a - b)            [por la distributiva]
--    = (aa - ab) + b(a - b)           [por la distributiva]
--    = (a^2 - ab) + b(a - b)          [por def. de cuadrado]
--    = (a^2 - ab) + (ba - bb)         [por la distributiva]
--    = (a^2 - ab) + (ba - b^2)        [por def. de cuadrado]
--    = (a^2 + -(ab)) + (ba - b^2)     [por def. de resta]
--    = a^2 + (-(ab) + (ba - b^2))     [por la asociativa]
--    = a^2 + (-(ab) + (ba + -b^2))    [por def. de resta]
--    = a^2 + ((-(ab) + ba) + -b^2)    [por la asociativa]
--    = a^2 + ((-(ab) + ab) + -b^2)    [por la conmutativa]
--    = a^2 + (0 + -b^2)               [por def. de opuesto]
--    = (a^2 + 0) + -b^2               [por asociativa]
--    = a^2 + -b^2                     [por def. de cero]
--    = a^2 - b^2                      [por def. de resta]

-- Demostraciones con Lean4
-- ========================

import Mathlib.Data.Real.Basic
import Mathlib.Tactic

variable (a b : ℝ)

-- 1ª demostración
-- ===============

example : (a + b) * (a - b) = a^2 - b^2 :=
calc
  (a + b) * (a - b)
    = a * (a - b) + b * (a - b)         := by rw [add_mul]
  _ = (a * a - a * b) + b * (a - b)     := by rw [mul_sub]
  _ = (a^2 - a * b) + b * (a - b)       := by rw [← pow_two]
  _ = (a^2 - a * b) + (b * a - b * b)   := by rw [mul_sub]
  _ = (a^2 - a * b) + (b * a - b^2)     := by rw [← pow_two]
  _ = (a^2 + -(a * b)) + (b * a - b^2)  := by ring
  _ = a^2 + (-(a * b) + (b * a - b^2))  := by rw [add_assoc]
  _ = a^2 + (-(a * b) + (b * a + -b^2)) := by ring
  _ = a^2 + ((-(a * b) + b * a) + -b^2) := by rw [← add_assoc
                                              (-(a * b)) (b * a) (-b^2)]
  _ = a^2 + ((-(a * b) + a * b) + -b^2) := by rw [mul_comm]
  _ = a^2 + (0 + -b^2)                  := by rw [neg_add_self (a * b)]
  _ = (a^2 + 0) + -b^2                  := by rw [← add_assoc]
  _ = a^2 + -b^2                        := by rw [add_zero]
  _ = a^2 - b^2                         := by linarith

-- 2ª demostración
-- ===============

example : (a + b) * (a - b) = a^2 - b^2 :=
calc
  (a + b) * (a - b)
    = a * (a - b) + b * (a - b)         := by ring
  _ = (a * a - a * b) + b * (a - b)     := by ring
  _ = (a^2 - a * b) + b * (a - b)       := by ring
  _ = (a^2 - a * b) + (b * a - b * b)   := by ring
  _ = (a^2 - a * b) + (b * a - b^2)     := by ring
  _ = (a^2 + -(a * b)) + (b * a - b^2)  := by ring
  _ = a^2 + (-(a * b) + (b * a - b^2))  := by ring
  _ = a^2 + (-(a * b) + (b * a + -b^2)) := by ring
  _ = a^2 + ((-(a * b) + b * a) + -b^2) := by ring
  _ = a^2 + ((-(a * b) + a * b) + -b^2) := by ring
  _ = a^2 + (0 + -b^2)                  := by ring
  _ = (a^2 + 0) + -b^2                  := by ring
  _ = a^2 + -b^2                        := by ring
  _ = a^2 - b^2                         := by ring

-- 3ª demostración
-- ===============

example : (a + b) * (a - b) = a^2 - b^2 :=
by ring

-- 4ª demostración
-- ===============

-- El lema anterior es
lemma aux : (a + b) * (c + d) = a * c + a * d + b * c + b * d :=
by ring

-- La demostración es
example : (a + b) * (a - b) = a^2 - b^2 :=
by
  rw [sub_eq_add_neg]
  rw [aux]
  rw [mul_neg]
  rw [add_assoc (a * a)]
  rw [mul_comm b a]
  rw [neg_add_self]
  rw [add_zero]
  rw [← pow_two]
  rw [mul_neg]
  rw [← pow_two]
  rw [← sub_eq_add_neg]

-- Lemas usados
-- ============

-- #check (add_assoc : ∀ (a b c : ℝ), (a + b) + c = a + (b + c))
-- #check (add_zero : ∀ (a : ℝ), a + 0 = a)
-- #check (add_mul : ∀ (a b c : ℝ), (a + b) * c = a * c + b * c)
-- #check (mul_comm : ∀ (a b : ℝ), a * b = b * a)
-- #check (mul_neg : ∀ (a b : ℝ), a * -b = -(a * b))
-- #check (mul_sub : ∀ (a b c : ℝ), a * (b - c) = a * b - a * c)
-- #check (neg_add_self : ∀ (a : ℝ), -a + a = 0)
-- #check (pow_two : ∀ (a : ℝ), a ^ 2 = a * a)
-- #check (sub_eq_add_neg : ∀ (a b : ℝ), a - b = a + -b)
\end{verbatim}
Se puede interactuar con las pruebas anteriores en \href{https://lean.math.hhu.de/\#url=https://raw.githubusercontent.com/jaalonso/Calculemus2/main/src/(a+b)(a-b)\_eq\_aa-bb.lean}{Lean 4 Web}.

\section{En ℝ, si c = da+b y b = ad, entonces c = 2ad}
\label{sec:orgbbec026}
\begin{verbatim}
-- ---------------------------------------------------------------------
-- Demostrar que si a, b, c y d son números reales tales que
--    c = d * a + b
--    b = a * d
-- entonces
--    c = 2 * a * d
-- ---------------------------------------------------------------------

-- Demostración en lenguaje natural
-- ================================

-- Por la siguiente cadena de igualdades
--    c = da + b     [por la primera hipótesis]
--      = da + ad    [por la segunda hipótesis]
--      = ad + ad    [por la conmutativa]
--      = 2(ad)      [por la def. de doble]
--      = 2ad        [por la asociativa]

-- Demostraciones con Lean4
-- ========================

import Mathlib.Data.Real.Basic
import Mathlib.Tactic

variable (a b c d : ℝ)

-- 1ª demostración
example
  (h1 : c = d * a + b)
  (h2 : b = a * d)
  : c = 2 * a * d :=
calc
  c = d * a + b     := by rw [h1]
  _ = d * a + a * d := by rw [h2]
  _ = a * d + a * d := by rw [mul_comm d a]
  _ = 2 * (a * d)   := by rw [← two_mul (a * d)]
  _ = 2 * a * d     := by rw [mul_assoc]

-- 2ª demostración
example
  (h1 : c = d * a + b)
  (h2 : b = a * d)
  : c = 2 * a * d :=
by
  rw [h2] at h1
  clear h2
  rw [mul_comm d a] at h1
  rw [← two_mul (a*d)] at h1
  rw [← mul_assoc 2 a d] at h1
  exact h1

-- 3ª demostración
example
  (h1 : c = d * a + b)
  (h2 : b = a * d)
  : c = 2 * a * d :=
by rw [h1, h2, mul_comm d a, ← two_mul (a * d), mul_assoc]

-- 4ª demostración
example
  (h1 : c = d * a + b)
  (h2 : b = a * d)
  : c = 2 * a * d :=
by
  rw [h1]
  rw [h2]
  ring

-- 5ª demostración
example
  (h1 : c = d * a + b)
  (h2 : b = a * d)
  : c = 2 * a * d :=
by
  rw [h1, h2]
  ring

-- 6ª demostración
example
  (h1 : c = d * a + b)
  (h2 : b = a * d)
  : c = 2 * a * d :=
by rw [h1, h2] ; ring

-- 7ª demostración
example
  (h1 : c = d * a + b)
  (h2 : b = a * d)
  : c = 2 * a * d :=
by linarith

-- Lemas usados
-- ============

-- #check (mul_assoc : ∀ (a b c : ℝ), (a * b) * c = a * (b * c))
-- #check (mul_comm : ∀ (a b : ℝ), a * b = b * a)
-- #check (two_mul : ∀ (a : ℝ), 2 * a = a + a)
\end{verbatim}
Se puede interactuar con las pruebas anteriores en \href{https://lean.math.hhu.de/\#url=https://raw.githubusercontent.com/jaalonso/Calculemus2/main/src/Si\_c\_eq\_da+b\_y\_b\_eq\_ad\_entonces\_c\_eq\_2ad.lean}{Lean 4 Web}.

\section{En ℝ, si a+b = c, entonces (a+b)(a+b) = ac+bc}
\label{sec:org02066e7}
\begin{verbatim}
-- ---------------------------------------------------------------------
-- Demostrar que si a, b y c son números reales tales que
--    a + b = c,
-- entonces
--    (a + b) * (a + b) = a * c + b * c
-- ---------------------------------------------------------------------

-- Demostración en lenguaje natural
-- ================================

-- Por la siguiente cadena de igualdades
--    (a + b)(a + b)
--    = (a + b)c        [por la hipótesis]
--    = ac + bc         [por la distributiva]

-- Demostraciones con Lean4
-- ========================

import Mathlib.Data.Real.Basic
import Mathlib.Tactic

variable (a b c : ℝ)

-- 1ª demostración
example
  (h : a + b = c)
  : (a + b) * (a + b) = a * c + b * c :=
calc
  (a + b) * (a + b)
    = (a + b) * c   := by exact congrArg (HMul.hMul (a + b)) h
  _ = a * c + b * c := by rw [add_mul]

-- 2ª demostración
example
  (h : a + b = c)
  : (a + b) * (a + b) = a * c + b * c :=
by
  nth_rewrite 2 [h]
  rw [add_mul]

-- Lemas usados
-- ============

-- #check (add_mul : ∀ (a b c : ℝ), (a + b) * c = a * c + b * c)
\end{verbatim}
Se puede interactuar con las pruebas anteriores en \href{https://lean.math.hhu.de/\#url=https://raw.githubusercontent.com/jaalonso/Calculemus2/main/src/Sia+b\_eq\_c\_entonces\_(a+b)(a+b)\_eq\_ac+bc.lean}{Lean 4 Web}.

\section{Si x e y son sumas de dos cuadrados, entonces xy también lo es}
\label{sec:org0f72155}
\begin{verbatim}
-- ---------------------------------------------------------------------
-- Demostrar que si x e y son sumas de dos cuadrados, entonces xy
-- también lo es.
-- ----------------------------------------------------------------------

-- Demostración en lenguaje natural
-- ================================

-- Puesto que x e y se pueden escribir como la suma de dos cuadrados,
-- existen a, b , c y d tales que
--    x = a² + b²
--    y = c² + d²
-- Entonces,
--    xy = (ac - bd)² + (ad + bc)²
-- En efecto,
--    xy = (a² + b²)(c² + d²)
--       = a²c² + b²d² + a²d² + b²c²
--       = a²c² - 2acbd + b²d² + a²d² + 2adbc + b²c²
--       = (ac - bd)² + (ad + bc)²
-- Por tanto, xy es la suma de dos cuadrados.

-- Demostraciones con Lean4
-- ========================

import Mathlib.Tactic
variable {α : Type _} [CommRing α]
variable {x y : α}

-- (suma_de_cuadrados x) afirma que x se puede escribir como la suma
-- de dos cuadrados.
def suma_de_cuadrados (x : α) :=
  ∃ a b, x = a^2 + b^2

-- 1ª demostración
example
  (hx : suma_de_cuadrados x)
  (hy : suma_de_cuadrados y)
  : suma_de_cuadrados (x * y) :=
by
  rcases hx with ⟨a, b, xeq : x = a^2 + b^2⟩
  -- a b : α
  -- xeq : x = a ^ 2 + b ^ 2
  rcases hy with ⟨c, d, yeq : y = c^2 + d^2⟩
  -- c d : α
  -- yeq : y = c ^ 2 + d ^ 2
  have h1: x * y = (a*c - b*d)^2 + (a*d + b*c)^2 :=
    calc x * y
         = (a^2 + b^2) * (c^2 + d^2) :=
                by rw [xeq, yeq]
       _ = a^2*c^2 + b^2*d^2 + a^2*d^2 + b^2*c^2 :=
                by ring
       _ = a^2*c^2 - 2*a*c*b*d + b^2*d^2 + a^2*d^2 + 2*a*d*b*c + b^2*c^2 :=
                by ring
       _ = (a*c - b*d)^2 + (a*d + b*c)^2 :=
                by ring
  have h2 : ∃ f, x * y = (a*c - b*d)^2 + f^2 :=
    Exists.intro (a*d + b*c) h1
  have h3 : ∃ e f, x * y = e^2 + f^2 :=
    Exists.intro (a*c - b*d) h2
  show suma_de_cuadrados (x * y)
  exact h3

-- 2ª demostración
example
  (hx : suma_de_cuadrados x)
  (hy : suma_de_cuadrados y)
  : suma_de_cuadrados (x * y) :=
by
  rcases hx with ⟨a, b, xeq : x = a^2 + b^2⟩
  -- a b : α
  -- xeq : x = a ^ 2 + b ^ 2
  rcases hy with ⟨c, d, yeq : y = c^2 + d^2⟩
  -- c d : α
  -- yeq : y = c ^ 2 + d ^ 2
  have h1: x * y = (a*c - b*d)^2 + (a*d + b*c)^2 :=
    calc x * y
         = (a^2 + b^2) * (c^2 + d^2)     := by rw [xeq, yeq]
       _ = (a*c - b*d)^2 + (a*d + b*c)^2 := by ring
  have h2 : ∃ e f, x * y = e^2 + f^2 :=
    by tauto
  show suma_de_cuadrados (x * y)
  exact h2

-- 3ª demostración
example
  (hx : suma_de_cuadrados x)
  (hy : suma_de_cuadrados y)
  : suma_de_cuadrados (x * y) :=
by
  rcases hx with ⟨a, b, xeq⟩
  -- a b : α
  -- xeq : x = a ^ 2 + b ^ 2
  rcases hy with ⟨c, d, yeq⟩
  -- c d : α
  -- yeq : y = c ^ 2 + d ^ 2
  rw [xeq, yeq]
  -- ⊢ suma_de_cuadrados ((a ^ 2 + b ^ 2) * (c ^ 2 + d ^ 2))
  use a*c - b*d, a*d + b*c
  -- ⊢ (a ^ 2 + b ^ 2) * (c ^ 2 + d ^ 2)
  --   = (a * c - b * d) ^ 2 + (a * d + b * c) ^ 2
  ring

-- 4ª demostración
example
  (hx : suma_de_cuadrados x)
  (hy : suma_de_cuadrados y)
  : suma_de_cuadrados (x * y) :=
by
  rcases hx with ⟨a, b, rfl⟩
  -- ⊢ suma_de_cuadrados ((a ^ 2 + b ^ 2) * y)
  rcases hy with ⟨c, d, rfl⟩
  -- ⊢ suma_de_cuadrados ((a ^ 2 + b ^ 2) * (c ^ 2 + d ^ 2))
  use a*c - b*d, a*d + b*c
  -- ⊢ (a ^ 2 + b ^ 2) * (c ^ 2 + d ^ 2)
  --   = (a * c - b * d) ^ 2 + (a * d + b * c) ^ 2
  ring
\end{verbatim}
Se puede interactuar con las pruebas anteriores en \href{https://lean.math.hhu.de/\#url=https://raw.githubusercontent.com/jaalonso/Calculemus2/main/src/Producto\_de\_suma\_de\_cuadrados.lean}{Lean 4 Web}

\section{En ℝ, x² + y² = 0 ↔ x = 0 ∧ y = 0}
\label{sec:org9a59799}
\begin{verbatim}
-- ---------------------------------------------------------------------
-- Demostrar que si x, y ∈ ℝ, entonces
--    x^2 + y^2 = 0 ↔ x = 0 ∧ y = 0
-- ----------------------------------------------------------------------

-- Demostración en lenguaje natural
-- ================================

-- En la demostración usaremos el siguiente lema auxiliar
--    (∀ x, y ∈ ℝ)[x² + y² = 0 → x = 0]
--
-- Para la primera implicación, supongamos que
--    x² + y² = 0                                                    (1)
-- Entonces, por el lema auxiliar,
--    x = 0                                                          (2)
-- Además, aplicando la conmutativa a (1), se tiene
--    y² + x² = 0
-- y, por el lema auxiliar,
--    y = 0                                                          (3)
-- De (2) y (3) se tiene
--    x = 0 ∧ y = 0
--
-- Para la segunda implicación, supongamos que
--    x = 0 ∧ y = 0
-- Por tanto,
--    x² + y² = 0² + 0²
--            = 0
--
-- En la demostración del lema auxiliar se usarán los siguientes lemas
--    (∀ x ∈ ℝ)(∀ n ∈ ℕ)[x^n = 0 → x = 0]                            (L1)
--    (∀ x, y ∈ ℝ)[x ≤ y → y ≤ x → x = y]                            (L2)
--    (∀ x, y ∈ ℝ)[0 ≤ y → x ≤ x + y]                                (L3)
--    (∀ x ∈ ℝ)[0 ≤ x²]                                              (L4)
--
-- Por el lema L1, para demostrar el lema auxiliar basta demostrar
--    x² = 0                                                         (1)
-- y, por el lema L2, basta demostrar las siguientes desigualdades
--     x² ≤ 0                                                        (2)
--     0 ≤ x²                                                        (3)
--
-- La prueba de la (2) es
--    x² ≤ x² + y²   [por L3 y L4]
--       = 0         [por la hipótesis]
--
-- La (3) se tiene por el lema L4.

-- Demostraciones con Lean 4
-- =========================

import Mathlib.Data.Real.Basic
variable {x y : ℝ}

-- 1ª demostración del lema auxiliar
-- =================================

example
  (h : x^2 + y^2 = 0)
  : x = 0 :=
by
  have h' : x^2 = 0 := by
  { apply le_antisymm
    . show x ^ 2 ≤ 0
      calc x ^ 2 ≤ x^2 + y^2 := by simp [le_add_of_nonneg_right,
                                         pow_two_nonneg]
               _ = 0         := by exact h
    . show 0 ≤ x ^ 2
      apply pow_two_nonneg }
  show x = 0
  exact pow_eq_zero h'

-- 2ª demostración lema auxiliar
-- =============================

example
  (h : x^2 + y^2 = 0)
  : x = 0 :=
by
  have h' : x^2 = 0 := by
  { apply le_antisymm
    . -- ⊢ x ^ 2 ≤ 0
      calc x ^ 2 ≤ x^2 + y^2 := by simp [le_add_of_nonneg_right,
                                         pow_two_nonneg]
               _ = 0         := by exact h
    . -- ⊢ 0 ≤ x ^ 2
      apply pow_two_nonneg }
  exact pow_eq_zero h'

-- 3ª demostración lema auxiliar
-- =============================

lemma aux
  (h : x^2 + y^2 = 0)
  : x = 0 :=
  have h' : x ^ 2 = 0 := by linarith [pow_two_nonneg x, pow_two_nonneg y]
  pow_eq_zero h'

-- 1ª demostración
-- ===============

example : x^2 + y^2 = 0 ↔ x = 0 ∧ y = 0 :=
by
  constructor
  . -- ⊢ x ^ 2 + y ^ 2 = 0 → x = 0 ∧ y = 0
    intro h
    -- h : x ^ 2 + y ^ 2 = 0
    -- ⊢ x = 0 ∧ y = 0
    constructor
    . -- ⊢ x = 0
      exact aux h
    . -- ⊢ y = 0
      rw [add_comm] at h
      -- h : x ^ 2 + y ^ 2 = 0
      exact aux h
  . -- ⊢ x = 0 ∧ y = 0 → x ^ 2 + y ^ 2 = 0
    intro h1
    -- h1 : x = 0 ∧ y = 0
    -- ⊢ x ^ 2 + y ^ 2 = 0
    rcases h1 with ⟨h2, h3⟩
    -- h2 : x = 0
    -- h3 : y = 0
    rw [h2, h3]
    -- ⊢ 0 ^ 2 + 0 ^ 2 = 0
    norm_num

-- 2ª demostración
-- ===============

example : x^2 + y^2 = 0 ↔ x = 0 ∧ y = 0 :=
by
  constructor
  . -- ⊢ x ^ 2 + y ^ 2 = 0 → x = 0 ∧ y = 0
    intro h
    -- h : x ^ 2 + y ^ 2 = 0
    -- ⊢ x = 0 ∧ y = 0
    constructor
    . -- ⊢ x = 0
      exact aux h
    . -- ⊢ y = 0
      rw [add_comm] at h
      -- h : x ^ 2 + y ^ 2 = 0
      exact aux h
  . -- ⊢ x = 0 ∧ y = 0 → x ^ 2 + y ^ 2 = 0
    rintro ⟨h1, h2⟩
    -- h1 : x = 0
    -- h2 : y = 0
    -- ⊢ x ^ 2 + y ^ 2 = 0
    rw [h1, h2]
    -- ⊢ 0 ^ 2 + 0 ^ 2 = 0
    norm_num

-- 3ª demostración
-- ===============

example : x ^ 2 + y ^ 2 = 0 ↔ x = 0 ∧ y = 0 := by
  constructor
  · -- ⊢ x ^ 2 + y ^ 2 = 0 → x = 0 ∧ y = 0
    intro h
    -- h : x ^ 2 + y ^ 2 = 0
    -- ⊢ x = 0 ∧ y = 0
    constructor
    · -- x = 0
      exact aux h
    . -- ⊢ y = 0
      rw [add_comm] at h
      -- h : y ^ 2 + x ^ 2 = 0
      exact aux h
  . -- ⊢ x = 0 ∧ y = 0 → x ^ 2 + y ^ 2 = 0
    rintro ⟨rfl, rfl⟩
    -- ⊢ 0 ^ 2 + 0 ^ 2 = 0
    norm_num

-- Lemas usados
-- ============

-- #check (add_comm x y : x + y = y + x)
-- #check (le_add_of_nonneg_right : 0 ≤ y → x ≤ x + y)
-- #check (le_antisymm : x ≤ y → y ≤ x → x = y)
-- #check (pow_eq_zero : ∀ {n : ℕ}, x ^ n = 0 → x = 0)
-- #check (pow_two_nonneg x : 0 ≤ x ^ 2)
\end{verbatim}
Se puede interactuar con las pruebas anteriores en \href{https://lean.math.hhu.de/\#url=https://raw.githubusercontent.com/jaalonso/Calculemus2/main/src/Suma\_nula\_de\_dos\_cuadrados.lean}{Lean 4 Web}

\section{En ℝ, x² = 1 → x = 1 ∨ x = -1}
\label{sec:org2bfa879}
\begin{verbatim}
-- ---------------------------------------------------------------------
-- Demostrar que si
--    x^2 = 1
-- entonces
--    x = 1 ∨ x = -1
-- ----------------------------------------------------------------------

-- Demostración en lenguaje natural
-- ================================

-- Usaremos los siguientes lemas
--    (∀ x ∈ ℝ)[x - x = 0]                                           (L1)
--    (∀ x, y ∈ ℝ)[xy = 0 → x = 0 ∨ y = 0]                           (L2)
--    (∀ x, y ∈ ℝ)[x - y = 0 ↔ x = y]                                (L3)
--    (∀ x, y ∈ ℝ)[x + y = 0 → x = -y]                               (L4)
--
-- Se tiene que
--    (x - 1)(x + 1) = x² - 1
--                   = 1 - 1      [por la hipótesis]
--                   = 0          [por L1]
-- y, por el lema L2, se tiene que
--    x - 1 = 0 ∨ x + 1 = 0
-- Acabaremos la demostración por casos.
--
-- Primer caso:
--   x - 1 = 0 ⟹ x = 1             [por L3]
--             ⟹ x = 1 ∨ x = -1
--
-- Segundo caso:
--   x + 1 = 0 ⟹ x = -1            [por L4]
--             ⟹ x = 1 ∨ x = -1

-- Demostraciones con Lean4
-- ========================

import Mathlib.Data.Real.Basic
variable (x y : ℝ)

-- 1ª demostración
-- ===============

example
  (h : x^2 = 1)
  : x = 1 ∨ x = -1 :=
by
  have h1 : (x - 1) * (x + 1) = 0 := by
    calc (x - 1) * (x + 1) = x^2 - 1 := by ring
                         _ = 1 - 1   := by rw [h]
                         _ = 0       := sub_self 1
  have h2 : x - 1 = 0 ∨ x + 1 = 0 := by
    apply eq_zero_or_eq_zero_of_mul_eq_zero h1
  rcases h2 with h3 | h4
  . -- h3 : x - 1 = 0
    left
    -- ⊢ x = 1
    exact sub_eq_zero.mp h3
  . -- h4 : x + 1 = 0
    right
    -- ⊢ x = -1
    exact eq_neg_of_add_eq_zero_left h4

-- 2ª demostración
-- ===============

example
  (h : x^2 = 1)
  : x = 1 ∨ x = -1 :=
by
  have h1 : (x - 1) * (x + 1) = 0 := by nlinarith
  have h2 : x - 1 = 0 ∨ x + 1 = 0 := by aesop
  rcases h2 with h3 | h4
  . -- h3 : x - 1 = 0
    left
    -- ⊢ x = 1
    linarith
  . -- h4 : x + 1 = 0
    right
    -- ⊢ x = -1
    linarith

-- 3ª demostración
-- ===============

example
  (h : x^2 = 1)
  : x = 1 ∨ x = -1 :=
sq_eq_one_iff.mp h

-- 3ª demostración
-- ===============

example
  (h : x^2 = 1)
  : x = 1 ∨ x = -1 :=
by aesop

-- Lemas usados
-- ============

-- #check (eq_neg_of_add_eq_zero_left : x + y = 0 → x = -y)
-- #check (eq_zero_or_eq_zero_of_mul_eq_zero : x * y = 0 → x = 0 ∨ y = 0)
-- #check (sq_eq_one_iff : x ^ 2 = 1 ↔ x = 1 ∨ x = -1)
-- #check (sub_eq_zero : x - y = 0 ↔ x = y)
-- #check (sub_self x : x - x = 0)
\end{verbatim}
Se puede interactuar con las pruebas anteriores en \href{https://lean.math.hhu.de/\#url=https://raw.githubusercontent.com/jaalonso/Calculemus2/main/src/Cuadrado\_igual\_a\_uno.lean.lean}{Lean 4 Web}

\section{En ℝ, x² = y² → x = y ∨ x = -y}
\label{sec:orgff04bde}
\begin{verbatim}
-- ---------------------------------------------------------------------
-- Demostrar que si
--    x^2 = y^2
-- entonces
--    x = y ∨ x = -y
-- ----------------------------------------------------------------------

-- Usaremos los siguientes lemas
--    (∀ x ∈ ℝ)[x - x = 0]                                           (L1)
--    (∀ x, y ∈ ℝ)[xy = 0 → x = 0 ∨ y = 0]                           (L2)
--    (∀ x, y ∈ ℝ)[x - y = 0 ↔ x = y]                                (L3)
--    (∀ x, y ∈ ℝ)[x + y = 0 → x = -y]                               (L4)
--
-- Se tiene que
--    (x - y)(x + y) = x² - y²
--                   = y² - y²    [por la hipótesis]
--                   = 0          [por L1]
-- y, por el lema L2, se tiene que
--    x - y = 0 ∨ x + y = 0
--
-- Acabaremos la demostración por casos.
--
-- Primer caso:
--   x - y = 0 ⟹ x = y             [por L3]
--             ⟹ x = y ∨ x = -y
--
-- Segundo caso:
--   x + y = 0 ⟹ x = -y            [por L4]
--             ⟹ x = y ∨ x = -y

-- Demostraciones con Lean4
-- ========================

import Mathlib.Data.Real.Basic
variable (x y : ℝ)

-- 1ª demostración
-- ===============

example
  (h : x^2 = y^2)
  : x = y ∨ x = -y :=
by
  have h1 : (x - y) * (x + y) = 0 := by
    calc (x - y) * (x + y) = x^2 - y^2 := by ring
                         _ = y^2 - y^2 := by rw [h]
                         _ = 0         := sub_self (y ^ 2)
  have h2 : x - y = 0 ∨ x + y = 0 := by
    apply eq_zero_or_eq_zero_of_mul_eq_zero h1
  rcases h2 with h3 | h4
  . -- h3 : x - y = 0
    left
    -- ⊢ x = y
    exact sub_eq_zero.mp h3
  . -- h4 : x + y = 0
    right
    -- ⊢ x = -y
    exact eq_neg_of_add_eq_zero_left h4

-- 2ª demostración
-- ===============

example
  (h : x^2 = y^2)
  : x = y ∨ x = -y :=
by
  have h1 : (x - y) * (x + y) = 0 := by nlinarith
  have h2 : x - y = 0 ∨ x + y = 0 := by aesop
  rcases h2 with h3 | h4
  . -- h3 : x - y = 0
    left
    -- ⊢ x = y
    linarith
  . -- h4 : x + y = 0
    right
    -- ⊢ x = -y
    linarith

-- 2ª demostración
-- ===============

example
  (h : x^2 = y^2)
  : x = y ∨ x = -y :=
sq_eq_sq_iff_eq_or_eq_neg.mp h

-- Lemas usados
-- ============

-- #check (eq_neg_of_add_eq_zero_left : x + y = 0 → x = -y)
-- #check (eq_zero_or_eq_zero_of_mul_eq_zero : x * y = 0 → x = 0 ∨ y = 0)
-- #check (sq_eq_sq_iff_eq_or_eq_neg : x ^ 2 = y ^ 2 ↔ x = y ∨ x = -y)
-- #check (sub_eq_zero : x - y = 0 ↔ x = y)
-- #check (sub_self x : x - x = 0)
\end{verbatim}
Se puede interactuar con las pruebas anteriores en \href{https://lean.math.hhu.de/\#url=https://raw.githubusercontent.com/jaalonso/Calculemus2/main/src/Cuadrado\_igual\_a\_cuadrado.lean}{Lean 4 Web}

\section{En ℝ, |a| = |a - b + b|}
\label{sec:org4f2f527}
\begin{verbatim}
-- ---------------------------------------------------------------------
-- Demostrar que
--    |a| = |a - b + b|
-- ----------------------------------------------------------------------

import Mathlib.Data.Real.Basic
variable (a b : ℝ)

-- 1ª demostración
-- ===============

example
  : |a| = |a - b + b| :=
by
  congr
  -- a = a - b + b
  ring

-- Comentario: La táctica congr sustituye una conclusión de la forma
-- A = B por las igualdades de sus subtérminos que no no iguales por
-- definición. Por ejemplo, sustituye la conclusión (x * f y = g w * f z)
-- por las conclusiones (x = g w) y (y = z).

-- 2ª demostración
-- ===============

example
  (a b : ℝ)
  : |a| = |a - b + b| :=
by { congr ; ring }

-- 3ª demostración
-- ===============

example
  (a b : ℝ)
  : |a| = |a - b + b| :=
by ring_nf
\end{verbatim}
Se puede interactuar con las pruebas anteriores en \href{https://lean.math.hhu.de/\#url=https://raw.githubusercontent.com/jaalonso/Calculemus2/main/src/Demostracion\_por\_congruencia.lean}{Lean 4 Web}

\chapter{Propiedades elementales de los anillos}
\label{sec:orgc55cd7f}

\section{Si R es un anillo y a ∈ R, entonces a + 0 = a}
\label{sec:org80be8c8}
\begin{verbatim}
-- ---------------------------------------------------------------------
-- Demostrar en Lean4 que si R es un anillo, entonces
--    ∀ a : R, a + 0 = a
-- ---------------------------------------------------------------------

-- Demostración en lenguaje natural
-- ================================

-- Por la siguiente cadena de igualdades
--    a + 0 = 0 + a    [por la conmutativa de la suma]
--          = a        [por el axioma del cero por la izquierda]

import Mathlib.Algebra.Ring.Defs

variable {R : Type _} [Ring R]
variable (a : R)

-- Demostraciones con Lean4
-- ========================

-- 1ª demostración
example : a + 0 = a :=
calc a + 0
     = 0 + a := by rw [add_comm]
   _ = a     := by rw [zero_add]

-- 2ª demostración
example : a + 0 = a :=
by
  rw [add_comm]
  rw [zero_add]

-- 3ª demostración
example : a + 0 = a :=
by rw [add_comm, zero_add]

-- 4ª demostración
example : a + 0 = a :=
by exact add_zero a

-- 5ª demostración
example : a + 0 = a :=
  add_zero a

-- 5ª demostración
example : a + 0 = a :=
by simp

-- Lemas usados
-- ============

variable (a b : R)

-- #check (add_comm a b : a + b = b + a)
-- #check (zero_add a :  0 + a = a)
\end{verbatim}
Se puede interactuar con las pruebas anteriores en \href{https://lean.math.hhu.de/\#url=https://raw.githubusercontent.com/jaalonso/Calculemus2/main/src/Suma\_con\_cero.lean}{Lean 4 Web}.

\section{Si R es un anillo y a ∈ R, entonces a + -a = 0}
\label{sec:org8821866}
\begin{verbatim}
-- ---------------------------------------------------------------------
-- Demostrar en Lean4 que si R es un anillo, entonces
--    ∀ a : R, a + -a = 0
-- ---------------------------------------------------------------------

-- Demostración en lenguaje natural
-- ================================

-- Por la siguiente cadena de igualdades
--    a + -a = -a + a    [por la conmutativa de la suma]
--           = 0         [por el axioma de inverso por la izquierda]

import Mathlib.Algebra.Ring.Defs

variable {R : Type _} [Ring R]
variable (a : R)

-- 1ª demostración
-- ===============

example : a + -a = 0 :=
calc a + -a = -a + a := by rw [add_comm]
          _ = 0      := by rw [add_left_neg]

-- 2ª demostración
-- ===============

example : a + -a = 0 :=
by
  rw [add_comm]
  rw [add_left_neg]

-- 3ª demostración
-- ===============

example : a + -a = 0 :=
by rw [add_comm, add_left_neg]

-- 4ª demostración
-- ===============

example : a + -a = 0 :=
by exact add_neg_self a

-- 5ª demostración
-- ===============

example : a + -a = 0 :=
  add_neg_self a

-- 6ª demostración
-- ===============

example : a + -a = 0 :=
by simp

-- Lemas usados
-- ============

-- variable (a b : R)
-- #check (add_comm a b : a + b = b + a)
-- #check (add_left_neg a : -a + a = 0)
-- #check (add_neg_self a : a + -a = 0)
\end{verbatim}
Se puede interactuar con las pruebas anteriores en \href{https://lean.math.hhu.de/\#url=https://raw.githubusercontent.com/jaalonso/Calculemus2/main/src/Suma\_con\_opuesto.lean}{Lean 4 Web}.

\section{Si R es un anillo y a, b ∈ R, entonces -a + (a + b) = b}
\label{sec:org962e83f}
\begin{verbatim}
-- ---------------------------------------------------------------------
-- Demostrar en Lean4 que si R es un anillo, entonces
--    ∀ a, b : R, -a + (a + b) = b
-- ---------------------------------------------------------------------

-- Demostración en lenguaje natural
-- ================================

-- Por la siguiente cadena de igualdades
--    -a + (a + b) = (-a + a) + b [por la asociativa]
--                 = 0 + b        [por inverso por la izquierda]
--                 = b            [por cero por la izquierda]

import Mathlib.Algebra.Ring.Defs

variable {R : Type _} [Ring R]
variable (a b : R)

-- Demostraciones con Lean4
-- ========================

-- 1ª demostración
example : -a + (a + b) = b :=
calc -a + (a + b) = (-a + a) + b := by rw [← add_assoc]
                _ = 0 + b        := by rw [add_left_neg]
                _ = b            := by rw [zero_add]

-- 2ª demostración
example : -a + (a + b) = b :=
by
  rw [←add_assoc]
  rw [add_left_neg]
  rw [zero_add]

-- 3ª demostración
example : -a + (a + b) = b :=
by rw [←add_assoc, add_left_neg, zero_add]

-- 4ª demostración
example : -a + (a + b) = b :=
by exact neg_add_cancel_left a b

-- 5ª demostración
example : -a + (a + b) = b :=
  neg_add_cancel_left a b

-- 6ª demostración
example : -a + (a + b) = b :=
by simp

-- Lemas usados
-- ============

-- variable (c : R)
-- #check (add_assoc a b c : (a + b) + c = a + (b + c))
-- #check (add_left_neg a : -a + a = 0)
-- #check (neg_add_cancel_left a b : -a + (a + b) = b)
-- #check (zero_add a :  0 + a = a)
\end{verbatim}
Se puede interactuar con las pruebas anteriores en \href{https://lean.math.hhu.de/\#url=https://raw.githubusercontent.com/jaalonso/Calculemus2/main/src/Opuesto\_se\_cancela\_con\_la\_suma\_por\_la\_izquierda.lean}{Lean 4 Web}.

\section{Si R es un anillo y a, b ∈ R, entonces (a + b) + -b = a}
\label{sec:orgb0ce930}
\begin{verbatim}
-- ---------------------------------------------------------------------
-- Demostrar en Lean4 que si R es un anillo, entonces
--    ∀ a, b : R, (a + b) + -b = a
-- ---------------------------------------------------------------------

-- Demostración en lenguaje natural
-- ================================

-- Por la siguiente cadena de igualdades
--    (a + b) + -b = a + (b + -b)    [por la asociativa]
--               _ = a + 0           [por suma con opuesto]
--               _ = a               [por suma con cero]

-- Demostraciones con Lean4
-- ========================

import Mathlib.Algebra.Ring.Defs

variable {R : Type _} [Ring R]
variable (a b : R)

-- 1ª demostración
example : (a + b) + -b = a :=
calc
  (a + b) + -b = a + (b + -b) := by rw [add_assoc]
             _ = a + 0        := by rw [add_right_neg]
             _ = a            := by rw [add_zero]

-- 2ª demostración
example : (a + b) + -b = a :=
by
  rw [add_assoc]
  rw [add_right_neg]
  rw [add_zero]

-- 3ª demostración
example : (a + b) + -b = a :=
by rw [add_assoc, add_right_neg, add_zero]

-- 4ª demostración
example : (a + b) + -b = a :=
  add_neg_cancel_right a b

-- 5ª demostración
example : (a + b) + -b = a :=
  add_neg_cancel_right _ _

-- 6ª demostración
example : (a + b) + -b = a :=
by simp

-- Lemas usados
-- ============

-- variable (c : R)
-- #check (add_assoc a b c : (a + b) + c = a + (b + c))
-- #check (add_neg_cancel_right a b : (a + b) + -b = a)
-- #check (add_right_neg a : a + -a = 0)
-- #check (add_zero a :  a + 0 = a)
\end{verbatim}
Se puede interactuar con las pruebas anteriores en \href{https://lean.math.hhu.de/\#url=https://raw.githubusercontent.com/jaalonso/Calculemus2/main/src/Opuesto\_se\_cancela\_con\_la\_suma\_por\_la\_derecha.lean}{Lean 4 Web}.

\section{Si R es un anillo y a, b, c ∈ R tales que a+b=a+c, entonces b=c}
\label{sec:orgf2abff5}
\begin{verbatim}
-- ---------------------------------------------------------------------
-- Demostrar que si R es un anillo y a, b, c ∈ R tales que
--    a + b = a + c
-- entonces
--    b = c
-- ----------------------------------------------------------------------

-- Demostraciones en lenguaje natural (LN)
-- ======================================

-- 1ª demostración en LN
-- =====================

-- Por la siguiente cadena de igualdades
--    b = 0 + b           [por suma con cero]
--      = (-a + a) + b    [por suma con opuesto]
--      = -a + (a + b)    [por asociativa]
--      = -a + (a + c)    [por hipótesis]
--      = (-a + a) + c    [por asociativa]
--      = 0 + c           [por suma con opuesto]
--      = c               [por suma con cero]

-- 2ª demostración en LN
-- =====================

-- Por la siguiente cadena de implicaciones
--    a + b = a + c
--    ==> -a + (a + b) = -a + (a + c)     [sumando -a]
--    ==>  (-a + a) + b = (-a + a) + c    [por la asociativa]
--    ==>  0 + b = 0 + b                  [suma con opuesto]
--    ==>  b = c                          [suma con cero]

-- 3ª demostración en LN
-- =====================

-- Por la siguiente cadena de igualdades
--    b = -a + (a + b)
--      = -a + (a + c)   [por la hipótesis]
--      = c

-- Demostraciones con Lean4
-- ========================

import Mathlib.Algebra.Ring.Defs
import Mathlib.Tactic

variable {R : Type _} [Ring R]
variable {a b c : R}

-- 1ª demostración
example
  (h : a + b = a + c)
  : b = c :=
calc
  b = 0 + b        := by rw [zero_add]
  _ = (-a + a) + b := by rw [add_left_neg]
  _ = -a + (a + b) := by rw [add_assoc]
  _ = -a + (a + c) := by rw [h]
  _ = (-a + a) + c := by rw [←add_assoc]
  _ = 0 + c        := by rw [add_left_neg]
  _ = c            := by rw [zero_add]

-- 2ª demostración
example
  (h : a + b = a + c)
  : b = c :=
by
  have h1 : -a + (a + b) = -a + (a + c) :=
    congrArg (HAdd.hAdd (-a)) h
  clear h
  rw [← add_assoc] at h1
  rw [add_left_neg] at h1
  rw [zero_add] at h1
  rw [← add_assoc] at h1
  rw [add_left_neg] at h1
  rw [zero_add] at h1
  exact h1

-- 3ª demostración
example
  (h : a + b = a + c)
  : b = c :=
calc
  b = -a + (a + b) := by rw [neg_add_cancel_left a b]
  _ = -a + (a + c) := by rw [h]
  _ = c            := by rw [neg_add_cancel_left]

-- 4ª demostración
example
  (h : a + b = a + c)
  : b = c :=
by
  rw [← neg_add_cancel_left a b]
  rw [h]
  rw [neg_add_cancel_left]

-- 5ª demostración
example
  (h : a + b = a + c)
  : b = c :=
by
  rw [← neg_add_cancel_left a b, h, neg_add_cancel_left]

-- 6ª demostración
example
  (h : a + b = a + c)
  : b = c :=
add_left_cancel h

-- Lemas usados
-- ============

-- #check (add_assoc a b c : (a + b) + c = a + (b + c))
-- #check (add_left_cancel : a + b = a + c → b = c)
-- #check (add_left_neg a : -a + a = 0)
-- #check (neg_add_cancel_left a b : -a + (a + b) = b)
-- #check (zero_add a :  0 + a = a)
\end{verbatim}
Se puede interactuar con las pruebas anteriores en \href{https://lean.math.hhu.de/\#url=https://raw.githubusercontent.com/jaalonso/Calculemus2/main/src/Cancelativa\_izquierda.lean}{Lean 4 Web}.

\section{Si R es un anillo y a, b, c ∈ R tales que a+b=c+b, entonces a=c}
\label{sec:orgff8adfe}
\begin{verbatim}
-- ---------------------------------------------------------------------
-- Demostrar que si R es un anillo y a, b, c ∈ R tales que
--    a + b = c + b
-- entonces
--    a = c
-- ----------------------------------------------------------------------

-- Demostraciones en lenguaje natural (LN)
-- =======================================

-- 1ª demostración en LN
-- =====================

-- Por la siguiente cadena de igualdades
--    a = a + 0           [por suma con cero]
--      = a + (b + -b)    [por suma con opuesto]
--      = (a + b) + -b    [por asociativa]
--      = (c + b) + -b    [por hipótesis]
--      = c + (b + -b)    [por asociativa]
--      = c + 0           [por suma con opuesto]
--      = c               [por suma con cero]

-- 2ª demostración en LN
-- =====================

-- Por la siguiente cadena de igualdades
--    a = (a + b) + -b
--      = (c + b) + -b    [por hipótesis]
--      = c

-- Demostraciones con Lean4
-- ========================

import Mathlib.Algebra.Ring.Defs
import Mathlib.Tactic

variable {R : Type _} [Ring R]
variable {a b c : R}

-- 1ª demostración con Lean4
-- =========================

example
  (h : a + b = c + b)
  : a = c :=
calc
  a = a + 0        := by rw [add_zero]
  _ = a + (b + -b) := by rw [add_right_neg]
  _ = (a + b) + -b := by rw [add_assoc]
  _ = (c + b) + -b := by rw [h]
  _ = c + (b + -b) := by rw [← add_assoc]
  _ = c + 0        := by rw [← add_right_neg]
  _ = c            := by rw [add_zero]

-- 2ª demostración con Lean4
-- =========================

example
  (h : a + b = c + b)
  : a = c :=
calc
  a = (a + b) + -b := (add_neg_cancel_right a b).symm
  _ = (c + b) + -b := by rw [h]
  _ = c            := add_neg_cancel_right c b

-- 3ª demostración con Lean4
-- =========================

example
  (h : a + b = c + b)
  : a = c :=
by
  rw [← add_neg_cancel_right a b]
  rw [h]
  rw [add_neg_cancel_right]

-- 4ª demostración con Lean4
-- =========================

example
  (h : a + b = c + b)
  : a = c :=
by
  rw [← add_neg_cancel_right a b, h, add_neg_cancel_right]

-- 5ª demostración con Lean4
-- =========================

example
  (h : a + b = c + b)
  : a = c :=
add_right_cancel h

-- Lemas usados
-- ============

-- #check (add_assoc a b c : (a + b) + c = a + (b + c))
-- #check (add_neg_cancel_right a b : (a + b) + -b = a)
-- #check (add_right_cancel : a + b = c + b → a = c)
-- #check (add_right_neg a : a + -a = 0)
-- #check (add_zero a :  a + 0 = a)
\end{verbatim}
Se puede interactuar con las pruebas anteriores en \href{https://lean.math.hhu.de/\#url=https://raw.githubusercontent.com/jaalonso/Calculemus2/main/src/Cancelativa\_derecha.lean}{Lean 4 Web}.

\section{Si R es un anillo y a ∈ R, entonces a.0 = 0}
\label{sec:org563cb06}
\begin{verbatim}
-- ---------------------------------------------------------------------
-- Demostrar que si R es un anillo y a ∈ R, entonces
--    a * 0 = 0
-- ----------------------------------------------------------------------

-- Demostración en lenguaje natural
-- ================================

-- Basta aplicar la propiedad cancelativa a
--    a.0 + a.0 = a.0 + 0
-- que se demuestra mediante la siguiente cadena de igualdades
--    a.0 + a.0 = a.(0 + 0)    [por la distributiva]
--              = a.0          [por suma con cero]
--              = a.0 + 0      [por suma con cero]

-- Demostraciones con Lean4
-- ========================

import Mathlib.Algebra.Ring.Defs
import Mathlib.Tactic

variable {R : Type _} [Ring R]
variable (a : R)

-- 1ª demostración
-- ===============

example : a * 0 = 0 :=
by
  have h : a * 0 + a * 0 = a * 0 + 0 :=
    calc a * 0 + a * 0 = a * (0 + 0) := by rw [mul_add a 0 0]
                     _ = a * 0       := by rw [add_zero 0]
                     _ = a * 0 + 0   := by rw [add_zero (a * 0)]
  rw [add_left_cancel h]

-- 2ª demostración
-- ===============

example : a * 0 = 0 :=
by
  have h : a * 0 + a * 0 = a * 0 + 0 :=
    calc a * 0 + a * 0 = a * (0 + 0) := by rw [← mul_add]
                     _ = a * 0       := by rw [add_zero]
                     _ = a * 0 + 0   := by rw [add_zero]
  rw [add_left_cancel h]

-- 3ª demostración
-- ===============

example : a * 0 = 0 :=
by
  have h : a * 0 + a * 0 = a * 0 + 0 :=
    by rw [← mul_add, add_zero, add_zero]
  rw [add_left_cancel h]

-- 4ª demostración
-- ===============

example : a * 0 = 0 :=
by
  have : a * 0 + a * 0 = a * 0 + 0 :=
    calc a * 0 + a * 0 = a * (0 + 0) := by simp
                     _ = a * 0       := by simp
                     _ = a * 0 + 0   := by simp
  simp

-- 5ª demostración
-- ===============

example : a * 0 = 0 :=
  mul_zero a

-- 6ª demostración
-- ===============

example : a * 0 = 0 :=
by simp

-- Lemas usados
-- ============

-- variable (b c : R)
-- #check (add_left_cancel : a + b = a + c → b = c)
-- #check (add_zero a :  a + 0 = a)
-- #check (mul_add a b c :  a * (b + c) = a * b + a * c)
-- #check (mul_zero a : a * 0 = 0)
\end{verbatim}
Se puede interactuar con las pruebas anteriores en \href{https://lean.math.hhu.de/\#url=https://raw.githubusercontent.com/jaalonso/Calculemus2/main/src/Multiplicacion\_por\_cero.lean}{Lean 4 Web}.

\section{Si R es un anillo y a ∈ R, entonces 0.a = 0}
\label{sec:org361a09a}
\begin{verbatim}
-- ---------------------------------------------------------------------
-- Demostrar que si R es un anillo y a ∈ R, entonces
--    0 * a = 0
-- ----------------------------------------------------------------------

-- Demostración en lenguaje natural
-- ================================

-- Basta aplicar la propiedad cancelativa a
--    0.a + 0.a = 0.a + 0
-- que se demuestra mediante la siguiente cadena de igualdades
--    0.a + 0.a = (0 + 0).a    [por la distributiva]
--              = 0.a          [por suma con cero]
--              = 0.a + 0      [por suma con cero]

-- Demostraciones con Lean4
-- ========================

import Mathlib.Algebra.Ring.Defs
import Mathlib.Tactic

variable {R : Type _} [Ring R]
variable (a : R)

-- 1ª demostración
example : 0 * a = 0 :=
by
  have h : 0 * a + 0 * a = 0 * a + 0 :=
    calc 0 * a + 0 * a = (0 + 0) * a := by rw [add_mul]
                     _ = 0 * a       := by rw [add_zero]
                     _ = 0 * a + 0   := by rw [add_zero]
  rw [add_left_cancel h]

-- 2ª demostración
example : 0 * a = 0 :=
by
  have h : 0 * a + 0 * a = 0 * a + 0 :=
    by rw [←add_mul, add_zero, add_zero]
  rw [add_left_cancel h]

-- 3ª demostración
example : 0 * a = 0 :=
by
  have : 0 * a + 0 * a = 0 * a + 0 :=
    calc 0 * a + 0 * a = (0 + 0) * a := by simp
                     _ = 0 * a       := by simp
                     _ = 0 * a + 0   := by simp
  simp

-- 4ª demostración
example : 0 * a = 0 :=
by
  have : 0 * a + 0 * a = 0 * a + 0 := by simp
  simp

-- 5ª demostración
example : 0 * a = 0 :=
by simp

-- 6ª demostración
example : 0 * a = 0 :=
zero_mul a

-- Lemas usados
-- ============

-- variable (b c : R)
-- #check (add_mul a b c :  (a + b) * c = a * c + b * c)
-- #check (add_zero a :  a + 0 = a)
-- #check (zero_mul a : 0 * a = 0)
\end{verbatim}
Se puede interactuar con las pruebas anteriores en \href{https://lean.math.hhu.de/\#url=https://raw.githubusercontent.com/jaalonso/Calculemus2/main/src/Multiplicacion\_por\_cero\_izquierda.lean}{Lean 4 Web}.

\section{Si R es un anillo y a, b ∈ R tales que a+b=0, entonces -a=b}
\label{sec:org9558154}
\begin{verbatim}
-- ---------------------------------------------------------------------
-- Demostrar que si es un anillo y a, b ∈ R tales que
--    a + b = 0
-- entonces
--    -a = b
-- ----------------------------------------------------------------------

-- Demostraciones en lenguaje natural (LN)
-- =======================================

-- 1ª demostración en LN
-- ---------------------

-- Por la siguiente cadena de igualdades
--    -a = -a + 0          [por suma cero]
--       = -a + (a + b)    [por hipótesis]
--       = b               [por cancelativa]

-- 2ª demostración en LN
-- ---------------------

-- Sumando -a a ambos lados de la hipótesis, se tiene
--    -a + (a + b) = -a + 0
-- El término de la izquierda se reduce a b (por la cancelativa) y el de
-- la derecha a -a (por la suma con cero). Por tanto, se tiene
--     b = -a
-- Por la simetría de la igualdad, se tiene
--     -a = b

-- Demostraciones con Lean 4
-- =========================

import Mathlib.Algebra.Ring.Defs
import Mathlib.Tactic

variable {R : Type _} [Ring R]
variable {a b : R}

-- 1ª demostración (basada en la 1º en LN)
example
  (h : a + b = 0)
  : -a = b :=
calc
  -a = -a + 0       := by rw [add_zero]
   _ = -a + (a + b) := by rw [h]
   _ = b            := by rw [neg_add_cancel_left]

-- 2ª demostración (basada en la 1º en LN)
example
  (h : a + b = 0)
  : -a = b :=
calc
  -a = -a + 0       := by simp
   _ = -a + (a + b) := by rw [h]
   _ = b            := by simp

-- 3ª demostración (basada en la 2º en LN)
example
  (h : a + b = 0)
  : -a = b :=
by
  have h1 : -a + (a + b) = -a + 0 := congrArg (HAdd.hAdd (-a)) h
  have h2 : -a + (a + b) = b := neg_add_cancel_left a b
  have h3 : -a + 0 = -a := add_zero (-a)
  rw [h2, h3] at h1
  exact h1.symm

-- 4ª demostración
example
  (h : a + b = 0)
  : -a = b :=
neg_eq_iff_add_eq_zero.mpr h

-- Lemas usados
-- ============

-- #check (add_zero a : a + 0 = a)
-- #check (neg_add_cancel_left a b : -a + (a + b) = b)
-- #check (neg_eq_iff_add_eq_zero : -a = b ↔ a + b = 0)
\end{verbatim}
Se puede interactuar con las pruebas anteriores en \href{https://lean.math.hhu.de/\#url=https://raw.githubusercontent.com/jaalonso/Calculemus2/main/src/Opuesto\_ig\_si\_suma\_ig\_cero.lean}{Lean 4 Web}.

\section{Si R es un anillo y a, b ∈ R tales que a+b=0, entonces a=-b}
\label{sec:org0585a57}
\begin{verbatim}
-- ---------------------------------------------------------------------
-- Demostrar que si R es un anillo y a, b ∈ R tales que
--    a + b = 0
-- entonces
--    a = -b
-- ----------------------------------------------------------------------

-- Demostraciones en lenguaje natural (LN)
-- =======================================

-- 1ª demostración en LN
-- ---------------------

-- Por la siguiente cadena de igualdades
--    a = (a + b) + -b    [por la concelativa]
--      = 0 + -b          [por la hipótesis]
--      = -b              [por la suma con cero]

-- 2ª demostración en LN
-- ---------------------

-- Sumando -a a ambos lados de la hipótesis, se tiene
--    (a + b) + -b = 0 + -b
-- El término de la izquierda se reduce a a (por la cancelativa) y el de
-- la derecha a -b (por la suma con cero). Por tanto, se tiene
--     a = -b

-- Demostraciones con Lean4
-- ========================

import Mathlib.Algebra.Ring.Defs
import Mathlib.Tactic

variable {R : Type _} [Ring R]
variable {a b : R}

-- 1ª demostración (basada en la 1ª en LN)
example
  (h : a + b = 0)
  : a = -b :=
calc
  a = (a + b) + -b := by rw [add_neg_cancel_right]
  _ = 0 + -b       := by rw [h]
  _ = -b           := by rw [zero_add]

-- 2ª demostración (basada en la 1ª en LN)
example
  (h : a + b = 0)
  : a = -b :=
calc
  a = (a + b) + -b := by simp
  _ = 0 + -b       := by rw [h]
  _ = -b           := by simp

-- 3ª demostración (basada en la 1ª en LN)
example
  (h : a + b = 0)
  : a = -b :=
by
  have h1 : (a + b) + -b = 0 + -b := by rw [h]
  have h2 : (a + b) + -b = a := add_neg_cancel_right a b
  have h3 : 0 + -b = -b := zero_add (-b)
  rwa [h2, h3] at h1

-- 4ª demostración
example
  (h : a + b = 0)
  : a = -b :=
add_eq_zero_iff_eq_neg.mp h

-- Lemas usados
-- ============

-- #check (add_eq_zero_iff_eq_neg : a + b = 0 ↔ a = -b)
-- #check (add_neg_cancel_right a b : (a + b) + -b = a)
-- #check (zero_add a : 0 + a = a)
\end{verbatim}
Se puede interactuar con las pruebas anteriores en \href{https://lean.math.hhu.de/\#url=https://raw.githubusercontent.com/jaalonso/Calculemus2/main/src/Ig\_opuesto\_si\_suma\_ig\_cero.lean}{Lean 4 Web}.

\section{Si R es un anillo, entonces -0 = 0}
\label{sec:orgaa5c010}
\begin{verbatim}
-- ---------------------------------------------------------------------
-- Demostrar que si R es un anillo, entonces
--    -0 = 0
-- ----------------------------------------------------------------------

-- Demostraciones en lenguaje natural (LN)
-- =======================================

-- 1ª demostración en LN
-- =====================

-- Por la suma con cero se tiene
--    0 + 0 = 0
-- Aplicándole la propiedad
--    ∀ a b ∈ R, a + b = 0 → -a = b
-- se obtiene
--    -0 = 0

-- 2ª demostración en LN
-- =====================

-- Puesto que
--    ∀ a b ∈ R, a + b = 0 → -a = b
-- basta demostrar que
--    0 + 0 = 0
-- que es cierta por la suma con cero.

-- Demostraciones con Lean4
-- ========================

import Mathlib.Algebra.Ring.Defs
import Mathlib.Tactic

variable {R : Type _} [Ring R]

-- 1ª demostración (basada en la 1ª en LN)
example : (-0 : R) = 0 :=
by
  have h1 : (0 : R) + 0 = 0 := add_zero 0
  show (-0 : R) = 0
  exact neg_eq_of_add_eq_zero_left h1

-- 2ª demostración (basada en la 2ª en LN)
example : (-0 : R) = 0 :=
by
  apply neg_eq_of_add_eq_zero_left
  rw [add_zero]

-- 3ª demostración
example : (-0 : R) = 0 :=
  neg_zero

-- 4ª demostración
example : (-0 : R) = 0 :=
by simp

-- Lemas usados
-- ============

-- variable (a b : R)
-- #check (add_zero a : a + 0 = a)
-- #check (neg_eq_of_add_eq_zero_left : a + b = 0 → -b = a)
-- #check (neg_zero : -0 = 0)
\end{verbatim}
Se puede interactuar con las pruebas anteriores en \href{https://lean.math.hhu.de/\#url=https://raw.githubusercontent.com/jaalonso/Calculemus2/main/src/Opuesto\_del\_cero.lean}{Lean 4 Web}.

\section{Si R es un anillo y a ∈ R, entonces -(-a) = a}
\label{sec:org1cd8ae3}
\begin{verbatim}
-- ---------------------------------------------------------------------
-- Demostrar que si R es un anillo y a ∈ R, entonces
--     -(-a) = a
-- ----------------------------------------------------------------------

-- Demostración en lenguaje natural
-- ================================

-- Es consecuencia de las siguiente propiedades demostradas en
-- ejercicios anteriores:
--    ∀ a b ∈ R, a + b = 0 → -a = b
--    ∀ a ∈ R, -a + a = 0

-- Demostraciones con Lean4
-- ========================

import Mathlib.Algebra.Ring.Defs
import Mathlib.Tactic

variable {R : Type _} [Ring R]
variable {a : R}

-- 1ª demostración
example : -(-a) = a :=
by
  have h1 : -a + a = 0 := add_left_neg a
  show -(-a) = a
  exact neg_eq_of_add_eq_zero_right h1

-- 2ª demostración
example : -(-a) = a :=
by
  apply neg_eq_of_add_eq_zero_right
  rw [add_left_neg]

-- 3ª demostración
example : -(-a) = a :=
neg_neg a

-- 4ª demostración
example : -(-a) = a :=
by simp

-- Lemas usados
-- ============

-- variable (b : R)
-- #check (add_left_neg a : -a + a = 0)
-- #check (neg_eq_of_add_eq_zero_right : a + b = 0 → -a = b)
-- #check (neg_neg a : -(-a) = a)
\end{verbatim}
Se puede interactuar con las pruebas anteriores en \href{https://lean.math.hhu.de/\#url=https://raw.githubusercontent.com/jaalonso/Calculemus2/main/src/Opuesto\_del\_opuesto.lean}{Lean 4 Web}.

\section{Si R es un anillo y a, b ∈ R, entonces a - b = a + -b}
\label{sec:org9e395af}
\begin{verbatim}
-- ---------------------------------------------------------------------
-- Demostrar que si R es un anillo y a, b ∈ R, entonces
--    a - b = a + -b
-- ----------------------------------------------------------------------

-- Demostración en lenguaje natural
-- ================================

-- Por la definición de la resta.

-- Demostración en Lean4
-- =====================

import Mathlib.Algebra.Ring.Defs

variable {R : Type _} [Ring R]
variable (a b : R)

example : a - b = a + -b :=
-- by exact?
sub_eq_add_neg a b

-- Lemas usados
-- ============

-- #check (sub_eq_add_neg a b : a - b = a + -b)
\end{verbatim}
Se puede interactuar con las pruebas anteriores en \href{https://lean.math.hhu.de/\#url=https://raw.githubusercontent.com/jaalonso/Calculemus2/main/src/Resta\_igual\_suma\_opuesto.lean}{Lean 4 Web}.

\section{Si R es un anillo y a ∈ R, entonces a - a = 0}
\label{sec:org07e454a}
\begin{verbatim}
-- ---------------------------------------------------------------------
-- Demostrar que si R es un anillo y a ∈ R, entonces
--     a - a = 0
-- ----------------------------------------------------------------------

-- Demostración en lenguaje natural
-- ================================

-- Por la siguiente cadena de igualdades:
--    a - a = a + -a    [por definición de resta]
--          = 0         [por suma con opuesto]

-- Demostraciones con Lean4
-- ========================

import Mathlib.Algebra.Ring.Defs
variable {R : Type _} [Ring R]
variable (a : R)

-- 1ª demostración
example : a - a = 0 :=
calc
  a - a = a + -a := by rw [sub_eq_add_neg a a]
      _ = 0      := by rw [add_right_neg]

-- 2ª demostración
example : a - a = 0 :=
sub_self a

-- 3ª demostración
example : a - a = 0 :=
by simp

-- Lemas usados
-- ============

-- #check (add_right_neg a : a + -a = 0)
-- #check (sub_eq_add_neg a b : a - b = a + -b)
-- #check (sub_self a : a - a = 0)
\end{verbatim}
Se puede interactuar con las pruebas anteriores en \href{https://lean.math.hhu.de/\#url=https://raw.githubusercontent.com/jaalonso/Calculemus2/main/src/Resta\_consigo\_mismo.lean}{Lean 4 Web}.

\section{En los anillos, 1 + 1 = 2}
\label{sec:org18575d2}
\begin{verbatim}
-- ---------------------------------------------------------------------
-- Demostrar que en los anillos,
--    1 + 1 = 2
-- ----------------------------------------------------------------------

-- Demostración en lenguaje natural
-- ================================

-- Por cálculo.

-- Demostración con Lean4
-- ======================

import Mathlib.Algebra.Ring.Defs
import Mathlib.Tactic
variable {R : Type _} [Ring R]

-- Demostraciones con Lean4
-- ========================

-- 1ª demostración
example : 1 + 1 = (2 : R) :=
by norm_num

-- 2ª demostración
example : 1 + 1 = (2 : R) :=
one_add_one_eq_two

-- Lemas usados
-- ============

-- #check (one_add_one_eq_two : 1 + 1 = 2)
\end{verbatim}
Se puede interactuar con las pruebas anteriores en \href{https://lean.math.hhu.de/\#url=https://raw.githubusercontent.com/jaalonso/Calculemus2/main/src/Uno\_mas\_uno\_es\_dos.lean}{Lean 4 Web}.

\section{Si R es un anillo y a ∈ R, entonces 2a = a+a}
\label{sec:org6649e90}
\begin{verbatim}
-- ---------------------------------------------------------------------
-- Demostrar que si R es un anillo y a ∈ R, entonces
--    2 * a = a + a
-- ----------------------------------------------------------------------

-- Demostración en lenguaje natural
-- ================================

-- Por la siguiente cadena de igualdades
--    2·a = (1 + 1)·a    [por la definición de 2]
--        = 1·a + 1·a    [por la distributiva]
--        = a + a        [por producto con uno]

-- Demostraciones con Lean4
-- ========================

import Mathlib.Algebra.Ring.Defs

variable {R : Type _} [Ring R]
variable (a : R)

-- 1ª demostración
example : 2 * a = a + a :=
calc
  2 * a = (1 + 1) * a   := by rw [one_add_one_eq_two]
      _ = 1 * a + 1 * a := by rw [add_mul]
      _ = a + a         := by rw [one_mul]

-- 2ª demostración
example : 2 * a = a + a :=
by exact two_mul a

-- Lemas usados
-- ============

-- variable (b c : R)
-- #check (add_mul a b c : (a + b) * c = a * c + b * c)
-- #check (one_add_one_eq_two : (1 : R) + 1 = 2)
-- #check (one_mul a : 1 * a = a)
-- #check (two_mul a : 2 * a = a + a)
\end{verbatim}
Se puede interactuar con las pruebas anteriores en \href{https://lean.math.hhu.de/\#url=https://raw.githubusercontent.com/jaalonso/Calculemus2/main/src/Producto\_por\_dos.lean}{Lean 4 Web}.

\chapter{Propiedades elementales de los grupos}
\label{sec:org5634173}

\section{Si G es un grupo y a ∈ G, entonces aa⁻¹ = 1}
\label{sec:org94abc6a}
\begin{verbatim}
-- ---------------------------------------------------------------------
-- En Lean4, se declara que G es un grupo mediante la expresión
--    variable {G : Type _} [Group G]
--
-- Como consecuencia, se tiene los siguientes axiomas
--    mul_assoc :    ∀ a b c : G, a * b * c = a * (b * c)
--    one_mul :      ∀ a : G, 1 * a = a
--    mul_left_inv : ∀ a : G, a⁻¹ * a = 1
--
-- Demostrar que si G es un grupo y a ∈ G, entonces
--    a * a⁻¹ = 1
-- ----------------------------------------------------------------------

-- Demostración en lenguaje natural
-- ================================

-- Por la siguiente cadena de igualdades
--    a·a⁻¹ = 1·(a·a⁻¹)                 [por producto con uno]
--          = (1·a)·a⁻¹                 [por asociativa]
--          = (((a⁻¹)⁻¹·a⁻¹) ·a)·a⁻¹    [por producto con inverso]
--          = ((a⁻¹)⁻¹·(a⁻¹ ·a))·a⁻¹    [por asociativa]
--          = ((a⁻¹)⁻¹·1)·a⁻¹           [por producto con inverso]
--          = (a⁻¹)⁻¹·(1·a⁻¹)           [por asociativa]
--          = (a⁻¹)⁻¹·a⁻¹               [por producto con uno]
--          = 1                         [por producto con inverso]

-- Demostraciones con Lean4
-- ========================

import Mathlib.Algebra.Group.Defs

variable {G : Type _} [Group G]
variable (a b : G)

-- 1ª demostración
example : a * a⁻¹ = 1 :=
calc
  a * a⁻¹ = 1 * (a * a⁻¹)                := by rw [one_mul]
        _ = (1 * a) * a⁻¹                := by rw [mul_assoc]
        _ = (((a⁻¹)⁻¹ * a⁻¹)  * a) * a⁻¹ := by rw [mul_left_inv]
        _ = ((a⁻¹)⁻¹ * (a⁻¹  * a)) * a⁻¹ := by rw [← mul_assoc]
        _ = ((a⁻¹)⁻¹ * 1) * a⁻¹          := by rw [mul_left_inv]
        _ = (a⁻¹)⁻¹ * (1 * a⁻¹)          := by rw [mul_assoc]
        _ = (a⁻¹)⁻¹ * a⁻¹                := by rw [one_mul]
        _ = 1                            := by rw [mul_left_inv]

-- 2ª demostración
example : a * a⁻¹ = 1 :=
calc
  a * a⁻¹ = 1 * (a * a⁻¹)                := by simp
        _ = (1 * a) * a⁻¹                := by simp
        _ = (((a⁻¹)⁻¹ * a⁻¹)  * a) * a⁻¹ := by simp
        _ = ((a⁻¹)⁻¹ * (a⁻¹  * a)) * a⁻¹ := by simp
        _ = ((a⁻¹)⁻¹ * 1) * a⁻¹          := by simp
        _ = (a⁻¹)⁻¹ * (1 * a⁻¹)          := by simp
        _ = (a⁻¹)⁻¹ * a⁻¹                := by simp
        _ = 1                            := by simp

-- 3ª demostración
example : a * a⁻¹ = 1 :=
by simp

-- 4ª demostración
example : a * a⁻¹ = 1 :=
by exact mul_inv_self a

-- Lemas usados
-- ============

-- variable (c : G)
-- #check (mul_assoc a b c : (a * b) * c = a * (b * c))
-- #check (mul_inv_self a : a * a⁻¹ = 1)
-- #check (mul_left_inv a : a⁻¹  * a = 1)
-- #check (one_mul a : 1 * a = a)
\end{verbatim}
Se puede interactuar con las pruebas anteriores en \href{https://lean.math.hhu.de/\#url=https://raw.githubusercontent.com/jaalonso/Calculemus2/main/src/Producto\_por\_dos.lean}{Lean 4 Web}.

\section{Si G es un grupo y a ∈ G, entonces a·1 = a}
\label{sec:org2802290}
\begin{verbatim}
-- ---------------------------------------------------------------------
-- Demostrar que si G es un grupo y a ∈ G, entonces
--    a * 1 = a
-- ----------------------------------------------------------------------

-- Demostración en lenguaje natural
-- ================================

-- Se tiene por la siguiente cadena de igualdades
--    a·1 = a·(a⁻¹·a)    [por producto con inverso]
--        = (a·a⁻¹)·a    [por asociativa]
--        = 1·a          [por producto con inverso]
--        = a            [por producto con uno]

-- Demostraciones con Lean4
-- ========================

import Mathlib.Algebra.Group.Defs

variable {G : Type _} [Group G]
variable (a b : G)

-- 1ª demostración
example : a * 1 = a :=
calc
  a * 1 = a * (a⁻¹ * a) := by rw [mul_left_inv]
      _ = (a * a⁻¹) * a := by rw [mul_assoc]
      _ = 1 * a         := by rw [mul_right_inv]
      _ = a             := by rw [one_mul]

-- 2ª demostración
example : a * 1 = a :=
calc
  a * 1 = a * (a⁻¹ * a) := by simp
      _ = (a * a⁻¹) * a := by simp
      _ = 1 * a         := by simp
      _ = a             := by simp

-- 3ª demostración
example : a * 1 = a :=
by simp

-- 4ª demostración
example : a * 1 = a :=
by exact mul_one a

-- Lemas usados
-- ============

-- variable (c : G)
-- #check (mul_left_inv a : a⁻¹  * a = 1)
-- #check (mul_assoc a b c : (a * b) * c = a * (b * c))
-- #check (mul_right_inv a : a * a⁻¹ = 1)
-- #check (one_mul a : 1 * a = a)
-- #check (mul_one a : a * 1 = a)
\end{verbatim}
Se puede interactuar con las pruebas anteriores en \href{https://lean.math.hhu.de/\#url=https://raw.githubusercontent.com/jaalonso/Calculemus2/main/src/Producto\_por\_uno.lean}{Lean 4 Web}.

\section{Si G es un grupo y a, b ∈ G tales que ab = 1 entonces a⁻¹ = b}
\label{sec:orge24567e}
\begin{verbatim}
-- ---------------------------------------------------------------------
-- Demostrar que si G es un grupo y a, b ∈ G, tales que
--    a * b = 1
-- entonces
--    a⁻¹ = b
-- ----------------------------------------------------------------------

-- Demostración en lenguaje natural
-- ================================

-- Se tiene a partir de la siguiente cadena de igualdades
--    a⁻¹ = a⁻¹ * 1           [por producto por uno]
--        = a⁻¹ * (a * b)     [por hipótesis]
--        = (a⁻¹ * a) * b     [por asociativa]
--        = 1 * b             [por producto con inverso]
--        = b                 [por producto por uno]

-- Demostraciones con Lean4
-- ========================

import Mathlib.Algebra.Group.Defs

variable {G : Type _} [Group G]
variable (a b : G)

-- 1º demostración
example
  (h : a * b = 1)
  : a⁻¹ = b :=
calc
  a⁻¹ = a⁻¹ * 1       := by rw [mul_one]
    _ = a⁻¹ * (a * b) := by rw [h]
    _ = (a⁻¹ * a) * b := by rw [mul_assoc]
    _ = 1 * b         := by rw [mul_left_inv]
    _ = b             := by rw [one_mul]

-- 2º demostración
example
  (h : a * b = 1)
  : a⁻¹ = b :=
calc
  a⁻¹ = a⁻¹ * 1       := by simp
    _ = a⁻¹ * (a * b) := by simp [h]
    _ = (a⁻¹ * a) * b := by simp
    _ = 1 * b         := by simp
    _ = b             := by simp

-- 3º demostración
example
  (h : a * b = 1)
  : a⁻¹ = b :=
calc
  a⁻¹ =  a⁻¹ * (a * b) := by simp [h]
    _ =  b             := by simp

-- 4º demostración
example
  (h : a * b = 1)
  : a⁻¹ = b :=
by exact inv_eq_of_mul_eq_one_right h

-- Lemas usados
-- ============

-- variable (c : G)
-- #check (mul_assoc a b c : (a * b) * c = a * (b * c))
-- #check (mul_left_inv a : a⁻¹  * a = 1)
-- #check (mul_one a : a * 1 = a)
-- #check (one_mul a : 1 * a = a)
-- #check (inv_eq_of_mul_eq_one_right : a * b = 1 → a⁻¹ = b)
\end{verbatim}
Se puede interactuar con las pruebas anteriores en \href{https://lean.math.hhu.de/\#url=https://raw.githubusercontent.com/jaalonso/Calculemus2/main/src/CS\_de\_inverso.lean}{Lean 4 Web}.

\section{Si G es un grupo y a, b ∈ G, entonces (ab)⁻¹ = b⁻¹a⁻¹}
\label{sec:org83c40d8}
\begin{verbatim}
-- ---------------------------------------------------------------------
-- Demostrar que si G es un grupo y a, b ∈ G, entonces
--    (a * b)⁻¹ = b⁻¹ * a⁻¹
-- ----------------------------------------------------------------------

-- Demostración en lenguaje natural
-- ================================

-- Teniendo en cuenta la propiedad
--    ∀ a b ∈ R, ab = 1 → a⁻¹ = b,
-- basta demostrar que
--    (a·b)·(b⁻¹·a⁻¹) = 1.
-- La identidad anterior se demuestra mediante la siguiente cadena de
-- igualdades
--    (a·b)·(b⁻¹·a⁻¹) =  a·(b·(b⁻¹·a⁻¹))   [por la asociativa]
--                    =  a·((b·b⁻¹)·a⁻¹)   [por la asociativa]
--                    =  a·(1·a⁻¹)         [por producto con inverso]
--                    =  a·a⁻¹             [por producto con uno]
--                    =  1                 [por producto con inverso]

-- Demostraciones con Lean4
-- ========================

import Mathlib.Algebra.Group.Defs

variable {G : Type _} [Group G]
variable (a b : G)

lemma aux : (a * b) * (b⁻¹ * a⁻¹) = 1 :=
calc
  (a * b) * (b⁻¹ * a⁻¹)
    = a * (b * (b⁻¹ * a⁻¹)) := by rw [mul_assoc]
  _ = a * ((b * b⁻¹) * a⁻¹) := by rw [mul_assoc]
  _ = a * (1 * a⁻¹)         := by rw [mul_right_inv]
  _ = a * a⁻¹               := by rw [one_mul]
  _ = 1                     := by rw [mul_right_inv]

-- 1ª demostración
example : (a * b)⁻¹ = b⁻¹ * a⁻¹ :=
by
  have h1 : (a * b) * (b⁻¹ * a⁻¹) = 1 :=
    aux a b
  show (a * b)⁻¹ = b⁻¹ * a⁻¹
  exact inv_eq_of_mul_eq_one_right h1

-- 3ª demostración
example : (a * b)⁻¹ = b⁻¹ * a⁻¹ :=
by
  have h1 : (a * b) * (b⁻¹ * a⁻¹) = 1 :=
    aux a b
  show (a * b)⁻¹ = b⁻¹ * a⁻¹
  simp [h1]

-- 4ª demostración
example : (a * b)⁻¹ = b⁻¹ * a⁻¹ :=
by
  have h1 : (a * b) * (b⁻¹ * a⁻¹) = 1 :=
    aux a b
  simp [h1]

-- 5ª demostración
example : (a * b)⁻¹ = b⁻¹ * a⁻¹ :=
by
  apply inv_eq_of_mul_eq_one_right
  rw [aux]

-- 6ª demostración
example : (a * b)⁻¹ = b⁻¹ * a⁻¹ :=
by exact mul_inv_rev a b

-- 7ª demostración
example : (a * b)⁻¹ = b⁻¹ * a⁻¹ :=
by simp

-- Lemas usados
-- ============

-- variable (c : G)
-- #check (inv_eq_of_mul_eq_one_right : a * b = 1 → a⁻¹ = b)
-- #check (mul_assoc a b c : (a * b) * c = a * (b * c))
-- #check (mul_inv_rev a b : (a * b)⁻¹ = b⁻¹ * a⁻¹)
-- #check (mul_right_inv a : a * a⁻¹ = 1)
-- #check (one_mul a : 1 * a = a)
\end{verbatim}
Se puede interactuar con las pruebas anteriores en \href{https://lean.math.hhu.de/\#url=https://raw.githubusercontent.com/jaalonso/Calculemus2/main/src/Inverso\_del\_producto.lean}{Lean 4 Web}.

\chapter{Propiedades de orden en los números reales}
\label{sec:org6b407d0}

\section{En ℝ, si a ≤ b, b < c, c ≤ d y d < e, entonces a < e}
\label{sec:org8a134b8}
\begin{verbatim}
-- ---------------------------------------------------------------------
-- Demostrar que si a, b, c, d y e son números reales tales que
--    a ≤ b,
--    b < c,
--    c ≤ d y
--    d < e,
-- entonces
--    a < e.
-- ----------------------------------------------------------------------

-- Demostraciones en lenguaje natural (LN)
-- =======================================

-- 1ª demostración en LN
-- =====================

-- Por la siguiente cadena de desigualdades
--    a ≤ b    [por h1]
--      < c    [por h2]
--      ≤ d    [por h3]
--      < e    [por h4]

-- 2ª demostración en LN
-- =====================

-- A partir de las hipótesis 1 (a ≤ b) y 2 (b < c) se tiene
--    a < c
-- que, junto la hipótesis 3 (c ≤ d) da
--    a < d
-- que, junto la hipótesis 4 (d < e) da
--    a < e.

-- 3ª demostración en LN
-- =====================

-- Para demostrar a < e, por la hipótesis 1 (a ≤ b) se reduce a probar
--    b < e
-- que, por la hipótesis 2 (b < c), se reduce a
--    c < e
-- que, por la hipótesis 3 (c ≤ d), se reduce a
--    d < e
-- que es cierto, por la hipótesis 4.

-- Demostraciones con Lean4
-- ========================

import Mathlib.Data.Real.Basic

variable (a b c d e : ℝ)

-- 1ª demostración
-- ===============

example
  (h1 : a ≤ b)
  (h2 : b < c)
  (h3 : c ≤ d)
  (h4 : d < e) :
  a < e :=
calc
  a ≤ b := h1
  _ < c := h2
  _ ≤ d := h3
  _ < e := h4

-- 2ª demostración
-- ===============

example
  (h1 : a ≤ b)
  (h2 : b < c)
  (h3 : c ≤ d)
  (h4 : d < e) :
  a < e :=
by
  have h5 : a < c := lt_of_le_of_lt h1 h2
  have h6 : a < d := lt_of_lt_of_le h5 h3
  show a < e
  exact lt_trans h6 h4

-- 3ª demostración
-- ===============

example
  (h1 : a ≤ b)
  (h2 : b < c)
  (h3 : c ≤ d)
  (h4 : d < e) :
  a < e :=
by
  apply lt_of_le_of_lt h1
  apply lt_trans h2
  apply lt_of_le_of_lt h3
  exact h4

-- El desarrollo de la prueba es
--
--    a b c d e : ℝ,
--    h1 : a ≤ b,
--    h2 : b < c,
--    h3 : c ≤ d,
--    h4 : d < e
--    ⊢ a < e
-- apply lt_of_le_of_lt h1,
--    ⊢ b < e
-- apply lt_trans h2,
--    ⊢ c < e
-- apply lt_of_le_of_lt h3,
--    ⊢ d < e
-- exact h4,
--    no goals

-- 4ª demostración
-- ===============

example
  (h1 : a ≤ b)
  (h2 : b < c)
  (h3 : c ≤ d)
  (h4 : d < e) :
  a < e :=
by linarith

-- Lemas usados
-- ============

-- #check (lt_of_le_of_lt : a ≤ b → b < c → a < c)
-- #check (lt_of_lt_of_le : a < b → b ≤ c → a < c)
-- #check (lt_trans : a < b → b < c → a < c)
\end{verbatim}
Se puede interactuar con las pruebas anteriores en \href{https://lean.math.hhu.de/\#url=https://raw.githubusercontent.com/jaalonso/Calculemus2/main/src/Cadena\_de\_desigualdades.lean}{Lean 4 Web}.

\section{En ℝ, si 2a ≤ 3b, 1 ≤ a y d = 2, entonces d + a ≤ 5b}
\label{sec:org6131a02}
\begin{verbatim}
-- ---------------------------------------------------------------------
-- Demostrar que si a, b y c son números reales tales que
--    2 * a ≤ 3 * b
--    1 ≤ a
--    c = 2
-- entonces
--    c + a ≤ 5 * b
-- ----------------------------------------------------------------------

-- Demostración en lenguaje natural
-- ================================

-- Por la siguiente cadena de desigualdades
--    c + a = 2 + a      [por la hipótesis 3 (c = 2)]
--          ≤ 2·a + a    [por la hipótesis 2 (1 ≤ a)]
--          = 3·a
--          ≤ 9/2·b      [por la hipótesis 1 (2·a ≤ 3·b)]
--          ≤ 5·b

-- Demostraciones con Lean4
-- ========================

import Mathlib.Data.Real.Basic

variable (a b c : ℝ)

-- 1ª demostración
example
  (h1 : 2 * a ≤ 3 * b)
  (h2 : 1 ≤ a)
  (h3 : c = 2)
  : c + a ≤ 5 * b :=
calc
  c + a = 2 + a     := by rw [h3]
      _ ≤ 2 * a + a := by linarith only [h2]
      _ = 3 * a     := by linarith only []
      _ ≤ 9/2 * b   := by linarith only [h1]
      _ ≤ 5 * b     := by linarith

-- 2ª demostración
example
  (h1 : 2 * a ≤ 3 * b)
  (h2 : 1 ≤ a)
  (h3 : c = 2)
  : c + a ≤ 5 * b :=
by linarith
\end{verbatim}
Se puede interactuar con las pruebas anteriores en \href{https://lean.math.hhu.de/\#url=https://raw.githubusercontent.com/jaalonso/Calculemus2/main/src/Inecuaciones.lean}{Lean 4 Web}.

\section{En ℝ, si 1 ≤ a y b ≤ d, entonces 2 + a + eᵇ ≤ 3a + eᵈ}
\label{sec:org274cd1b}
\begin{verbatim}
-- ---------------------------------------------------------------------
-- Sean a, b, y d números reales. Demostrar  que si
--    1 ≤ a
--    b ≤ d
-- entonces
--    2 + a + exp b ≤ 3 * a + exp d
-- ----------------------------------------------------------------------

-- Demostración en lenguaje natural
-- ================================

-- De la primera hipótesis (1 ≤ a), multiplicando por 2, se obtiene
--    2 ≤ 2a
-- y, sumando a ambos lados, se tiene
--    2 + a ≤ 3a             (1)
-- De la hipótesis 2 (b ≤ d) y de la monotonía de la función exponencial
-- se tiene
--    e^b ≤ e^d              (2)
-- Finalmente, de (1) y (2) se tiene
--    2 + a + e^b ≤ 3a + e^d

-- Demostraciones con Lean4
-- ========================

import Mathlib.Analysis.SpecialFunctions.Log.Basic

open Real

variable (a b d : ℝ)

-- 1ª demostración
example
  (h1 : 1 ≤ a)
  (h2 : b ≤ d)
  : 2 + a + exp b ≤ 3 * a + exp d :=
by
  have h3 : 2 + a ≤ 3 * a := calc
    2 + a = 2 * 1 + a := by linarith only []
        _ ≤ 2 * a + a := by linarith only [h1]
        _ ≤ 3 * a     := by linarith only []
  have h4 : exp b ≤ exp d := by
    linarith only [exp_le_exp.mpr h2]
  show 2 + a + exp b ≤ 3 * a + exp d
  exact add_le_add h3 h4

-- 2ª demostración
example
  (h1 : 1 ≤ a)
  (h2 : b ≤ d)
  : 2 + a + exp b ≤ 3 * a + exp d :=
calc
  2 + a + exp b
    ≤ 3 * a + exp b := by linarith only [h1]
  _ ≤ 3 * a + exp d := by linarith only [exp_le_exp.mpr h2]

-- 3ª demostración
example
  (h1 : 1 ≤ a)
  (h2 : b ≤ d)
  : 2 + a + exp b ≤ 3 * a + exp d :=
by linarith [exp_le_exp.mpr h2]

-- Lemas usados
-- ============

-- variable (c : ℝ)
-- #check (add_le_add : a ≤ b → c ≤ d → a + c ≤ b + d)
-- #check (exp_le_exp : exp a ≤ exp b ↔ a ≤ b)
\end{verbatim}
Se puede interactuar con las pruebas anteriores en \href{https://lean.math.hhu.de/\#url=https://raw.githubusercontent.com/jaalonso/Calculemus2/main/src/Inecuaciones\_con\_exponenciales.lean}{Lean 4 Web}.

\section{En ℝ, si a ≤ b y c < d, entonces a + eᶜ + f ≤ b + eᵈ + f}
\label{sec:org0bb0a3c}
\begin{verbatim}
-- ---------------------------------------------------------------------
-- Demostrar que si a, b, c, d y f son números reales tales que
--    a ≤ b
--    c < d
-- entonces
--    a + eᶜ + f ≤ b + eᵈ + f
-- ----------------------------------------------------------------------

-- Demostraciones en lenguaje natural (LN)
-- =======================================

-- 1ª demostración en LN
-- =====================

-- Aplicando a la hipótesis 3 (c < d) la monotonía de la exponencial, se
-- tiene
--    e^c < e^d
-- que, junto a la hipótesis 1 (a ≤ b) y la monotonía de la suma da
--    a + e^c < b + e^d
-- y, de nuevo por la monotonía de la suma, se tiene
--    a + e^c + f < b + e^d + f

-- 2ª demostración en LN
-- =====================

-- Tenemos que demostrar que
--    (a + e^c) + f < (b + e^d) + f
-- que, por la monotonía de la suma, se reduce a las siguientes dos
-- desigualdades:
--    a + e^c < b + e^d                                               (1)
--    f ≤ f                                                           (2)
--
-- La (1), de nuevo por la monotonía de la suma, se reduce a las
-- siguientes dos:
--    a ≤ b                                                         (1.1)
--    e^c < e^d                                                     (1.2)
--
-- La (1.1) se tiene por la hipótesis 1.
--
-- La (1.2) se tiene aplicando la monotonía de la exponencial a la
-- hipótesis 2.
--
-- La (2) se tiene por la propiedad reflexiva.

-- Demostraciones con Lean4
-- ========================

import Mathlib.Analysis.SpecialFunctions.Log.Basic
open Real
variable (a b c d f : ℝ)

-- 1ª demostración
example
  (h1 : a ≤ b)
  (h2 : c < d)
  : a + exp c + f < b + exp d + f :=
by
  have h3 : exp c < exp d :=
    exp_lt_exp.mpr h2
  have h4 : a + exp c < b + exp d :=
    add_lt_add_of_le_of_lt h1 h3
  show a + exp c + f < b + exp d + f
  exact add_lt_add_right h4 f

-- 2ª demostración
example
  (h1 : a ≤ b)
  (h2 : c < d)
  : a + exp c + f < b + exp d + f :=
by
  apply add_lt_add_of_lt_of_le
  { apply add_lt_add_of_le_of_lt
    { exact h1 }
    { apply exp_lt_exp.mpr
      exact h2 } }
  { apply le_refl }

-- 3ª demostración
example
  (h1 : a ≤ b)
  (h2 : c < d)
  : a + exp c + f < b + exp d + f :=
by
  apply add_lt_add_of_lt_of_le
  . apply add_lt_add_of_le_of_lt h1
    apply exp_lt_exp.mpr h2
  rfl

-- Lemas usados
-- ============

-- #check (add_lt_add_of_le_of_lt : a ≤ b → c < d → a + c < b + d)
-- #check (add_lt_add_of_lt_of_le : a < b → c ≤ d → a + c < b + d)
-- #check (add_lt_add_right : b < c → ∀ (a : ℝ), b + a < c + a)
-- #check (exp_lt_exp : exp a < exp b ↔ a < b)
-- #check (le_refl a : a ≤ a)
\end{verbatim}
Se puede interactuar con las pruebas anteriores en \href{https://lean.math.hhu.de/\#url=https://raw.githubusercontent.com/jaalonso/Calculemus2/main/src/Inecuaciones\_con\_exponenciales\_2.lean}{Lean 4 Web}

\section{En ℝ, si d ≤ f, entonces c + e\^{}(a + d) ≤ c + e\^{}(a + f)}
\label{sec:org550d0dc}
\begin{verbatim}
-- ---------------------------------------------------------------------
-- Demostrar que si a, c, d y f son números reales tales que
--    d ≤ f
-- entonces
--    c + exp (a + d) ≤ c + exp (a + f)
-- ----------------------------------------------------------------------

-- Demostraciones en lenguaje natural (LN)
-- =======================================

-- 1ª demostración en LN
-- =====================

-- De la hipótesis, por la monotonia de la suma, se tiene
--    a + d ≤ a + f
-- que, por la monotonía de la exponencial, da
--    exp (a + d) ≤ exp (a + f)
-- y, por la monotonía de la suma, se tiene
--    c + exp (a + d) ≤ c + exp (a + f)

-- 2ª demostración en LN
-- =====================

-- Tenemos que demostrar que
--    c + exp (a + d) ≤ c + exp (a + f)
-- Por la monotonía de la suma, se reduce a
--    exp (a + d) ≤ exp (a + f)
-- que, por la monotonía de la exponencial, se reduce a
--    a + d ≤ a + f
-- que, por la monotonía de la suma, se reduce a
--    d ≤ f
-- que es la hipótesis.

-- Demostraciones con Lean4
-- ========================

import Mathlib.Analysis.SpecialFunctions.Log.Basic
open Real
variable (a c d f : ℝ)

-- 1ª demostración
example
  (h : d ≤ f)
  : c + exp (a + d) ≤ c + exp (a + f) :=
by
  have h1 : a + d ≤ a + f :=
    add_le_add_left h a
  have h2 : exp (a + d) ≤ exp (a + f) :=
    exp_le_exp.mpr h1
  show c + exp (a + d) ≤ c + exp (a + f)
  exact add_le_add_left h2 c

-- 2ª demostración
example
  (h : d ≤ f)
  : c + exp (a + d) ≤ c + exp (a + f) :=
by
  apply add_le_add_left
  apply exp_le_exp.mpr
  apply add_le_add_left
  exact h

-- Lemas usados
-- ============

-- variable (b : ℝ)
-- #check (add_le_add_left : b ≤ c → ∀ (a : ℝ), a + b ≤ a + c)
-- #check (exp_le_exp : exp a ≤ exp b ↔ a ≤ b)
\end{verbatim}
Se puede interactuar con las pruebas anteriores en \href{https://lean.math.hhu.de/\#url=https://raw.githubusercontent.com/jaalonso/Calculemus2/main/src/Inecuaciones\_con\_exponenciales\_3.lean}{Lean 4 Web}

\section{En ℝ, si a ≤ b, entonces log(1+e\^{}a) ≤ log(1+e\^{}b)}
\label{sec:orgac9fbc1}
\begin{verbatim}
-- ---------------------------------------------------------------------
-- Demostrar que si a y b son números reales tales que
--    a ≤ b
-- entonces
--    log(1+e^a) ≤ log(1+e^b)
-- ----------------------------------------------------------------------

-- Demostración en lenguaje natural
-- ================================

-- Por la monotonía del logaritmo, basta demostrar que
--    0 < 1 + e^a                 (1)
--    1 + e^a ≤ 1 + e^b           (2)
--
-- La (1), por la suma de positivos, se reduce a
--    0 < 1                       (1.1)
--    0 < e^a                     (1.2)
-- La (1.1) es una propiedad de los números naturales y la (1.2) de la
-- función exponencial.
--
-- La (2), por la monotonía de la suma, se reduce a
--    e^a ≤ e^b
-- que, por la monotonía de la exponencial, se reduce a
--    a ≤ b
-- que es la hipótesis.

-- Demostraciones con Lean4
-- ========================

import Mathlib.Analysis.SpecialFunctions.Log.Basic
open Real
variable (a b : ℝ)

-- 1ª demostración
example
  (h : a ≤ b)
  : log (1 + exp a) ≤ log (1 + exp b) :=
by
  have h1 : (0 : ℝ) < 1 :=
    zero_lt_one
  have h2 : 0 < exp a :=
    exp_pos a
  have h3 : 0 < 1 + exp a :=
    add_pos h1 h2
  have h4 : exp a ≤ exp b :=
    exp_le_exp.mpr h
  have h5 : 1 + exp a ≤ 1 + exp b :=
    add_le_add_left h4 1
  show log (1 + exp a) ≤ log (1 + exp b)
  exact log_le_log' h3 h5

-- 2ª demostraciṕn
example
  (h : a ≤ b)
  : log (1 + exp a) ≤ log (1 + exp b) :=
by
  apply log_le_log'
  { apply add_pos
    { exact zero_lt_one }
    { exact exp_pos a }}
  { apply add_le_add_left
    exact exp_le_exp.mpr h }

-- Lemas usados
-- ============

-- variable (c : ℝ)
-- #check (add_le_add_left : b ≤ c → ∀ (a : ℝ), a + b ≤ a + c)
-- #check (add_pos : 0 < a → 0 < b → 0 < a + b)
-- #check (exp_le_exp : exp a ≤ exp b ↔ a ≤ b)
-- #check (exp_pos a : 0 < exp a)
-- #check (log_le_log' : 0 < a → a ≤ b → log a ≤ log b)
-- #check (zero_lt_one : 0 < 1)
\end{verbatim}
Se puede interactuar con las pruebas anteriores en \href{https://lean.math.hhu.de/\#url=https://raw.githubusercontent.com/jaalonso/Calculemus2/main/src/Desigualdad\_logaritmica.lean}{Lean 4 Web}

\section{En ℝ, si a ≤ b, entonces c - e\^{}b ≤ c - e\^{}a}
\label{sec:orgcc6a24e}
\begin{verbatim}
-- ---------------------------------------------------------------------
-- Sean a, b y c números reales. Demostrar que si
--    a ≤ b
-- entonces
--    c - e^b ≤ c - e^a
-- ----------------------------------------------------------------------

-- Demostración en lenguaje natural
-- ================================

-- Aplicando la monotonía de la exponencial a la hipótesis, se tiene
--    e^a ≤ e^b
-- y, restando de c, se invierte la desigualdad
--    c - e^b ≤ c - e^a

-- Demostraciones con Lean4
-- ========================

import Mathlib.Analysis.SpecialFunctions.Log.Basic

open Real

variable (a b c : ℝ)

-- 1ª demostración
example
  (h : a ≤ b)
  : c - exp b ≤ c - exp a :=
by
   have h1 : exp a ≤ exp b :=
     exp_le_exp.mpr h
   show c - exp b ≤ c - exp a
   exact sub_le_sub_left h1 c

-- 2ª demostración
example
  (h : a ≤ b)
  : c - exp b ≤ c - exp a :=
by
   apply sub_le_sub_left _ c
   apply exp_le_exp.mpr h

-- 3ª demostración
example
  (h : a ≤ b)
  : c - exp b ≤ c - exp a :=
sub_le_sub_left (exp_le_exp.mpr h) c

-- 4ª demostración
example
  (h : a ≤ b)
  : c - exp b ≤ c - exp a :=
by linarith [exp_le_exp.mpr h]

-- Lemas usados
-- ============

-- #check (exp_le_exp : exp a ≤ exp b ↔ a ≤ b)
-- #check (sub_le_sub_left : a ≤ b → ∀ (c : ℝ), c - b ≤ c - a)
\end{verbatim}
Se puede interactuar con las pruebas anteriores en \href{https://lean.math.hhu.de/\#url=https://raw.githubusercontent.com/jaalonso/Calculemus2/main/src/Desigualdad\_logaritmica.lean}{Lean 4 Web}

\section{En ℝ, 2ab ≤ a² + b²}
\label{sec:orge51683e}
\begin{verbatim}
-- ---------------------------------------------------------------------
-- Sean a y b números reales. Demostrar que
--    2ab ≤ a² + b²
-- ----------------------------------------------------------------------

-- Demostración en lenguaje natural
-- ================================

-- Puesto que los cuadrados son positivos, se tiene
--    (a - b)² ≥ 0
-- Desarrollando el cuadrado, se obtiene
--    a² - 2ab + b² ≥ 0
-- Sumando 2ab a ambos lados, queda
--    a² + b² ≥ 2ab

-- Demostraciones con Lean4
-- ========================

import Mathlib.Data.Real.Basic

variable (a b : ℝ)

-- 1ª demostración
example : 2*a*b ≤ a^2 + b^2 :=
by
  have h1 : 0 ≤ (a - b)^2         := sq_nonneg (a - b)
  have h2 : 0 ≤ a^2 - 2*a*b + b^2 := by linarith only [h1]
  show 2*a*b ≤ a^2 + b^2
  linarith

-- 2ª demostración
example : 2*a*b ≤ a^2 + b^2 :=
by
  have h : 0 ≤ a^2 - 2*a*b + b^2
  { calc a^2 - 2*a*b + b^2
         = (a - b)^2                 := (sub_sq a b).symm
       _ ≥ 0                         := sq_nonneg (a - b) }
  calc 2*a*b
       = 2*a*b + 0                   := (add_zero (2*a*b)).symm
     _ ≤ 2*a*b + (a^2 - 2*a*b + b^2) := add_le_add (le_refl _) h
     _ = a^2 + b^2                   := by ring

-- 3ª demostración
example : 2*a*b ≤ a^2 + b^2 :=
by
  have h : 0 ≤ a^2 - 2*a*b + b^2
  { calc a^2 - 2*a*b + b^2
         = (a - b)^2       := (sub_sq a b).symm
       _ ≥ 0               := sq_nonneg (a - b) }
  linarith only [h]

-- 4ª demostración
example : 2*a*b ≤ a^2 + b^2 :=
-- by apply?
two_mul_le_add_sq a b

-- Lemas usados
-- ============

-- variable (c : ℝ)
-- #check (add_le_add : a ≤ b → c ≤ d → a + c ≤ b + d)
-- #check (add_zero a : a + 0 = a)
-- #check (sq_nonneg a : 0 ≤ a ^ 2)
-- #check (sub_sq a b : (a - b) ^ 2 = a ^ 2 - 2 * a * b + b ^ 2)
-- #check (two_mul_le_add_sq a b : 2 * a * b ≤ a ^ 2 + b ^ 2)
\end{verbatim}
Se puede interactuar con las pruebas anteriores en \href{https://lean.math.hhu.de/\#url=https://raw.githubusercontent.com/jaalonso/Calculemus2/main/src/Desigualdad\_logaritmica.lean}{Lean 4 Web}

\section{En ℝ, |ab| ≤ (a²+b²)/2]}
\label{sec:org232c33a}
\begin{verbatim}
-- ---------------------------------------------------------------------
-- Sean a y b números reales. Demostrar que
--    |a*b| ≤ (a^2 + b^2) / 2
-- ----------------------------------------------------------------------

-- Demostración en lenguaje natural
-- ================================

-- Para demostrar
--    |ab| ≤ (a² + b² / 2
-- basta demostrar estas dos desigualdades
--    ab ≤ (a² + b²) / 2                                              (1)
--    -(ab) ≤ (a² + b²) / 2                                           (2)
--
-- Para demostrar (1) basta demostrar que
--    2ab ≤ a² + b²
-- que se prueba como sigue. En primer lugar, como los cuadrados son no
-- negativos, se tiene
--   (a - b)² ≥ 0
-- Desarrollando el cuandrado,
--   a² - 2ab + b² ≥ 0
-- Sumando 2ab,
--   a² + b² ≥ 2ab
--
-- Para demostrar (2) basta demostrar que
--    -2ab ≤ a² + b²
-- que se prueba como sigue. En primer lugar, como los cuadrados son no
-- negativos, se tiene
--   (a + b)² ≥ 0
-- Desarrollando el cuandrado,
--   a² + 2ab + b² ≥ 0
-- Restando 2ab,
--   a² + b² ≥ -2ab

-- Demostraciones con Lean4
-- ========================

import Mathlib.Data.Real.Basic

variable (a b : ℝ)

-- Lemas auxiliares
-- ================

lemma aux1 : a * b * 2 ≤ a ^ 2 + b ^ 2 := by
  have h : 0 ≤ a ^ 2 - 2 * a * b + b ^ 2
  calc
    a ^ 2 - 2 * a * b + b ^ 2
      = (a - b) ^ 2            := by ring
    _ ≥ 0                      := pow_two_nonneg (a - b)
  linarith only [h]

lemma aux2 : -(a * b) * 2 ≤ a ^ 2 + b ^ 2 := by
  have h : 0 ≤ a ^ 2 + 2 * a * b + b ^ 2
  calc
    a ^ 2 + 2 * a * b + b ^ 2
      = (a + b) ^ 2            := by ring
    _ ≥ 0                      := pow_two_nonneg (a + b)
  linarith only [h]

-- 1ª demostración
-- ===============

example : |a * b| ≤ (a ^ 2 + b ^ 2) / 2 := by
  have h : (0 : ℝ) < 2 := by norm_num
  apply abs_le'.mpr
  constructor
  { have h1 : a * b * 2 ≤ a ^ 2 + b ^ 2 := aux1 a b
    show a * b ≤ (a ^ 2 + b ^ 2) / 2
    exact (le_div_iff h).mpr h1 }
  { have h2 : -(a * b) * 2 ≤ a ^ 2 + b ^ 2 := aux2 a b
    show -(a * b) ≤ (a ^ 2 + b ^ 2) / 2
    exact (le_div_iff h).mpr h2 }

-- 2ª demostración
-- ===============

example : |a * b| ≤ (a ^ 2 + b ^ 2) / 2 := by
  have h : (0 : ℝ) < 2 := by norm_num
  apply abs_le'.mpr
  constructor
  { exact (le_div_iff h).mpr (aux1 a b) }
  { exact (le_div_iff h).mpr (aux2 a b) }

-- 3ª demostración
-- ===============

example : |a * b| ≤ (a ^ 2 + b ^ 2) / 2 := by
  have h : (0 : ℝ) < 2 := by norm_num
  apply abs_le'.mpr
  constructor
  { rw [le_div_iff h]
    apply aux1 }
  { rw [le_div_iff h]
    apply aux2 }

-- Lemas usados
-- ============

-- variable (c : ℝ)
-- #check (abs_le' : |a| ≤ b ↔ a ≤ b ∧ -a ≤ b)
-- #check (le_div_iff : 0 < c → (a ≤ b / c ↔ a * c ≤ b))
-- #check (pow_two_nonneg a : 0 ≤ a ^ 2)
\end{verbatim}
Se puede interactuar con las pruebas anteriores en \href{https://lean.math.hhu.de/\#url=https://raw.githubusercontent.com/jaalonso/Calculemus2/main/src/Ejercicio\_desigualdades\_absolutas.lean}{Lean 4 Web}

\section{En ℝ, min(a,b) = min(b,a)}
\label{sec:org88fc131}
\begin{verbatim}
-- ---------------------------------------------------------------------
-- Sean a y b números reales. Demostrar que
--    min a b = min b a
-- ----------------------------------------------------------------------

-- Demostración en lenguaje natural
-- ================================

-- Es consecuencia de la siguiente propiedad
--    min(a, b) ≤ min(b, a)                                          (1)
-- En efecto, intercambiando las variables en (1) se obtiene
--    min(b, a) ≤ min(a, b)                                          (2)
-- Finalmente de (1) y (2) se obtiene
--    min(b, a) = min(a, b)
--
-- Para demostrar (1), se observa que
--    min(a, b) ≤ b
--    min(a, b) ≤ a
-- y, por tanto,
--    min(a, b) = min(b, a)

-- Demostraciones con Lean4
-- ========================

import Mathlib.Data.Real.Basic

variable (a b : ℝ)

-- Lema auxiliar
-- =============

-- 1ª demostración del lema auxiliar
-- =================================

example : min a b ≤ min b a :=
by
  have h1 : min a b ≤ b := min_le_right a b
  have h2 : min a b ≤ a := min_le_left a b
  show min a b ≤ min b a
  exact le_min h1 h2

-- 2ª demostración del lema auxiliar
-- =================================

example : min a b ≤ min b a :=
by
  apply le_min
  { apply min_le_right }
  { apply min_le_left }

-- 3ª demostración del lema auxiliar
-- =================================

lemma aux : min a b ≤ min b a :=
by exact le_min (min_le_right a b) (min_le_left a b)

-- 1ª demostración
-- ===============

example : min a b = min b a :=
by
  apply le_antisymm
  { exact aux a b}
  { exact aux b a}

-- 2ª demostración
-- ===============

example : min a b = min b a :=
le_antisymm (aux a b) (aux b a)

-- 3ª demostración
-- ===============

example : min a b = min b a :=
min_comm a b

-- Lemas usados
-- ============

-- variable (c : ℝ)
-- #check (le_antisymm : a ≤ b → b ≤ a → a = b)
-- #check (le_min : c ≤ a → c ≤ b → c ≤ min a b)
-- #check (min_comm a b : min a b = min b a)
-- #check (min_le_left a b : min a b ≤ a)
-- #check (min_le_right a b : min a b ≤ b)
\end{verbatim}
Se puede interactuar con las pruebas anteriores en \href{https://lean.math.hhu.de/\#url=https://raw.githubusercontent.com/jaalonso/Calculemus2/main/src/Conmutatividad\_del\_minimo.lean}{Lean 4 Web}

\section{En ℝ, max(a,b) = max(b,a)}
\label{sec:org734e0df}
\begin{verbatim}
-- ---------------------------------------------------------------------
-- Sean a y b números reales. Demostrar que
--    max a b = max b a
-- ----------------------------------------------------------------------

-- Demostración en lenguaje natural
-- ================================

-- Es consecuencia de la siguiente propiedad
--    max(a, b) ≤ max(b, a)                                          (1)
-- En efecto, intercambiando las variables en (1) se obtiene
--    max(b, a) ≤ max(a, b)                                          (2)
-- Finalmente de (1) y (2) se obtiene
--    max(b, a) = max(a, b)
--
-- Para demostrar (1), se observa que
--    a ≤ max(b, a)
--    b ≤ max(b, a)
-- y, por tanto,
--    max(a, b) ≤ max(b, a)

-- Demostraciones con Lean4
-- ========================

import Mathlib.Data.Real.Basic

variable (a b : ℝ)

-- Lema auxiliar
-- =============

-- 1ª demostración del lema auxiliar
-- =================================

example : max a b ≤ max b a :=
by
  have h1 : a ≤ max b a := le_max_right b a
  have h2 : b ≤ max b a := le_max_left b a
  show max a b ≤ max b a
  exact max_le h1 h2

-- 2ª demostración del lema auxiliar
-- =================================

example : max a b ≤ max b a :=
by
  apply max_le
  { apply le_max_right }
  { apply le_max_left }

-- 3ª demostración del lema auxiliar
-- =================================

lemma aux : max a b ≤ max b a :=
by exact max_le (le_max_right b a) (le_max_left b a)

-- 1ª demostración
-- ===============

example : max a b = max b a :=
by
  apply le_antisymm
  { exact aux a b}
  { exact aux b a}

-- 2ª demostración
-- ===============

example : max a b = max b a :=
le_antisymm (aux a b) (aux b a)

-- 3ª demostración
-- ===============

example : max a b = max b a :=
max_comm a b

-- Lemas usados
-- ============

-- variable (c : ℝ)
-- #check (le_antisymm : a ≤ b → b ≤ a → a = b)
-- #check (le_max_left a b : a ≤ max a b)
-- #check (le_max_right a b : b ≤ max a b)
-- #check (max_comm a b : max a b = max b a)
-- #check (max_le : a ≤ c → b ≤ c → max a b ≤ c)
\end{verbatim}
Se puede interactuar con las pruebas anteriores en \href{https://lean.math.hhu.de/\#url=https://raw.githubusercontent.com/jaalonso/Calculemus2/main/src/Conmutatividad\_del\_maximo.lean}{Lean 4 Web}

\section{En ℝ, min(min(a,b),c) = min(a,min(b,c))}
\label{sec:org6d5cf63}
\begin{verbatim}
-- ---------------------------------------------------------------------
-- Sean a, b y c números reales. Demostrar que
--    min (min a b) c = min a (min b c)
-- ----------------------------------------------------------------------

-- Demostración en lenguaje natural
-- ================================

-- Por la propiedad antisimétrica, la igualdad es consecuencia de las
-- siguientes desigualdades
--    min(min(a, b), c) ≤ min(a, min(b, c))                           (1)
--    min(a, min(b, c)) ≤ min(min(a, b), c)                           (2)
--
-- La (1) es consecuencia de las siguientes desigualdades
--    min(min(a, b), c) ≤ a                                          (1a)
--    min(min(a, b), c) ≤ b                                          (1b)
--    min(min(a, b), c) ≤ c                                          (1c)
-- En efecto, de (1b) y (1c) se obtiene
--    min(min(a, b), c) ≤ min(b,c)
-- que, junto con (1a) da (1).
--
-- La (2) es consecuencia de las siguientes desigualdades
--    min(a, min(b, c)) ≤ a                                          (2a)
--    min(a, min(b, c)) ≤ b                                          (2b)
--    min(a, min(b, c)) ≤ c                                          (2c)
-- En efecto, de (2a) y (2b) se obtiene
--    min(a, min(b, c)) ≤ min(a, b)
-- que, junto con (2c) da (2).
--
-- La demostración de (1a) es
--    min(min(a, b), c) ≤ min(a, b) ≤ a
-- La demostración de (1b) es
--    min(min(a, b), c) ≤ min(a, b) ≤ b
-- La demostración de (2b) es
--    min(a, min(b, c)) ≤ min(b, c) ≤ b
-- La demostración de (2c) es
--    min(a, min(b, c)) ≤ min(b, c) ≤ c
-- La (1c) y (2a) son inmediatas.

-- Demostraciones con Lean4
-- ========================

import Mathlib.Data.Real.Basic

variable {a b c : ℝ}

-- Lemas auxiliares
-- ================

lemma aux1a : min (min a b) c ≤ a :=
calc min (min a b) c
     ≤ min a b := by exact min_le_left (min a b) c
   _ ≤ a       := min_le_left a b

lemma aux1b : min (min a b) c ≤ b :=
calc min (min a b) c
     ≤ min a b := by exact min_le_left (min a b) c
   _ ≤ b       := min_le_right a b

lemma aux1c : min (min a b) c ≤ c :=
by exact min_le_right (min a b) c

-- 1ª demostración del lema aux1
lemma aux1 : min (min a b) c ≤ min a (min b c) :=
by
  apply le_min
  { show min (min a b) c ≤ a
    exact aux1a }
  { show min (min a b) c ≤ min b c
    apply le_min
    { show min (min a b) c ≤ b
      exact aux1b }
    { show min (min a b) c ≤ c
      exact aux1c }}

-- 2ª demostración del lema aux1
lemma aux1' : min (min a b) c ≤ min a (min b c) :=
le_min aux1a (le_min aux1b aux1c)

lemma aux2a : min a (min b c) ≤ a :=
by exact min_le_left a (min b c)

lemma aux2b : min a (min b c) ≤ b :=
calc min a (min b c)
     ≤ min b c        := by exact min_le_right a (min b c)
   _ ≤ b              := min_le_left b c

lemma aux2c : min a (min b c) ≤ c :=
calc min a (min b c)
     ≤ min b c        := by exact min_le_right a (min b c)
   _ ≤ c              := min_le_right b c

-- 1ª demostración del lema aux2
lemma aux2 : min a (min b c) ≤ min (min a b) c :=
by
  apply le_min
  { show min a (min b c) ≤ min a b
    apply le_min
    { show min a (min b c) ≤ a
      exact aux2a }
    { show min a (min b c) ≤ b
      exact aux2b }}
  { show min a (min b c) ≤ c
    exact aux2c }

-- 2ª demostración del lema aux2
lemma aux2' : min a (min b c) ≤ min (min a b) c :=
le_min (le_min aux2a aux2b) aux2c

-- 1ª demostración
-- ===============

example :
  min (min a b) c = min a (min b c) :=
by
  apply le_antisymm
  { show min (min a b) c ≤ min a (min b c)
    exact aux1 }
  { show min a (min b c) ≤ min (min a b) c
    exact aux2 }

-- 2ª demostración
-- ===============

example : min (min a b) c = min a (min b c) :=
by
  apply le_antisymm
  { exact aux1 }
  { exact aux2 }

-- 3ª demostración
-- ===============

example : min (min a b) c = min a (min b c) :=
le_antisymm aux1 aux2

-- 4ª demostración
-- ===============

example : min (min a b) c = min a (min b c) :=
min_assoc a b c

-- Lemas usados
-- ============

-- #check (le_antisymm : a ≤ b → b ≤ a → a = b)
-- #check (le_min : c ≤ a → c ≤ b → c ≤ min a b)
-- #check (min_assoc a b c : min (min a b) c = min a (min b c))
-- #check (min_le_left a b : min a b ≤ a)
-- #check (min_le_right a b : min a b ≤ b)
\end{verbatim}
Se puede interactuar con las pruebas anteriores en \href{https://lean.math.hhu.de/\#url=https://raw.githubusercontent.com/jaalonso/Calculemus2/main/src/Asociatividad\_del\_minimo.lean}{Lean 4 Web}

\section{En ℝ, min(a,b)+c = min(a+c,b+c)}
\label{sec:orgf25d1a6}
\begin{verbatim}
-- ---------------------------------------------------------------------
-- Sean a, b y c números reales. Demostrar que
--    min a b + c = min (a + c) (b + c)
-- ----------------------------------------------------------------------

-- Demostraciones en lenguaje natural (LN)
-- =======================================

-- 1ª demostración en LN
-- =====================

-- Aplicando la propiedad antisimétrica a las siguientes desigualdades
--    min(a, b) + c ≤ min(a + c, b + c)                               (1)
--    min(a + c, b + c) ≤ min(a, b) + c                               (2)
--
-- Para demostrar (1) basta demostrar que se verifican las siguientes
-- desigualdades
--    min(a, b) + c ≤ a + c                                          (1a)
--    min(a, b) + c ≤ b + c                                          (1b)
-- que se tienen porque se verifican las siguientes desigualdades
--    min(a, b) ≤ a
--    min(a, b) ≤ b
--
-- Para demostrar (2) basta demostrar que se verifica
--    min(a + c, b + c) - c ≤ min(a, b)
-- que se demuestra usando (1); en efecto,
--    min(a + c, b + c) - c ≤ min(a + c - c, b + c - c)    [por (1)]
--                          = min(a, b)

-- 2ª demostración en LN
-- =====================

-- Por casos según a ≤ b.
--
-- 1º caso: Supongamos que a ≤ b. Entonces,
--    min(a, b) + c = a + c
--                  = min(a + c, b + c)
--
-- 2º caso: Supongamos que a ≰ b. Entonces,
--    min(a, b) + c = b + c
--                  = min(a + c, b + c)

-- Demostraciones con Lean4
-- ========================

import Mathlib.Data.Real.Basic

variable {a b c : ℝ}

-- En las demostraciones se usarán los siguientes lemas auxiliares
--    aux1 : min a b + c ≤ min (a + c) (b + c)
--    aux2 : min (a + c) (b + c) ≤ min a b + c
-- cuyas demostraciones se exponen a continuación.

-- 1ª demostración de aux1
lemma aux1 :
  min a b + c ≤ min (a + c) (b + c) :=
by
  have h1 : min a b ≤ a :=
    min_le_left a b
  have h2 : min a b + c ≤ a + c :=
    add_le_add_right h1 c
  have h3 : min a b  ≤ b :=
    min_le_right a b
  have h4 : min a b + c ≤ b + c :=
    add_le_add_right h3 c
  show min a b + c ≤ min (a + c) (b + c)
  exact le_min h2 h4

-- 2ª demostración de aux1
example :
  min a b + c ≤ min (a + c) (b + c) :=
by
  apply le_min
  { apply add_le_add_right
    exact min_le_left a b }
  { apply add_le_add_right
    exact min_le_right a b }

-- 3ª demostración de aux1
example :
  min a b + c ≤ min (a + c) (b + c) :=
le_min (add_le_add_right (min_le_left a b) c)
       (add_le_add_right (min_le_right a b) c)

-- 1ª demostración de aux2
lemma aux2 :
  min (a + c) (b + c) ≤ min a b + c :=
by
  have h1 : min (a + c) (b + c) + -c ≤ min a b
  { calc min (a + c) (b + c) + -c
         ≤ min (a + c + -c) (b + c + -c) := aux1
       _ = min a b                       := by ring_nf }
  show min (a + c) (b + c) ≤ min a b + c
  exact add_neg_le_iff_le_add.mp h1

-- 1ª demostración del ejercicio
example :
  min a b + c = min (a + c) (b + c) :=
by
  have h1 : min a b + c ≤ min (a + c) (b + c) := aux1
  have h2 : min (a + c) (b + c) ≤ min a b + c := aux2
  show min a b + c = min (a + c) (b + c)
  exact le_antisymm h1 h2

-- 2ª demostración del ejercicio
example :
  min a b + c = min (a + c) (b + c) :=
by
  apply le_antisymm
  { show min a b + c ≤ min (a + c) (b + c)
    exact aux1 }
  { show min (a + c) (b + c) ≤ min a b + c
    exact aux2 }

-- 3ª demostración del ejercicio
example :
  min a b + c = min (a + c) (b + c) :=
by
  apply le_antisymm
  { exact aux1 }
  { exact aux2 }

-- 4ª demostración del ejercicio
example :
  min a b + c = min (a + c) (b + c) :=
le_antisymm aux1 aux2

-- 5ª demostración del ejercicio
example : min a b + c = min (a + c) (b + c) :=
by
  by_cases h : a ≤ b
  { have h1 : a + c ≤ b + c := add_le_add_right h c
    calc min a b + c = a + c               := by simp [min_eq_left h]
                   _ = min (a + c) (b + c) := by simp [min_eq_left h1]}
  { have h2: b ≤ a := le_of_not_le h
    have h3 : b + c ≤ a + c := add_le_add_right h2 c
    calc min a b + c = b + c               := by simp [min_eq_right h2]
                   _ = min (a + c) (b + c) := by simp [min_eq_right h3]}

-- 6ª demostración del ejercicio
example : min a b + c = min (a + c) (b + c) :=
(min_add_add_right a b c).symm

-- Lemas usados
-- ============

-- #check (add_le_add_right : b ≤ c → ∀ (a : ℝ),  b + a ≤ c + a)
-- #check (add_neg_le_iff_le_add : a - b ≤ c ↔ a ≤ c + b)
-- #check (le_antisymm : a ≤ b → b ≤ a → a = b)
-- #check (le_min : c ≤ a → c ≤ b → c ≤ min a b)
-- #check (min_add_add_right a b c : min (a + c) (b + c) = min a b + c)
-- #check (min_eq_left : a ≤ b → min a b = a)
-- #check (min_eq_right : b ≤ a → min a b = b)
-- #check (min_le_left a b : min a b ≤ a)
-- #check (min_le_right a b : min a b ≤ b)
\end{verbatim}
Se puede interactuar con las pruebas anteriores en \href{https://lean.math.hhu.de/\#url=https://raw.githubusercontent.com/jaalonso/Calculemus2/main/src/Minimo\_de\_suma.lean}{Lean 4 Web}

\section{En ℝ, |a| - |b| ≤ |a - b|}
\label{sec:orgf516a3e}
\begin{verbatim}
-- ---------------------------------------------------------------------
-- Sean a y b números reales. Demostrar que
--    |a| - |b| ≤ |a - b|
-- ----------------------------------------------------------------------

-- Demostraciones en lenguaje natural (LN)
-- =======================================

-- 1ª demostración en LN
-- =====================

-- Por la siguiente cadena de desigualdades
--    |a| - |b| = |a - b + b| - |b|
--              ≤ (|a - b| + |b|) - |b|   [por la desigualdad triangular]
--              = |a - b|

-- 2ª demostración en LN
-- =====================

-- Por la desigualdad triangular
--    |a - b + b| ≤ |a - b| + |b|
-- simplificando en la izquierda
--    |a| ≤ |a - b| + |b|
-- y, pasando |b| a la izquierda
--    |a| - |b| ≤ |a - b|

-- Demostraciones con Lean4
-- ========================

import Mathlib.Data.Real.Basic

variable (a b : ℝ)

-- 1ª demostración (basada en la 1ª en LN)
example : |a| - |b| ≤ |a - b| :=
calc |a| - |b|
     = |a - b + b| - |b| :=
          congrArg (fun x => |x| - |b|) (sub_add_cancel a b).symm
   _ ≤ (|a - b| + |b|) - |b| :=
           sub_le_sub_right (abs_add (a - b) b) (|b|)
   _ = |a - b| :=
          add_sub_cancel (|a - b|) (|b|)

-- 2ª demostración (basada en la 1ª en LN)
example : |a| - |b| ≤ |a - b| :=
calc |a| - |b|
     = |a - b + b| - |b| := by
          rw [sub_add_cancel]
   _ ≤ (|a - b| + |b|) - |b| := by
          apply sub_le_sub_right
          apply abs_add
   _ = |a - b| := by
          rw [add_sub_cancel]

-- 3ª demostración (basada en la 2ª en LN)
example : |a| - |b| ≤ |a - b| :=
by
  have h1 : |a - b + b| ≤ |a - b| + |b| := abs_add (a - b) b
  rw [sub_add_cancel] at h1
  exact abs_sub_abs_le_abs_sub a b

-- 4ª demostración
example : |a| - |b| ≤ |a - b| :=
abs_sub_abs_le_abs_sub a b

-- Lemas usados
-- ============

-- #check (abs_add a b : |a + b| ≤ |a| + |b|)
-- #check (abs_sub_abs_le_abs_sub a b : |a| - |b| ≤ |a - b|)
-- #check (add_sub_cancel a b : a + b - b = a)
-- #check (sub_add_cancel a b : a - b + b = a)
-- #check (sub_le_sub_right : a ≤ b → ∀ (c : ℝ), a - c ≤ b - c)
\end{verbatim}
Se puede interactuar con las pruebas anteriores en \href{https://lean.math.hhu.de/\#url=https://raw.githubusercontent.com/jaalonso/Calculemus2/main/src/abs\_sub.lean}{Lean 4 Web}

\section{En ℝ, \{0 < ε, ε ≤ 1, |x| < ε, |y| < ε\} ⊢ |xy| < ε}
\label{sec:orgff31351}
\begin{verbatim}
-- ---------------------------------------------------------------------
-- Demostrar que para todos los números reales x, y, ε si
--    0 < ε
--    ε ≤ 1
--    |x| < ε
--    |y| < ε
-- entonces
--    |x * y| < ε
-- ----------------------------------------------------------------------

-- Demostración en lenguaje natural
-- ================================

-- Se usarán los siguientes lemas
--    abs_mul          : |a * b| = |a| * |b|
--    zero_mul         : 0 * a = 0
--    abs_nonneg a     : 0 ≤ |a|
--    lt_of_le_of_ne   : a ≤ b → a ≠ b → a < b
--    ne_comm          : a ≠ b ↔ b ≠ a
--    mul_lt_mul_left  : 0 < a → (a * b < a * c ↔ b < c)
--    mul_lt_mul_right : 0 < a → (b * a < c * a ↔ b < c)
--    mul_le_mul_right : 0 < a → (b * a ≤ c * a ↔ b ≤ c)
--    one_mul          : 1 * a = a
--
-- Sean x y ε ∈ ℝ tales que
--    0 < ε                                                         (he1)
--    ε ≤ 1                                                         (he2)
--    |x| < ε                                                       (hx)
--    |y| < ε                                                       (hy)
-- y tenemos que demostrar que
--    |x * y| < ε
-- Lo haremos distinguiendo caso según |x| = 0.
--
-- 1º caso. Supongamos que
--    |x| = 0                                                        (1)
-- Entonces,
--    |x * y| = |x| * |y|    [por abs_mul]
--            = 0 * |y|      [por h1]
--            = 0            [por zero_mul]
--            < ε            [por he1]
--
-- 2º caso. Supongamos que
--    |x| ≠ 0                                                        (2)
-- Entonces, por lt_of_le_of_ne, abs_nonneg y ne_comm, se tiene
--    0 < x                                                          (3)
-- y, por tanto,
--    |x * y| = |x| * |y|    [por abs_mul]
--            < |x| * ε      [por mul_lt_mul_left, (3) y (hy)]
--            < ε * ε        [por mul_lt_mul_right, (he1) y (hx)]
--            ≤ 1 * ε        [por mul_le_mul_right, (he1) y (he2)]
--            = ε            [por one_mul]

-- Demostraciones con Lean4
-- ========================

import Mathlib.Data.Real.Basic

-- 1ª demostración
-- ===============

example :
  ∀ {x y ε : ℝ}, 0 < ε → ε ≤ 1 → |x| < ε → |y| < ε → |x * y| < ε :=
by
  intros x y ε he1 he2 hx hy
  by_cases h : (|x| = 0)
  . -- h : |x| = 0
    show |x * y| < ε
    calc
      |x * y|
         = |x| * |y| := abs_mul x y
      _  = 0 * |y|   := by rw [h]
      _  = 0         := zero_mul (abs y)
      _  < ε         := he1
  . -- h : ¬|x| = 0
    have h1 : 0 < |x| := by
      have h2 : 0 ≤ |x| := abs_nonneg x
      show 0 < |x|
      exact lt_of_le_of_ne h2 (ne_comm.mpr h)
    show |x * y| < ε
    calc |x * y|
         = |x| * |y| := abs_mul x y
       _ < |x| * ε   := (mul_lt_mul_left h1).mpr hy
       _ < ε * ε     := (mul_lt_mul_right he1).mpr hx
       _ ≤ 1 * ε     := (mul_le_mul_right he1).mpr he2
       _ = ε         := one_mul ε

-- 2ª demostración
-- ===============

example :
  ∀ {x y ε : ℝ}, 0 < ε → ε ≤ 1 → |x| < ε → |y| < ε → |x * y| < ε :=
by
  intros x y ε he1 he2 hx hy
  by_cases (|x| = 0)
  . -- h : |x| = 0
    show |x * y| < ε
    calc
      |x * y| = |x| * |y| := by apply abs_mul
            _ = 0 * |y|   := by rw [h]
            _ = 0         := by apply zero_mul
            _ < ε         := by apply he1
  . -- h : ¬|x| = 0
    have h1 : 0 < |x| := by
      have h2 : 0 ≤ |x| := by apply abs_nonneg
      exact lt_of_le_of_ne h2 (ne_comm.mpr h)
    show |x * y| < ε
    calc
      |x * y| = |x| * |y| := by rw [abs_mul]
            _ < |x| * ε   := by apply (mul_lt_mul_left h1).mpr hy
            _ < ε * ε     := by apply (mul_lt_mul_right he1).mpr hx
            _ ≤ 1 * ε     := by apply (mul_le_mul_right he1).mpr he2
            _ = ε         := by rw [one_mul]

-- 3ª demostración
-- ===============

example :
  ∀ {x y ε : ℝ}, 0 < ε → ε ≤ 1 → |x| < ε → |y| < ε → |x * y| < ε :=
by
  intros x y ε he1 he2 hx hy
  by_cases (|x| = 0)
  . -- h : |x| = 0
    show |x * y| < ε
    calc |x * y| = |x| * |y| := by simp only [abs_mul]
               _ = 0 * |y|   := by simp only [h]
               _ = 0         := by simp only [zero_mul]
               _ < ε         := by simp only [he1]
  . -- h : ¬|x| = 0
    have h1 : 0 < |x| := by
      have h2 : 0 ≤ |x| := by simp only [abs_nonneg]
      exact lt_of_le_of_ne h2 (ne_comm.mpr h)
    show |x * y| < ε
    calc
      |x * y| = |x| * |y| := by simp [abs_mul]
            _ < |x| * ε   := by simp only [mul_lt_mul_left, h1, hy]
            _ < ε * ε     := by simp only [mul_lt_mul_right, he1, hx]
            _ ≤ 1 * ε     := by simp only [mul_le_mul_right, he1, he2]
            _ = ε         := by simp only [one_mul]

-- Lemas usados
-- ============

-- variable (a b c : ℝ)
-- #check (abs_mul a b : |a * b| = |a| * |b|)
-- #check (abs_nonneg a : 0 ≤ |a|)
-- #check (lt_of_le_of_ne : a ≤ b → a ≠ b → a < b)
-- #check (mul_le_mul_right : 0 < a → (b * a ≤ c * a ↔ b ≤ c))
-- #check (mul_lt_mul_left : 0 < a → (a * b < a * c ↔ b < c))
-- #check (mul_lt_mul_right : 0 < a → (b * a < c * a ↔ b < c))
-- #check (ne_comm : a ≠ b ↔ b ≠ a)
-- #check (one_mul a : 1 * a = a)
-- #check (zero_mul a : 0 * a = 0)
\end{verbatim}
Se puede interactuar con las pruebas anteriores en \href{https://lean.math.hhu.de/\#url=https://raw.githubusercontent.com/jaalonso/Calculemus2/main/src/Acotacion\_del\_producto.lean}{Lean 4 Web}

\section{En ℝ, a < b → ¬(b < a)}
\label{sec:orgd6a9559}
\begin{verbatim}
-- ---------------------------------------------------------------------
-- Demostrar que para todo par de numero reales a y b, si a < b entonces
-- no se tiene que b < a.
-- ----------------------------------------------------------------------

-- Demostración en lenguaje natural
-- ================================

-- Por hipótesis a < b y tenemos que demostrar que ¬(b < a). Supongamos
-- que b < a. Entonces, por la propiedad transiva a < a que es una
-- contradicción con la propiedad irreflexiva.

-- Demostraciones con Lean4
-- ========================

import Mathlib.Data.Real.Basic

variable (a b : ℝ)

-- 1ª demostración
example
  (h : a < b)
  : ¬ b < a :=
by
  intro h1
  -- h1 : b < a
  -- ⊢ False
  have : a < a := lt_trans h h1
  apply lt_irrefl a this

-- 2ª demostración
example
  (h : a < b)
  : ¬ b < a :=
by
  intro h1
  -- h1 : b < a
  -- ⊢ False
  exact lt_irrefl a (lt_trans h h1)

-- 3ª demostración
example
  (h : a < b)
  : ¬ b < a :=
fun h1 ↦ lt_irrefl a (lt_trans h h1)

-- 4ª demostración
example
  (h : a < b)
  : ¬ b < a :=
lt_asymm h

-- Lemas usados
-- ============

-- variable (c : ℝ)
-- #check (lt_asymm : a < b → ¬b < a)
-- #check (lt_irrefl a : ¬a < a)
-- #check (lt_trans : a < b → b < c → a < c)
\end{verbatim}
Se puede interactuar con las pruebas anteriores en \href{https://lean.math.hhu.de/\#url=https://raw.githubusercontent.com/jaalonso/Calculemus2/main/src/Asimetria\_de\_menor.lean}{Lean 4 Web}

\section{Hay algún número real entre 2 y 3}
\label{sec:orgf7e0b6c}
\begin{verbatim}
-- ---------------------------------------------------------------------
-- Demostrar que hay algún número real entre 2 y 3.
-- ----------------------------------------------------------------------

-- Demostración en lenguaje natural
-- ================================

-- Puesto que 2 < 5/2 < 3, basta elegir 5/2.

-- Demostracione con Lean4
-- =======================

import Mathlib.Data.Real.Basic

-- 1ª demostración
example : ∃ x : ℝ, 2 < x ∧ x < 3 :=
by
  have h : 2 < (5 : ℝ) / 2 ∧ (5 : ℝ) / 2 < 3 :=
    by norm_num
  show ∃ x : ℝ, 2 < x ∧ x < 3
  exact Exists.intro (5 / 2) h

-- 2ª demostración
example : ∃ x : ℝ, 2 < x ∧ x < 3 :=
by
  have h : 2 < (5 : ℝ) / 2 ∧ (5 : ℝ) / 2 < 3 :=
    by norm_num
  show ∃ x : ℝ, 2 < x ∧ x < 3
  exact ⟨5 / 2, h⟩

-- 3ª demostración
example : ∃ x : ℝ, 2 < x ∧ x < 3 :=
by
  use 5 / 2
  norm_num

-- 4ª demostración
example : ∃ x : ℝ, 2 < x ∧ x < 3 :=
⟨5 / 2, by norm_num⟩
\end{verbatim}
Se puede interactuar con las pruebas anteriores en \href{https://lean.math.hhu.de/\#url=https://raw.githubusercontent.com/jaalonso/Calculemus2/main/src/Existencia\_de\_valor\_intermedio.lean}{Lean 4 Web}

\section{Si (∀ε > 0)[x ≤ ε], entonces x ≤ 0}
\label{sec:orgf68218a}
\begin{verbatim}
-- ---------------------------------------------------------------------
-- Sea x un número real tal que para todo número positivo ε, x ≤ ε
-- Demostrar que x ≤ 0.
-- ----------------------------------------------------------------------

-- Demostración en lenguaje natural
-- ================================

-- Basta demostrar que x ≯ 0. Para ello, supongamos que x > 0 y vamos a
-- demostrar que
--    ¬(∀ε)[ε > 0 → x ≤ ε]                                       (1)
-- que es una contradicción con la hipótesis. Interiorizando la
-- negación, (1) es equivalente a
--    (∃ε)[ε > 0 ∧ ε < x]                                        (2)
-- Para demostrar (2) se puede elegir ε = x/2 ya que, como x > 0, se
-- tiene
--    0 < x/2 < x.

-- Demostraciones con Lean4
-- ========================

import Mathlib.Data.Real.Basic

variable (x : ℝ)

-- 1ª demostración
-- ===============

example
  (h : ∀ ε > 0, x ≤ ε)
  : x ≤ 0 :=
by
  apply le_of_not_gt
  -- ⊢ ¬x > 0
  intro hx0
  -- hx0 : x > 0
  -- ⊢ False
  apply absurd h
  -- ⊢ ¬∀ (ε : ℝ), ε > 0 → x ≤ ε
  push_neg
  -- ⊢ ∃ ε, ε > 0 ∧ ε < x
  use x /2
  -- ⊢ x / 2 > 0 ∧ x / 2 < x
  constructor
  { show x / 2 > 0
    exact half_pos hx0 }
  { show x / 2 < x
    exact half_lt_self hx0 }

-- 2ª demostración
-- ===============

example
  (x : ℝ)
  (h : ∀ ε > 0, x ≤ ε)
  : x ≤ 0 :=
by
  contrapose! h
  -- ⊢ ∃ ε, ε > 0 ∧ ε < x
  use x / 2
  -- ⊢ x / 2 > 0 ∧ x / 2 < x
  constructor
  { show x / 2 > 0
    exact half_pos h }
  { show x / 2 < x
    exact half_lt_self h }

-- Lemas usados
-- ============

-- variable (a b : ℝ)
-- variable (p q : Prop)
-- #check (le_of_not_gt : ¬a > b → a ≤ b)
-- #check (half_lt_self : 0 < a → a / 2 < a)
-- #check (half_pos : 0 < a → 0 < a / 2)
-- #check (absurd : p → ¬p → q)
\end{verbatim}
Se puede interactuar con las pruebas anteriores en \href{https://lean.math.hhu.de/\#url=https://raw.githubusercontent.com/jaalonso/Calculemus2/main/src/Condicion\_para\_no\_positivo.lean}{Lean 4 Web}

\section{Si 0 < 0, entonces a > 37 para cualquier número a}
\label{sec:org974f96c}
\begin{verbatim}
-- ---------------------------------------------------------------------
-- Demostrar que si 0 < 0, entonces a > 37 para cualquier número a.
-- ----------------------------------------------------------------------

-- Demostración en lenguaje natural
-- ================================

-- Basta demostrar una contradicción, ya que de una contradicción se
-- sigue cualquier cosa.
--
-- La hipótesis es una contradicción con la propiedad irreflexiva de la
-- relación <.

-- Demostraciones con Lean4
-- ========================

import Mathlib.Tactic

variable (a : ℝ)

-- 1ª demostración
-- ===============

example
  (h : 0 < 0)
  : a > 37 :=
by
  exfalso
  -- ⊢ False
  show False
  exact lt_irrefl 0 h

-- 2ª demostración
-- ===============

example
  (h : 0 < 0)
  : a > 37 :=
by
  exfalso
  -- ⊢ False
  apply lt_irrefl 0 h

-- 3ª demostración
-- ===============

example
  (h : 0 < 0)
  : a > 37 :=
absurd h (lt_irrefl 0)

-- 4ª demostración
-- ===============

example
  (h : 0 < 0)
  : a > 37 :=
by
  have : ¬ 0 < 0 :=  lt_irrefl 0
  contradiction

-- 5ª demostración
-- ===============

example
  (h : 0 < 0)
  : a > 37 :=
by linarith

-- Lemas usados
-- ============

-- variable (p q : Prop)
-- #check (lt_irrefl a : ¬a < a)
-- #check (absurd : p → ¬p → q)
\end{verbatim}
Se puede interactuar con las pruebas anteriores en \href{https://lean.math.hhu.de/\#url=https://raw.githubusercontent.com/jaalonso/Calculemus2/main/src/Principio\_de\_explosion.lean}{Lean 4 Web}

\section{\{x ≤ y, y ≰ x\} ⊢ x ≤ y ∧ x ≠ y}
\label{sec:org3862a3a}
\begin{verbatim}
-- ---------------------------------------------------------------------
-- Sean x e y dos números tales que
--    x ≤ y
--    ¬ y ≤ x
-- entonces
--    x ≤ y ∧ x ≠ y
-- ----------------------------------------------------------------------

-- Demostración en lenguaje natural
-- ================================

-- Como la conclusión es una conjunción, tenemos que desmostrar sus dos
-- partes. La primera parte (x ≤ y) coincide con la hipótesis. Para
-- demostrar la segunda parte (x ≠ y), supongamos que x = y; entonces
-- y ≤ x en contradicción con la segunda hipótesis.

-- Demostraciones con Lean4
-- ========================

import Mathlib.Data.Real.Basic

variable {x y : ℝ}

-- 1ª demostración
-- ===============

example
  (h1 : x ≤ y)
  (h2 : ¬y ≤ x)
  : x ≤ y ∧ x ≠ y :=
by
  constructor
  . -- ⊢ x ≤ y
    exact h1
  . -- ⊢ x ≠ y
    intro h3
    -- h3 : x = y
    -- ⊢ False
    have h4 : y ≤ x := h3.symm.le
    show False
    exact h2 h4

-- 2ª demostración
-- ===============

example
  (h1 : x ≤ y)
  (h2 : ¬y ≤ x)
  : x ≤ y ∧ x ≠ y :=
by
  constructor
  . -- ⊢ x ≤ y
    exact h1
  . -- ⊢ x ≠ y
    intro h3
    -- h3 : x = y
    -- ⊢ False
    exact h2 (h3.symm.le)

-- 3ª demostración
-- ===============

example
  (h1 : x ≤ y)
  (h2 : ¬y ≤ x)
  : x ≤ y ∧ x ≠ y :=
⟨h1, fun h3 ↦ h2 (h3.symm.le)⟩

-- 4ª demostración
-- ===============

example
  (h1 : x ≤ y)
  (h2 : ¬y ≤ x)
  : x ≤ y ∧ x ≠ y :=
by
  constructor
  . -- ⊢ x ≤ y
    exact h1
  . -- ⊢ x ≠ y
    intro h3
    -- h3 : x = y
    -- ⊢ False
    apply h2
    -- ⊢ y ≤ x
    rw [h3]

-- 5ª demostración
-- ===============

example
  (h1 : x ≤ y)
  (h2 : ¬y ≤ x)
  : x ≤ y ∧ x ≠ y :=
by
  constructor
  . -- ⊢ x ≤ y
    exact h1
  . -- ⊢ x ≠ y
    intro h3
    -- h3 : x = y
    -- ⊢ False
    exact h2 (by rw [h3])

-- 6ª demostración
-- ===============

example
  (h1 : x ≤ y)
  (h2 : ¬ y ≤ x)
  : x ≤ y ∧ x ≠ y :=
⟨h1, fun h ↦ h2 (by rw [h])⟩

-- 7ª demostración
-- ===============

example
  (h1 : x ≤ y)
  (h2 : ¬ y ≤ x)
  : x ≤ y ∧ x ≠ y :=
by
  have h3 : x ≠ y
  . contrapose! h2
    -- ⊢ y ≤ x
    rw [h2]
  exact ⟨h1, h3⟩

-- 8ª demostración
-- ===============

example
  (h1 : x ≤ y)
  (h2 : ¬ y ≤ x)
  : x ≤ y ∧ x ≠ y :=
by aesop
\end{verbatim}
Se puede interactuar con las pruebas anteriores en \href{https://lean.math.hhu.de/\#url=https://raw.githubusercontent.com/jaalonso/Calculemus2/main/src/Introduccion\_de\_la\_conjuncion.lean}{Lean 4 Web}

\section{x ≤ y ∧ x ≠ y ⊢ y ≰ x}
\label{sec:org9fdab5e}
\begin{verbatim}
-- ---------------------------------------------------------------------
-- Demostrar que en los reales, si
--    x ≤ y ∧ x ≠ y
-- entonces
--    ¬ y ≤ x
-- ----------------------------------------------------------------------

-- Demostración en lenguaje natural
-- ================================

-- Supongamos que y ≤ x. Entonces, por la antisimetría y la primera
-- parte de la hipótesis, se tiene que x = y que contradice la segunda
-- parte de la hipótesis.

-- Demostraciones con Lean4
-- ========================

import Mathlib.Data.Real.Basic

variable {x y : ℝ}

-- 1ª demostración
-- ===============

example
  (h : x ≤ y ∧ x ≠ y)
  : ¬ y ≤ x :=
by
  intro h1
  cases' h with h2 h3
  -- h2 : x ≤ y
  -- h3 : x ≠ y
  have h4 : x = y := le_antisymm h2 h1
  show False
  exact h3 h4

-- 2ª demostración
-- ===============

example
  (h : x ≤ y ∧ x ≠ y)
  : ¬ y ≤ x :=
by
  intro h1
  have h4 : x = y := le_antisymm h.1 h1
  show False
  exact h.2 h4

-- 3ª demostración
-- ===============

example
  (h : x ≤ y ∧ x ≠ y)
  : ¬ y ≤ x :=
by
  intro h1
  show False
  exact h.2 (le_antisymm h.1 h1)

-- 4ª demostración
-- ===============

example
  (h : x ≤ y ∧ x ≠ y)
  : ¬ y ≤ x :=
fun h1 ↦ h.2 (le_antisymm h.1 h1)

-- 5ª demostración
-- ===============

example
  (h : x ≤ y ∧ x ≠ y)
  : ¬ y ≤ x :=
by
  intro h'
  -- h' : y ≤ x
  -- ⊢ False
  apply h.right
  -- ⊢ x = y
  exact le_antisymm h.left h'

-- 6ª demostración
-- ===============

example
  (h : x ≤ y ∧ x ≠ y)
  : ¬ y ≤ x :=
by
  cases' h with h1 h2
  -- h1 : x ≤ y
  -- h2 : x ≠ y
  contrapose! h2
  -- h2 : y ≤ x
  -- ⊢ x = y
  exact le_antisymm h1 h2

-- 7ª demostración
-- ===============

example : x ≤ y ∧ x ≠ y → ¬ y ≤ x :=
by
  rintro ⟨h1, h2⟩ h'
  -- h1 : x ≤ y
  -- h2 : x ≠ y
  -- h' : y ≤ x
  -- ⊢ False
  exact h2 (le_antisymm h1 h')

-- 8ª demostración
-- ===============

example : x ≤ y ∧ x ≠ y → ¬ y ≤ x :=
fun ⟨h1, h2⟩ h' ↦ h2 (le_antisymm h1 h')

-- Lemas usados
-- ============

-- #check (le_antisymm : x ≤ y → y ≤ x → x = y)
\end{verbatim}
Se puede interactuar con las pruebas anteriores en \href{https://lean.math.hhu.de/\#url=https://raw.githubusercontent.com/jaalonso/Calculemus2/main/src/Eliminacion\_de\_la\_conjuncion.lean}{Lean 4 Web}

\section{(∃x ∈ ℝ)[2 < x < 3]}
\label{sec:org8f4c6dc}
\begin{verbatim}
-- ---------------------------------------------------------------------
-- Demostrar que (∃x ∈ ℝ)[2 < x < 3]
-- ---------------------------------------------------------------------

-- Demostración en lenguaje natural
-- ================================

-- Podemos usar el número 5/2 y comprobar que 2 < 5/2 < 3.

-- Demostraciones con Lean4
-- ========================

import Mathlib.Data.Real.Basic

-- 1ª demostración
-- ===============

example : ∃ x : ℝ, 2 < x ∧ x < 3 :=
by
  use 5 / 2
  show 2 < 5 / 2 ∧ 5 / 2 < 3
  constructor
  . show 2 < 5 / 2
    norm_num
  . show 5 / 2 < 3
    norm_num

-- 2ª demostración
-- ===============

example : ∃ x : ℝ, 2 < x ∧ x < 3 :=
by
  use 5 / 2
  constructor
  . norm_num
  . norm_num

-- 3ª demostración
-- ===============

example : ∃ x : ℝ, 2 < x ∧ x < 3 :=
by
  use 5 / 2
  constructor <;> norm_num

-- 4ª demostración
-- ===============

example : ∃ x : ℝ, 2 < x ∧ x < 3 :=
⟨5/2, by norm_num⟩
\end{verbatim}
Se puede interactuar con las pruebas anteriores en \href{https://lean.math.hhu.de/\#url=https://raw.githubusercontent.com/jaalonso/Calculemus2/main/src/Entre\_2\_y\_3.lean}{Lean 4 Web}

\section{Si (∃z ∈ ℝ)[x < z < y], entonces x < y}
\label{sec:orga36e2c3}
\begin{verbatim}
-- ---------------------------------------------------------------------
-- Demostrar que si (∃z ∈ ℝ)[x < z < y], entonces x < y.
-- ---------------------------------------------------------------------

-- Demostración en lenguaje natural
-- ================================

-- Sea z tal que verifica las siguientes relaciones:
--    x < z                                                          (1)
--    z < y                                                          (2)
-- Aplicando la propiedad transitiva a (1) y (2) se tiene que
--    x < y.

-- Demostraciones con Lean4
-- ========================

import Mathlib.Data.Real.Basic
variable (x y : ℝ)

-- 1ª demostración
-- ===============

example : (∃ z : ℝ, x < z ∧ z < y) → x < y :=
by
  rintro ⟨z, h1 : x < z, h2 : z < y⟩
  show x < y
  exact lt_trans h1 h2

-- 2ª demostración
-- ===============

example : (∃ z : ℝ, x < z ∧ z < y) → x < y :=
by
  rintro ⟨z, h1, h2⟩
  exact lt_trans h1 h2

-- 3ª demostración
-- ===============

example : (∃ z : ℝ, x < z ∧ z < y) → x < y :=
fun ⟨_, h1, h2⟩ ↦ lt_trans h1 h2

-- Lemas usados
-- ============

-- variable (a b c : ℝ)
-- #check (lt_trans : a < b → b < c → a < c)
\end{verbatim}
Se puede interactuar con las pruebas anteriores en \href{https://lean.math.hhu.de/\#url=https://raw.githubusercontent.com/jaalonso/Calculemus2/main/src/Menor\_por\_intermedio.lean}{Lean 4 Web}

\section{En ℝ, x ≤ y ∧ x ≠ y → x ≤ y ∧ y ≰ x}
\label{sec:org70d73db}
\begin{verbatim}
-- ---------------------------------------------------------------------
-- Demostrar que, en ℝ, x ≤ y ∧ x ≠ y → x ≤ y ∧ y ≰ x
-- ---------------------------------------------------------------------

-- Demostración en lenguaje natural
-- ================================

-- Supongamos que
--    x ≤ y                                                          (1)
--    x ≠ y                                                          (2)
-- Entonces, se tiene x ≤ y (por (1)) y, para probar y ≰ x, supongamos
-- que y ≤ x. Por (1), se tiene que x = y, en contradicción con (2).

-- Demostraciones con Lean4
-- ========================

import Mathlib.Data.Real.Basic
variable (x y : ℝ)

-- 1ª demostración
-- ===============

example : x ≤ y ∧ x ≠ y → x ≤ y ∧ ¬ y ≤ x :=
by
  rintro ⟨h1 : x ≤ y, h2 : x ≠ y⟩
  constructor
  . show x ≤ y
    exact h1
  . show ¬ y ≤ x
    rintro h3 : y ≤ x
    -- ⊢ False
    have h4 : x = y := le_antisymm h1 h3
    show False
    exact h2 h4

-- 2ª demostración
-- ===============

example : x ≤ y ∧ x ≠ y → x ≤ y ∧ ¬ y ≤ x :=
by
  rintro ⟨h1 : x ≤ y, h2 : x ≠ y⟩
  -- ⊢ x ≤ y ∧ ¬y ≤ x
  constructor
  . show x ≤ y
    exact h1
  . show ¬ y ≤ x
    rintro h3 : y ≤ x
    -- ⊢ False
    show False
    exact h2 (le_antisymm h1 h3)

-- 3ª demostración
-- ===============

example : x ≤ y ∧ x ≠ y → x ≤ y ∧ ¬ y ≤ x :=
by
  rintro ⟨h1 : x ≤ y, h2 : x ≠ y⟩
  constructor
  . show x ≤ y
    exact h1
  . show ¬ y ≤ x
    exact fun h3 ↦ h2 (le_antisymm h1 h3)

-- 4ª demostración
-- ===============

example : x ≤ y ∧ x ≠ y → x ≤ y ∧ ¬ y ≤ x :=
by
  rintro ⟨h1, h2⟩
  exact ⟨h1, fun h3 ↦ h2 (le_antisymm h1 h3)⟩

-- 5ª demostración
-- ===============

example : x ≤ y ∧ x ≠ y → x ≤ y ∧ ¬ y ≤ x :=
  fun ⟨h1, h2⟩ ↦ ⟨h1, fun h3 ↦ h2 (le_antisymm h1 h3)⟩

-- 6ª demostración
-- ===============

example : x ≤ y ∧ x ≠ y → x ≤ y ∧ ¬ y ≤ x :=
by
  rintro ⟨h1 : x ≤ y, h2 : x ≠ y⟩
  use h1
  exact fun h3 ↦ h2 (le_antisymm h1 h3)

-- 7ª demostración
-- ===============

example : x ≤ y ∧ x ≠ y → x ≤ y ∧ ¬ y ≤ x :=
by
  rintro ⟨h1, h2⟩
  -- h1 : x ≤ y
  -- h2 : x ≠ y
  -- ⊢ x ≤ y ∧ ¬y ≤ x
  use h1
  -- ¬y ≤ x
  contrapose! h2
  -- h2 : y ≤ x
  -- ⊢ x = y
  apply le_antisymm h1 h2

-- Lemas usados
-- ============

-- #check (le_antisymm : x ≤ y → y ≤ x → x = y)
\end{verbatim}
Se puede interactuar con las pruebas anteriores en \href{https://lean.math.hhu.de/\#url=https://raw.githubusercontent.com/jaalonso/Calculemus2/main/src/Entre\_desigualdades.lean}{Lean 4 Web}

\section{En ℝ, si x ≤ y, entonces y ≰ x ↔ x ≠ y}
\label{sec:org33c0c63}
\begin{verbatim}
-- ---------------------------------------------------------------------
-- Sean x, y números reales tales que x ≤ y. Entonces, y ≰ x ↔ x ≠ y.
-- ---------------------------------------------------------------------

-- Demostración en lenguaje natural
-- ================================

-- Para demostrar la equivalencia, demostraremos cada una de las
-- implicaciones.
--
-- Para demostrar la primera, supongamos que y ≰ x y que x =
-- y. Entonces, y ≤ x que es una contradicción.
--
-- Para demostrar la segunda, supongamos que x ≠ y y que y ≤
-- x. Entonces, por la hipótesis y la antisimetría, se tiene que x = y
-- lo que es una contradicción.

-- Demostraciones con Lean4
-- ========================

import Mathlib.Data.Real.Basic
variable {x y : ℝ}

-- 1ª demostración
-- ===============

example
  (h : x ≤ y)
  : ¬y ≤ x ↔ x ≠ y :=
by
  constructor
  . show ¬y ≤ x → x ≠ y
    { intro h1
      -- h1 : ¬y ≤ x
      -- ⊢ x ≠ y
      intro h2
      -- h2 : x = y
      -- ⊢ False
      have h3 : y ≤ x := by rw [h2]
      show False
      exact h1 h3 }
  . show x ≠ y → ¬y ≤ x
    { intro h1
      -- h1 : x ≠ y
      -- ⊢ ¬y ≤ x
      intro h2
      -- h2 : y ≤ x
      -- ⊢ False
      have h3 : x = y := le_antisymm h h2
      show False
      exact h1 h3 }

-- 2ª demostración
-- ===============

example
  (h : x ≤ y)
  : ¬y ≤ x ↔ x ≠ y :=
by
  constructor
  . show ¬y ≤ x → x ≠ y
    { intro h1
      -- h1 : ¬y ≤ x
      -- ⊢ x ≠ y
      intro h2
      -- h2 : x = y
      -- ⊢ False
      show False
      exact h1 (by rw [h2]) }
  . show x ≠ y → ¬y ≤ x
    { intro h1
      -- h1 : x ≠ y
      -- ⊢ ¬y ≤ x
      intro h2
      -- h2 : y ≤ x
      -- ⊢ False
      show False
      exact h1 (le_antisymm h h2) }

-- 3ª demostración
-- ===============

example
  (h : x ≤ y)
  : ¬y ≤ x ↔ x ≠ y :=
by
  constructor
  . show ¬y ≤ x → x ≠ y
    { intro h1 h2
      exact h1 (by rw [h2]) }
  . show x ≠ y → ¬y ≤ x
    { intro h1 h2
      exact h1 (le_antisymm h h2) }

-- 4ª demostración
-- ===============

example
  (h : x ≤ y)
  : ¬y ≤ x ↔ x ≠ y :=
by
  constructor
  . intro h1 h2
    exact h1 (by rw [h2])
  . intro h1 h2
    exact h1 (le_antisymm h h2)

-- 5ª demostración
-- ===============

example
  (h : x ≤ y)
  : ¬y ≤ x ↔ x ≠ y :=
by
  constructor
  . exact fun h1 h2 ↦ h1 (by rw [h2])
  . exact fun h1 h2 ↦ h1 (le_antisymm h h2)

-- 6ª demostración
-- ===============

example
  (h : x ≤ y)
  : ¬y ≤ x ↔ x ≠ y :=
  ⟨fun h1 h2 ↦ h1 (by rw [h2]),
   fun h1 h2 ↦ h1 (le_antisymm h h2)⟩

-- 7ª demostración
-- ===============

example
  (h : x ≤ y)
  : ¬y ≤ x ↔ x ≠ y :=
by
  constructor
  . show ¬y ≤ x → x ≠ y
    { intro h1
      -- h1 : ¬y ≤ x
      -- ⊢ x ≠ y
      contrapose! h1
      -- h1 : x = y
      -- ⊢ y ≤ x
      calc y = x := h1.symm
           _ ≤ x := by rfl }
  . show x ≠ y → ¬y ≤ x
    { intro h2
      -- h2 : x ≠ y
      -- ⊢ ¬y ≤ x
      contrapose! h2
      -- h2 : y ≤ x
      -- ⊢ x = y
      show x = y
      exact le_antisymm h h2 }

-- 8ª demostración
-- ===============

example
  (h : x ≤ y)
  : ¬y ≤ x ↔ x ≠ y :=
by
  constructor
  · -- ⊢ ¬y ≤ x → x ≠ y
    contrapose!
    -- ⊢ x = y → y ≤ x
    rintro rfl
    -- ⊢ x ≤ x
    rfl
  . -- ⊢ x ≠ y → ¬y ≤ x
    contrapose!
    -- ⊢ y ≤ x → x = y
    exact le_antisymm h
\end{verbatim}
Se puede interactuar con las pruebas anteriores en \href{https://lean.math.hhu.de/\#url=https://raw.githubusercontent.com/jaalonso/Calculemus2/main/src/CNS\_de\_distintos.lean}{Lean 4 Web}

\section{Si |x + 3| < 5, entonces -8 < x < 2}
\label{sec:org7feca95}
\begin{verbatim}
-- ---------------------------------------------------------------------
-- Demostrar que si
--    |x + 3| < 5
-- entonces
--    -8 < x < 2
-- ----------------------------------------------------------------------

-- Demostración en lenguaje natural
-- ================================

-- Supongamos que
--    |x + 3| < 5
-- entonces
--    -5 < x + 3 < 5
-- por tanto
--    -8 < x < 2

-- Demostraciones con Lean4
-- ========================

import Mathlib.Data.Real.Basic
variable (x y : ℝ)

-- 1ª demostración
-- ===============

example
  : |x + 3| < 5 → -8 < x ∧ x < 2 :=
by
  rw [abs_lt]
  -- ⊢ -5 < x + 3 ∧ x + 3 < 5 → -8 < x ∧ x < 2
  intro h
  -- h : -5 < x + 3 ∧ x + 3 < 5
  -- ⊢ -8 < x ∧ x < 2
  constructor
  . -- ⊢ -8 < x
    linarith
  . -- x < 2
    linarith

-- 2ª demostración
-- ===============

example
  : |x + 3| < 5 → -8 < x ∧ x < 2 :=
by
  rw [abs_lt]
  intro h
  constructor <;> linarith

-- Comentario: La composición (constructor <;> linarith) aplica constructor y a
-- continuación le aplica linarith a cada subojetivo.

-- 3ª demostración
-- ===============

example
  : |x + 3| < 5 → -8 < x ∧ x < 2 :=
by
  rw [abs_lt]
  exact fun _ ↦ ⟨by linarith, by linarith⟩

-- Lemas usados
-- ============

-- #check (abs_lt: |x| < y ↔ -y < x ∧ x < y)
\end{verbatim}
Se puede interactuar con las pruebas anteriores en \href{https://lean.math.hhu.de/\#url=https://raw.githubusercontent.com/jaalonso/Calculemus2/main/src/Acotacion\_del\_valor\_absoluto.lean}{Lean 4 Web}

\section{En ℝ, y > x² ⊢ y > 0 ∨ y < -1}
\label{sec:orgea1a9b0}
\begin{verbatim}
-- ---------------------------------------------------------------------
-- Demostrar que si
--    y > x^2
-- entonces
--    y > 0 ∨ y < -1
-- ----------------------------------------------------------------------

-- Demostración en lenguaje natural
-- ================================

-- Puesto que
--    (∀ x ∈ ℝ)[x² ≥ 0]
-- se tiene que
--    y > x²
--      ≥ 0
-- Por tanto, y > 0 y, al verificar la primera parte de la diyunción, se
-- verifica la disyunción.

-- Demostraciones con Lean4
-- ========================

import Mathlib.Data.Real.Basic
variable {x y : ℝ}

-- 1ª demostración
-- ===============

example
  (h : y > x^2)
  : y > 0 ∨ y < -1 :=
by
  have h1 : y > 0 := by
    calc y > x^2 := h
         _ ≥ 0   := pow_two_nonneg x
  show y > 0 ∨ y < -1
  exact Or.inl h1

-- 2ª demostración
-- ===============

example
  (h : y > x^2)
  : y > 0 ∨ y < -1 :=
by
  left
  -- ⊢ y > 0
  calc y > x^2 := h
       _ ≥ 0   := pow_two_nonneg x

-- 3ª demostración
-- ===============

example
  (h : y > x^2)
  : y > 0 ∨ y < -1 :=
by
  left
  -- ⊢ y > 0
  linarith [pow_two_nonneg x]

-- 4ª demostración
-- ===============

example
  (h : y > x^2)
  : y > 0 ∨ y < -1 :=
by { left ; linarith [pow_two_nonneg x] }

-- Lema usado
-- ==========

-- #check (pow_two_nonneg x : 0 ≤ x ^ 2)
\end{verbatim}
Se puede interactuar con las pruebas anteriores en \href{https://lean.math.hhu.de/\#url=https://raw.githubusercontent.com/jaalonso/Calculemus2/main/src/Introduccion\_de\_la\_disyuncion\_1.lean}{Lean 4 Web}

\section{En ℝ, -y > x² + 1 ⊢ y > 0 ∨ y < -1}
\label{sec:orgc6dd146}
\begin{verbatim}
-- ---------------------------------------------------------------------
-- Demostrar que si
--    -y > x^2 + 1
-- entonces
--    y > 0 ∨ y < -1
-- ----------------------------------------------------------------------

-- Demostración en lenguaje natural
-- ================================

-- Usaremos los siguientes lemas
--    (∀ b, c ∈ ℝ)[b ≤ c → ∀ (a : ℝ),  b + a ≤ c + a)]               (L1)
--    (∀ a ∈ ℝ)[0 ≤ a^2]                                             (L2)
--    (∀ a  ∈ ℝ)[0 + a = a]                                          (L3)
--    (∀ a, b ∈ ℝ)[a < -b ↔ b < -a]                                  (L4)

-- Se tiene
--    -y > x^2 + 1    [por la hipótesis]
--       ≥ 0 + 1      [por L1 y L2]
--       = 1          [por L3]
-- Por tanto,
--    -y > 1
-- y, aplicando el lema L4, se tiene
--    y < -1
-- Como se verifica la segunda parte de la diyunción, se verifica la
-- disyunción.

-- Demostraciones con Lean4
-- ========================

import Mathlib.Data.Real.Basic
variable {x y : ℝ}

-- 1ª demostración
-- ===============

example
  (h : -y > x^2 + 1)
  : y > 0 ∨ y < -1 :=
by
  have h1 : -y > 1 := by
    calc -y > x^2 + 1 := by exact h
          _ ≥ 0 + 1   := add_le_add_right (pow_two_nonneg x) 1
          _ = 1       := zero_add 1
  have h2: y < -1 := lt_neg.mp h1
  show y > 0 ∨ y < -1
  exact Or.inr h2

-- 2ª demostración
-- ===============

example
  (h : -y > x^2 + 1)
  : y > 0 ∨ y < -1 :=
by
  have h1 : -y > 1 := by linarith [pow_two_nonneg x]
  have h2: y < -1 := lt_neg.mp h1
  show y > 0 ∨ y < -1
  exact Or.inr h2

-- 3ª demostración
-- ===============

example
  (h : -y > x^2 + 1)
  : y > 0 ∨ y < -1 :=
by
  have h1: y < -1 := by linarith [pow_two_nonneg x]
  show y > 0 ∨ y < -1
  exact Or.inr h1

-- 4ª demostración
-- ===============

example
  (h : -y > x^2 + 1)
  : y > 0 ∨ y < -1 :=
by
  right
  -- ⊢ y < -1
  linarith [pow_two_nonneg x]

-- 5ª demostración
-- ===============

example
  (h : -y > x^2 + 1)
  : y > 0 ∨ y < -1 :=
by { right ; linarith [pow_two_nonneg x] }

-- Lemas usados
-- ============

-- variable (a b c : ℝ)
-- #check (add_le_add_right : b ≤ c → ∀ (a : ℝ),  b + a ≤ c + a)
-- #check (lt_neg : a < -b ↔ b < -a)
-- #check (pow_two_nonneg a : 0 ≤ a ^ 2)
-- #check (zero_add a : 0 + a = a)
\end{verbatim}
Se puede interactuar con las pruebas anteriores en \href{https://lean.math.hhu.de/\#url=https://raw.githubusercontent.com/jaalonso/Calculemus2/main/src/Introduccion\_de\_la\_disyuncion\_2.lean}{Lean 4 Web}

\section{En ℝ, si x < |y|, entonces x < y ó x < -y}
\label{sec:orgf974e89}
\begin{verbatim}
-- ---------------------------------------------------------------------
-- Demostrar que para todo par de números reales x e y, si x < |y|,
-- entonces x < y ó x < -y.
-- ----------------------------------------------------------------------

-- Demostración en lenguaje natural
-- ================================

-- Se demostrará por casos según y ≥ 0.
--
-- Primer caso: Supongamos que y ≥ 0. Entonces, |y| = y y, por tanto,
-- x < y.
--
-- Segundo caso: Supongamos que y < 0. Entonces, |y| = -y y, por tanto,
-- x < -y.

-- Demostraciones con Lean4
-- ========================

import Mathlib.Data.Real.Basic
variable {x y : ℝ}

-- 1ª demostración
-- ===============

example : x < |y| → x < y ∨ x < -y :=
by
  intro h1
  -- h1 : x < |y|
  -- ⊢ x < y ∨ x < -y
  cases' le_or_gt 0 y with h2 h3
  . -- h2 : 0 ≤ y
    left
    -- ⊢ x < y
    rwa [abs_of_nonneg h2] at h1
  . -- h3 : 0 > y
    right
    -- ⊢ x < -y
    rwa [abs_of_neg h3] at h1

-- 2ª demostración
-- ===============

example : x < |y| → x < y ∨ x < -y :=
lt_abs.mp

-- Lemas usados
-- ============

-- #check (le_or_gt x y : x ≤ y ∨ x > y)
-- #check (abs_of_nonneg : 0 ≤ x → abs x = x)
-- #check (abs_of_neg : x < 0 → abs x = -x)
-- #check (lt_abs : x < |y| ↔ x < y ∨ x < -y)
\end{verbatim}
Se puede interactuar con las pruebas anteriores en \href{https://lean.math.hhu.de/\#url=https://raw.githubusercontent.com/jaalonso/Calculemus2/main/src/Eliminacion\_de\_la\_disyuncion.lean}{Lean 4 Web}

\section{En ℝ, x ≤ |x|}
\label{sec:orgef266ba}
\begin{verbatim}
-- ---------------------------------------------------------------------
-- Demostrar que en ℝ,
--    x ≤ |x|
-- ----------------------------------------------------------------------

-- Demostración en lenguaje natural
-- ================================

-- Se usarán los siguientes lemas
--    (∀ x ∈ ℝ)[0 ≤ x → |x| = x]                                     (L1)
--    (∀ x, y ∈ ℝ)[x < y → x ≤ y]                                    (L2)
--    (∀ x ∈ ℝ)[x ≤ 0 → x ≤ -x]                                      (L3)
--    (∀ x ∈ ℝ)[x < 0 → |x| = -x]                                    (L4)
--
-- Se demostrará por casos según x ≥ 0:
--
-- Primer caso: Supongamos que x ≥ 0. Entonces,
--    x ≤ x
--      = |x|    [por L1]
--
-- Segundo caso: Supongamos que x < 0. Entonces, por el L2, se tiene
--    x ≤ 0                                                          (1)
-- Por tanto,
--    x ≤ -x     [por L3 y (1)]
--      = |x|    [por L4]

-- Demostraciones con Lean4
-- ========================

import Mathlib.Data.Real.Basic
variable {x : ℝ}

-- 1ª demostración
-- ===============

example : x ≤ |x| :=
by
  cases' le_or_gt 0 x with h1 h2
  . -- h1 : 0 ≤ x
    show x ≤ |x|
    calc x ≤ x   := le_refl x
         _ = |x| := (abs_of_nonneg h1).symm
  . -- h2 : 0 > x
    have h3 : x ≤ 0 := le_of_lt h2
    show x ≤ |x|
    calc x ≤ -x  := le_neg_self_iff.mpr h3
         _ = |x| := (abs_of_neg h2).symm

-- 2ª demostración
-- ===============

example : x ≤ |x| :=
by
  cases' le_or_gt 0 x with h1 h2
  . -- h1 : 0 ≤ x
    rw [abs_of_nonneg h1]
  . -- h2 : 0 > x
    rw [abs_of_neg h2]
    -- ⊢ x ≤ -x
    apply Left.self_le_neg
    -- ⊢ x ≤ 0
    exact le_of_lt h2

-- 3ª demostración
-- ===============

example : x ≤ |x| :=
by
  rcases (le_or_gt 0 x) with h1 | h2
  . -- h1 : 0 ≤ x
    rw [abs_of_nonneg h1]
  . -- h1 : 0 ≤ x
    rw [abs_of_neg h2]
    linarith

-- 4ª demostración
-- ===============

example : x ≤ |x| :=
  le_abs_self x

-- Lemas usados
-- ============

-- variable (y : ℝ)
-- #check (Left.self_le_neg : x ≤ 0 → x ≤ -x)
-- #check (abs_of_neg : x < 0 → |x| = -x)
-- #check (abs_of_nonneg : 0 ≤ x → |x| = x)
-- #check (le_abs_self x : x ≤ |x|)
-- #check (le_neg_self_iff : x ≤ -x ↔ x ≤ 0)
-- #check (le_of_lt : x < y → x ≤ y)
-- #check (le_or_gt x y : x ≤ y ∨ x > y)
\end{verbatim}
Se puede interactuar con las pruebas anteriores en \href{https://lean.math.hhu.de/\#url=https://raw.githubusercontent.com/jaalonso/Calculemus2/main/src/Cota\_inf\_de\_abs.lean}{Lean 4 Web}

\section{En ℝ, -x ≤ |x|}
\label{sec:org1409e7b}
\begin{verbatim}
-- ---------------------------------------------------------------------
-- Demostrar que
--    -x ≤ |x|
-- ----------------------------------------------------------------------

-- Demostración en lenguaje natural
-- ================================

-- Se usarán los siguientes lemas
--    (∀ x ∈ ℝ)[0 ≤ x → -x ≤ x]                                      (L1)
--    (∀ x ∈ ℝ)[0 ≤ x → |x| = x]                                     (L2)
--    (∀ x ∈ ℝ)[x ≤ x]                                               (L3)
--    (∀ x ∈ ℝ)[x < 0 → |x| = -x]                                    (L4)
--
-- Se demostrará por casos según x ≥ 0:
--
-- Primer caso: Supongamos que x ≥ 0. Entonces,
--    -x ≤ x      [por L1]
--       = |x|    [por L2]
--
-- Segundo caso: Supongamos que x < 0. Entonces,
--    -x ≤ -x     [por L3]
--     _ = |x|    [por L4]

-- Demostraciones con Lean4
-- ========================

import Mathlib.Data.Real.Basic
variable {x : ℝ}

-- 1ª demostración
-- ===============

example : -x ≤ |x| :=
by
  cases' (le_or_gt 0 x) with h1 h2
  . -- h1 : 0 ≤ x
    show -x ≤ |x|
    calc -x ≤ x   := by exact neg_le_self h1
          _ = |x| := (abs_of_nonneg h1).symm
  . -- h2 : 0 > x
    show -x ≤ |x|
    calc -x ≤ -x  := by exact le_refl (-x)
          _ = |x| := (abs_of_neg h2).symm

-- 2ª demostración
-- ===============

example : -x ≤ |x| :=
by
  cases' (le_or_gt 0 x) with h1 h2
  . -- h1 : 0 ≤ x
    rw [abs_of_nonneg h1]
    -- ⊢ -x ≤ x
    exact neg_le_self h1
  . -- h2 : 0 > x
    rw [abs_of_neg h2]

-- 3ª demostración
-- ===============

example : -x ≤ |x| :=
by
  rcases (le_or_gt 0 x) with h1 | h2
  . -- h1 : 0 ≤ x
    rw [abs_of_nonneg h1]
    -- ⊢ -x ≤ x
    linarith
  . -- h2 : 0 > x
    rw [abs_of_neg h2]

-- 4ª demostración
-- ===============

example : -x ≤ |x| :=
  neg_le_abs_self x

-- Lemas usados
-- ============

-- variable (y : ℝ)
-- #check (abs_of_neg : x < 0 → |x| = -x)
-- #check (abs_of_nonneg : 0 ≤ x → |x| = x)
-- #check (le_or_gt x y : x ≤ y ∨ x > y)
-- #check (le_refl x : x ≤ x)
-- #check (neg_le_abs_self x : -x ≤ |x|)
-- #check (neg_le_self : 0 ≤ x → -x ≤ x)
\end{verbatim}
Se puede interactuar con las pruebas anteriores en \href{https://lean.math.hhu.de/\#url=https://raw.githubusercontent.com/jaalonso/Calculemus2/main/src/Cota\_inf2\_de\_abs.lean}{Lean 4 Web}

\section{En ℝ, |x + y| ≤ |x| + |y|}
\label{sec:orgf01b38b}
\begin{verbatim}
-- ---------------------------------------------------------------------
-- Demostrar que
--    |x + y| ≤ |x| + |y|
-- ----------------------------------------------------------------------

-- Demostración en lenguaje natural
-- ================================

-- Se usarán los siguientes lemas
--     (∀ x ∈ ℝ)[0 ≤ x → |x| = x]                          (L1)
--     (∀ a, b, c, d ∈ ℝ)[a ≤ b ∧ c ≤ d → a + c ≤ b + d]   (L2)
--     (∀ x ∈ ℝ)[x ≤ |x|]                                  (L3)
--     (∀ x ∈ ℝ)[x < 0 → |x| = -x]                         (L4)
--     (∀ x, y ∈ ℝ)[-(x + y) = -x + -y]                    (L5)
--     (∀ x ∈ ℝ)[-x ≤ |x|]                                 (L6)
--
-- Se demostrará por casos según x + y ≥ 0:
--
-- Primer caso: Supongamos que x + y ≥ 0. Entonces,
--    |x + y| = x + y        [por L1]
--          _ ≤ |x| + |y|    [por L2 y L3]
--
-- Segundo caso: Supongamos que x + y < 0. Entonces,
--    |x + y| = -(x + y)     [por L4]
--            = -x + -y      [por L5]
--            ≤ |x| + |y|    [por L2 y L6]

-- Demostraciones con Lean4
-- ========================

import Mathlib.Data.Real.Basic
variable {x y : ℝ}

-- 1ª demostración
-- ===============

example : |x + y| ≤ |x| + |y| :=
by
  rcases le_or_gt 0 (x + y) with h1 | h2
  · -- h1 : 0 ≤ x + y
    show |x + y| ≤ |x| + |y|
    calc |x + y| = x + y     := by exact abs_of_nonneg h1
               _ ≤ |x| + |y| := add_le_add (le_abs_self x) (le_abs_self y)
  . -- h2 : 0 > x + y
    show |x + y| ≤ |x| + |y|
    calc |x + y| = -(x + y)  := by exact abs_of_neg h2
               _ = -x + -y   := by exact neg_add x y
               _ ≤ |x| + |y| := add_le_add (neg_le_abs_self x) (neg_le_abs_self y)

-- 2ª demostración
-- ===============

example : |x + y| ≤ |x| + |y| := by
  rcases le_or_gt 0 (x + y) with h1 | h2
  · -- h1 : 0 ≤ x + y
    rw [abs_of_nonneg h1]
    -- ⊢ x + y ≤ |x| + |y|
    exact add_le_add (le_abs_self x) (le_abs_self y)
  . -- h2 : 0 > x + y
    rw [abs_of_neg h2]
    -- ⊢ -(x + y) ≤ |x| + |y|
    calc -(x + y) = -x + -y    := by exact neg_add x y
                _ ≤ |x| + |y|  := add_le_add (neg_le_abs_self x) (neg_le_abs_self y)

-- 2ª demostración
-- ===============

example : |x + y| ≤ |x| + |y| := by
  rcases le_or_gt 0 (x + y) with h1 | h2
  · -- h1 : 0 ≤ x + y
    rw [abs_of_nonneg h1]
    -- ⊢ x + y ≤ |x| + |y|
    linarith [le_abs_self x, le_abs_self y]
  . -- h2 : 0 > x + y
    rw [abs_of_neg h2]
    -- ⊢ -(x + y) ≤ |x| + |y|
    linarith [neg_le_abs_self x, neg_le_abs_self y]

-- 3ª demostración
-- ===============

example : |x + y| ≤ |x| + |y| :=
  abs_add x y

-- Lemas usados
-- ============

-- variable (a b c d : ℝ)
-- #check (abs_add x y : |x + y| ≤ |x| + |y|)
-- #check (abs_of_neg : x < 0 → |x| = -x)
-- #check (abs_of_nonneg : 0 ≤ x → |x| = x)
-- #check (add_le_add : a ≤ b → c ≤ d → a + c ≤ b + d)
-- #check (le_abs_self a : a ≤ |a|)
-- #check (le_or_gt x y : x ≤ y ∨ x > y)
-- #check (neg_add x y : -(x + y) = -x + -y)
-- #check (neg_le_abs_self x : -x ≤ |x|)
\end{verbatim}
Se puede interactuar con las pruebas anteriores en \href{https://lean.math.hhu.de/\#url=https://raw.githubusercontent.com/jaalonso/Calculemus2/main/src/Desigualdad\_triangular\_para\_valor\_absoluto.lean.lean}{Lean 4 Web}

\section{En ℝ, si x ≠ 0 entonces x < 0 ó x > 0}
\label{sec:org8b911ed}
\begin{verbatim}
-- ---------------------------------------------------------------------
-- Ejercicio. Sea x un número real. Demostrar que si
--    x ≠ 0
-- entonces
--    x < 0 ∨ x > 0
 -- ----------------------------------------------------------------------

-- Demostración en lenguaje natural
-- ================================

-- Usando el siguiente lema
--    (∀ x y ∈ ℝ)[x < y ∨ x = y ∨ y < x]
-- se demuestra distinguiendo tres casos.
--
-- Caso 1: Supongamos que x < 0. Entonces, se verifica la disyunción ya
-- que se verifica su primera parte.
--
-- Caso 2: Supongamos que x = 0. Entonces, se tiene una contradicción
-- con la hipótesis.
--
-- Caso 3: Supongamos que x > 0. Entonces, se verifica la disyunción ya
-- que se verifica su segunda parte.

-- Demostraciones con Lean4
-- ========================

import Mathlib.Data.Real.Basic
variable {x : ℝ}

-- 1ª demostración
-- ===============

example
  (h : x ≠ 0)
  : x < 0 ∨ x > 0 :=
by
  rcases lt_trichotomy x 0 with hx1 | hx2 | hx3
  . -- hx1 : x < 0
    left
    -- ⊢ x < 0
    exact hx1
  . -- hx2 : x = 0
    contradiction
  . -- hx3 : 0 < x
    right
    -- ⊢ x > 0
    exact hx3

-- 2ª demostración
-- ===============

example
  (h : x ≠ 0)
  : x < 0 ∨ x > 0 :=
Ne.lt_or_lt h

-- 3ª demostración
-- ===============

example
  (h : x ≠ 0)
  : x < 0 ∨ x > 0 :=
by aesop

-- Lemas usados
-- ============

-- variable (y : ℝ)
-- #check (lt_trichotomy x y : x < y ∨ x = y ∨ y < x)
\end{verbatim}
Se puede interactuar con las pruebas anteriores en \href{https://lean.math.hhu.de/\#url=https://raw.githubusercontent.com/jaalonso/Calculemus2/main/src/Eliminacion\_de\_la\_disyuncion\_con\_rcases.lean.lean}{Lean 4 Web}

\section{Si (∃ x, y ∈ ℝ)[z = x² + y² ∨ z = x² + y² + 1], entonces z ≥ 0}
\label{sec:org734cdf8}
\begin{verbatim}
-- ---------------------------------------------------------------------
-- Demostrar que si
--    ∃ x y, z = x^2 + y^2 ∨ z = x^2 + y^2 + 1
-- entonces
--    z ≥ 0
-- ----------------------------------------------------------------------

-- Demostración en lenguaje natural
-- ================================

-- Usaremos los siguientes lemas
--    (∀ x ∈ ℝ)[x² ≥ 0]                                              (L1)
--    (∀ x, y ∈ ℝ)[x ≥ 0 → y ≥ 0 → x + y ≥ 0]                        (L2)
--    1 ≥ 0                                                          (L3)

-- Sean a y b tales que
--    z = a² + b² ∨ z = a² + b² + 1
-- Entonces, por L1, se tiene que
--    a² ≥ 0                                                         (1)
--    b² ≥ 0                                                         (2)
--
-- En el primer caso, z = a² + b² y se tiene que z ≥ 0 por el lema L2
-- aplicado a (1) y (2).
--
-- En el segundo caso, z = a² + b² y se tiene que z ≥ 0 por el lema L2
-- aplicado a (1), (2) y L3.

-- Demostraciones con Lean4
-- ========================

import Mathlib.Data.Real.Basic
import Mathlib.Tactic
variable {z : ℝ}

-- 1ª demostración
-- ===============

example
  (h : ∃ x y, z = x^2 + y^2 ∨ z = x^2 + y^2 + 1)
  : z ≥ 0 :=
by
  rcases h with ⟨a, b, h1⟩
  -- a b : ℝ
  -- h1 : z = a ^ 2 + b ^ 2 ∨ z = a ^ 2 + b ^ 2 + 1
  have h2 : a ^ 2 ≥ 0 := pow_two_nonneg a
  have h3 : b ^ 2 ≥ 0 := pow_two_nonneg b
  have h4 : a ^ 2 + b ^ 2 ≥ 0 := add_nonneg h2 h3
  rcases h1 with h5 | h6
  . -- h5 : z = a ^ 2 + b ^ 2
    show z ≥ 0
    calc z = a ^ 2 + b ^ 2 := h5
         _ ≥ 0             := add_nonneg h2 h3
  . -- h6 : z = a ^ 2 + b ^ 2 + 1
    show z ≥ 0
    calc z = (a ^ 2 + b ^ 2) + 1 := h6
         _ ≥ 0                   := add_nonneg h4 zero_le_one

-- 2ª demostración
-- ===============

example
  (h : ∃ x y, z = x^2 + y^2 ∨ z = x^2 + y^2 + 1)
  : z ≥ 0 :=
by
  rcases h with ⟨a, b, h1 | h2⟩
  . -- h1 : z = a ^ 2 + b ^ 2
    have h1a : a ^ 2 ≥ 0 := pow_two_nonneg a
    have h1b : b ^ 2 ≥ 0 := pow_two_nonneg b
    show z ≥ 0
    calc z = a ^ 2 + b ^ 2 := h1
         _ ≥ 0             := add_nonneg h1a h1b
  . -- h2 : z = a ^ 2 + b ^ 2 + 1
    have h2a : a ^ 2 ≥ 0         := pow_two_nonneg a
    have h2b : b ^ 2 ≥ 0         := pow_two_nonneg b
    have h2c : a ^ 2 + b ^ 2 ≥ 0 := add_nonneg h2a h2b
    show z ≥ 0
    calc z = (a ^ 2 + b ^ 2) + 1 := h2
         _ ≥ 0                   := add_nonneg h2c zero_le_one

-- 3ª demostración
-- ===============

example
  (h : ∃ x y, z = x^2 + y^2 ∨ z = x^2 + y^2 + 1)
  : z ≥ 0 :=
by
  rcases h with ⟨a, b, h1 | h2⟩
  . -- h1 : z = a ^ 2 + b ^ 2
    rw [h1]
    -- ⊢ a ^ 2 + b ^ 2 ≥ 0
    apply add_nonneg
    . -- ⊢ 0 ≤ a ^ 2
      apply pow_two_nonneg
    . -- ⊢ 0 ≤ b ^ 2
      apply pow_two_nonneg
  . -- h2 : z = a ^ 2 + b ^ 2 + 1
    rw [h2]
    -- ⊢ a ^ 2 + b ^ 2 + 1 ≥ 0
    apply add_nonneg
    . -- ⊢ 0 ≤ a ^ 2 + b ^ 2
      apply add_nonneg
      . -- ⊢ 0 ≤ a ^ 2
        apply pow_two_nonneg
      . -- ⊢ 0 ≤ b ^ 2
        apply pow_two_nonneg
    . -- ⊢ 0 ≤ 1
      exact zero_le_one

-- 4ª demostración
-- ===============

example
  (h : ∃ x y, z = x^2 + y^2 ∨ z = x^2 + y^2 + 1)
  : z ≥ 0 :=
by
  rcases h with ⟨a, b, rfl | rfl⟩
  . -- ⊢ a ^ 2 + b ^ 2 ≥ 0
    apply add_nonneg
    . -- ⊢ 0 ≤ a ^ 2
      apply pow_two_nonneg
    . -- ⊢ 0 ≤ b ^ 2
      apply pow_two_nonneg
  . -- ⊢ a ^ 2 + b ^ 2 + 1 ≥ 0
    apply add_nonneg
    . -- ⊢ 0 ≤ a ^ 2 + b ^ 2
      apply add_nonneg
      . -- ⊢ 0 ≤ a ^ 2
        apply pow_two_nonneg
      . -- ⊢ 0 ≤ b ^ 2
        apply pow_two_nonneg
    . -- ⊢ 0 ≤ 1
      exact zero_le_one

-- 5ª demostración
-- ===============

example
  (h : ∃ x y, z = x^2 + y^2 ∨ z = x^2 + y^2 + 1)
  : z ≥ 0 :=
by
  rcases h with ⟨a, b, rfl | rfl⟩
  . -- ⊢ a ^ 2 + b ^ 2 ≥ 0
    nlinarith
  . -- ⊢ a ^ 2 + b ^ 2 + 1 ≥ 0
    nlinarith

-- 6ª demostración
-- ===============

example
  (h : ∃ x y, z = x^2 + y^2 ∨ z = x^2 + y^2 + 1)
  : z ≥ 0 :=
by rcases h with ⟨a, b, rfl | rfl⟩ <;> nlinarith

-- Lemas usados
-- ============

-- variable (x y : ℝ)
-- #check (add_nonneg : 0 ≤ x → 0 ≤ y → 0 ≤ x + y)
-- #check (pow_two_nonneg x : 0 ≤ x ^ 2)
-- #check (zero_le_one : 0 ≤ 1)
\end{verbatim}
Se puede interactuar con las pruebas anteriores en \href{https://lean.math.hhu.de/\#url=https://raw.githubusercontent.com/jaalonso/Calculemus2/main/src/Desigualdad\_con\_rcases.lean.lean}{Lean 4 Web}

\section{En ℝ, si 1 < a, entonces a < aa}
\label{sec:org6cde2dc}
\begin{verbatim}
-- ---------------------------------------------------------------------
-- Demostrar, para todo a ∈ ℝ, si
--    1 < a
-- entonces
--    a < a * a
-- ----------------------------------------------------------------------

-- Demostración en lenguaje natural
-- ================================

-- Se usarán los siguientes lemas
--    L1: 0 < 1
--    L2: (∀ a ∈ ℝ[1a = a]
--    L3: (∀ a, b, c ∈ ℝ)[0 < a → (ba < ca ↔ b < c)]
--
-- En primer lugar, tenemos que
--    0 < a                                                          (1)
-- ya que
--    0 < 1    [por L1]
--      < a    [por la hipótesis]
-- Entonces,
--    a = 1a   [por L2]
--      < aa   [por L3, (1) y la hipótesis]

-- Demostraciones con Lean4
-- ========================

import Mathlib.Data.Real.Basic
variable {a : ℝ}

-- 1ª demostración
-- ===============

example
  (h : 1 < a)
  : a < a * a :=
by
  have h1 : 0 < a := calc
    0 < 1 := zero_lt_one
    _ < a := h
  show a < a * a
  calc a = 1 * a := (one_mul a).symm
       _ < a * a := (mul_lt_mul_right h1).mpr h

-- Comentarios: La táctica (convert e) genera nuevos subojetivos cuya
-- conclusiones son las diferencias entre el tipo de e y la conclusión.

-- 2ª demostración
-- ===============

example
  (h : 1 < a)
  : a < a * a :=
by
  convert (mul_lt_mul_right _).2 h
  . -- ⊢ a = 1 * a
    rw [one_mul]
  . -- ⊢ 0 < a
    exact lt_trans zero_lt_one h

-- Lemas usados
-- ============

-- variables (a b c : ℝ)
-- #check (mul_lt_mul_right : 0 < a → (b * a < c * a ↔ b < c))
-- #check (one_mul a : 1 * a = a)
-- #check (lt_trans : a < b → b < c → a < c)
-- #check (zero_lt_one : 0 < 1)
\end{verbatim}
Se puede interactuar con las pruebas anteriores en \href{https://lean.math.hhu.de/\#url=https://raw.githubusercontent.com/jaalonso/Calculemus2/main/src/Demostracion\_por\_conversion.lean}{Lean 4 Web}

\chapter{Divisibilidad}
\label{sec:org0ae29c8}

\section{Si x,y,z ∈ ℕ, entonces x ∣ yxz}
\label{sec:org5ac6ae3}
\begin{verbatim}
-- ---------------------------------------------------------------------
-- Demostrar que si x, y, z ∈ ℕ, entonces
--    x ∣ y * x * z
-- ----------------------------------------------------------------------

-- Demostración en lenguaje natural
-- ================================

-- Por la transitividad de la divisibilidad aplicada a las relaciones
--    x ∣ yx
--    yx ∣ yxz

-- Demostraciones con Lean4
-- ========================

import Mathlib.Data.Real.Basic
variable (x y z : ℕ)

-- 1ª demostración
-- ===============

example : x ∣ y * x * z :=
by
  have h1 : x ∣ y * x :=
    dvd_mul_left x y
  have h2 : (y * x) ∣ (y * x * z) :=
    dvd_mul_right (y * x) z
  show x ∣ y * x * z
  exact dvd_trans h1 h2

-- 2ª demostración
-- ===============

example : x ∣ y * x * z :=
dvd_trans (dvd_mul_left x y) (dvd_mul_right (y * x) z)

-- 3ª demostración
-- ===============

example : x ∣ y * x * z :=
by
  apply dvd_mul_of_dvd_left
  apply dvd_mul_left

-- Lemas usados
-- ============

-- #check (dvd_mul_left x y : x ∣ y * x)
-- #check (dvd_mul_right x y : x ∣ x * y)
-- #check (dvd_trans : x ∣ y → y ∣ z → x ∣ z)
\end{verbatim}
Se puede interactuar con las pruebas anteriores en \href{https://lean.math.hhu.de/\#url=https://raw.githubusercontent.com/jaalonso/Calculemus2/main/src/Divisibilidad\_de\_producto.lean}{Lean 4 Web}

\section{Si x divide a w, también divide a y(xz)+x²+w²}
\label{sec:orgee566aa}
\begin{verbatim}
-- ---------------------------------------------------------------------
-- Demostrar que si
--    x ∣ w
-- entonces
--    x ∣ y * (x * z) + x^2 + w^2
-- ----------------------------------------------------------------------

-- Demostración en lenguaje natural
-- ================================

-- Por la divisibilidad de la suma basta probar que
--    x | yxz                                                         (1)
--    x | x²                                                          (2)
--    x | w²                                                          (3)
--
-- Para demostrar (1), por la divisibilidad del producto se tiene
--    x | xz
-- y, de nuevo por la divisibilidad del producto,
--    x | y(xz).
--
-- La propiedad (2) se tiene por la definición de cuadrado y la
-- divisibilidad del producto.
--
-- La propiedad (3) se tiene por la definición de cuadrado, la hipótesis
-- y la divisibilidad del producto.

-- Demostraciones con Lean4
-- ========================

import Mathlib.Data.Real.Basic
variable (w x y z : ℕ)

-- 1ª demostración
example
  (h : x ∣ w)
  : x ∣ y * (x * z) + x^2 + w^2 :=
by
  have h1 : x ∣ x * z :=
    dvd_mul_right x z
  have h2 : x ∣ y * (x * z) :=
    dvd_mul_of_dvd_right h1 y
  have h3 : x ∣ x^2 := by
    apply dvd_mul_left
  have h4 : x ∣ w * w :=
    dvd_mul_of_dvd_left h w
  have h5 : x ∣ w^2 := by
    rwa [← pow_two w] at h4
  have h6 : x ∣ y * (x * z) + x^2 :=
    dvd_add h2 h3
  show x ∣ y * (x * z) + x^2 + w^2
  exact dvd_add h6 h5

-- 2ª demostración
example
  (h : x ∣ w)
  : x ∣ y * (x * z) + x^2 + w^2 :=
by
  apply dvd_add
  { apply dvd_add
    { apply dvd_mul_of_dvd_right
      apply dvd_mul_right }
    { rw [pow_two]
      apply dvd_mul_right }}
  { rw [pow_two]
    apply dvd_mul_of_dvd_left h }

-- 3ª demostración
example
  (h : x ∣ w)
  : x ∣ y * (x * z) + x^2 + w^2 :=
by
  repeat' apply dvd_add
  { apply dvd_mul_of_dvd_right
    apply dvd_mul_right }
  { rw [pow_two]
    apply dvd_mul_right }
  { rw [pow_two]
    apply dvd_mul_of_dvd_left h }

-- Lemas usados
-- ============

-- #check (dvd_add : x ∣ y → x ∣ z → x ∣ y + z)
-- #check (dvd_mul_left x y : x ∣ y * x)
-- #check (dvd_mul_right x y : x ∣ x * y)
-- #check (dvd_mul_of_dvd_left : x ∣ y → ∀ (c : ℕ), x ∣ y * c)
-- #check (dvd_mul_of_dvd_right : x ∣ y → ∀ (c : ℕ), x ∣ c * y)
-- #check (pow_two x : x ^ 2 = x * x)
\end{verbatim}
Se puede interactuar con las pruebas anteriores en \href{https://lean.math.hhu.de/\#url=https://raw.githubusercontent.com/jaalonso/Calculemus2/main/src/Ejercicio\_de\_divisibilidad.lean}{Lean 4 Web}

\section{Transitividad de la divisibilidad}
\label{sec:org1c5be65}
\begin{verbatim}
-- ---------------------------------------------------------------------
-- Demostrar que la relación de divisibilidad es transitiva.
-- ----------------------------------------------------------------------

-- Demostración en lenguaje natural
-- ================================

-- Supongamos que a | b y b | c. Entonces, existen d y e tales que
--    b = ad                                                         (1)
--    c = be                                                         (2)
-- Por tanto,
--    c = be       [por (2)]
--      = (ad)e    [por (1)]
--      = a(de)
-- Por consiguiente, a | c.

-- Demostraciones con Lean4
-- ========================

import Mathlib.Tactic

variable {a b c : ℕ}

-- 1ª demostración
example
  (divab : a ∣ b)
  (divbc : b ∣ c) :
  a ∣ c :=
by
  rcases divab with ⟨d, beq : b = a * d⟩
  rcases divbc with ⟨e, ceq : c = b * e⟩
  have h1 : c = a * (d * e) :=
    calc c = b * e      := ceq
        _ = (a * d) * e := congrArg (. * e) beq
        _ = a * (d * e) := mul_assoc a d e
  show a ∣ c
  exact Dvd.intro (d * e) h1.symm

-- 2ª demostración
example
  (divab : a ∣ b)
  (divbc : b ∣ c) :
  a ∣ c :=
by
  rcases divab with ⟨d, beq : b = a * d⟩
  rcases divbc with ⟨e, ceq : c = b * e⟩
  use (d * e)
  -- ⊢ c = a * (d * e)
  rw [ceq, beq]
  -- ⊢ (a * d) * e = a * (d * e)
  exact mul_assoc a d e

-- 3ª demostración
example
  (divab : a ∣ b)
  (divbc : b ∣ c) :
  a ∣ c :=
by
  rcases divbc with ⟨e, rfl⟩
  -- ⊢ a ∣ b * e
  rcases divab with ⟨d, rfl⟩
  -- ⊢ a ∣ a * d * e
  use (d * e)
  -- ⊢ a * d * e = a * (d * e)
  ring

-- 4ª demostración
example
  (divab : a ∣ b)
  (divbc : b ∣ c) :
  a ∣ c :=
by
  cases' divab with d beq
  -- d : ℕ
  -- beq : b = a * d
  cases' divbc with e ceq
  -- e : ℕ
  -- ceq : c = b * e
  rw [ceq, beq]
  -- ⊢ a ∣ a * d * e
  use (d * e)
  -- ⊢ (a * d) * e = a * (d * e)
  exact mul_assoc a d e

-- 5ª demostración
example
  (divab : a ∣ b)
  (divbc : b ∣ c) :
  a ∣ c :=
by exact dvd_trans divab divbc

-- Lemas usados
-- ============

-- #check (mul_assoc a b c : (a * b) * c = a * (b * c))
-- #check (Dvd.intro c : a * c = b → a ∣ b)
-- #check (dvd_trans : a ∣ b → b ∣ c → a ∣ c)
\end{verbatim}
Se puede interactuar con las pruebas anteriores en \href{https://lean.math.hhu.de/\#url=https://raw.githubusercontent.com/jaalonso/Calculemus2/main/src/Transitividad\_de\_la\_divisibilidad.lean}{Lean 4 Web}

\section{Si a divide a b y a c, entonces divide a b+c}
\label{sec:orgccd0de7}
\begin{verbatim}
-- ---------------------------------------------------------------------
-- Demostrar que si a es un divisor de b y de c, tambien lo es de b + c.
-- ----------------------------------------------------------------------

-- Demostración en lenguaje natural
-- ================================

-- Puesto que a divide a b y a c, existen d y e tales que
--    b = ad                                                         (1)
--    c = ae                                                         (2)
-- Por tanto,
--    b + c = ad + c     [por (1)]
--          = ad + ae    [por (2)]
--          = a(d + e)   [por la distributiva]
-- Por consiguiente, a divide a b + c.

-- Demostraciones con Lean4
-- ========================

import Mathlib.Tactic

variable {a b c : ℕ}

-- 1ª demostración
example
  (h1 : a ∣ b)
  (h2 : a ∣ c)
  : a ∣ (b + c) :=
by
  rcases h1 with ⟨d, beq : b = a * d⟩
  rcases h2 with ⟨e, ceq: c = a * e⟩
  have h1 : b + c = a * (d + e) :=
    calc b + c
         = (a * d) + c       := congrArg (. + c) beq
       _ = (a * d) + (a * e) := congrArg ((a * d) + .) ceq
       _ = a * (d + e)       := by rw [← mul_add]
  show a ∣ (b + c)
  exact Dvd.intro (d + e) h1.symm

-- 2ª demostración
example
  (h1 : a ∣ b)
  (h2 : a ∣ c)
  : a ∣ (b + c) :=
by
  rcases h1 with ⟨d, beq : b = a * d⟩
  rcases h2 with ⟨e, ceq: c = a * e⟩
  have h1 : b + c = a * (d + e) := by linarith
  show a ∣ (b + c)
  exact Dvd.intro (d + e) h1.symm

-- 3ª demostración
example
  (h1 : a ∣ b)
  (h2 : a ∣ c)
  : a ∣ (b + c) :=
by
  rcases h1 with ⟨d, beq : b = a * d⟩
  rcases h2 with ⟨e, ceq: c = a * e⟩
  show a ∣ (b + c)
  exact Dvd.intro (d + e) (by linarith)

-- 4ª demostración
example
  (h1 : a ∣ b)
  (h2 : a ∣ c)
  : a ∣ (b + c) :=
by
  cases' h1 with d beq
  -- d : ℕ
  -- beq : b = a * d
  cases' h2 with e ceq
  -- e : ℕ
  -- ceq : c = a * e
  rw [ceq, beq]
  -- ⊢ a ∣ a * d + a * e
  use (d + e)
  -- ⊢ a * d + a * e = a * (d + e)
  ring

-- 5ª demostración
example
  (h1 : a ∣ b)
  (h2 : a ∣ c)
  : a ∣ (b + c) :=
by
  rcases h1 with ⟨d, rfl⟩
  -- ⊢ a ∣ a * d + c
  rcases h2 with ⟨e, rfl⟩
  -- ⊢ a ∣ a * d + a * e
  use (d + e)
  -- ⊢ a * d + a * e = a * (d + e)
  ring

-- 6ª demostración
example
  (h1 : a ∣ b)
  (h2 : a ∣ c)
  : a ∣ (b + c) :=
dvd_add h1 h2

-- Lemas usados
-- ============

-- #check (Dvd.intro c : a * c = b → a ∣ b)
-- #check (dvd_add : a ∣ b →  a ∣ c → a ∣ (b + c))
-- #check (mul_add a b c : a * (b + c) = a * b + a * c)
\end{verbatim}
Se puede interactuar con las pruebas anteriores en \href{https://lean.math.hhu.de/\#url=https://raw.githubusercontent.com/jaalonso/Calculemus2/main/src/Suma\_divisible.lean}{Lean 4 Web}

\section{Conmutatividad del máximo común divisor}
\label{sec:org8044603}
\begin{verbatim}
-- ---------------------------------------------------------------------
-- Demostrar que si m y n son números naturales, entonces
--    gcd m n = gcd n m
-- ----------------------------------------------------------------------

-- Demostración en lenguaje natural
-- ================================

-- Es consecuencia del siguiente lema auxiliar
--    (∀ x, y ∈ ℕ)[gcd(x,y) ∣ gcd(y,x)]                               (1)
-- En efecto, sustituyendo en (1) x por m e y por n, se tiene
--    gcd(m, n) ∣ gcd(n, m)                                           (2)
-- y sustituyendo en (1) x por n e y por m, se tiene
--    gcd(n, m) ∣ gcd(m, n)                                           (3)
-- Finalmente, aplicando la propiedad antisimétrica de la divisibilidad
-- a (2) y (3), se tiene
--    gcd(m, n) = gcd(n, m)
--
-- Para demostrar (1), por la definición del máximo común divisor, basta
-- demostrar las siguientes relaciones
--    gcd(m, n) ∣ n
--    gcd(m, n) ∣ m
-- y ambas se tienen por la definición del máximo común divisor.

-- Demostraciones con Lean4
-- ========================

import Mathlib.Data.Real.Basic
variable (k m n : ℕ)

open Nat

-- 1ª demostración del lema auxiliar
lemma aux : gcd m n ∣ gcd n m :=
by
  have h1 : gcd m n ∣ n :=
    gcd_dvd_right m n
  have h2 : gcd m n ∣ m :=
    gcd_dvd_left m n
  show gcd m n ∣ gcd n m
  exact dvd_gcd h1 h2

-- 2ª demostración del lema auxiliar
example : gcd m n ∣ gcd n m :=
dvd_gcd (gcd_dvd_right m n) (gcd_dvd_left m n)

-- 1ª demostración
example : gcd m n = gcd n m :=
by
  have h1 : gcd m n ∣ gcd n m := aux m n
  have h2 : gcd n m ∣ gcd m n := aux n m
  show gcd m n = gcd n m
  exact _root_.dvd_antisymm h1 h2

-- 2ª demostración
example : gcd m n = gcd n m :=
by
  apply _root_.dvd_antisymm
  { exact aux m n }
  { exact aux n m }

-- 3ª demostración
example : gcd m n = gcd n m :=
_root_.dvd_antisymm (aux m n) (aux n m)

-- 4ª demostración
example : gcd m n = gcd n m :=
-- by apply?
gcd_comm m n

-- Lemas usados
-- ============

-- #check (_root_.dvd_antisymm : m ∣ n → n ∣ m → m = n)
-- #check (dvd_gcd : k ∣ m → k ∣ n → k ∣ gcd m n)
-- #check (gcd_comm m n : gcd m n = gcd n m)
-- #check (gcd_dvd_left  m n: gcd m n ∣ m)
-- #check (gcd_dvd_right m n : gcd m n ∣ n)
\end{verbatim}
Se puede interactuar con las pruebas anteriores en \href{https://lean.math.hhu.de/\#url=https://raw.githubusercontent.com/jaalonso/Calculemus2/main/src/Conmutatividad\_del\_gcd.lean}{Lean 4 Web}

\section{Si (m ∣ n ∧ m ≠ n), entonces (m ∣ n ∧ ¬(n ∣ m))}
\label{sec:org46b1d2f}
\begin{verbatim}
-- ---------------------------------------------------------------------
-- Sean m y n números naturales. Demostrar que si
--    m ∣ n ∧ m ≠ n
-- entonces
--    m ∣ n ∧ ¬(n ∣ m)
-- ----------------------------------------------------------------------

-- Demostración en lenguaje natural
-- ================================

-- La primera parte de la conclusión coincide con la primera de la
-- hipótesis. Nos queda demostrar la segunda parte; es decir, que
-- ¬(n | m). Para ello, supongamos que n | m. Entonces, por la propiedad
-- antisimétrica de la divisibilidad y la primera parte de la hipótesis,
-- se tiene que m = n en contradicción con la segunda parte de la
-- hipótesis.

-- Demostraciones con Lean4
-- ========================

import Mathlib.Data.Nat.GCD.Basic

variable {m n : ℕ}

-- 1ª demostración
-- ===============

example
  (h : m ∣ n ∧ m ≠ n)
  : m ∣ n ∧ ¬ n ∣ m :=
by
  constructor
  . show m ∣ n
    exact h.left
  . show ¬n ∣ m
    { intro (h1 : n ∣ m)
      have h2 : m = n := dvd_antisymm h.left h1
      show False
      exact h.right h2 }

-- 2ª demostración
-- ===============

example
  (h : m ∣ n ∧ m ≠ n)
  : m ∣ n ∧ ¬ n ∣ m :=
by
  constructor
  . exact h.left
  . intro (h1 : n ∣ m)
    exact h.right (dvd_antisymm h.left h1)

-- 3ª demostración
-- ===============

example
  (h : m ∣ n ∧ m ≠ n)
  : m ∣ n ∧ ¬ n ∣ m :=
⟨h.left, fun h1 ↦ h.right (dvd_antisymm h.left h1)⟩

-- 4ª demostración
-- ===============

example
  (h : m ∣ n ∧ m ≠ n)
  : m ∣ n ∧ ¬ n ∣ m :=
by
  cases' h with h1 h2
  -- h1 : m ∣ n
  -- h2 : m ≠ n
  constructor
  . -- ⊢ m ∣ n
    exact h1
  . -- ⊢ ¬n ∣ m
    contrapose! h2
    -- h2 : n ∣ m
    -- ⊢ m = n
    apply dvd_antisymm h1 h2

-- 5ª demostración
-- ===============

example
  (h : m ∣ n ∧ m ≠ n)
  : m ∣ n ∧ ¬ n ∣ m :=
by
  rcases h with ⟨h1 : m ∣ n, h2 : m ≠ n⟩
  constructor
  . -- ⊢ m ∣ n
    exact h1
  . -- ⊢ ¬n ∣ m
    contrapose! h2
    -- h2 : n ∣ m
    -- ⊢ m = n
    apply dvd_antisymm h1 h2

-- Lemas usados
-- ============

-- #check (dvd_antisymm : m ∣ n → n ∣ m → m = n)
\end{verbatim}
Se puede interactuar con las pruebas anteriores en \href{https://lean.math.hhu.de/\#url=https://raw.githubusercontent.com/jaalonso/Calculemus2/main/src/Uso\_de\_conjuncion.lean}{Lean 4 Web}

\section{Existen números primos m y n tales que 4 < m < n < 10}
\label{sec:org7471e2c}
\begin{verbatim}
-- ---------------------------------------------------------------------
-- Demostrar que existen números primos m y n tales que
-- 4 < m < n < 10.
-- ---------------------------------------------------------------------

-- Demostración en lenguaje natural
-- ================================

-- Basta considerar los números 5 y 7, ya que son primos y
-- 4 < 5 < 7 < 10.

-- Demostración con Lean4
-- ======================

import Mathlib.Data.Nat.Prime
import Mathlib.Tactic

example :
  ∃ m n : ℕ, 4 < m ∧ m < n ∧ n < 10 ∧ Nat.Prime m ∧ Nat.Prime n :=
by
  use 5, 7
  norm_num
\end{verbatim}
Se puede interactuar con las pruebas anteriores en \href{https://lean.math.hhu.de/\#url=https://raw.githubusercontent.com/jaalonso/Calculemus2/main/src/Primos\_intermedios.lean}{Lean 4 Web}

\section{3 divide al máximo común divisor de 6 y 15}
\label{sec:org6c93b4e}
\begin{verbatim}
-- ---------------------------------------------------------------------
-- Demostrar que 3 divide al máximo común divisor de 6 y 15.
-- ----------------------------------------------------------------------

-- Demostración en lenguaje natural
-- ================================

-- Se usará el siguiente lema
--    (∀ k, m, n ∈ ℕ)[k ∣ gcd m n ↔ k ∣ m ∧ k ∣ n]
--
-- Por el lema,
--    3 ∣ gcd 6 15
-- se reduce a
--    3 ∣ 6 ∧ 3 ∣ 15
-- que se verifican fácilmente.

-- Demostraciones con Lean4
-- ========================

import Mathlib.Data.Real.Basic
import Mathlib.Data.Nat.GCD.Basic

open Nat

-- 1ª demostración
-- ===============

example : 3 ∣ gcd 6 15 :=
by
  rw [dvd_gcd_iff]
  -- ⊢ 3 ∣ 6 ∧ 3 ∣ 15
  constructor
  . -- 3 ∣ 6
    norm_num
  . -- ⊢ 3 ∣ 15
    norm_num

-- 2ª demostración
-- ===============

example : 3 ∣ gcd 6 15 :=
by
  rw [dvd_gcd_iff]
  -- ⊢ 3 ∣ 6 ∧ 3 ∣ 15
  constructor <;> norm_num

-- Lemas usados
-- ============

-- variable (k m n : ℕ)
-- #check (dvd_gcd_iff : k ∣ gcd m n ↔ k ∣ m ∧ k ∣ n)
\end{verbatim}
Se puede interactuar con las pruebas anteriores en \href{https://lean.math.hhu.de/\#url=https://raw.githubusercontent.com/jaalonso/Calculemus2/main/src/Divisor\_del\_mcd.lean}{Lean 4 Web}

\section{Si m divide a n o a k, entonces m divide a nk}
\label{sec:orgb9dd66c}
\begin{verbatim}
-- ---------------------------------------------------------------------
-- Demostrar que si m divide a n o a k, entonces m divide a nk.
-- ----------------------------------------------------------------------

-- Demostración en lenguaje natural
-- ================================

-- Se demuestra por casos.
--
-- Caso 1: Supongamos que m ∣ n. Entonces, existe un a ∈ ℕ tal que
--    n = ma
-- Por tanto,
--    nk = (ma)k
--       = m(ak)
-- que es divisible por m.
--
-- Caso 2: Supongamos que m ∣ k. Entonces, existe un b ∈ ℕ tal que
--    k = mb
-- Por tanto,
--    nk = n(mb)
--       = m(nb)
-- que es divisible por m.

-- Demostraciones con Lean4
-- ========================

import Mathlib.Tactic
variable {m n k : ℕ}

-- 1ª demostración
-- ===============

example
  (h : m ∣ n ∨ m ∣ k)
  : m ∣ n * k :=
by
  rcases h with h1 | h2
  . -- h1 : m ∣ n
    rcases h1 with ⟨a, ha⟩
    -- a : ℕ
    -- ha : n = m * a
    rw [ha]
    -- ⊢ m ∣ (m * a) * k
    rw [mul_assoc]
    -- ⊢ m ∣ m * (a * k)
    exact dvd_mul_right m (a * k)
  . -- h2 : m ∣ k
    rcases h2 with ⟨b, hb⟩
    -- b : ℕ
    -- hb : k = m * b
    rw [hb]
    -- ⊢ m ∣ n * (m * b)
    rw [mul_comm]
    -- ⊢ m ∣ (m * b) * n
    rw [mul_assoc]
    -- ⊢ m ∣ m * (b * n)
    exact dvd_mul_right m (b * n)

-- 2ª demostración
-- ===============

example
  (h : m ∣ n ∨ m ∣ k)
  : m ∣ n * k :=
by
  rcases h with h1 | h2
  . -- h1 : m ∣ n
    rcases h1 with ⟨a, ha⟩
    -- a : ℕ
    -- ha : n = m * a
    rw [ha, mul_assoc]
    -- ⊢ m ∣ m * (a * k)
    exact dvd_mul_right m (a * k)
  . -- h2 : m ∣ k
    rcases h2 with ⟨b, hb⟩
    -- b : ℕ
    -- hb : k = m * b
    rw [hb, mul_comm, mul_assoc]
    -- ⊢ m ∣ m * (b * n)
    exact dvd_mul_right m (b * n)

-- 3ª demostración
-- ===============

example
  (h : m ∣ n ∨ m ∣ k)
  : m ∣ n * k :=
by
  rcases h with ⟨a, rfl⟩ | ⟨b, rfl⟩
  . -- a : ℕ
    -- ⊢ m ∣ (m * a) * k
    rw [mul_assoc]
    -- ⊢ m ∣ m * (a * k)
    exact dvd_mul_right m (a * k)
  . -- ⊢ m ∣ n * (m * b)
    rw [mul_comm, mul_assoc]
    -- ⊢ m ∣ m * (b * n)
    exact dvd_mul_right m (b * n)

-- 4ª demostración
-- ===============

example
  (h : m ∣ n ∨ m ∣ k)
  : m ∣ n * k :=
by
  rcases h with h1 | h2
  . -- h1 : m ∣ n
    exact dvd_mul_of_dvd_left h1 k
  . -- h2 : m ∣ k
    exact dvd_mul_of_dvd_right h2 n

-- Lemas usados
-- ============

-- #check (dvd_mul_of_dvd_left : m ∣ n → ∀ (c : ℕ), m ∣ n * c)
-- #check (dvd_mul_of_dvd_right : m ∣ n → ∀ (c : ℕ), m ∣ c * n)
-- #check (dvd_mul_right m n : m ∣ m * n)
-- #check (mul_assoc m n k : m * n * k = m * (n * k))
-- #check (mul_comm m n : m * n = n * m)
\end{verbatim}
Se puede interactuar con las pruebas anteriores en \href{https://lean.math.hhu.de/\#url=https://raw.githubusercontent.com/jaalonso/Calculemus2/main/src/CS\_de\_divisibilidad\_del\_producto.lean.lean}{Lean 4 Web}

\section{Existen infinitos números primos}
\label{sec:org6f23c5b}
\begin{verbatim}
-- ---------------------------------------------------------------------
-- Demostrar que hay infinitos números primos.
-- ---------------------------------------------------------------------

-- Demostración en lenguaje natural
-- ================================

-- Se usarán los siguientes lemas de los números naturales
--    n ≠ 1 → el menor factor primo de n es primo                    (L1)
--    n! > 0                                                         (L2)
--    0 < k → n < k + n                                              (L3)
--    k < n → n ≠ k                                                  (L4)
--    k ≱ n → k ≤ n                                                  (L5)
--    0 < k → k ≤ n → k ∣ n!                                         (L6)
--    0 < minFac(n)                                                  (L7)
--    k ∣ m → (k ∣ n ↔ k ∣ m + n)                                    (L8)
--    minFac(n) ∣ n                                                  (L9)
--    Prime(n) → ¬n ∣ 1                                              (L10)
--
-- Sea p el menor factor primo de n! + 1. Tenemos que demostrar que n ≤
-- p y que p es primo.
--
-- Para demostrar que p es primo, por el lema L1, basta demostrar que
--    n! + 1 ≠ 1
-- Su demostración es
--    n ! > 0           [por L2]
--    ==> n ! + 1 > 1   [por L3]
--    ==> n ! + 1 ≠ 1   [por L4]
--
-- Para demostrar n ≤ p, por el lema L5, basta demostrar que
--    n ≱ p
-- Su demostración es
--    n ≥ p
--    ==> p ∣ n!    [por L6 y L7]
--    ==> p | 1     [por L8 y (p | n! + 1) por L9]
--    ==> Falso     [por L10 y p es primo]

-- Demostración con Lean4
-- ======================

import Mathlib.Tactic
import Mathlib.Data.Nat.Prime
open Nat

-- 1ª demostración
-- ===============

example
  (n : ℕ) :
  ∃ p, n ≤ p ∧ Nat.Prime p :=
by
  let p := minFac (n !  + 1)
  have h1 : Nat.Prime p := by
    apply minFac_prime
    -- ⊢ n ! + 1 ≠ 1
    have h3 : n ! > 0     := factorial_pos n
    have h4 : n ! + 1 > 1 := Nat.lt_add_of_pos_left h3
    show n ! + 1 ≠ 1
    exact Nat.ne_of_gt h4
  use p
  constructor
  . -- ⊢ n ≤ p
    apply le_of_not_ge
    -- ⊢ ¬n ≥ p
    intro h5
    -- h5 : n ≥ p
    -- ⊢ False
    have h6 : p ∣ n ! := dvd_factorial (minFac_pos _) h5
    have h7 : p ∣ 1   := (Nat.dvd_add_iff_right h6).mpr (minFac_dvd _)
    show False
    exact (Nat.Prime.not_dvd_one h1) h7
  . -- ⊢ Nat.Prime p
    exact h1
  done

-- 2ª demostración
-- ===============

example
  (n : ℕ) :
  ∃ p, n ≤ p ∧ Nat.Prime p :=
exists_infinite_primes n

-- Lemas usados
-- ============

-- variable (k m n : ℕ)
-- #check (Nat.Prime.not_dvd_one : Nat.Prime n → ¬n ∣ 1)
-- #check (Nat.dvd_add_iff_right : k ∣ m → (k ∣ n ↔ k ∣ m + n))
-- #check (Nat.dvd_one : n ∣ 1 ↔ n = 1)
-- #check (Nat.lt_add_of_pos_left : 0 < k → n < k + n)
-- #check (Nat.ne_of_gt : k < n → n ≠ k)
-- #check (dvd_factorial : 0 < k → k ≤ n → k ∣ n !)
-- #check (factorial_pos n: n ! > 0)
-- #check (le_of_not_ge : ¬k ≥ n → k ≤ n)
-- #check (minFac_dvd n : minFac n ∣ n)
-- #check (minFac_pos n : 0 < minFac n)
-- #check (minFac_prime : n ≠ 1 → Nat.Prime (minFac n))
\end{verbatim}
Se puede interactuar con las pruebas anteriores en \href{https://lean.math.hhu.de/\#url=https://raw.githubusercontent.com/jaalonso/Calculemus2/main/src/Infinitud\_de\_primos.lean}{Lean 4 Web}

\section{Si n² es par, entonces n es par}
\label{sec:org5b0aed4}
\begin{verbatim}
-- ---------------------------------------------------------------------
-- Demostrar que si n² es par, entonces n es par.
-- ---------------------------------------------------------------------

-- Demostración en lenguaje natural
-- ================================

-- Se usara el siguiente lema: "Si p es primo, entonces
--    (∀ a, b ∈ ℕ)[p ∣ ab ↔ p ∣ a ∨ p ∣ b].
--
-- Si n² es par, entonces 2 divide a n.n y, por el lema, 2 divide a n.

-- Demostraciones con Lean4
-- ========================

import Mathlib.Tactic
open Nat
variable (n : ℕ)

-- 1ª demostración
-- ===============

example
  (h : 2 ∣ n ^ 2)
  : 2 ∣ n :=
by
  rw [pow_two] at h
  -- h : 2 ∣ n * n
  have h1 : Nat.Prime 2 := prime_two
  have h2 : 2 ∣ n ∨ 2 ∣ n := (Prime.dvd_mul h1).mp h
  rcases h2 with h3 | h4
  · -- h3 : 2 ∣ n
    exact h3
  · -- h4 : 2 ∣ n
    exact h4
  done

-- 2ª demostración
-- ===============

example
  (h : 2 ∣ n ^ 2)
  : 2 ∣ n :=
by
  rw [pow_two] at h
  -- h : 2 ∣ n * n
  have h2 : 2 ∣ n ∨ 2 ∣ n := (Prime.dvd_mul prime_two).mp h
  rcases h2 with h3 | h4
  · exact h3
  · exact h4
  done

-- 3ª demostración
-- ===============

example
  (h : 2 ∣ n ^ 2)
  : 2 ∣ n :=
by
  rw [pow_two] at h
  -- h : 2 ∣ n * n
  have h2 : 2 ∣ n ∨ 2 ∣ n := (Prime.dvd_mul prime_two).mp h
  tauto
  done

-- Lemas usados
-- ============

-- variable (p a b : ℕ)
-- #check (prime_two : Nat.Prime 2)
-- #check (Prime.dvd_mul : Nat.Prime p → (p ∣ a * b ↔ p ∣ a ∨ p ∣ b))
\end{verbatim}
Se puede interactuar con las pruebas anteriores en \href{https://lean.math.hhu.de/\#url=https://raw.githubusercontent.com/jaalonso/Calculemus2/main/src/Par\_si\_cuadrado\_par.lean}{Lean 4 Web}

\section{La raíz cuadrada de 2 es irracional}
\label{sec:org4c205ba}
\begin{verbatim}
-- ---------------------------------------------------------------------
-- Demostrar que la raíz cuadrada de 2 es irracional; es decir, que no
-- existen m, n ∈ ℕ tales que m y n son coprimos (es decir, que no
-- factores comunes distintos de uno) y m² = 2n².
-- ---------------------------------------------------------------------

-- Demostración en lenguaje natural
-- ================================

-- Usaremos el lema del ejercicio anterior:
--    (∀ n ∈ ℕ)[2 ∣ n² → 2 | n]
--
-- Supongamos que existen existen m, n ∈ ℕ tales que m y n son coprimos y
-- m² = 2n² y tenemos que demostrar una contradicción. Puesto que 2 no
-- divide a 1, para tener la contradicción basta demostrar que 2 divide
-- a 1 y (puesto que m y n son coprimos), para ello es suficiente
-- demostrar que 2 divide al máximo común divisor de m y n. En
-- definitiva, basta demostrar que 2 divide a m y a n.
--
-- La demostración de que 2 divide a m es
--    m² = 2n² ⟹ 2 | m²
--             ⟹ 2 | m    [por el lema]
--
-- Para demostrar que 2 divide a n, observamos que, puesto que 2 divide
-- a m, existe un k ∈ ℕ tal que m = 2k. Sustituyendo en
--    m² = 2n²
-- se tiene
--    (2k)² = 2n²
-- Simplificando, queda
--    2k = n²
-- Por tanto, 2 divide a n² y, por el lema, 2 divide a n.

-- Demostraciones con Lean4
-- ========================

import Mathlib.Tactic
import Mathlib.Data.Nat.Prime
import Std.Data.Nat.Gcd
open Nat
variable {m n : ℕ}

lemma par_si_cuadrado_par
  (h : 2 ∣ n ^ 2)
  : 2 ∣ n :=
by
  rw [pow_two] at h
  -- h : 2 ∣ n * n
  have h2 : 2 ∣ n ∨ 2 ∣ n := (Prime.dvd_mul prime_two).mp h
  tauto

example : ¬∃ m n, coprime m n ∧ m ^ 2 = 2 * n ^ 2 :=
by
  rintro ⟨m, n, ⟨h1, h2⟩⟩
  -- m n : ℕ
  -- h1 : coprime m n
  -- h2 : m ^ 2 = 2 * n ^ 2
  -- ⊢ False
  have h3 : ¬(2 ∣ 1) := by norm_num
  have h4 : 2 ∣ 1 := by
    have h5 : Nat.gcd m n = 1 := h1
    rw [← h5]
    -- ⊢ 2 ∣ Nat.gcd m n
    have h6 : 2 ∣ m := by
      apply par_si_cuadrado_par
      -- ⊢ 2 ∣ m ^ 2
      rw [h2]
      -- ⊢ 2 ∣ 2 * n ^ 2
      exact Nat.dvd_mul_right 2 (n ^ 2)
    have h7 : 2 ∣ n := by
      have h8 : ∃ k, m = 2 * k := h6
      rcases h8 with ⟨k, h9⟩
      -- k : ℕ
      -- h9 : m = 2 * k
      have h10 : 2 * k ^ 2 = n ^ 2 := by
        have h10a : 2 * (2 * k ^ 2) = 2 * n ^ 2 := calc
          2 * (2 * k ^ 2) = (2 * k) ^ 2 := by nlinarith
                        _ = m ^ 2       := by rw [← h9]
                        _ = 2 * n ^ 2   := h2
        show 2 * k ^ 2 = n ^ 2
        exact (mul_right_inj' (by norm_num : 2 ≠ 0)).mp h10a
      have h11 : 2 ∣ n ^ 2 := by
        rw [← h10]
        -- ⊢ 2 ∣ 2 * k ^ 2
        exact Nat.dvd_mul_right 2 (k ^ 2)
      show 2 ∣ n
      exact par_si_cuadrado_par h11
    show 2 ∣ Nat.gcd m n
    exact Nat.dvd_gcd h6 h7
  show False
  exact h3 h4

-- Lemas usados
-- ============

-- variable (p k : ℕ)
-- #check (pow_two n : n ^ 2 = n * n)
-- #check (Prime.dvd_mul : Nat.Prime p → (p ∣ m * n ↔ p ∣ m ∨ p ∣ n))
-- #check (prime_two : Nat.Prime 2)
-- #check (Nat.dvd_gcd : k ∣ m → k ∣ n → k ∣ Nat.gcd m n)
-- #check (Nat.dvd_mul_right m n :  m ∣ m * n)
-- #check (mul_right_inj' : k ≠ 0 → (k * m = k * n ↔ m = n))
\end{verbatim}
Se puede interactuar con las pruebas anteriores en \href{https://lean.math.hhu.de/\#url=https://raw.githubusercontent.com/jaalonso/Calculemus2/main/src/Irracionalidad\_de\_la\_raiz\_cuadrada\_de\_2.lean}{Lean 4 Web}

\chapter{Retículos}
\label{sec:org6f18c61}

\section{En los retículos, x ⊓ y = y ⊓ x}
\label{sec:org9dd351e}
\begin{verbatim}
-- ---------------------------------------------------------------------
-- Demostrar que en los retículos se verifica que
--     x ⊓ y = y ⊓ x
-- ----------------------------------------------------------------------

-- Demostración en lenguaje natural
-- ================================

-- Es consecuencia del siguiente lema auxiliar
--    (∀ a, b)[a ⊓ b ≤ b ⊓ a]                                         (1)
-- En efecto, sustituyendo en (1) a por x y b por y, se tiene
--    x ⊓ y ≤ y ⊓ x                                                   (2)
-- y sustituyendo en (1) a por y y b por x, se tiene
--    y ⊓ x ≤ x ⊓ y                                                   (3)
-- Finalmente, aplicando la propiedad antisimétrica de la divisibilidad
-- a (2) y (3), se tiene
--    x ⊓ y = y ⊓ x
--
-- Para demostrar (1), por la definición del ínfimo, basta demostrar
-- las siguientes relaciones
--    y ⊓ x ≤ x
--    y ⊓ x ≤ y
-- y ambas se tienen por la definición del ínfimo.

-- Demostraciones con Lean4
-- ========================

import Mathlib.Order.Lattice
variable {α : Type _} [Lattice α]
variable (x y z : α)

-- 1ª demostración del lema auxiliar
lemma aux : x ⊓ y ≤ y ⊓ x :=
by
  have h1 : x ⊓ y ≤ y :=
    inf_le_right
  have h2 : x ⊓ y ≤ x :=
    inf_le_left
  show x ⊓ y ≤ y ⊓ x
  exact le_inf h1 h2

-- 2ª demostración del lema auxiliar
example : x ⊓ y ≤ y ⊓ x :=
by
  apply le_inf
  { apply inf_le_right }
  { apply inf_le_left }

-- 3ª demostración del lema auxiliar
example : x ⊓ y ≤ y ⊓ x :=
le_inf inf_le_right inf_le_left

-- 1ª demostración
example : x ⊓ y = y ⊓ x :=
by
  have h1 : x ⊓ y ≤ y ⊓ x :=
    aux x y
  have h2 : y ⊓ x ≤ x ⊓ y :=
    aux y x
  show x ⊓ y = y ⊓ x
  exact le_antisymm h1 h2

-- 2ª demostración
example : x ⊓ y = y ⊓ x :=
by
  apply le_antisymm
  { apply aux }
  { apply aux }

-- 3ª demostración
example : x ⊓ y = y ⊓ x :=
le_antisymm (aux x y) (aux y x)

-- 4ª demostración
example : x ⊓ y = y ⊓ x :=
by apply le_antisymm; simp ; simp

-- 5ª demostración
example : x ⊓ y = y ⊓ x :=
-- by apply?
inf_comm

-- Lemas usados
-- ============

-- #check (inf_comm : x ⊓ y = y ⊓ x)
-- #check (inf_le_left : x ⊓ y ≤ x)
-- #check (inf_le_right : x ⊓ y ≤ y)
-- #check (le_antisymm : x ≤ y → y ≤ x → x = y)
-- #check (le_inf : z ≤ x → z ≤ y → z ≤ x ⊓ y)
\end{verbatim}
Se puede interactuar con las pruebas anteriores en \href{https://lean.math.hhu.de/\#url=https://raw.githubusercontent.com/jaalonso/Calculemus2/main/src/Conmutatividad\_del\_infimo.lean}{Lean 4 Web}

\section{En los retículos, x ⊔ y = y ⊔ x}
\label{sec:org7441831}
\begin{verbatim}
-- ---------------------------------------------------------------------
-- Demostrar que en los retículos se verifica que
--    x ⊔ y = y ⊔ x
-- para todo x e y en el retículo.
-- ----------------------------------------------------------------------

-- Demostración en lenguaje natural
-- ================================

-- Es consecuencia del siguiente lema auxiliar
--    (∀ a, b)[a ⊔ b ≤ b ⊔ a]                                         (1)
-- En efecto, sustituyendo en (1) a por x y b por y, se tiene
--    x ⊔ y ≤ y ⊔ x                                                   (2)
-- y sustituyendo en (1) a por y y b por x, se tiene
--    y ⊔ x ≤ x ⊔ y                                                   (3)
-- Finalmente, aplicando la propiedad antisimétrica de la divisibilidad
-- a (2) y (3), se tiene
--    x ⊔ y = y ⊔ x
--
-- Para demostrar (1), por la definición del supremo, basta demostrar
-- las siguientes relaciones
--    x ≤ y ⊔ x
--    y ≤ y ⊔ x
-- y ambas se tienen por la definición del supremo.

-- Demostraciones con Lean4
-- ========================

import Mathlib.Order.Lattice
variable {α : Type _} [Lattice α]
variable (x y z : α)

-- 1ª demostración del lema auxiliar
lemma aux : x ⊔ y ≤ y ⊔ x :=
by
  have h1 : x ≤ y ⊔ x :=
    le_sup_right
  have h2 : y ≤ y ⊔ x :=
    le_sup_left
  show x ⊔ y ≤ y ⊔ x
  exact sup_le h1 h2

-- 2ª demostración del lema auxiliar
example : x ⊔ y ≤ y ⊔ x :=
by
  apply sup_le
  { apply le_sup_right }
  { apply le_sup_left }

-- 3ª demostración del lema auxiliar
example : x ⊔ y ≤ y ⊔ x :=
sup_le le_sup_right le_sup_left

-- 1ª demostración
example : x ⊔ y = y ⊔ x :=
by
  have h1 : x ⊔ y ≤ y ⊔ x :=
    aux x y
  have h2 : y ⊔ x ≤ x ⊔ y :=
    aux y x
  show x ⊔ y = y ⊔ x
  exact le_antisymm h1 h2

-- 2ª demostración
example : x ⊔ y = y ⊔ x :=
by
  apply le_antisymm
  { apply aux }
  { apply aux }

-- 3ª demostración
example : x ⊔ y = y ⊔ x :=
le_antisymm (aux x y) (aux y x)

-- 4ª demostración
example : x ⊔ y = y ⊔ x :=
by apply le_antisymm; simp ; simp

-- 5ª demostración
example : x ⊔ y = y ⊔ x :=
-- by apply?
sup_comm

-- Lemas usados
-- ============

-- #check (le_antisymm : x ≤ y → y ≤ x → x = y)
-- #check (le_sup_left : x ≤ x ⊔ y)
-- #check (le_sup_right : y ≤ x ⊔ y)
-- #check (sup_comm : x ⊔ y = y ⊔ x)
-- #check (sup_le : x ≤ z → y ≤ z → x ⊔ y ≤ z)
\end{verbatim}
Se puede interactuar con las pruebas anteriores en \href{https://lean.math.hhu.de/\#url=https://raw.githubusercontent.com/jaalonso/Calculemus2/main/src/Conmutatividad\_del\_supremo.lean}{Lean 4 Web}

\section{En los retículos, (x ⊓ y) ⊓ z = x ⊓ (y ⊓ z)}
\label{sec:org7ded3a6}
\begin{verbatim}
-- ---------------------------------------------------------------------
-- Demostrar que en los retículos se verifica que
--     (x ⊓ y) ⊓ z = x ⊓ (y ⊓ z)
-- ----------------------------------------------------------------------

-- Demostración en lenguaje natural
-- ================================

-- En la demostración se usarán los siguientes lemas
--    le_antisymm : x ≤ y → y ≤ x → x = y
--    le_inf : z ≤ x → z ≤ y → z ≤ x ⊓ y
--    inf_le_left : x ⊓ y ≤ x
--    inf_le_right : x ⊓ y ≤ y)
--
-- Por le_antisym, es suficiente demostrar las siguientes relaciones:
--    (x ⊓ y) ⊓ z ≤ x ⊓ (y ⊓ z)                                      (1)
--    x ⊓ (y ⊓ z) ≤ (x ⊓ y) ⊓ z                                      (2)
--
-- Para demostrar (1), por le_inf, basta probar que
--    (x ⊓ y) ⊓ z ≤ x                                               (1a)
--    (x ⊓ y) ⊓ z ≤ y ⊓ z                                           (1b)
--
-- La (1a) se demuestra por la siguiente cadena de desigualdades
--    (x ⊓ y) ⊓ z ≤ x ⊓ y   [por inf_le_left]
--                ≤ x       [por inf_le_left]
--
-- Para demostrar (1b), por le_inf, basta probar que
--    (x ⊓ y) ⊓ z ≤ y                                              (1b1)
--    (x ⊓ y) ⊓ z ≤ z                                              (1b2)
--
-- La (1b1) se demuestra por la siguiente cadena de desigualdades
--    (x ⊓ y) ⊓ z ≤ x ⊓ y   [por inf_le_left]
--                ≤ y       [por inf_le_right]
--
-- La (1b2) se tiene por inf_le_right.
--
-- Para demostrar (2), por le_inf, basta probar que
--    x ⊓ (y ⊓ z) ≤ x ⊓ y                                           (2a)
--    x ⊓ (y ⊓ z) ≤ z                                               (2b)
--
-- Para demostrar (2a), por le_inf, basta probar que
--    x ⊓ (y ⊓ z) ≤ x                                              (2a1)
--    x ⊓ (y ⊓ z) ≤ y                                              (2a2)
--
-- La (2a1) se tiene por inf_le_left.
--
-- La (2a2) se demuestra por la siguiente cadena de desigualdades
--    x ⊓ (y ⊓ z) ≤ y ⊓ z   [por inf_le_right]
--                ≤ y       [por inf_le_left]
--
-- La (2b) se demuestra por la siguiente cadena de desigualdades
--    x ⊓ (y ⊓ z) ≤ y ⊓ z   [por inf_le_right]
--                ≤ z       [por inf_le_right]

-- Demostraciones con Lean4
-- ========================

import Mathlib.Order.Lattice
variable {α : Type _} [Lattice α]
variable (x y z : α)

-- 1ª demostración
-- ===============

example : (x ⊓ y) ⊓ z = x ⊓ (y ⊓ z) :=
by
  have h1 : (x ⊓ y) ⊓ z ≤ x ⊓ (y ⊓ z) := by
  { have h1a : (x ⊓ y) ⊓ z ≤ x := calc
      (x ⊓ y) ⊓ z ≤ x ⊓ y := by exact inf_le_left
                _ ≤ x     := by exact inf_le_left
    have h1b : (x ⊓ y) ⊓ z ≤ y ⊓ z := by
    { have h1b1 : (x ⊓ y) ⊓ z ≤ y := calc
        (x ⊓ y) ⊓ z ≤ x ⊓ y := by exact inf_le_left
                  _ ≤ y     := by exact inf_le_right
      have h1b2 : (x ⊓ y) ⊓ z ≤ z :=
        inf_le_right
      show (x ⊓ y) ⊓ z ≤ y ⊓ z
      exact le_inf h1b1 h1b2 }
    show (x ⊓ y) ⊓ z ≤ x ⊓ (y ⊓ z)
    exact le_inf h1a h1b }
  have h2 : x ⊓ (y ⊓ z) ≤ (x ⊓ y) ⊓ z := by
  { have h2a : x ⊓ (y ⊓ z) ≤ x ⊓ y := by
    { have h2a1 : x ⊓ (y ⊓ z) ≤ x :=
        inf_le_left
      have h2a2 : x ⊓ (y ⊓ z) ≤ y := calc
        x ⊓ (y ⊓ z) ≤ y ⊓ z := by exact inf_le_right
                  _ ≤ y     := by exact inf_le_left
      show x ⊓ (y ⊓ z) ≤ x ⊓ y
      exact le_inf h2a1 h2a2 }
    have h2b : x ⊓ (y ⊓ z) ≤ z := by calc
      x ⊓ (y ⊓ z) ≤ y ⊓ z := by exact inf_le_right
                _ ≤ z     := by exact inf_le_right
    show x ⊓ (y ⊓ z) ≤ (x ⊓ y) ⊓ z
    exact le_inf h2a h2b }
  show (x ⊓ y) ⊓ z = x ⊓ (y ⊓ z)
  exact le_antisymm h1 h2

-- 2ª demostración
-- ===============

example : x ⊓ y ⊓ z = x ⊓ (y ⊓ z) := by
  apply le_antisymm
  · apply le_inf
    · apply le_trans
      apply inf_le_left
      apply inf_le_left
    . apply le_inf
      · apply le_trans
        apply inf_le_left
        apply inf_le_right
      . apply inf_le_right
  . apply le_inf
    · apply le_inf
      · apply inf_le_left
      . apply le_trans
        apply inf_le_right
        apply inf_le_left
    . apply le_trans
      apply inf_le_right
      apply inf_le_right

-- 3ª demostración
-- ===============

example : (x ⊓ y) ⊓ z = x ⊓ (y ⊓ z) :=
by
  apply le_antisymm
  . apply le_inf
    . apply inf_le_of_left_le inf_le_left
    . apply le_inf (inf_le_of_left_le inf_le_right) inf_le_right
  . apply le_inf
    . apply le_inf inf_le_left (inf_le_of_right_le inf_le_left)
    . apply inf_le_of_right_le inf_le_right

-- 4ª demostración
-- ===============

example : (x ⊓ y) ⊓ z = x ⊓ (y ⊓ z) :=
le_antisymm
  (le_inf
    (inf_le_of_left_le inf_le_left)
    (le_inf (inf_le_of_left_le inf_le_right) inf_le_right))
  (le_inf
    (le_inf inf_le_left (inf_le_of_right_le inf_le_left))
    (inf_le_of_right_le inf_le_right))

-- 5ª demostración
-- ===============

example : (x ⊓ y) ⊓ z = x ⊓ (y ⊓ z) :=
-- by apply?
inf_assoc

-- Lemas usados
-- ============

-- #check (inf_assoc : (x ⊓ y) ⊓ z = x ⊓ (y ⊓ z))
-- #check (inf_le_left : x ⊓ y ≤ x)
-- #check (inf_le_of_left_le : x ≤ z → x ⊓ y ≤ z)
-- #check (inf_le_of_right_le : y ≤ z → x ⊓ y ≤ z)
-- #check (inf_le_right : x ⊓ y ≤ y)
-- #check (le_antisymm : x ≤ y → y ≤ x → x = y)
-- #check (le_inf : z ≤ x → z ≤ y → z ≤ x ⊓ y)
-- #check (le_trans : x ≤ y → y ≤ z → x ≤ z)
\end{verbatim}
Se puede interactuar con las pruebas anteriores en \href{https://lean.math.hhu.de/\#url=https://raw.githubusercontent.com/jaalonso/Calculemus2/main/src/Asociatividad\_del\_infimo.lean}{Lean 4 Web}

\section{En los retículos, (x ⊔ y) ⊔ z = x ⊔ (y ⊔ z)}
\label{sec:orgd0fb5c0}
\begin{verbatim}
-- ---------------------------------------------------------------------
-- Demostrar que en los retículos se verifica que
--    (x ⊔ y) ⊔ z = x ⊔ (y ⊔ z)
-- ---------------------------------------------------------------------

-- Demostración en lenguaje natural
-- ================================

-- En la demostración se usarán los siguientes lemas
--    le_antisymm  : x ≤ y → y ≤ x → x = y
--    le_sup_left  : x ≤ x ⊔ y
--    le_sup_right : y ≤ x ⊔ y
--    sup_le       : x ≤ z → y ≤ z → x ⊔ y ≤ z
--
-- Por le_antisymm, basta demostrar las siguientes relaciones:
--    (x ⊔ y) ⊔ z ≤ x ⊔ (y ⊔ z)                                      (1)
--    x ⊔ (y ⊔ z) ≤ (x ⊔ y) ⊔ z                                      (2)
--
-- Para demostrar (1), por sup_le, basta probar
--    x ⊔ y ≤ x ⊔ (y ⊔ z)                                           (1a)
--    z ≤ x ⊔ (y ⊔ z)                                               (1b)
--
-- Para demostrar (1a), por sup_le, basta probar
--    x ≤ x ⊔ (y ⊔ z)                                              (1a1)
--    y ≤ x ⊔ (y ⊔ z)                                              (1a2)
--
-- La (1a1) se tiene por le_sup_left.
--
-- La (1a2) se tiene por la siguiente cadena de desigualdades:
--    y ≤ y ⊔ z          [por le_sup_left]
--      ≤ x ⊔ (y ⊔ z)    [por le_sup_right]
--
-- La (1b) se tiene por la siguiente cadena de desigualdades
--    z ≤ y ⊔ z          [por le_sup_right]
--      ≤ x ⊔ (y ⊔ z)    [por le_sup_right]
--
-- Para demostrar (2), por sup_le, basta probar
--    x ≤ (x ⊔ y) ⊔ z                                               (2a)
--    y ⊔ z ≤ (x ⊔ y) ⊔ z                                           (2b)
--
-- La (2a) se demuestra por la siguiente cadena de desigualdades:
--    x ≤ x ⊔ y          [por le_sup_left]
--      ≤ (x ⊔ y) ⊔ z    [por le_sup_left]
--
-- Para demostrar (2b), por sup_le, basta probar
--    y ≤ (x ⊔ y) ⊔ z                                              (2b1)
--    z ≤ (x ⊔ y) ⊔ z                                              (2b2)
--
-- La (2b1) se demuestra por la siguiente cadena de desigualdades:
--    y ≤ x ⊔ y          [por le_sup_right]
--      ≤ (x ⊔ y) ⊔ z    [por le_sup_left]
--
-- La (2b2) se tiene por le_sup_right.

-- Demostraciones con Lean 4
-- =========================

import Mathlib.Order.Lattice

variable {α : Type _} [Lattice α]
variable (x y z : α)

-- 1ª demostración
-- ===============

example : (x ⊔ y) ⊔ z = x ⊔ (y ⊔ z) :=
by
  have h1 : (x ⊔ y) ⊔ z ≤ x ⊔ (y ⊔ z) := by
  { have h1a : x ⊔ y ≤ x ⊔ (y ⊔ z) := by
    { have h1a1 : x ≤ x ⊔ (y ⊔ z) := by exact le_sup_left
      have h1a2 : y ≤ x ⊔ (y ⊔ z) := calc
        y ≤ y ⊔ z       := by exact le_sup_left
        _ ≤ x ⊔ (y ⊔ z) := by exact le_sup_right
      show x ⊔ y ≤ x ⊔ (y ⊔ z)
      exact sup_le h1a1 h1a2 }
    have h1b : z ≤ x ⊔ (y ⊔ z) := calc
      z ≤ y ⊔ z       := by exact le_sup_right
      _ ≤ x ⊔ (y ⊔ z) := by exact le_sup_right
    show (x ⊔ y) ⊔ z ≤ x ⊔ (y ⊔ z)
    exact sup_le h1a h1b }
  have h2 : x ⊔ (y ⊔ z) ≤ (x ⊔ y) ⊔ z := by
  { have h2a : x ≤ (x ⊔ y) ⊔ z := calc
      x ≤ x ⊔ y       := by exact le_sup_left
      _ ≤ (x ⊔ y) ⊔ z := by exact le_sup_left
    have h2b : y ⊔ z ≤ (x ⊔ y) ⊔ z := by
    { have h2b1 : y ≤ (x ⊔ y) ⊔ z := calc
        y ≤ x ⊔ y       := by exact le_sup_right
        _ ≤ (x ⊔ y) ⊔ z := by exact le_sup_left
      have h2b2 : z ≤ (x ⊔ y) ⊔ z := by
        exact le_sup_right
      show  y ⊔ z ≤ (x ⊔ y) ⊔ z
      exact sup_le h2b1 h2b2 }
    show x ⊔ (y ⊔ z) ≤ (x ⊔ y) ⊔ z
    exact sup_le h2a h2b }
  show (x ⊔ y) ⊔ z = x ⊔ (y ⊔ z)
  exact le_antisymm h1 h2

-- 2ª demostración
-- ===============

example : x ⊔ y ⊔ z = x ⊔ (y ⊔ z) :=
by
  apply le_antisymm
  · -- (x ⊔ y) ⊔ z ≤ x ⊔ (y ⊔ z)
    apply sup_le
    · -- x ⊔ y ≤ x ⊔ (y ⊔ z)
      apply sup_le
      . -- x ≤ x ⊔ (y ⊔ z)
        apply le_sup_left
      · -- y ≤ x ⊔ (y ⊔ z)
        apply le_trans
        . -- y ≤ y ⊔ z
          apply @le_sup_left _ _ y z
        . -- y ⊔ z ≤ x ⊔ (y ⊔ z)
          apply le_sup_right
    . -- z ≤ x ⊔ (y ⊔ z)
      apply le_trans
      . -- z ≤ x ⊔ (y ⊔ z)
        apply @le_sup_right _ _ y z
      . -- y ⊔ z ≤ x ⊔ (y ⊔ z)
        apply le_sup_right
  . -- x ⊔ (y ⊔ z) ≤ (x ⊔ y) ⊔ z
    apply sup_le
    · -- x ≤ (x ⊔ y) ⊔ z
      apply le_trans
      . -- x ≤ x ⊔ y
        apply @le_sup_left _ _ x y
      . -- x ⊔ y ≤ (x ⊔ y) ⊔ z
        apply le_sup_left
    . -- y ⊔ z ≤ (x ⊔ y) ⊔ z
      apply sup_le
      · -- y ≤ (x ⊔ y) ⊔ z
        apply le_trans
        . -- y ≤ x ⊔ y
          apply @le_sup_right _ _ x y
        . -- x ⊔ y ≤ (x ⊔ y) ⊔ z
          apply le_sup_left
      . -- z ≤ (x ⊔ y) ⊔ z
        apply le_sup_right

-- 3ª demostración
-- ===============

example : x ⊔ y ⊔ z = x ⊔ (y ⊔ z) :=
by
  apply le_antisymm
  · apply sup_le
    · apply sup_le
      . apply le_sup_left
      · apply le_trans
        . apply @le_sup_left _ _ y z
        . apply le_sup_right
    . apply le_trans
      . apply @le_sup_right _ _ y z
      . apply le_sup_right
  . apply sup_le
    · apply le_trans
      . apply @le_sup_left _ _ x y
      . apply le_sup_left
    . apply sup_le
      · apply le_trans
        . apply @le_sup_right _ _ x y
        . apply le_sup_left
      . apply le_sup_right

-- 4ª demostración
-- ===============

example : (x ⊔ y) ⊔ z = x ⊔ (y ⊔ z) :=
by
  apply le_antisymm
  . -- (x ⊔ y) ⊔ z ≤ x ⊔ (y ⊔ z)
    apply sup_le
    . -- x ⊔ y ≤ x ⊔ (y ⊔ z)
      apply sup_le le_sup_left (le_sup_of_le_right le_sup_left)
    . -- z ≤ x ⊔ (y ⊔ z)
      apply le_sup_of_le_right le_sup_right
  . -- x ⊔ (y ⊔ z) ≤ (x ⊔ y) ⊔ z
    apply sup_le
    . -- x ≤ (x ⊔ y) ⊔ z
      apply le_sup_of_le_left le_sup_left
    . -- y ⊔ z ≤ (x ⊔ y) ⊔ z
      apply sup_le (le_sup_of_le_left le_sup_right) le_sup_right

-- 5ª demostración
-- ===============

example : (x ⊔ y) ⊔ z = x ⊔ (y ⊔ z) :=
by
  apply le_antisymm
  . apply sup_le
    . apply sup_le le_sup_left (le_sup_of_le_right le_sup_left)
    . apply le_sup_of_le_right le_sup_right
  . apply sup_le
    . apply le_sup_of_le_left le_sup_left
    . apply sup_le (le_sup_of_le_left le_sup_right) le_sup_right

-- 6ª demostración
-- ===============

example : (x ⊔ y) ⊔ z = x ⊔ (y ⊔ z) :=
le_antisymm
  (sup_le
    (sup_le le_sup_left (le_sup_of_le_right le_sup_left))
    (le_sup_of_le_right le_sup_right))
  (sup_le
    (le_sup_of_le_left le_sup_left)
    (sup_le (le_sup_of_le_left le_sup_right) le_sup_right))

-- 7ª demostración
-- ===============

example : (x ⊔ y) ⊔ z = x ⊔ (y ⊔ z) :=
-- by apply?
sup_assoc

-- Lemas usados
-- ============

-- #check (le_antisymm : x ≤ y → y ≤ x → x = y)
-- #check (le_sup_left : x ≤ x ⊔ y)
-- #check (le_sup_of_le_left : z ≤ x → z ≤ x ⊔ y)
-- #check (le_sup_of_le_right : z ≤ y → z ≤ x ⊔ y)
-- #check (le_sup_right : y ≤ x ⊔ y)
-- #check (le_trans : x ≤ y → y ≤ z → x ≤ z)
-- #check (sup_assoc : (x ⊔ y) ⊔ z = x ⊔ (y ⊔ z))
-- #check (sup_le : x ≤ z → y ≤ z → x ⊔ y ≤ z)
\end{verbatim}
Se puede interactuar con las pruebas anteriores en \href{https://lean.math.hhu.de/\#url=https://raw.githubusercontent.com/jaalonso/Calculemus2/main/src/Asociatividad\_del\_supremo.lean}{Lean 4 Web}

\section{En los retículos, x ⊓ (x ⊔ y) = x}
\label{sec:org4894ded}
\begin{verbatim}
-- ---------------------------------------------------------------------
-- Demostrar que en los retículos se verifica que
--    x ⊓ (x ⊔ y) = x
-- ---------------------------------------------------------------------

-- Demostración en lenguaje natural
-- ================================

-- En la demostración se usarán los siguientes lemas
--    le_antisymm  : x ≤ y → y ≤ x → x = y
--    inf_le_left  : x ⊓ y ≤ x
--    le_inf       : z ≤ x → z ≤ y → z ≤ x ⊓ y
--    le_rfl       : x ≤ x
--    le_sup_left  : x ≤ x ⊔ y
--
-- Por le_antisymm, basta demostrar las siguientes relaciones:
--    x ⊓ (x ⊔ y) ≤ x                                                (1)
--    x ≤ x ⊓ (x ⊔ y)                                                (2)
--
-- La (1) se tiene por inf_le_left.
--
-- Para demostrar la (2), por le_inf, basta probar las relaciones:
--    x ≤ x                                                         (2a)
--    x ≤ x ⊔ y                                                     (2b)
--
-- La (2a) se tiene por le_rfl.
--
-- La (2b) se tiene por le_sup_left

-- Demostraciones con Lean4
-- ========================

import Mathlib.Order.Lattice
variable {α : Type _} [Lattice α]
variable (x y : α)

-- 1ª demostración
-- ===============

example : x ⊓ (x ⊔ y) = x :=
by
  have h1 : x ⊓ (x ⊔ y) ≤ x := inf_le_left
  have h2 : x ≤ x ⊓ (x ⊔ y)
  { have h2a : x ≤ x := le_rfl
    have h2b : x ≤ x ⊔ y := le_sup_left
    show x ≤ x ⊓ (x ⊔ y)
    exact le_inf h2a h2b }
  show x ⊓ (x ⊔ y) = x
  exact le_antisymm h1 h2

-- 2ª demostración
-- ===============

example : x ⊓ (x ⊔ y) = x :=
by
  have h1 : x ⊓ (x ⊔ y) ≤ x := by simp
  have h2 : x ≤ x ⊓ (x ⊔ y) := by simp
  show x ⊓ (x ⊔ y) = x
  exact le_antisymm h1 h2

-- 3ª demostración
-- ===============

example : x ⊓ (x ⊔ y) = x :=
by
  apply le_antisymm
  . -- x ⊓ (x ⊔ y) ≤ x
    apply inf_le_left
  . -- x ≤ x ⊓ (x ⊔ y)
    apply le_inf
    . -- x ≤ x
      apply le_rfl
    . -- x ≤ x ⊔ y
      apply le_sup_left

-- 4ª demostración
-- ===============

example : x ⊓ (x ⊔ y) = x :=
le_antisymm inf_le_left (le_inf le_rfl le_sup_left)

-- 5ª demostración
-- ===============

example : x ⊓ (x ⊔ y) = x :=
-- by apply?
inf_sup_self

-- 6ª demostración
-- ===============

example : x ⊓ (x ⊔ y) = x :=
by simp

-- Lemas usados
-- ============

-- variable (z : α)
-- #check (inf_le_left : x ⊓ y ≤ x)
-- #check (inf_sup_self : x ⊓ (x ⊔ y) = x)
-- #check (le_antisymm : x ≤ y → y ≤ x → x = y)
-- #check (le_inf : z ≤ x → z ≤ y → z ≤ x ⊓ y)
-- #check (le_rfl : x ≤ x)
-- #check (le_sup_left : x ≤ x ⊔ y)
\end{verbatim}
Se puede interactuar con las pruebas anteriores en \href{https://lean.math.hhu.de/\#url=https://raw.githubusercontent.com/jaalonso/Calculemus2/main/src/Leyes\_de\_absorcion\_1.lean}{Lean 4 Web}

\section{En los retículos, x ⊔ (x ⊓ y) = x}
\label{sec:org06bdfd3}
\begin{verbatim}
-- ---------------------------------------------------------------------
-- Demostrar que en los retículos se verifica que
--    x ⊔ (x ⊓ y) = x
-- ---------------------------------------------------------------------

-- Demostración en lenguaje natural
-- ================================

-- En la demostración se usarán los siguientes lemas
--   le_antisymm  : x ≤ y → y ≤ x → x = y
--   inf_le_left  : x ⊓ y ≤ x
--   le_rfl       : x ≤ x
--   le_sup_left  : x ≤ x ⊔ y
--   sup_le       : x ≤ z → y ≤ z → x ⊔ y ≤ z
--
-- Por le_antisymm, basta demostrar las siguientes relaciones:
--    x ⊔ (x ⊓ y) ≤ x                                                (1)
--    x ≤ x ⊔ (x ⊓ y)    [que se tiene por le_sup_left]
--
-- Para demostrar (1), por sup_le, basta probar las relaciones:
--    x ≤ x              [que se tiene por le_rfl]
--    x ⊓ y ≤ x          [que se tiene por inf_le_left]

-- Demostraciones con Lean4
-- ========================

import Mathlib.Order.Lattice
variable {α : Type _} [Lattice α]--
variable (x y : α)

-- 1ª demostración
-- ===============

example : x ⊔ (x ⊓ y) = x :=
by
  have h1 : x ⊔ (x ⊓ y) ≤ x
  { have h1a : x ≤ x := le_rfl
    have h1b : x ⊓ y ≤ x := inf_le_left
    show x ⊔ (x ⊓ y) ≤ x
    exact sup_le h1a h1b }
  have h2 : x ≤ x ⊔ (x ⊓ y) := le_sup_left
  show x ⊔ (x ⊓ y) = x
  exact le_antisymm h1 h2

-- 2ª demostración
-- ===============

example : x ⊔ (x ⊓ y) = x :=
by
  have h1 : x ⊔ (x ⊓ y) ≤ x := by simp
  have h2 : x ≤ x ⊔ (x ⊓ y) := by simp
  show x ⊔ (x ⊓ y) = x
  exact le_antisymm h1 h2

-- 3ª demostración
-- ===============

example : x ⊔ (x ⊓ y) = x :=
by
  apply le_antisymm
  . -- x ⊔ (x ⊓ y) ≤ x
    apply sup_le
    . -- x ≤ x
      apply le_rfl
    . -- x ⊓ y ≤ x
      apply inf_le_left
  . -- x ≤ x ⊔ (x ⊓ y)
    apply le_sup_left

-- 4ª demostración
-- ===============

example : x ⊔ (x ⊓ y) = x :=
-- by apply?
sup_inf_self

-- 5ª demostración
-- ===============

example : x ⊔ (x ⊓ y) = x :=
by simp

-- Lemas usados
-- ============

-- variable (z : α)
-- #check (le_rfl : x ≤ x)
-- #check (inf_le_left : x ⊓ y ≤ x)
-- #check (sup_le : x ≤ z → y ≤ z → x ⊔ y ≤ z)
-- #check (le_sup_left : x ≤ x ⊔ y)
-- #check (le_antisymm : x ≤ y → y ≤ x → x = y)
-- #check (sup_inf_self : x ⊔ (x ⊓ y) = x)
\end{verbatim}
Se puede interactuar con las pruebas anteriores en \href{https://lean.math.hhu.de/\#url=https://raw.githubusercontent.com/jaalonso/Calculemus2/main/src/Leyes\_de\_absorcion\_2.lean}{Lean 4 Web}

\section{En los retículos, una distributiva del ínfimo implica la otra}
\label{sec:org82ba7ec}
\begin{verbatim}
-- ---------------------------------------------------------------------
-- Demostrar que si α es un retículo tal que
--    ∀ x y z : α, x ⊓ (y ⊔ z) = (x ⊓ y) ⊔ (x ⊓ z))
-- entonces
--    (a ⊔ b) ⊓ c = (a ⊓ c) ⊔ (b ⊓ c)
-- para todos los elementos de α.
-- ----------------------------------------------------------------------

-- Demostración en lenguaje natural
-- ================================

-- Se demuestra por la siguiente cadena de igualdades
--    (a ⊔ b) ⊓ c = c ⊓ (a ⊔ b)          [por conmutatividad de ⊓]
--                = (c ⊓ a) ⊔ (c ⊓ b)    [por la hipótesis]
--                = (a ⊓ c) ⊔ (c ⊓ b)    [por conmutatividad de ⊓]
--                = (a ⊓ c) ⊔ (b ⊓ c)    [por conmutatividad de ⊓]

-- Demostraciones con Lean4
-- ========================

import Mathlib.Order.Lattice
variable {α : Type _} [Lattice α]
variable (a b c : α)

-- 1ª demostración
example
  (h : ∀ x y z : α, x ⊓ (y ⊔ z) = (x ⊓ y) ⊔ (x ⊓ z))
  : (a ⊔ b) ⊓ c = (a ⊓ c) ⊔ (b ⊓ c) :=
calc
  (a ⊔ b) ⊓ c = c ⊓ (a ⊔ b)       := by rw [inf_comm]
            _ = (c ⊓ a) ⊔ (c ⊓ b) := by rw [h]
            _ = (a ⊓ c) ⊔ (c ⊓ b) := by rw [@inf_comm _ _ c a]
            _ = (a ⊓ c) ⊔ (b ⊓ c) := by rw [@inf_comm _ _ c b]

-- 2ª demostración
example
  (h : ∀ x y z : α, x ⊓ (y ⊔ z) = (x ⊓ y) ⊔ (x ⊓ z))
  : (a ⊔ b) ⊓ c = (a ⊓ c) ⊔ (b ⊓ c) :=
by simp [h, inf_comm]

-- Lemas usados
-- ============

-- #check (inf_comm : a ⊓ b = b ⊓ a)
\end{verbatim}
Se puede interactuar con las pruebas anteriores en \href{https://lean.math.hhu.de/\#url=https://raw.githubusercontent.com/jaalonso/Calculemus2/main/src/Propiedad\_distributiva\_1.lean}{Lean 4 Web}

\section{En los retículos, una distributiva del supremos implica la otra}
\label{sec:orgfc1af8e}
\begin{verbatim}
-- ---------------------------------------------------------------------
-- Demostrar que si α es un retículo tal que
--    ∀ x y z : α, x ⊔ (y ⊓ z) = (x ⊔ y) ⊓ (x ⊔ z)
-- entonces
--    (a ⊓ b) ⊔ c = (a ⊔ c) ⊓ (b ⊔ c)
-- ----------------------------------------------------------------------

-- Demostración en lenguaje natural
-- ================================

-- Se demuestra por la siguiente cadena de igualdades
--    (a ⊓ b) ⊔ c = c ⊔ (a ⊓ b)          [por la conmutatividad de ⊔]
--                = (c ⊔ a) ⊓ (c ⊔ b)    [por la hipótesis]
--                = (a ⊔ c) ⊓ (c ⊔ b)    [por la conmutatividad de ⊔]
--                = (a ⊔ c) ⊓ (b ⊔ c)    [por la conmutatividad de ⊔]

-- Demostraciones con Lean4
-- ========================

import Mathlib.Order.Lattice
variable {α : Type _} [Lattice α]
variable (a b c : α)

-- 1ª demostración
example
  (h : ∀ x y z : α, x ⊔ (y ⊓ z) = (x ⊔ y) ⊓ (x ⊔ z))
  : (a ⊓ b) ⊔ c = (a ⊔ c) ⊓ (b ⊔ c) :=
calc
  (a ⊓ b) ⊔ c = c ⊔ (a ⊓ b)       := by rw [sup_comm]
            _ = (c ⊔ a) ⊓ (c ⊔ b) := by rw [h]
            _ = (a ⊔ c) ⊓ (c ⊔ b) := by rw [@sup_comm _ _ c a]
            _ = (a ⊔ c) ⊓ (b ⊔ c) := by rw [@sup_comm _ _ c b]

-- 2ª demostración
example
  (h : ∀ x y z : α, x ⊔ (y ⊓ z) = (x ⊔ y) ⊓ (x ⊔ z))
  : (a ⊓ b) ⊔ c = (a ⊔ c) ⊓ (b ⊔ c) :=
by simp [h, sup_comm]

-- Lemas usados
-- ============

-- #check (sup_comm : a ⊔ b = b ⊔ a)
\end{verbatim}
Se puede interactuar con las pruebas anteriores en \href{https://lean.math.hhu.de/\#url=https://raw.githubusercontent.com/jaalonso/Calculemus2/main/src/Propiedad\_distributiva\_2.lean}{Lean 4 Web}

\chapter{Relaciones de orden}
\label{sec:org30b6c86}

\section{En los órdenes parciales, a < b ↔ a ≤ b ∧ a ≠ b}
\label{sec:org1a782be}
\begin{verbatim}
-- ---------------------------------------------------------------------
-- Demostrar que en un orden parcial
--     a < b ↔ a ≤ b ∧ a ≠ b
-- ----------------------------------------------------------------------

-- Demostración en lenguaje natural
-- ================================

-- Usaremos los siguientes lemas
--    (∀ a, b)[a < b ↔ a ≤ b ∧ b ≰ a]                               (L1)
--    (∀ a, b)[a ≤ b → b ≤ a → a = b]                               (L2)
--
-- Por el lema L1, lo que tenemos que demostrar es
--    a ≤ b ∧ b ≰ a ↔ a ≤ b ∧ a ≠ b
-- Lo haremos demostrando las dos implicaciones.
--
-- (⇒) Supongamos que a ≤ b y b ≰ a. Tenemos que demostrar que
-- a ≠ b. Lo haremos por reducción al absurdo. Para ello, supongamos que
-- a = b. Entonces, b ≤ a que contradice a b ≰ a.
--
-- (⇐) Supongamos que a ≤ b y a ≠ b. Tenemos que demostrar que
-- b ≰ a. Lo haremos por reducción al absurdo. Para ello, supongamos que
-- b ≤ a. Entonces, junto con a ≤ b, se tiene que a = b que es una
-- contradicicción con a ≠ b.

-- Demostraciones con Lean4
-- ========================

import Mathlib.Tactic

variable {α : Type _} [PartialOrder α]
variable (a b : α)

-- 1ª demostración
-- ===============

example : a < b ↔ a ≤ b ∧ a ≠ b :=
by
  rw [lt_iff_le_not_le]
  -- ⊢ a ≤ b ∧ ¬b ≤ a ↔ a ≤ b ∧ a ≠ b
  constructor
  . -- ⊢ a ≤ b ∧ ¬b ≤ a → a ≤ b ∧ a ≠ b
    rintro ⟨h1 : a ≤ b, h2 : ¬b ≤ a⟩
    -- ⊢ a ≤ b ∧ a ≠ b
    constructor
    . -- ⊢ a ≤ b
      exact h1
    . -- ⊢ a ≠ b
      rintro (h3 : a = b)
      -- ⊢ False
      have h4: b = a := h3.symm
      have h5: b ≤ a := le_of_eq h4
      show False
      exact h2 h5
  . -- ⊢ a ≤ b ∧ a ≠ b → a ≤ b ∧ ¬b ≤ a
    rintro ⟨h5 : a ≤ b , h6 : a ≠ b⟩
    -- ⊢ a ≤ b ∧ ¬b ≤ a
    constructor
    . -- ⊢ a ≤ b
      exact h5
    . -- ⊢ ¬b ≤ a
      rintro (h7 : b ≤ a)
      have h8 : a = b := le_antisymm h5 h7
      show False
      exact h6 h8

-- 2ª demostración
-- ===============

example : a < b ↔ a ≤ b ∧ a ≠ b :=
by
  rw [lt_iff_le_not_le]
  -- ⊢ a ≤ b ∧ ¬b ≤ a ↔ a ≤ b ∧ a ≠ b
  constructor
  . -- ⊢ a ≤ b ∧ ¬b ≤ a → a ≤ b ∧ a ≠ b
    rintro ⟨h1 : a ≤ b, h2 : ¬b ≤ a⟩
    -- ⊢ a ≤ b ∧ a ≠ b
    constructor
    . -- ⊢ a ≤ b
      exact h1
    . -- ⊢ a ≠ b
      rintro (h3 : a = b)
      -- ⊢ False
      exact h2 (le_of_eq h3.symm)
  . -- ⊢ a ≤ b ∧ a ≠ b → a ≤ b ∧ ¬b ≤ a
    rintro ⟨h4 : a ≤ b , h5 : a ≠ b⟩
    -- ⊢ a ≤ b ∧ ¬b ≤ a
    constructor
    . -- ⊢ a ≤ b
      exact h4
    . -- ⊢ ¬b ≤ a
      rintro (h6 : b ≤ a)
      exact h5 (le_antisymm h4 h6)

-- 3ª demostración
-- ===============

example : a < b ↔ a ≤ b ∧ a ≠ b :=
by
  rw [lt_iff_le_not_le]
  -- ⊢ a ≤ b ∧ ¬b ≤ a ↔ a ≤ b ∧ a ≠ b
  constructor
  . -- ⊢ a ≤ b ∧ ¬b ≤ a → a ≤ b ∧ a ≠ b
    rintro ⟨h1 : a ≤ b, h2 : ¬b ≤ a⟩
    -- ⊢ a ≤ b ∧ a ≠ b
    constructor
    . -- ⊢ a ≤ b
      exact h1
    . -- ⊢ a ≠ b
      exact fun h3 ↦ h2 (le_of_eq h3.symm)
  . -- ⊢ a ≤ b ∧ a ≠ b → a ≤ b ∧ ¬b ≤ a
    rintro ⟨h4 : a ≤ b , h5 : a ≠ b⟩
    -- ⊢ a ≤ b ∧ ¬b ≤ a
    constructor
    . -- ⊢ a ≤ b
      exact h4
    . -- ⊢ ¬b ≤ a
      exact fun h6 ↦ h5 (le_antisymm h4 h6)

-- 4ª demostración
-- ===============

example : a < b ↔ a ≤ b ∧ a ≠ b :=
by
  rw [lt_iff_le_not_le]
  -- ⊢ a ≤ b ∧ ¬b ≤ a ↔ a ≤ b ∧ a ≠ b
  constructor
  . -- ⊢ a ≤ b ∧ ¬b ≤ a → a ≤ b ∧ a ≠ b
    rintro ⟨h1 : a ≤ b, h2 : ¬b ≤ a⟩
    -- ⊢ a ≤ b ∧ a ≠ b
    exact ⟨h1, fun h3 ↦ h2 (le_of_eq h3.symm)⟩
  . -- ⊢ a ≤ b ∧ a ≠ b → a ≤ b ∧ ¬b ≤ a
    rintro ⟨h4 : a ≤ b , h5 : a ≠ b⟩
    -- ⊢ a ≤ b ∧ ¬b ≤ a
    exact ⟨h4, fun h6 ↦ h5 (le_antisymm h4 h6)⟩

-- 5ª demostración
-- ===============

example : a < b ↔ a ≤ b ∧ a ≠ b :=
by
  rw [lt_iff_le_not_le]
  -- ⊢ a ≤ b ∧ ¬b ≤ a ↔ a ≤ b ∧ a ≠ b
  constructor
  . -- ⊢ a ≤ b ∧ ¬b ≤ a → a ≤ b ∧ a ≠ b
    exact fun ⟨h1, h2⟩ ↦ ⟨h1, fun h3 ↦ h2 (le_of_eq h3.symm)⟩
  . -- ⊢ a ≤ b ∧ a ≠ b → a ≤ b ∧ ¬b ≤ a
    exact fun ⟨h4, h5⟩ ↦ ⟨h4, fun h6 ↦ h5 (le_antisymm h4 h6)⟩

-- 6ª demostración
-- ===============

example : a < b ↔ a ≤ b ∧ a ≠ b :=
by
  rw [lt_iff_le_not_le]
  -- ⊢ a ≤ b ∧ ¬b ≤ a ↔ a ≤ b ∧ a ≠ b
  exact ⟨fun ⟨h1, h2⟩ ↦ ⟨h1, fun h3 ↦ h2 (le_of_eq h3.symm)⟩,
         fun ⟨h4, h5⟩ ↦ ⟨h4, fun h6 ↦ h5 (le_antisymm h4 h6)⟩⟩

-- 7ª demostración
-- ===============

example : a < b ↔ a ≤ b ∧ a ≠ b :=
by
  constructor
  . -- ⊢ a < b → a ≤ b ∧ a ≠ b
    intro h
    -- h : a < b
    -- ⊢ a ≤ b ∧ a ≠ b
    constructor
    . -- ⊢ a ≤ b
      exact le_of_lt h
    . -- ⊢ a ≠ b
      exact ne_of_lt h
  . -- ⊢ a ≤ b ∧ a ≠ b → a < b
    rintro ⟨h1, h2⟩
    -- h1 : a ≤ b
    -- h2 : a ≠ b
    -- ⊢ a < b
    exact lt_of_le_of_ne h1 h2

-- 8ª demostración
-- ===============

example : a < b ↔ a ≤ b ∧ a ≠ b :=
  ⟨fun h ↦ ⟨le_of_lt h, ne_of_lt h⟩,
   fun ⟨h1, h2⟩ ↦ lt_of_le_of_ne h1 h2⟩

-- 9ª demostración
-- ===============

example : a < b ↔ a ≤ b ∧ a ≠ b :=
  lt_iff_le_and_ne

-- Lemas usados
-- ============

-- #check (le_antisymm : a ≤ b → b ≤ a → a = b)
-- #check (le_of_eq : a = b → a ≤ b)
-- #check (lt_iff_le_and_ne : a < b ↔ a ≤ b ∧ a ≠ b)
-- #check (lt_iff_le_not_le : a < b ↔ a ≤ b ∧ ¬b ≤ a)
-- #check (lt_of_le_of_ne : a ≤ b → a ≠ b → a < b)
\end{verbatim}
Se puede interactuar con las pruebas anteriores en \href{https://lean.math.hhu.de/\#url=https://raw.githubusercontent.com/jaalonso/Calculemus2/main/src/Caracterizacion\_de\_menor\_en\_ordenes\_parciales.lean}{Lean 4 Web}

\section{Si ≤ es un preorden, entonces < es irreflexiva}
\label{sec:org74c0b0d}
\begin{verbatim}
-- ---------------------------------------------------------------------
-- Demostrar que si ≤ es un preorden, entonces < es irreflexiva.
-- ----------------------------------------------------------------------

-- Demostración en lenguaje natural
-- ================================

-- Se usará la siguiente propiedad de lo preórdenes
--    (∀ a, b)[a < b ↔ a ≤ b ∧ b ≰ a]
-- Con dicha propiedad, lo que tenemos que demostrar se transforma en
--    ¬(a ≤ a ∧ a ≰ a)
-- Para demostrarla, supongamos que
--    a ≤ a ∧ a ≰ a
-- lo que es una contradicción.

-- Demostraciones con Lean4
-- ========================

import Mathlib.Tactic
variable {α : Type _} [Preorder α]
variable (a : α)

-- 1ª demostración
-- ===============

example : ¬a < a :=
by
  rw [lt_iff_le_not_le]
  -- ⊢ ¬(a ≤ a ∧ ¬a ≤ a)
  rintro ⟨h1, h2⟩
  -- h1 : a ≤ a
  -- h2 : ¬a ≤ a
  -- ⊢ False
  exact h2 h1

-- 2ª demostración
-- ===============

example : ¬a < a :=
  irrefl a

-- Lemas usados
-- ============

-- variable (b : α)
-- #check (lt_iff_le_not_le : a < b ↔ a ≤ b ∧ ¬b ≤ a)
-- #check (irrefl a : ¬a < a)
\end{verbatim}
Se puede interactuar con las pruebas anteriores en \href{https://lean.math.hhu.de/\#url=https://raw.githubusercontent.com/jaalonso/Calculemus2/main/src/Preorden\_es\_irreflexivo.lean}{Lean 4 Web}

\section{Si ≤ es un preorden, entonces < es transitiva}
\label{sec:orge33c2a0}
\begin{verbatim}
-- ---------------------------------------------------------------------
-- Demostrar que si ≤ es un preorden, entonces < es transitiva.
-- ----------------------------------------------------------------------

-- Demostración en lenguaje natural
-- ================================

-- Se usará la siguiente propiedad de los preórdenes
--    (∀ a, b)[a < b ↔ a ≤ b ∧ b ≰ a]
-- Con dicha propiedad, lo que tenemos que demostrar se transforma en
--    a ≤ b ∧ b ≰ a → b ≤ c ∧ c ≰ b → a ≤ c ∧ c ≰ a
-- Para demostrarla, supongamos que
--    a ≤ b                                                          (1)
--    b ≰ a                                                          (2)
--    b ≤ c                                                          (3)
--    c ≰ b                                                          (4)
-- y tenemos que demostrar las siguientes relaciones
--    a ≤ c                                                          (5)
--    c ≰ a                                                          (6)
--
-- La (5) se tiene aplicando la propiedad transitiva a (1) y (3).
--
-- Para demostrar la (6), supongamos que
--    c ≤ a                                                          (7)
-- entonces, junto a la (1), por la propieda transitiva se tiene
--    c ≤ b
-- que es una contradicción con la (4).

-- Demostraciones con Lean4
-- ========================

import Mathlib.Tactic
variable {α : Type _} [Preorder α]
variable (a b c : α)

-- 1ª demostración
-- ===============

example : a < b → b < c → a < c :=
by
  simp only [lt_iff_le_not_le]
  -- ⊢ a ≤ b ∧ ¬b ≤ a → b ≤ c ∧ ¬c ≤ b → a ≤ c ∧ ¬c ≤ a
  rintro ⟨h1 : a ≤ b, _h2 : ¬b ≤ a⟩ ⟨h3 : b ≤ c, h4 : ¬c ≤ b⟩
  -- ⊢ a ≤ c ∧ ¬c ≤ a
  constructor
  . -- ⊢ a ≤ c
    exact le_trans h1 h3
  . -- ⊢ ¬c ≤ a
    contrapose! h4
    -- h4 : c ≤ a
    -- ⊢ c ≤ b
    exact le_trans h4 h1

-- 2ª demostración
-- ===============

example : a < b → b < c → a < c :=
by
  simp only [lt_iff_le_not_le]
  -- ⊢ a ≤ b ∧ ¬b ≤ a → b ≤ c ∧ ¬c ≤ b → a ≤ c ∧ ¬c ≤ a
  rintro ⟨h1 : a ≤ b, _h2 : ¬b ≤ a⟩ ⟨h3 : b ≤ c, h4 : ¬c ≤ b⟩
  -- ⊢ a ≤ c ∧ ¬c ≤ a
  constructor
  . -- ⊢ a ≤ c
    exact le_trans h1 h3
  . -- ⊢ ¬c ≤ a
    rintro (h5 : c ≤ a)
    -- ⊢ False
    have h6 : c ≤ b := le_trans h5 h1
    show False
    exact h4 h6

-- 3ª demostración
-- ===============

example : a < b → b < c → a < c :=
by
  simp only [lt_iff_le_not_le]
  -- ⊢ a ≤ b ∧ ¬b ≤ a → b ≤ c ∧ ¬c ≤ b → a ≤ c ∧ ¬c ≤ a
  rintro ⟨h1 : a ≤ b, _h2 : ¬b ≤ a⟩ ⟨h3 : b ≤ c, h4 : ¬c ≤ b⟩
  -- ⊢ a ≤ c ∧ ¬c ≤ a
  constructor
  . -- ⊢ a ≤ c
    exact le_trans h1 h3
  . -- ⊢ ¬c ≤ a
    exact fun h5 ↦ h4 (le_trans h5 h1)

-- 4ª demostración
-- ===============

example : a < b → b < c → a < c :=
by
  simp only [lt_iff_le_not_le]
  -- ⊢ a ≤ b ∧ ¬b ≤ a → b ≤ c ∧ ¬c ≤ b → a ≤ c ∧ ¬c ≤ a
  rintro ⟨h1 : a ≤ b, _h2 : ¬b ≤ a⟩ ⟨h3 : b ≤ c, h4 : ¬c ≤ b⟩
  -- ⊢ a ≤ c ∧ ¬c ≤ a
  exact ⟨le_trans h1 h3, fun h5 ↦ h4 (le_trans h5 h1)⟩

-- 5ª demostración
-- ===============

example : a < b → b < c → a < c :=
by
  simp only [lt_iff_le_not_le]
  -- ⊢ a ≤ b ∧ ¬b ≤ a → b ≤ c ∧ ¬c ≤ b → a ≤ c ∧ ¬c ≤ a
  exact fun ⟨h1, _h2⟩ ⟨h3, h4⟩ ↦ ⟨le_trans h1 h3,
                                  fun h5 ↦ h4 (le_trans h5 h1)⟩

-- 6ª demostración
-- ===============

example : a < b → b < c → a < c :=
  lt_trans

-- Lemas usados
-- ============

-- #check (lt_iff_le_not_le : a < b ↔ a ≤ b ∧ ¬b ≤ a)
-- #check (le_trans : a ≤ b → b ≤ c → a ≤ c)
-- #check (lt_trans : a < b → b < c → a < c)
\end{verbatim}
Se puede interactuar con las pruebas anteriores en \href{https://lean.math.hhu.de/\#url=https://raw.githubusercontent.com/jaalonso/Calculemus2/main/src/Preorden\_transitiva.lean}{Lean 4 Web}

\chapter{Anillos ordenados}
\label{sec:org1736a62}

\section{En los anillos ordenados, a ≤ b → 0 ≤ b - a}
\label{sec:org2613e4d}
\begin{verbatim}
-- ---------------------------------------------------------------------
-- Demostrar que en los anillos ordenados se verifica que
--    a ≤ b → 0 ≤ b - a
-- ----------------------------------------------------------------------

-- Demostración en lenguaje natural
-- ================================

-- Se usarán los siguientes lemas:
--    sub_self         : a - a = 0
--    sub_le_sub_right : a ≤ b → ∀ (c : R), a - c ≤ b - c
--
-- Supongamos que
--    a ≤ b                                                          (1)
-- La demostración se tiene por la siguiente cadena de desigualdades:
--    0 = a - a    [por sub_self]
--      ≤ b - a    [por (1) y sub_le_sub_right]

-- Demostraciones con Lean4
-- ========================

import Mathlib.Algebra.Order.Ring.Defs
variable {R : Type _} [StrictOrderedRing R]
variable (a b c : R)

-- 1ª demostración
example : a ≤ b → 0 ≤ b - a :=
by
  intro h
  calc
    0 = a - a := (sub_self a).symm
    _ ≤ b - a := sub_le_sub_right h a

-- 2ª demostración
example : a ≤ b → 0 ≤ b - a :=
sub_nonneg.mpr

-- 3ª demostración
example : a ≤ b → 0 ≤ b - a :=
by simp

-- Lemas usados
-- ============

-- #check (sub_le_sub_right : a ≤ b → ∀ (c : R), a - c ≤ b - c)
-- #check (sub_nonneg : 0 ≤ a - b ↔ b ≤ a)
-- #check (sub_self a : a - a = 0)
\end{verbatim}
Se puede interactuar con las pruebas anteriores en \href{https://lean.math.hhu.de/\#url=https://raw.githubusercontent.com/jaalonso/Calculemus2/main/src/Ejercicio\_sobre\_anillos\_ordenados\_1.lean}{Lean 4 Web}

\section{En los anillos ordenados, 0 ≤ b - a → a ≤ b}
\label{sec:org088b19e}
\begin{verbatim}
-- ---------------------------------------------------------------------
-- Demostrar que en los anillos ordenados
--    0 ≤ b - a → a ≤ b
-- ----------------------------------------------------------------------

-- Demostración en lenguaje natural
-- ================================

-- Se usarán los siguientes lemas:
--    zero_add a : 0 + a = a
--    add_le_add_right : b ≤ c → ∀ (a : R),  b + a ≤ c + a
--    sub_add_cancel a b : a - b + b = -a
-- Supongamos que
--    0 ≤ b - a                                                      (1)
-- La demostración se tiene por la siguiente cadena de desigualdades:
--    a = 0 + a          [por zero_add]
--      ≤ (b - a) + a    [por (1) y add_le_add_right]
--      = b              [por sub_add_cancel]

-- Demostraciones con Lean4
-- ========================

import Mathlib.Algebra.Order.Ring.Defs
variable {R : Type _} [StrictOrderedRing R]
variable (a b c : R)

-- 1ª demostración
-- ===============

example : 0 ≤ b - a → a ≤ b :=
by
  intro h
  calc
    a = 0 + a       := (zero_add a).symm
    _ ≤ (b - a) + a := add_le_add_right h a
    _ = b           := sub_add_cancel b a

-- 2ª demostración
-- ===============

example : 0 ≤ b - a → a ≤ b :=
-- by apply?
sub_nonneg.mp

-- 3ª demostración
-- ===============

example : 0 ≤ b - a → a ≤ b :=
by simp

-- Lemas usados
-- ============

-- #check (zero_add a : 0 + a = a)
-- #check (add_le_add_right : b ≤ c → ∀ (a : R),  b + a ≤ c + a)
-- #check (sub_add_cancel a b : a - b + b = a)
-- #check (sub_nonneg : 0 ≤ a - b ↔ b ≤ a)
\end{verbatim}
Se puede interactuar con las pruebas anteriores en \href{https://lean.math.hhu.de/\#url=https://raw.githubusercontent.com/jaalonso/Calculemus2/main/src/Ejercicio\_sobre\_anillos\_ordenados\_2.lean}{Lean 4 Web}

\section{En los anillos ordenados, \{a ≤ b, 0 ≤ c\} ⊢ ac ≤ bc}
\label{sec:orge466e6a}
\begin{verbatim}
-- ---------------------------------------------------------------------
-- Demostrar que, en los anillos ordenados, si
--    a ≤ b
--    0 ≤ c
-- entonces
--    a * c ≤ b * c
-- ----------------------------------------------------------------------

-- Demostración en lenguaje natural
-- ================================

-- Se usarán los siguientes lemas:
--    sub_nonneg                 : 0 ≤ a - b ↔ b ≤ a)
--    mul_nonneg                 : 0 ≤ a → 0 ≤ b → 0 ≤ a * b)
--    sub_mul a b c              : (a - b) * c = a * c - b * c)
--
-- Supongamos que
--    a ≤ b                                                          (1)
--    0 ≤ c
-- De (1), por sub_nonneg, se tiene
--    0 ≤ b - a
-- y con (2), por mul_nonneg, se tiene
--    0 ≤ (b - a) * c
-- que, por sub_mul, da
--    0 ≤ b * c - a * c
-- y, aplicándole sub_nonneg, se tiene
--    a * c ≤ b * c

-- Demostraciones con Lean4
-- ========================

import Mathlib.Algebra.Order.Ring.Defs
variable {R : Type _} [StrictOrderedRing R]
variable (a b c : R)

-- 1ª demostración
-- ===============

example
  (h1 : a ≤ b)
  (h2 : 0 ≤ c)
  : a * c ≤ b * c :=
by
  have h3 : 0 ≤ b - a :=
    sub_nonneg.mpr h1
  have h4 : 0 ≤ b * c - a * c := calc
    0 ≤ (b - a) * c   := mul_nonneg h3 h2
    _ = b * c - a * c := sub_mul b a c
  show a * c ≤ b * c
  exact sub_nonneg.mp h4

-- 2ª demostración
-- ===============

example
  (h1 : a ≤ b)
  (h2 : 0 ≤ c)
  : a * c ≤ b * c :=
by
  have h3 : 0 ≤ b - a := sub_nonneg.mpr h1
  have h4 : 0 ≤ (b - a) * c := mul_nonneg h3 h2
  -- h4 : 0 ≤ b * c - a * c
  rw [sub_mul] at h4
  -- a * c ≤ b * c
  exact sub_nonneg.mp h4

-- 3ª demostración
-- ===============

example
  (h1 : a ≤ b)
  (h2 : 0 ≤ c)
  : a * c ≤ b * c :=
by
  -- 0 ≤ b * c - a * c
  apply sub_nonneg.mp
  -- 0 ≤ (b - a) * c
  rw [← sub_mul]
  apply mul_nonneg
  . -- 0 ≤ b - a
    exact sub_nonneg.mpr h1
  . -- 0 ≤ c
    exact h2

-- 4ª demostración
-- ===============

example
  (h1 : a ≤ b)
  (h2 : 0 ≤ c)
  : a * c ≤ b * c :=
by
  apply sub_nonneg.mp
  rw [← sub_mul]
  apply mul_nonneg (sub_nonneg.mpr h1) h2

-- 5ª demostración
example
  (h1 : a ≤ b)
  (h2 : 0 ≤ c)
  : a * c ≤ b * c :=
-- by apply?
mul_le_mul_of_nonneg_right h1 h2

-- Lemas usados
-- ============

-- #check (mul_le_mul_of_nonneg_right : a ≤ b → 0 ≤ c → a * c ≤ b * c)
-- #check (mul_nonneg : 0 ≤ a → 0 ≤ b → 0 ≤ a * b)
-- #check (sub_mul a b c : (a - b) * c = a * c - b * c)
-- #check (sub_nonneg : 0 ≤ a - b ↔ b ≤ a)
\end{verbatim}
Se puede interactuar con las pruebas anteriores en \href{https://lean.math.hhu.de/\#url=https://raw.githubusercontent.com/jaalonso/Calculemus2/main/src/Ejercicio\_sobre\_anillos\_ordenados\_3.lean}{Lean 4 Web}

\chapter{Espacios métricos}
\label{sec:org192ad7a}

\section{En los espacios métricos, dist(x,y) ≥ 0}
\label{sec:org1aec468}
\begin{verbatim}
-- ---------------------------------------------------------------------
-- Ejercicio. Demostrar que en los espacios métricos
--    0 ≤ dist x y
-- ----------------------------------------------------------------------

-- Demostración en lenguaje natural
-- ================================

-- Se usarán los siguientes lemas:
--    dist_comm x y             : dist x y = dist y x
--    dist_self x               : dist x x = 0
--    dist_triangle x y z       : dist x z ≤ dist x y + dist y z
--    mul_two a                 : a * 2 = a + a
--    nonneg_of_mul_nonneg_left : 0 ≤ a * b → 0 < b → 0 ≤ a
--    zero_lt_two               : 0 < 2
--
-- Por nonneg_of_mul_nonneg_left es suficiente demostrar las siguientes
-- desigualdades:
--    0 ≤ dist x y * 2                                               (1)
--    0 < 2                                                          (2)
--
-- La (1) se demuestra por las siguiente cadena de desigualdades:
--    0 = dist x x               [por dist_self]
--      ≤ dist x y + dist y x    [por dist_triangle]
--      = dist x y + dist x y    [por dist_comm]
--      = dist x y * 2           [por mul_two]
--
-- La (2) se tiene por zero_lt_two.

-- Demostraciones con Lean4
-- ========================

import Mathlib.Topology.MetricSpace.Basic
variable {X : Type _} [MetricSpace X]
variable (x y : X)

-- 1ª demostración
example : 0 ≤ dist x y :=
by
  have h1 : 0 ≤ dist x y * 2 := calc
    0 = dist x x            := (dist_self x).symm
    _ ≤ dist x y + dist y x := dist_triangle x y x
    _ = dist x y + dist x y := by rw [dist_comm x y]
    _ = dist x y * 2        := (mul_two (dist x y)).symm
  show 0 ≤ dist x y
  exact nonneg_of_mul_nonneg_left h1 zero_lt_two

-- 2ª demostración
example : 0 ≤ dist x y :=
by
  apply nonneg_of_mul_nonneg_left
  . -- 0 ≤ dist x y * 2
    calc 0 = dist x x            := by simp only [dist_self]
         _ ≤ dist x y + dist y x := by simp only [dist_triangle]
         _ = dist x y + dist x y := by simp only [dist_comm]
         _ = dist x y * 2        := by simp only [mul_two]
  . -- 0 < 2
    exact zero_lt_two

-- 3ª demostración
example : 0 ≤ dist x y :=
by
  have : 0 ≤ dist x y + dist y x := by
    rw [← dist_self x]
    apply dist_triangle
  linarith [dist_comm x y]

-- 3ª demostración
example : 0 ≤ dist x y :=
-- by apply?
dist_nonneg

-- Lemas usados
-- ============

-- variable (a b : ℝ)
-- variable (z : X)
-- #check (dist_comm x y : dist x y = dist y x)
-- #check (dist_nonneg : 0 ≤ dist x y)
-- #check (dist_self x : dist x x = 0)
-- #check (dist_triangle x y z : dist x z ≤ dist x y + dist y z)
-- #check (mul_two a : a * 2 = a + a)
-- #check (nonneg_of_mul_nonneg_left : 0 ≤ a * b → 0 < b → 0 ≤ a)
-- #check (zero_lt_two : 0 < 2)
\end{verbatim}
Se puede interactuar con las pruebas anteriores en \href{https://lean.math.hhu.de/\#url=https://raw.githubusercontent.com/jaalonso/Calculemus2/main/src/Ejercicio\_en\_espacios\_metricos.lean}{Lean 4 Web}

\chapter{Funciones reales}
\label{sec:org6ef8965}

\section{La suma de una cota superior de f y una cota superior de g es una cota superior de f+g}
\label{sec:org1aff40e}
\begin{verbatim}
-- ---------------------------------------------------------------------
-- Demostrar que la suma de una cota superior de f y una cota superior
-- de g es una cota superior de f + g.
-- ----------------------------------------------------------------------

-- Demostración en lenguaje natural
-- ================================

-- Se usará el siguiente lema
--    add_le_add : a ≤ b → c ≤ d → a + c ≤ b + d
--
-- Por la definición de cota superior, hay que demostrar que
--    (∀ x ∈ ℝ) [f(x) + g(x) ≤ a + b]                                  (1)
-- Para ello, sea x ∈ R. Puesto que es a es una cota superior de f, se
-- tiene que
--    f(x) ≤ a                                                         (2)
-- y, puesto que b es una cota superior de g, se tiene que
--    g(x) ≤ b                                                         (3)
-- De (2) y (3), por add_le_add, se tiene que
--    f(x) + g(x) ≤ a + b
-- que es lo que había que demostrar.

-- Demostraciones con Lean4
-- ========================

import Mathlib.Data.Real.Basic

-- (CotaSuperior f a) se verifica si a es una cota superior de f.
def CotaSuperior (f : ℝ → ℝ) (a : ℝ) : Prop :=
  ∀ x, f x ≤ a

variable {f g : ℝ → ℝ}
variable {a b : ℝ}

-- 1ª demostración
-- ===============

example
  (hfa : CotaSuperior f a)
  (hgb : CotaSuperior g b)
  : CotaSuperior (f + g) (a + b) :=
by
  have h1 : ∀ x, (f + g) x  ≤ a + b := by
  { intro x
    have h2 : f x ≤ a := hfa x
    have h3 : g x ≤ b := hgb x
    show (f + g) x ≤ a + b
    exact add_le_add h2 h3 }
  show CotaSuperior (f + g) (a + b)
  exact h1

-- 2ª demostración
-- ===============

example
  (hfa : CotaSuperior f a)
  (hgb : CotaSuperior g b)
  : CotaSuperior (f + g) (a + b) :=
by
  have h1 : ∀ x, (f + g) x ≤ a + b := by
  { intro x
    show (f + g) x ≤ a + b
    exact add_le_add (hfa x) (hgb x) }
  show CotaSuperior (f + g) (a + b)
  exact h1

-- 3ª demostración
-- ===============

example
  (hfa : CotaSuperior f a)
  (hgb : CotaSuperior g b)
  : CotaSuperior (f + g) (a + b) :=
by
  intro x
  dsimp
  apply add_le_add
  . apply hfa
  . apply hgb

-- 4ª demostración
-- ===============

theorem sumaCotaSup
  (hfa : CotaSuperior f a)
  (hgb : CotaSuperior g b)
  : CotaSuperior (f + g) (a + b) :=
λ x ↦ add_le_add (hfa x) (hgb x)

-- Lemas usados
-- ============

-- variable (c d : ℝ)
-- #check (add_le_add : a ≤ b → c ≤ d → a + c ≤ b + d)
\end{verbatim}
Se puede interactuar con las pruebas anteriores en \href{https://lean.math.hhu.de/\#url=https://raw.githubusercontent.com/jaalonso/Calculemus2/main/src/Suma\_de\_cotas\_superiores.lean}{Lean 4 Web}

\section{La suma de una cota inferior de f y una cota inferior de g es una cota inferior de f+g}
\label{sec:orgaee0df3}
\begin{verbatim}
-- ---------------------------------------------------------------------
-- Demostrar que la suma de una cota inferior de f y una  cota inferior
-- de g es una cota inferior de f + g.
-- ----------------------------------------------------------------------

-- Demostración en lenguaje natural
-- ================================

-- Se usará el siguiente lema
--    add_le_add : a ≤ b → c ≤ d → a + c ≤ b + d
--
-- Por la definición de cota inferior, hay que demostrar que
--    (∀ x ∈ ℝ) [a + b ≤ f(x) + g(x)]                                  (1)
-- Para ello, sea x ∈ R. Puesto que es a es una cota inferior de f, se
-- tiene que
--    a ≤ f(x)                                                         (2)
-- y, puesto que b es una cota inferior de g, se tiene que
--    b ≤ g(x)                                                         (3)
-- De (2) y (3), por add_le_add, se tiene que
--    a + b ≤ f(x) + g(x)
-- que es lo que había que demostrar.

-- Demostraciones con Lean4
-- ========================

import Mathlib.Data.Real.Basic

-- (CotaInferior f a) expresa que a es una cota inferior de f.
def CotaInferior (f : ℝ → ℝ) (a : ℝ) : Prop :=
  ∀ x, a ≤ f x

variable {f g : ℝ → ℝ}
variable {a b : ℝ}

-- 1ª demostración
example
  (hfa : CotaInferior f a)
  (hgb : CotaInferior g b)
  : CotaInferior (f + g) (a + b) :=
by
  have h1 : ∀ x, a + b ≤ f x + g x
  { intro x
    have h1a : a ≤ f x := hfa x
    have h1b : b ≤ g x := hgb x
    show a + b ≤ f x + g x
    exact add_le_add h1a h1b }
  show CotaInferior (f + g) (a + b)
  exact h1

-- 2ª demostración
example
  (hfa : CotaInferior f a)
  (hgb : CotaInferior g b)
  : CotaInferior (f + g) (a + b) :=
by
  have h1 : ∀ x, a + b ≤ f x + g x
  { intro x
    show a + b ≤ f x + g x
    exact add_le_add (hfa x) (hgb x) }
  show CotaInferior (f + g) (a + b)
  exact h1

-- 3ª demostración
example
  (hfa : CotaInferior f a)
  (hgb : CotaInferior g b)
  : CotaInferior (f + g) (a + b) :=
by
  intro x
  dsimp
  apply add_le_add
  . apply hfa
  . apply hgb

-- 4ª demostración
theorem sumaCotaInf
  (hfa : CotaInferior f a)
  (hgb : CotaInferior g b)
  : CotaInferior (f + g) (a + b) :=
λ x ↦ add_le_add (hfa x) (hgb x)

-- Lemas usados
-- ============

-- variable (c d : ℝ)
-- #check (add_le_add : a ≤ b → c ≤ d → a + c ≤ b + d)
\end{verbatim}
Se puede interactuar con las pruebas anteriores en \href{https://lean.math.hhu.de/\#url=https://raw.githubusercontent.com/jaalonso/Calculemus2/main/src/Suma\_de\_cotas\_inferiores.lean}{Lean 4 Web}

\section{El producto de funciones no negativas es no negativo}
\label{sec:orgda5a6c5}
\begin{verbatim}
-- ---------------------------------------------------------------------
-- Demostrar que el producto de dos funciones no negativas es no
-- negativa.
-- ----------------------------------------------------------------------

-- Demostración en lenguaje natural
-- ================================

-- Se usará el siguiente lema
--    mul_nonneg : 0 ≤ a → 0 ≤ b → 0 ≤ a * b
--
-- Hay que demostrar que
--    (∀ x ∈ ℝ) [0 ≤ f(x) * g(x)]                                      (1)
-- Para ello, sea x ∈ R. Puesto que f es no negatica, se tiene que
--    0 ≤ f(x)                                                         (2)
-- y, puesto que g es no negativa, se tiene que
--    0 ≤ g(x)                                                         (3)
-- De (2) y (3), por mul_nonneg, se tiene que
--    0 ≤ f(x) * g(x)
-- que es lo que había que demostrar.

-- Demostraciones con Lean4
-- ========================

import Mathlib.Data.Real.Basic

-- (CotaInferior f a) expresa que a es una cota inferior de f.
def CotaInferior (f : ℝ → ℝ) (a : ℝ) : Prop :=
  ∀ x, a ≤ f x

variable (f g : ℝ → ℝ)

-- 1ª demostración
example
  (nnf : CotaInferior f 0)
  (nng : CotaInferior g 0)
  : CotaInferior (f * g) 0 :=
by
  have h1 : ∀x, 0 ≤ f x * g x
  { intro x
    have h2: 0 ≤ f x := nnf x
    have h3: 0 ≤ g x := nng x
    show 0 ≤ f x * g x
    exact mul_nonneg h2 h3 }
  show CotaInferior (f * g) 0
  exact h1

-- 2ª demostración
example
  (nnf : CotaInferior f 0)
  (nng : CotaInferior g 0)
  : CotaInferior (f * g) 0 :=
by
  have h1 : ∀x, 0 ≤ f x * g x
  { intro x
    show 0 ≤ f x * g x
    exact mul_nonneg (nnf x) (nng x) }
  show CotaInferior (f * g) 0
  exact h1

-- 3ª demostración
example
  (nnf : CotaInferior f 0)
  (nng : CotaInferior g 0)
  : CotaInferior (f * g) 0 :=
by
  intro x
  dsimp
  apply mul_nonneg
  . apply nnf
  . apply nng

-- 4ª demostración
example
  (nnf : CotaInferior f 0)
  (nng : CotaInferior g 0)
  : CotaInferior (f * g) 0 :=
λ x ↦ mul_nonneg (nnf x) (nng x)

-- Lemas usados
-- ============

-- variable (a b : ℝ)
-- #check (mul_nonneg : 0 ≤ a → 0 ≤ b → 0 ≤ a * b)
\end{verbatim}
Se puede interactuar con las pruebas anteriores en \href{https://lean.math.hhu.de/\#url=https://raw.githubusercontent.com/jaalonso/Calculemus2/main/src/Producto\_de\_funciones\_no\_negativas.lean}{Lean 4 Web}

\section{Si a es una cota superior no negativa de f y b es es una cota superior de la función no negativa g, entonces ab es una cota superior de fg}
\label{sec:orgaa83da1}
\begin{verbatim}
-- ---------------------------------------------------------------------

-- ---------------------------------------------------------------------
-- Demostrar que si a es una cota superior de f, b es una cota superior
-- de g, a es no negativa y g es no negativa, entonces ab es una cota
-- superior de fg.
-- ----------------------------------------------------------------------

-- Demostración en lenguaje natural
-- ================================

-- Se usará el siguiente lema
--    mul_le_mul : a ≤ b → c ≤ d → 0 ≤ c → 0 ≤ b → a * c ≤ b * d
--
-- Hay que demostrar que
--    (∀ x ∈ ℝ) [f x * g x ≤ a * b]                                    (1)
-- Para ello, sea x ∈ R. Puesto que a es una cota superior de f, se tiene que
--    f(x) ≤ a                                                         (2)
-- puesto que b es una cota superior de g, se tiene que
--    g(x) ≤ b                                                         (3)
-- puesto que g es no negativa, se tiene que
--    0 ≤ g(x)                                                         (4)
-- y, puesto que a es no negativa, se tiene que
--    0 ≤ a                                                            (5)
-- De (2), (3), (4) y (5), por mul_le_mul, se tiene que
--    f x * g x ≤ a * b
-- que es lo que había que demostrar.

-- Demostraciones con Lean4
-- ========================

import Mathlib.Data.Real.Basic

-- (CotaSuperior f a) se verifica si a es una cota superior de f.
def CotaSuperior (f : ℝ → ℝ) (a : ℝ) : Prop :=
  ∀ x, f x ≤ a

-- (CotaInferior f a) expresa que a es una cota inferior de f.
def CotaInferior (f : ℝ → ℝ) (a : ℝ) : Prop :=
  ∀ x, a ≤ f x

variable (f g : ℝ → ℝ)
variable (a b : ℝ)

-- 1ª demostración
example
  (hfa : CotaSuperior f a)
  (hgb : CotaSuperior g b)
  (nng : CotaInferior g 0)
  (nna : 0 ≤ a)
  : CotaSuperior (f * g) (a * b) :=
by
  have h1 : ∀ x, f x * g x ≤ a * b
  { intro x
    have h2 : f x ≤ a := hfa x
    have h3 : g x ≤ b := hgb x
    have h4 : 0 ≤ g x := nng x
    show f x * g x ≤ a * b
    exact mul_le_mul h2 h3 h4 nna }
  show CotaSuperior (f * g) (a * b)
  exact h1

-- 2ª demostración
example
  (hfa : CotaSuperior f a)
  (hgb : CotaSuperior g b)
  (nng : CotaInferior g 0)
  (nna : 0 ≤ a)
  : CotaSuperior (f * g) (a * b) :=
by
  intro x
  dsimp
  apply mul_le_mul
  . apply hfa
  . apply hgb
  . apply nng
  . apply nna

-- 3ª demostración
example
  (hfa : CotaSuperior f a)
  (hgb : CotaSuperior g b)
  (nng : CotaInferior g 0)
  (nna : 0 ≤ a)
  : CotaSuperior (f * g) (a * b) :=
by
  intro x
  have h1:= hfa x
  have h2:= hgb x
  have h3:= nng x
  exact mul_le_mul h1 h2 h3 nna

-- 4ª demostración
example
  (hfa : CotaSuperior f a)
  (hgb : CotaSuperior g b)
  (nng : CotaInferior g 0)
  (nna : 0 ≤ a)
  : CotaSuperior (f * g) (a * b) :=
by
  intro x
  specialize hfa x
  specialize hgb x
  specialize nng x
  exact mul_le_mul hfa hgb nng nna

-- 5ª demostración
example
  (hfa : CotaSuperior f a)
  (hgb : CotaSuperior g b)
  (nng : CotaInferior g 0)
  (nna : 0 ≤ a)
  : CotaSuperior (f * g) (a * b) :=
λ x ↦ mul_le_mul (hfa x) (hgb x) (nng x) nna

-- Lemas usados
-- ============

-- variable (c d : ℝ)
-- #check (mul_le_mul : a ≤ b → c ≤ d → 0 ≤ c → 0 ≤ b → a * c ≤ b * d)
\end{verbatim}
Se puede interactuar con las pruebas anteriores en \href{https://lean.math.hhu.de/\#url=https://raw.githubusercontent.com/jaalonso/Calculemus2/main/src/Cota\_superior\_del\_producto.lean}{Lean 4 Web}

\section{La suma de dos funciones acotadas superiormente también lo está}
\label{sec:org53a0b6c}
\begin{verbatim}
-- ---------------------------------------------------------------------
-- Demostrar que la suma de dos funciones acotadas superiormente también
-- lo está.
-- ----------------------------------------------------------------------

-- Demostración en lenguaje natural
-- ================================

-- Del ejercicio "La suma de una cota superior de f y una cota superior
-- de g es una cota superior de f+g" (que se encuentra en
-- https://bit.ly/3QauluK ) usaremos la definición de cota superior
-- (CotaSuperior) y el lema sumaCotaSup.
--
-- Puesto que f está acotada superiormente, tiene una cota superior. Sea
-- a una de dichas cotas. Análogamentte, puesto que g está acotada
-- superiormente, tiene una cota superior. Sea b una de dichas
-- cotas. Por el lema sumaCotaSup, a+b es una cota superior de f+g. or
-- consiguiente, f+g está acotada superiormente.

-- Demostraciones con Lean4
-- ========================

import src.Suma_de_cotas_superiores

variable {f g : ℝ → ℝ}

-- (acotadaSup f) afirma que f tiene cota superior.
def acotadaSup (f : ℝ → ℝ) :=
  ∃ a, CotaSuperior f a

-- 1ª demostración
example
  (hf : acotadaSup f)
  (hg : acotadaSup g)
  : acotadaSup (f + g) :=
by
  cases' hf with a ha
  -- a : ℝ
  -- ha : CotaSuperior f a
  cases' hg with b hb
  -- b : ℝ
  -- hb : CotaSuperior g b
  have h1 : CotaSuperior (f + g) (a + b) :=
    sumaCotaSup ha hb
  have h2 : ∃ z, CotaSuperior (f+g) z :=
    Exists.intro (a + b) h1
  show acotadaSup (f + g)
  exact h2

-- 2ª demostración
example
  (hf : acotadaSup f)
  (hg : acotadaSup g)
  : acotadaSup (f + g) :=
by
  cases' hf with a ha
  -- a : ℝ
  -- ha : FnUb f a
  cases' hg with b hb
  -- b : ℝ
  -- hb : FnUb g b
  use a + b
  apply sumaCotaSup ha hb

-- 4ª demostración
example
  (hf : acotadaSup f)
  (hg : acotadaSup g)
  : acotadaSup (f + g) :=
by
  rcases hf with ⟨a, ha⟩
  rcases hg with ⟨b, hb⟩
  exact ⟨a + b, sumaCotaSup ha hb⟩

-- 5ª demostración
example :
  acotadaSup f → acotadaSup g → acotadaSup (f + g) :=
by
  rintro ⟨a, ha⟩ ⟨b, hb⟩
  exact ⟨a + b, sumaCotaSup ha hb⟩

-- 6ª demostración
example :
  acotadaSup f → acotadaSup g → acotadaSup (f + g) :=
fun ⟨a, ha⟩ ⟨b, hb⟩ ↦ ⟨a + b, sumaCotaSup ha hb⟩

-- Lemas usados
-- ============

-- #check (sumaCotaSup : CotaSuperior f a → CotaSuperior g b → CotaSuperior (f + g) (a + b))
\end{verbatim}
Se puede interactuar con las pruebas anteriores en \href{https://lean.math.hhu.de/\#url=https://raw.githubusercontent.com/jaalonso/Calculemus2/main/src//Suma\_de\_funciones\_acotadas\_superiormente.lean}{Lean 4 Web}

\section{La suma de dos funciones acotadas inferiormente también lo está}
\label{sec:org839f110}
\begin{verbatim}
-- ---------------------------------------------------------------------
-- Demostrar que la suma de dos funciones acotadas inferiormente también
-- lo está.
-- ----------------------------------------------------------------------

-- Demostración en lenguaje natural
-- ================================

-- Del ejercicio "La suma de una cota inferior de f y una cota inferior
-- de g es una cota inferior de f+g" usaremos la definición de cota
-- inferior (CotaInferior) y el lema sumaCotaInf.
--
-- Puesto que f está acotada inferiormente, tiene una cota inferior. Sea
-- a una de dichas cotas. Análogamentte, puesto que g está acotada
-- inferiormente, tiene una cota inferior. Sea b una de dichas
-- cotas. Por el lema FnLb_add, a+b es una cota inferior de f+g. Por
-- consiguiente, f+g está acotada inferiormente.

-- Demostraciones con Lean4
-- ========================

import src.Suma_de_cotas_inferiores
variable {f g : ℝ → ℝ}

-- (acotadaInf f) afirma que f tiene cota inferior.
def acotadaInf (f : ℝ → ℝ) :=
  ∃ a, CotaInferior f a

-- 1ª demostración
example
  (hf : acotadaInf f)
  (hg : acotadaInf g)
  : acotadaInf (f + g) :=
by
  cases' hf with a ha
  -- a : ℝ
  -- ha : CotaInferior f a
  cases' hg with b hb
  -- b : ℝ
  -- hb : CotaInferior g b
  have h1 : CotaInferior (f + g) (a + b) := sumaCotaInf ha hb
  have h2 : ∃ z, CotaInferior (f + g) z :=
    Exists.intro (a + b) h1
  show acotadaInf (f + g)
  exact h2

-- 2ª demostración
example
  (hf : acotadaInf f)
  (hg : acotadaInf g)
  : acotadaInf (f + g) :=
by
  cases' hf with a ha
  -- a : ℝ
  -- ha : FnLb f a
  cases' hg with b hgb
  -- b : ℝ
  -- hgb : FnLb g b
  use a + b
  -- ⊢ FnLb (f + g) (a + b)
  apply sumaCotaInf ha hgb

-- 3ª demostración
example
  (hf : acotadaInf f)
  (hg : acotadaInf g)
  : acotadaInf (f + g) :=
by
  rcases hf with ⟨a, ha⟩
  -- a : ℝ
  -- ha : FnLb f a
  rcases hg with ⟨b, hb⟩
  -- b : ℝ
  -- hb : FnLb g b
  exact ⟨a + b, sumaCotaInf ha hb⟩

-- 4ª demostración
example :
  acotadaInf f → acotadaInf g → acotadaInf (f + g) :=
by
  rintro ⟨a, ha⟩ ⟨b, hb⟩
  -- a : ℝ
  -- ha : FnLb f a
  -- b : ℝ
  -- hb : FnLb g b
  exact ⟨a + b, sumaCotaInf ha hb⟩

-- 5ª demostración
example :
  acotadaInf f → acotadaInf g → acotadaInf (f + g) :=
fun ⟨a, ha⟩ ⟨b, hb⟩ ↦ ⟨a + b, sumaCotaInf ha hb⟩

-- Lemas usados
-- ============

-- #check (sumaCotaInf : FnLb f a → FnLb g b → FnLb (f + g) (a + b))
\end{verbatim}
Se puede interactuar con las pruebas anteriores en \href{https://lean.math.hhu.de/\#url=https://raw.githubusercontent.com/jaalonso/Calculemus2/main/src/Suma\_de\_funciones\_acotadas\_inferiormente.lean}{Lean 4 Web}

\section{Si a es una cota superior de f y c ≥ 0, entonces ca es una cota superior de cf}
\label{sec:orgac1063c}
\begin{verbatim}
-- ---------------------------------------------------------------------
-- Demostrar que si a es una cota superior de f y c ≥ 0,
-- entonces c * a es una cota superior de c * f.
-- ----------------------------------------------------------------------

-- Demostración en lenguaj natural
-- ===============================

-- Se usará el lema
--    {b ≤ c, 0 ≤ a} ⊢ ab ≤ ac                                      (L1)
--
-- Tenemos que demostrar que
--    (∀ y ∈ ℝ) cf(y) ≤ ca.
-- Sea y ∈ R. Puesto que a es una cota de f, se tiene que
--    f(y) ≤ a
-- que, junto con c ≥ 0, por el lema L1 nos da
--    cf(y) ≤ ca

-- Demostraciones con Lean4
-- ========================

import Mathlib.Data.Real.Basic

-- (CotaSuperior f a) se verifica si a es una cota superior de f.
def CotaSuperior (f : ℝ → ℝ) (a : ℝ) : Prop :=
  ∀ x, f x ≤ a

variable {f : ℝ → ℝ}
variable {c : ℝ}

-- Demostraciones con Lean4
-- ========================

-- 1ª demostración
example
  (hfa : CotaSuperior f a)
  (h : c ≥ 0)
  : CotaSuperior (fun x ↦ c * f x) (c * a) :=
by
  intro y
  -- y : ℝ
  -- ⊢ (fun x => c * f x) y ≤ c * a
  have ha : f y ≤ a := hfa y
  calc (fun x => c * f x) y
       = c * f y := by rfl
     _ ≤ c * a   := mul_le_mul_of_nonneg_left ha h

-- 2ª demostración
example
  (hfa : CotaSuperior f a)
  (h : c ≥ 0)
  : CotaSuperior (fun x ↦ c * f x) (c * a) :=
by
  intro y
  calc (fun x => c * f x) y
       = c * f y := by rfl
     _ ≤ c * a   := mul_le_mul_of_nonneg_left (hfa y) h

-- 3ª demostración
example
  (hfa : CotaSuperior f a)
  (h : c ≥ 0)
  : CotaSuperior (fun x ↦ c * f x) (c * a) :=
by
  intro y
  show (fun x => c * f x) y ≤ c * a
  exact mul_le_mul_of_nonneg_left (hfa y) h

-- 4ª demostración
lemma CotaSuperior_mul
  (hfa : CotaSuperior f a)
  (h : c ≥ 0)
  : CotaSuperior (fun x ↦ c * f x) (c * a) :=
fun y ↦ mul_le_mul_of_nonneg_left (hfa y) h

-- Lemas usados
-- ============

-- variable (c : ℝ)
-- #check (mul_le_mul_of_nonneg_left : b ≤ c → 0 ≤ a → a * b ≤ a * c)
\end{verbatim}
Se puede interactuar con las pruebas anteriores en \href{https://lean.math.hhu.de/\#url=https://raw.githubusercontent.com/jaalonso/Calculemus2/main/src/Cota\_superior\_de\_producto\_por\_escalar.lean}{Lean 4 Web}

\section{Si c ≥ 0 y f está acotada superiormente, entonces c·f también lo está}
\label{sec:orgfc64ba6}
\begin{verbatim}
-- ---------------------------------------------------------------------
-- Demostrar que si c ≥ 0 y f está acotada superiormente, entonces c * f
-- también lo está.
-- ----------------------------------------------------------------------

-- Demostración en lenguaje natural
-- ================================

-- Usaremos el siguiente lema:
--    CotaSuperior_mul : CotaSuperior f a → c ≥ 0 → CotaSuperior (fun x ↦ c * f x) (c * a)
--
-- Puesto que f está acotada superiormente, tiene una cota superior. Sea
-- a una de dichas cotas. Entonces, por el lema CotaSuperior_mul, ca es una cota
-- superior de cf. Por consiguiente, cf está acotada superiormente.

-- Demostraciones con Lean4
-- ========================

import src.Cota_superior_de_producto_por_escalar

variable {f : ℝ → ℝ}
variable {c : ℝ}

-- (acotadaSup f) afirma que f tiene cota superior.
def acotadaSup (f : ℝ → ℝ) :=
  ∃ a, CotaSuperior f a

-- 1ª demostración
example
  (hf : acotadaSup f)
  (hc : c ≥ 0)
  : acotadaSup (fun x ↦ c * f x) :=
by
  cases' hf with a ha
  -- a : ℝ
  -- ha : CotaSuperior f a
  have h1 : CotaSuperior (fun x ↦ c * f x) (c * a) :=
    CotaSuperior_mul ha hc
  have h2 : ∃ z, ∀ x, (fun x ↦ c * f x) x ≤ z :=
    Exists.intro (c * a) h1
  show acotadaSup (fun x ↦ c * f x)
  exact h2

-- 2ª demostración
example
  (hf : acotadaSup f)
  (hc : c ≥ 0)
  : acotadaSup (fun x ↦ c * f x) :=
by
  cases' hf with a ha
  -- a : ℝ
  -- ha : CotaSuperior f a
  use c * a
  -- ⊢ CotaSuperior (fun x => c * f x) (c * a)
  apply CotaSuperior_mul ha hc

-- 3ª demostración
example
  (hf : acotadaSup f)
  (hc : c ≥ 0)
  : acotadaSup (fun x ↦ c * f x) :=
by
  rcases hf with ⟨a, ha⟩
  -- a : ℝ
  -- ha : CotaSuperior f a
  exact ⟨c * a, CotaSuperior_mul ha hc⟩

-- 4ª demostración
example
  (hc : c ≥ 0)
  : acotadaSup f → acotadaSup (fun x ↦ c * f x) :=
by
  rintro ⟨a, ha⟩
  -- a : ℝ
  -- ha : CotaSuperior f a
  exact ⟨c * a, CotaSuperior_mul ha hc⟩

-- 5ª demostración
example
  (hc : c ≥ 0)
  : acotadaSup f → acotadaSup (fun x ↦ c * f x) :=
fun ⟨a, ha⟩ ↦ ⟨c * a, CotaSuperior_mul ha hc⟩

-- Lemas usados
-- ============

-- #check (CotaSuperior_mul : CotaSuperior f a → c ≥ 0 → CotaSuperior (fun x ↦ c * f x) (c * a))
\end{verbatim}
Se puede interactuar con las pruebas anteriores en \href{https://lean.math.hhu.de/\#url=https://raw.githubusercontent.com/jaalonso/Calculemus2/main/src/Producto\_por\_escalar\_acotado\_superiormente.lean}{Lean 4 Web}

\section{Si para cada a existe un x tal que f(x) > a, entonces f no tiene cota superior}
\label{sec:org7db7474}
\begin{verbatim}
-- ---------------------------------------------------------------------
-- Demostrar que si f es una función de ℝ en ℝ tal que para cada a,
-- existe un x tal que f x > a, entonces f no tiene cota superior.
-- ----------------------------------------------------------------------

-- Demostración en lenguaje natural
-- ================================

-- Supongamos que f tiene cota superior. Sea b una de dichas cotas
-- superiores. Por la hipótesis, existe un x tal que f(x) > b. Además,
-- como b es una cota superior de f, f(x) ≤ b que contradice la
-- desigualdad anterior.

-- Demostraciones con Lean4
-- ========================

import Mathlib.Data.Real.Basic

def CotaSuperior (f : ℝ → ℝ) (a : ℝ) : Prop :=
  ∀ x, f x ≤ a

def acotadaSup (f : ℝ → ℝ) : Prop :=
  ∃ a, CotaSuperior f a

variable (f : ℝ → ℝ)

-- 1ª demostración
example
  (h : ∀ a, ∃ x, f x > a)
  : ¬ acotadaSup f :=
by
  intros hf
  -- hf : acotadaSup f
  -- ⊢ False
  cases' hf with b hb
  -- b : ℝ
  -- hb : CotaSuperior f b
  cases' h b with x hx
  -- x : ℝ
  -- hx : f x > b
  have : f x ≤ b := hb x
  linarith

-- 2ª demostración
theorem sinCotaSup
  (h : ∀ a, ∃ x, f x > a)
  : ¬ acotadaSup f :=
by
  intros hf
  -- hf : acotadaSup f
  -- ⊢ False
  rcases hf with ⟨b, hb : CotaSuperior f b⟩
  rcases h b with ⟨x, hx : f x > b⟩
  have : f x ≤ b := hb x
  linarith
\end{verbatim}
Se puede interactuar con las pruebas anteriores en \href{https://lean.math.hhu.de/\#url=https://raw.githubusercontent.com/jaalonso/Calculemus2/main/src/Funcion\_no\_acotada\_superiormente.lean}{Lean 4 Web}

\section{Si para cada a existe un x tal que f(x) < a, entonces f no tiene cota inferior}
\label{sec:orgb412d7b}
\begin{verbatim}
-- ---------------------------------------------------------------------
-- Demostrar que si f es una función de ℝ en ℝ tal que para cada a,
-- existe un x tal que f x < a, entonces f no tiene cota inferior.
-- ----------------------------------------------------------------------

-- Demostración en lenguaje natural
-- ================================

-- Supongamos que f tiene cota inferior. Sea b una de dichas cotas
-- inferiores. Por la hipótesis, existe un x tal que f(x) < b. Además,
-- como b es una cota inferior de f, b ≤ f(x) que contradice la
-- desigualdad anterior.

-- Demostraciones con Lean4
-- ========================

import Mathlib.Data.Real.Basic

def CotaInferior (f : ℝ → ℝ) (a : ℝ) : Prop :=
  ∀ x, a ≤ f x

def acotadaInf (f : ℝ → ℝ) : Prop :=
  ∃ a, CotaInferior f a

variable (f : ℝ → ℝ)

-- 1ª demostración
example
  (h : ∀ a, ∃ x, f x < a)
  : ¬ acotadaInf f :=
by
  intros hf
  -- hf : acotadaInf f
  -- ⊢ False
  cases' hf with b hb
  -- b : ℝ
  -- hb : CotaInferior f b
  cases' h b with x hx
  -- x : ℝ
  -- hx : f x < b
  have : b ≤ f x := hb x
  linarith

-- 2ª demostración
example
  (h : ∀ a, ∃ x, f x < a)
  : ¬ acotadaInf f :=
by
  intros hf
  -- hf : acotadaInf f
  -- ⊢ False
  rcases hf with ⟨b, hb : CotaInferior f b⟩
  rcases h b with ⟨x, hx : f x < b⟩
  have : b ≤ f x := hb x
  linarith
\end{verbatim}
Se puede interactuar con las pruebas anteriores en \href{https://lean.math.hhu.de/\#url=https://raw.githubusercontent.com/jaalonso/Calculemus2/main/src/Funcion\_no\_acotada\_inferiormente.lean}{Lean 4 Web}

\section{La función identidad no está acotada superiormente}
\label{sec:org2497ce7}
\begin{verbatim}
-- ---------------------------------------------------------------------
-- Demostrar que la función identidad no está acotada superiormente.
-- ----------------------------------------------------------------------

-- Demostración en lenguaje natural
-- ================================

-- Usamos el lema de ejercicio anterior (que afirma que si para cada a,
-- existe un x tal que f x > a, entonces f no tiene cota superior) basta
-- demostrar que
--    (∀a ∈ ℝ)(∃x ∈ ℝ) [x > a]
-- Sea a ∈ ℝ. Entonces a + 1 > a y, por tanto, (∃x ∈ ℝ) [x > a].

-- Demostraciones con Lean4
-- ========================

import src.Funcion_no_acotada_superiormente

-- 1ª demostración
example : ¬ acotadaSup (fun x ↦ x) :=
by
  apply sinCotaSup
  -- ⊢ ∀ (a : ℝ), ∃ x, x > a
  intro a
  -- a : ℝ
  -- ⊢ ∃ x, x > a
  use a + 1
  -- ⊢ a + 1 > a
  linarith

-- 2ª demostración
example : ¬ acotadaSup (fun x ↦ x) :=
by
  apply sinCotaSup
  -- ⊢ ∀ (a : ℝ), ∃ x, x > a
  intro a
  -- a : ℝ
  -- ⊢ ∃ x, x > a
  exact ⟨a + 1, by linarith⟩

-- 3ª demostración
example : ¬ acotadaSup (fun x ↦ x) :=
by
  apply sinCotaSup
  -- ⊢ ∀ (a : ℝ), ∃ x, x > a
  exact fun a ↦ ⟨a + 1, by linarith⟩
\end{verbatim}
Se puede interactuar con las pruebas anteriores en \href{https://lean.math.hhu.de/\#url=https://raw.githubusercontent.com/jaalonso/Calculemus2/main/src/La\_identidad\_no\_esta\_acotada\_superiormente.lean}{Lean 4 Web}

\section{Si f no está acotada superiormente, entonces (∀a)(∃x)[f(x) > a]}
\label{sec:org885895e}
\begin{verbatim}
-- ---------------------------------------------------------------------
-- Sea f una función de ℝ en ℝ. Demostrar que si f no está acotada
-- superiormente, entonces (∀a)(∃x)[f(x) > a]​.
-- ----------------------------------------------------------------------

-- Demostraciones en lenguaje natural (LN)
-- =======================================

-- 1ª demostración en LN
-- =====================

-- Usaremos los siguientes lemas
--    ¬(∃x)P(x) → (∀x)¬P(x)                                          (L1)
--    ¬a > b → a ≤ b                                                 (L2)
--
-- Sea a ∈ ℝ. Tenemos que demostrar que
--    (∃x)[f(x) > a]
-- Lo haremos por reducción al absurdo. Para ello, suponemos que
--    ¬(∃x)[f(x) > a]                                                (1)
-- y tenemos que obtener una contradicción. Aplicando L1 a (1) se tiene
--    (∀x)[¬ f(x) > a]
-- y, aplicando L2, se tiene
--    (∀x)[f(x) ≤ a]
-- Lo que significa que a es una cota superior de f y, por tanto f está
-- acotada superiormente, en cotradicción con la hipótesis.

-- 2ª demostración en LN
-- =====================

-- Por la contrarecíproca, se supone que
--    ¬(∀a)(∃x)[f(x) > a]                                             (1)
-- y tenemos que demostrar que f está acotada superiormente.
--
-- Interiorizando la negación en (1) y simplificando, se tiene que
--    (∃a)(∀x)[f x ≤ a]
-- que es lo que teníamos que demostrar.

-- Demostraciones con Lean 4
-- =========================

import Mathlib.Data.Real.Basic

def CotaSuperior (f : ℝ → ℝ) (a : ℝ) : Prop :=
  ∀ x, f x ≤ a

def acotadaSup (f : ℝ → ℝ) :=
  ∃ a, CotaSuperior f a

variable (f : ℝ → ℝ)

-- 1ª demostración
-- ===============

example
  (h : ¬acotadaSup f)
  : ∀ a, ∃ x, f x > a :=
by
  intro a
  -- a : ℝ
  -- ⊢ ∃ x, f x > a
  by_contra h1
  -- h1 : ¬∃ x, f x > a
  -- ⊢ False
  have h2 : ∀ x, ¬ f x > a :=
    forall_not_of_not_exists h1
  have h3 : ∀ x, f x ≤ a := by
    intro x
    have h3a : ¬ f x > a := h2 x
    show f x ≤ a
    exact le_of_not_gt h3a
  have h4 : CotaSuperior f a := h3
  have h5 : ∃ b, CotaSuperior f b := ⟨a, h4⟩
  have h6 : acotadaSup f := h5
  show False
  exact h h6

-- 2ª demostración
-- ===============

example
  (h : ¬acotadaSup f)
  : ∀ a, ∃ x, f x > a :=
by
  intro a
  -- a : ℝ
  -- ⊢ ∃ x, f x > a
  by_contra h1
  -- h1 : ¬∃ x, f x > a
  -- ⊢ False
  apply h
  -- ⊢ acotadaSup f
  use a
  -- ⊢ CotaSuperior f a
  intro x
  -- x : ℝ
  -- ⊢ f x ≤ a
  apply le_of_not_gt
  -- ⊢ ¬f x > a
  intro h2
  -- h2 : f x > a
  -- ⊢ False
  apply h1
  -- ⊢ ∃ x, f x > a
  use x
  -- ⊢ f x > a
  exact h2

-- 3ª demostración
-- ===============

example
  (h : ¬acotadaSup f)
  : ∀ a, ∃ x, f x > a :=
by
  unfold acotadaSup at h
  -- h : ¬∃ a, CotaSuperior f a
  unfold CotaSuperior at h
  -- h : ¬∃ a, ∀ (x : ℝ), f x ≤ a
  push_neg at h
  -- ∀ (a : ℝ), ∃ x, f x > a
  exact h

-- 4ª demostración
-- ===============

example
  (h : ¬acotadaSup f)
  : ∀ a, ∃ x, f x > a :=
by
  simp only [acotadaSup, CotaSuperior] at h
  -- h : ¬∃ a, ∀ (x : ℝ), f x ≤ a
  push_neg at h
  -- ∀ (a : ℝ), ∃ x, f x > a
  exact h

-- 5ª demostración
-- ===============

example
  (h : ¬acotadaSup f) :
  ∀ a, ∃ x, f x > a :=
by
  contrapose h
  -- h : ¬∀ (a : ℝ), ∃ x, f x > a
  -- ⊢ ¬¬acotadaSup f
  push_neg at *
  -- h : ∃ a, ∀ (x : ℝ), f x ≤ a
  -- ⊢ acotadaSup f
  exact h

-- 6ª demostración
-- ===============

example
  (h : ¬acotadaSup f) :
  ∀ a, ∃ x, f x > a :=
by
  contrapose! h
  -- h : ∃ a, ∀ (x : ℝ), f x ≤ a
  -- ⊢ acotadaSup f
  exact h

-- Lemas usados
-- ============

-- variable {α : Type _}
-- variable (P : α → Prop)
-- #check (forall_not_of_not_exists : (¬∃ x, P x) → ∀ x, ¬P x)
--
-- variable (a b : ℝ)
-- #check (le_of_not_gt : ¬a > b → a ≤ b)
\end{verbatim}
Se puede interactuar con las pruebas anteriores en \href{https://lean.math.hhu.de/\#url=https://raw.githubusercontent.com/jaalonso/Calculemus2/main/src/CN\_no\_acotada\_superiormente.lean}{Lean 4 Web}

\section{Si ¬(∀a)(∃x)[f(x) > a]​, entonces f está acotada superiormente}
\label{sec:org136beb8}
\begin{verbatim}
-- ---------------------------------------------------------------------
-- Demostrar que si ¬(∀a)(∃x)[f(x) > a]​, entonces f está acotada
-- superiormente.
-- ----------------------------------------------------------------------

-- Demostración en lenguaje natural
-- ================================

-- Tenemos que demostrar que f es acotada superiormente; es decir, que
--    (∃a)(∀x)[f(x) ≤ a]
-- que es exactamente la fórmula obtenida interiorizando la negación en
-- la hipótesis.

-- Demostraciones con Lean4
-- ========================

import Mathlib.Data.Real.Basic

def CotaSuperior (f : ℝ → ℝ) (a : ℝ) : Prop :=
  ∀ x, f x ≤ a

def acotadaSup (f : ℝ → ℝ) :=
  ∃ a, CotaSuperior f a

variable (f : ℝ → ℝ)

-- 1ª demostración
-- ===============

example
  (h : ¬∀ a, ∃ x, f x > a)
  : acotadaSup f :=
by
  unfold acotadaSup
  -- ⊢ ∃ a, CotaSuperior f a
  unfold CotaSuperior
  -- ⊢ ∃ a, ∀ (x : ℝ), f x ≤ a
  push_neg at h
  -- h : ∃ a, ∀ (x : ℝ), f x ≤ a
  exact h

-- 2ª demostración
-- ===============

example
  (h : ¬∀ a, ∃ x, f x > a)
  : acotadaSup f :=
by
  unfold acotadaSup CotaSuperior
  -- ⊢ ∃ a, ∀ (x : ℝ), f x ≤ a
  push_neg at h
  -- h : ∃ a, ∀ (x : ℝ), f x ≤ a
  exact h

-- 3ª demostración
-- ===============

example
  (h : ¬∀ a, ∃ x, f x > a)
  : acotadaSup f :=
by
  push_neg at h
  -- h : ∃ a, ∀ (x : ℝ), f x ≤ a
  exact h
\end{verbatim}
Se puede interactuar con las pruebas anteriores en \href{https://lean.math.hhu.de/\#url=https://raw.githubusercontent.com/jaalonso/Calculemus2/main/src/CS\_de\_acotada\_superiormente.lean}{Lean 4 Web}

\section{Suma de funciones monótonas}
\label{sec:org8835ba9}
\begin{verbatim}
-- ---------------------------------------------------------------------
-- Demostrar que la suma de dos funciones monótonas es monótona.
-- ----------------------------------------------------------------------

-- Demostración en lenguaje natural
-- ================================

-- Se usará el siguiente lema:
--    add_le_add : a ≤ b → c ≤ d → a + c ≤ b + d
--
-- Supongamos que f y g son monótonas y teneno que demostrar que f+g
-- también lo es; que
--    ∀ a b, a ≤ b → (f + g)(a) ≤ (f + g)(b)
-- Sean a, b ∈ ℝ tales que
--    a ≤ b                                                          (1)
-- Entonces, por ser f y g monótonas se tiene
--    f(a) ≤ f(b)                                                    (2)
--    g(a) ≤ g(b)                                                    (3)
-- Entonces,
--    (f + g)(a) = f(a) + g(a)
--               ≤ f(b) + g(b)    [por add_le_add, (2) y (3)]
--               = (f + g)(b)

-- Demostraciones con Lean4
-- ========================

import Mathlib.Data.Real.Basic

variable (f g : ℝ → ℝ)

-- 1ª demostración
example
  (mf : Monotone f)
  (mg : Monotone g)
  : Monotone (f + g) :=
by
  have h1 : ∀ a b, a ≤ b → (f + g) a ≤ (f + g) b
  { intros a b hab
    have h2 : f a ≤ f b := mf hab
    have h3 : g a ≤ g b := mg hab
    calc (f + g) a
         = f a + g a := rfl
       _ ≤ f b + g b := add_le_add h2 h3
       _ = (f + g) b := rfl }
  show Monotone (f + g)
  exact h1

-- 2ª demostración
example
  (mf : Monotone f)
  (mg : Monotone g)
  : Monotone (f + g) :=
by
  have h1 : ∀ a b, a ≤ b → (f + g) a ≤ (f + g) b
  { intros a b hab
    calc (f + g) a
         = f a + g a := rfl
       _ ≤ f b + g b := add_le_add (mf hab) (mg hab)
       _ = (f + g) b := rfl }
  show Monotone (f + g)
  exact h1

-- 3ª demostración
example
  (mf : Monotone f)
  (mg : Monotone g)
  : Monotone (f + g) :=
by
  have h1 : ∀ a b, a ≤ b → (f + g) a ≤ (f + g) b
  { intros a b hab
    show (f + g) a ≤ (f + g) b
    exact add_le_add (mf hab) (mg hab) }
  show Monotone (f + g)
  exact h1

-- 4ª demostración
example
  (mf : Monotone f)
  (mg : Monotone g)
  : Monotone (f + g) :=
by
  -- a b : ℝ
  -- hab : a ≤ b
  intros a b hab
  apply add_le_add
  . -- f a ≤ f b
    apply mf hab
  . --  g a ≤ g b
    apply mg hab

-- 5ª demostración
example
  (mf : Monotone f)
  (mg : Monotone g)
  : Monotone (f + g) :=
λ _ _ hab ↦ add_le_add (mf hab) (mg hab)

-- Lemas usados
-- ============

-- variable (a b c d : ℝ)
-- #check (add_le_add : a ≤ b → c ≤ d → a + c ≤ b + d)
\end{verbatim}
Se puede interactuar con las pruebas anteriores en \href{https://lean.math.hhu.de/\#url=https://raw.githubusercontent.com/jaalonso/Calculemus2/main/src/Suma\_de\_funciones\_monotonas.lean}{Lean 4 Web}

\section{Si c es no negativo y f es monótona, entonces cf es monótona.}
\label{sec:org9e8107a}
\begin{verbatim}
-- ---------------------------------------------------------------------
-- Demostrar que si c es no negativo y f es monótona, entonces cf es
-- monótona.
-- ----------------------------------------------------------------------

-- Demostración en lenguaje natural
-- ================================

-- Se usará el Lema
--    mul_le_mul_of_nonneg_left : b ≤ c → 0 ≤ a → a * b ≤ a * c
--
-- Tenemos que demostrar que
--    (∀ a, b ∈ ℝ) [a ≤ b → (cf)(a) ≤ (cf)(b)]
-- Sean a, b ∈ ℝ tales que a ≤ b. Puesto que f es monótona, se tiene
--    f(a) ≤ f(b).
-- y, junto con la hipótesis de que c es no negativo, usando el lema
-- mul_le_mul_of_nonneg_left, se tiene que
--    cf(a) ≤ cf(b)
-- que es lo que había que demostrar.

-- Demostraciones con Lean4
-- ========================

import Mathlib.Data.Real.Basic

variable (f : ℝ → ℝ)
variable {c : ℝ}

-- 1ª demostración
example
  (mf : Monotone f)
  (nnc : 0 ≤ c)
  : Monotone (fun x ↦ c * f x) :=
by
  have h1 : ∀ a b, a ≤ b → (fun x ↦ c * f x) a ≤ (fun x ↦ c * f x) b
  { intros a b hab
    have h2 : f a ≤ f b := mf hab
    show (fun x ↦ c * f x) a ≤ (fun x ↦ c * f x) b
    exact mul_le_mul_of_nonneg_left h2 nnc }
  show Monotone (fun x ↦ c * f x)
  exact h1

-- 2ª demostración
example
  (mf : Monotone f)
  (nnc : 0 ≤ c)
  : Monotone (fun x ↦ c * f x) :=
by
  -- a b : ℝ
  -- hab : a ≤ b
  intros a b hab
  -- (fun x => c * f x) a ≤ (fun x => c * f x) b
  apply mul_le_mul_of_nonneg_left
  . -- f a ≤ f b
    apply mf hab
  . -- 0 ≤ c
    apply nnc

-- 3ª demostración
example (mf : Monotone f) (nnc : 0 ≤ c) :
  Monotone (fun x ↦ c * f x) :=
λ _ _ hab ↦ mul_le_mul_of_nonneg_left (mf hab) nnc

-- Lemas usados
-- ============

-- variable (a b : ℝ)
-- #check (mul_le_mul_of_nonneg_left : b ≤ c → 0 ≤ a → a * b ≤ a * c)
\end{verbatim}
Se puede interactuar con las pruebas anteriores en \href{https://lean.math.hhu.de/\#url=https://raw.githubusercontent.com/jaalonso/Calculemus2/main/src/Producto\_de\_un\_positivo\_por\_una\_funcion\_monotona.lean}{Lean 4 Web}

\section{La composición de dos funciones monótonas es monótona}
\label{sec:org77e02dc}
\begin{verbatim}
-- ---------------------------------------------------------------------
-- Demostrar que la composición de dos funciones monótonas es monótona.
-- ----------------------------------------------------------------------

-- Demostración en lenguaje natural
-- ================================

-- Sean f y g dos funciones monótonas de ℝ en ℝ. Tenemos que demostrar
-- que f ∘ g es monótona; es decir, que
--    (∀ a, b ∈ ℝ) [a ≤ b → (f ∘ g)(a) ≤ (f ∘ g)(b)]
-- Sean a, b ∈ ℝ tales que a ≤ b. Por ser g monótona, se tiene
--    g(a) ≤ g(b)
-- y, por ser f monótona, se tiene
--    f(g(a)) ≤ f(g(b))
-- Finalmente, por la definición de composición,
--    (f ∘ g)(a) ≤ (f ∘ g)(b)
-- que es lo que había que demostrar.

-- Demostraciones con Lean4
-- ========================

import Mathlib.Data.Real.Basic

variable (f g : ℝ → ℝ)

-- 1ª demostración
example
  (mf : Monotone f)
  (mg : Monotone g)
  : Monotone (f ∘ g) :=
by
  have h1 : ∀ a b, a ≤ b → (f ∘ g) a ≤ (f ∘ g) b
  { intros a b hab
    have h1 : g a ≤ g b := mg hab
    show (f ∘ g) a ≤ (f ∘ g) b
    exact mf h1 }
  show Monotone (f ∘ g)
  exact h1

-- 2ª demostración
example
  (mf : Monotone f)
  (mg : Monotone g)
  : Monotone (f ∘ g) :=
by
  have h1 : ∀ a b, a ≤ b → (f ∘ g) a ≤ (f ∘ g) b
  { intros a b hab
    show (f ∘ g) a ≤ (f ∘ g) b
    exact mf (mg hab) }
  show Monotone (f ∘ g)
  exact h1

-- 3ª demostración
example
  (mf : Monotone f)
  (mg : Monotone g)
  : Monotone (f ∘ g) :=
by
  -- a b : ℝ
  -- hab : a ≤ b
  intros a b hab
  -- (f ∘ g) a ≤ (f ∘ g) b
  apply mf
  -- g a ≤ g b
  apply mg
  -- a ≤ b
  apply hab

-- 4ª demostración
example (mf : Monotone f) (mg : Monotone g) :
  Monotone (f ∘ g) :=
λ _ _ hab ↦ mf (mg hab)
\end{verbatim}
Se puede interactuar con las pruebas anteriores en \href{https://lean.math.hhu.de/\#url=https://raw.githubusercontent.com/jaalonso/Calculemus2/main/src/Composicion\_de\_funciones\_monotonas.lean}{Lean 4 Web}

\section{Si f es monótona y f(a) < f(b), entonces a < b}
\label{sec:org5af672c}
\begin{verbatim}
-- ---------------------------------------------------------------------
-- Demostrar que si f es monótona y f(a) < f(b), entonces a < b
-- ----------------------------------------------------------------------

-- Demostración en lenguaje natural
-- ================================

-- Usaremos los lemas
--    a ≱ b → a < b                                                  (L1)
--    a ≥ b → a ≮ b                                                  (L2)
--
-- Usando el lema L1, basta demostrar que a ≱ b. Lo haremos por
-- reducción al absurdo. Para ello, supongamos que a ≥ b. Como f es
-- monótona, se tiene que f(a) ≥ f(b) y, aplicando el lema L2,
-- f(a) ≮ f(b), que contradice a la hipótesis.

-- Demostraciones con Lean4
-- ========================

import Mathlib.Data.Real.Basic
variable (f : ℝ → ℝ)
variable (a b : ℝ)

-- 1ª demostración
-- ===============

example
  (h1 : Monotone f)
  (h2 : f a < f b)
  : a < b :=
by
  apply lt_of_not_ge
  -- ⊢ ¬a ≥ b
  intro h3
  -- h3 : a ≥ b
  -- ⊢ False
  have h4 : f a ≥ f b := h1 h3
  have h5 : ¬ f a < f b := not_lt_of_ge h4
  exact h5 h2

-- 2ª demostración
-- ===============

example
  (h1 : Monotone f)
  (h2 : f a < f b)
  : a < b :=
by
  apply lt_of_not_ge
  -- ⊢ ¬a ≥ b
  intro h3
  -- h3 : a ≥ b
  -- ⊢ False
  have h5 : ¬ f a < f b := not_lt_of_ge (h1 h3)
  exact h5 h2

-- 3ª demostración
-- ===============

example
  (h1 : Monotone f)
  (h2 : f a < f b)
  : a < b :=
by
  apply lt_of_not_ge
  -- ⊢ ¬a ≥ b
  intro h3
  -- h3 : a ≥ b
  -- ⊢ False
  exact (not_lt_of_ge (h1 h3)) h2

-- 4ª demostración
-- ===============

example
  (h1 : Monotone f)
  (h2 : f a < f b)
  : a < b :=
by
  apply lt_of_not_ge
  -- ⊢ ¬a ≥ b
  exact fun h3 ↦ (not_lt_of_ge (h1 h3)) h2

-- 5ª demostración
-- ===============

example
  (h1 : Monotone f)
  (h2 : f a < f b)
  : a < b :=
lt_of_not_ge (fun h3 ↦ (not_lt_of_ge (h1 h3)) h2)

-- Lemas usados
-- ============

-- #check (lt_of_not_ge : ¬ a ≥ b → a < b)
-- #check (not_lt_of_ge : a ≥ b → ¬ a < b)
\end{verbatim}
Se puede interactuar con las pruebas anteriores en \href{https://lean.math.hhu.de/\#url=https://raw.githubusercontent.com/jaalonso/Calculemus2/main/src/CN\_de\_monotona.lean}{Lean 4 Web}

\section{Si a, b ∈ ℝ tales que a ≤ b y f(b) < f(a), entonces f no es monótona}
\label{sec:org8b5ceba}
\begin{verbatim}
-- ---------------------------------------------------------------------
-- Demostrar que si a, b ∈ ℝ tales que (a ≤ b) y (f b < f a), entonces f
-- no es monótona.
-- ----------------------------------------------------------------------

-- Demostración en lenguaje natural
-- ================================

-- Usaremos el lema
--    a ≥ b → a ≮ b                                                (L1)
--
-- Lo demostraremos por reducción al absurdo. Para ello, supongamos que
-- f es monótona. Entonces, como a ≤ b, se tiene f(a) ≤ f(b) y, por el
-- lema L1, f b ≮ f a, en contradicción con la hipótesis.

-- Demostraciones con Lean4
-- ========================

import Mathlib.Data.Real.Basic
variable (f : ℝ → ℝ)
variable (a b : ℝ)

-- 1ª demostración
-- ===============

example
  (h1 : a ≤ b)
  (h2 : f b < f a)
  : ¬ Monotone f :=
by
  intro h3
  -- h3 : Monotone f
  -- ⊢ False
  have h4 : f a ≤ f b := h3 h1
  have h5 : ¬(f b < f a) := not_lt_of_ge h4
  exact h5 h2

-- 2ª demostración
-- ===============

example
  (h1 : a ≤ b)
  (h2 : f b < f a)
  : ¬ Monotone f :=
by
  intro h3
  -- h3 : Monotone f
  -- ⊢ False
  have h5 : ¬(f b < f a) := not_lt_of_ge (h3 h1)
  exact h5 h2

-- 3ª demostración
-- ===============

example
  (h1 : a ≤ b)
  (h2 : f b < f a)
  : ¬ Monotone f :=
by
  intro h3
  -- h3 : Monotone f
  -- ⊢ False
  exact (not_lt_of_ge (h3 h1)) h2

-- 4ª demostración
-- ===============

example
  (h1 : a ≤ b)
  (h2 : f b < f a)
  : ¬ Monotone f :=
fun h3 ↦ (not_lt_of_ge (h3 h1)) h2

-- Lemas usados
-- ============

-- #check (not_lt_of_ge : a ≥ b → ¬a < b)
\end{verbatim}
Se puede interactuar con las pruebas anteriores en \href{https://lean.math.hhu.de/\#url=https://raw.githubusercontent.com/jaalonso/Calculemus2/main/src/CS\_de\_no\_monotona.lean}{Lean 4 Web}

\section{No para toda f : ℝ → ℝ monótona, (∀a, b)[f(a) ≤ f(b) → a ≤ b]}
\label{sec:orga494579}
\begin{verbatim}
-- ---------------------------------------------------------------------
-- Demostrar que no para toda f : ℝ → ℝ monótona,
--    (∀ a b)[f(a) ≤ f(b) → a ≤ b]
-- ----------------------------------------------------------------------

-- Demostración en lenguaje natural
-- ================================

-- Supongamos que
--    (∀f)[f es monótona → (∀a, b)[f(a) ≤ f(b) → a ≤ b]]             (1)
-- Sea f : ℝ → ℝ la función constante igual a cero (es decir,
--    (∀x ∈ ℝ)[f(x) = 0]
-- Entonces, f es monótona y f(1) ≤ f(0) (ya que
-- f(1) = 0 ≤ 0 = f(0)). Luego, por (1), 1 ≤ 0 que es una
-- contradicción.

-- Demostraciones con Lean4
-- ========================

import Mathlib.Data.Real.Basic

-- 1ª demostración
-- ===============

example :
  ¬∀ {f : ℝ → ℝ}, Monotone f → ∀ {a b}, f a ≤ f b → a ≤ b :=
by
  intro h1
  -- h1 : ∀ {f : ℝ → ℝ}, Monotone f → ∀ {a b : ℝ}, f a ≤ f b → a ≤ b
  -- ⊢ False
  let f := fun _ : ℝ ↦ (0 : ℝ)
  have h2 : Monotone f := monotone_const
  have h3 : f 1 ≤ f 0 := le_refl 0
  have h4 : 1 ≤ 0 := h1 h2 h3
  linarith

-- Lemas usados
-- ============

-- variable (a c : ℝ)
-- #check (le_refl a : a ≤ a)
-- #check (monotone_const : Monotone fun _ : ℝ ↦ c)
\end{verbatim}
Se puede interactuar con las pruebas anteriores en \href{https://lean.math.hhu.de/\#url=https://raw.githubusercontent.com/jaalonso/Calculemus2/main/src/Propiedad\_de\_monotona.lean}{Lean 4 Web}

\section{Si f no es monótona, entonces ∃x∃y[x ≤ y ∧ f(y) < f(x)]}
\label{sec:org2640110}
\begin{verbatim}
-- ---------------------------------------------------------------------
-- Demostrar que si f no es monótona, entonces existen x, y tales que
-- x ≤ y y f(y) < f(x).
-- ----------------------------------------------------------------------

-- Demostración en lenguaje natural
-- ================================

-- Usaremos los siguientes lemas.
--    ¬(∀x)P(x) ↔ (∃ x)¬P(x)                                         (L1)
--    ¬(p → q) ↔ p ∧ ¬q                                              (L2)
--    (∀a, b ∈ ℝ)[¬b ≤ a → a < b]                                    (L3)
--
-- Por la definición de función monótona,
--    ¬(∀x)(∀y)[x ≤ y → f(x) ≤ f(y)]
-- Aplicando L1 se tiene
--    (∃x)¬(∀y)[x ≤ y → f(x) ≤ f(y)]
-- Sea a tal que
--    ¬(∀y)[a ≤ y → f(a) ≤ f(y)]
-- Aplicando L1 se tiene
--    (∃y)¬[a ≤ y → f(a) ≤ f(y)]
-- Sea b tal que
--    ¬[a ≤ b → f(a) ≤ f(b)]
-- Aplicando L2 se tiene que
--    a ≤ b ∧ ¬(f(a) ≤ f(b))
-- Aplicando L3 se tiene que
--    a ≤ b ∧ f(b) < f(a)
-- Por tanto,
--    (∃x,y)[x ≤ y ∧ f(y) < f(x)]

-- Demostraciones con Lean4
-- ========================

import Mathlib.Tactic
variable (f : ℝ → ℝ)

-- 1ª demostración
-- ===============

example
  (h : ¬Monotone f)
  : ∃ x y, x ≤ y ∧ f y < f x :=
by
  have h1 : ¬∀ x y, x ≤ y → f x ≤ f y := h
  have h2 : ∃ x, ¬(∀ y, x ≤ y → f x ≤ f y) := not_forall.mp h1
  rcases h2 with ⟨a, ha : ¬∀ y, a ≤ y → f a ≤ f y⟩
  have h3 : ∃ y, ¬(a ≤ y → f a ≤ f y) := not_forall.mp ha
  rcases h3 with ⟨b, hb : ¬(a ≤ b → f a ≤ f b)⟩
  have h4 : a ≤ b ∧ ¬(f a ≤ f b) := not_imp.mp hb
  have h5 : a ≤ b ∧ f b < f a := ⟨h4.1, lt_of_not_le h4.2⟩
  use a, b
  -- ⊢ a ≤ b ∧ f b < f a
  exact h5

-- 2ª demostración
-- ===============

example
  (h : ¬Monotone f)
  : ∃ x y, x ≤ y ∧ f y < f x :=
by
  simp only [Monotone] at h
  -- h : ¬∀ ⦃a b : ℝ⦄, a ≤ b → f a ≤ f b
  push_neg at h
  -- h : Exists fun ⦃a⦄ => Exists fun ⦃b⦄ => a ≤ b ∧ f b < f a
  exact h

-- Lemas usados
-- ============

-- variable {α : Type _}
-- variable (P : α → Prop)
-- variable (p q : Prop)
-- variable (a b : ℝ)
-- #check (not_forall : (¬∀ x, P x) ↔ ∃ x, ¬P x)
-- #check (not_imp : ¬(p → q) ↔ p ∧ ¬q)
-- #check (lt_of_not_le : ¬b ≤ a → a < b)
\end{verbatim}
Se puede interactuar con las pruebas anteriores en \href{https://lean.math.hhu.de/\#url=https://raw.githubusercontent.com/jaalonso/Calculemus2/main/src/CN\_de\_no\_monotona.lean}{Lean 4 Web}

\section{f: ℝ → ℝ  no es monótona syss (∃x,y)(x ≤ y ∧ f(x) > f(y))}
\label{sec:orgd6553bd}
\begin{verbatim}
-- ---------------------------------------------------------------------
-- Demostrar que f : ℝ → ℝ no es monótona syss existen x e y tales
-- que x ≤ y y f(x) > f(y).
-- ----------------------------------------------------------------------

-- Demostración en lenguaje natural
-- ================================

-- Por la siguiente cadena de equivalencias:
--    f es no monótona ↔ ¬(∀ x y)[x ≤ y → f(x) ≤ f(y)]
--                     ↔ (∃ x y)[x ≤ y ∧ f(x) > f(y)]

-- Demostraciones con Lean4
-- ========================

import Mathlib.Data.Real.Basic
variable {f : ℝ → ℝ}

-- 1ª demostración
-- ===============

example :
  ¬Monotone f ↔ ∃ x y, x ≤ y ∧ f x > f y :=
calc
  ¬Monotone f
    ↔ ¬∀ x y, x ≤ y → f x ≤ f y := by rw [Monotone]
  _ ↔ ∃ x y, x ≤ y ∧ f y < f x  := by simp_all only [not_forall, not_le, exists_prop]
  _ ↔ ∃ x y, x ≤ y ∧ f x > f y  := by rfl

-- 2ª demostración
-- ===============

example :
  ¬Monotone f ↔ ∃ x y, x ≤ y ∧ f x > f y :=
calc
  ¬Monotone f
    ↔ ¬∀ x y, x ≤ y → f x ≤ f y := by rw [Monotone]
  _ ↔ ∃ x y, x ≤ y ∧ f x > f y  := by aesop

-- 3ª demostración
-- ===============

example :
  ¬Monotone f ↔ ∃ x y, x ≤ y ∧ f x > f y :=
by
  rw [Monotone]
  -- ⊢ (¬∀ ⦃a b : ℝ⦄, a ≤ b → f a ≤ f b) ↔ ∃ x y, x ≤ y ∧ f x > f y
  push_neg
  -- ⊢ (Exists fun ⦃a⦄ => Exists fun ⦃b⦄ => a ≤ b ∧ f b < f a) ↔ ∃ x y, x ≤ y ∧ f x > f y
  rfl

-- 4ª demostración
-- ===============

lemma not_Monotone_iff :
  ¬Monotone f ↔ ∃ x y, x ≤ y ∧ f x > f y :=
by
  rw [Monotone]
  -- ⊢ (¬∀ ⦃a b : ℝ⦄, a ≤ b → f a ≤ f b) ↔ ∃ x y, x ≤ y ∧ f x > f y
  aesop
\end{verbatim}
Se puede interactuar con las pruebas anteriores en \href{https://lean.math.hhu.de/\#url=https://raw.githubusercontent.com/jaalonso/Calculemus2/main/src/CNS\_de\_no\_monotona.lean}{Lean 4 Web}

\section{La función x ↦ -x no es monótona creciente}
\label{sec:org6e33dab}
\begin{verbatim}
-- ---------------------------------------------------------------------
-- Demostrar que la función opuesta no es monótona.
-- ----------------------------------------------------------------------

-- Demostración en lenguaje natural
-- ================================

-- Usando el lema del ejercicio anterior que afirma que una función f no
-- es monótona syss existen x e y tales que x ≤ y y f(x) > f(y), basta
-- demostrar que
--    (∃ x y)[x ≤ y ∧ -x > -y]
-- Basta elegir 2 y 3 ya que
--    2 ≤ 3 ∧ -2 > -3

-- Demostración con Lean4
-- ======================

import Mathlib.Data.Real.Basic
import src.CNS_de_no_monotona

example : ¬Monotone fun x : ℝ ↦ -x :=
by
  apply not_Monotone_iff.mpr
  -- ⊢ ∃ x y, x ≤ y ∧ -x > -y
  use 2, 3
  -- ⊢ 2 ≤ 3 ∧ -2 > -3
  norm_num
\end{verbatim}
Se puede interactuar con las pruebas anteriores en \href{https://lean.math.hhu.de/\#url=https://raw.githubusercontent.com/jaalonso/Calculemus2/main/src/La\_opuesta\_es\_no\_monotona.lean}{Lean 4 Web}

\section{La suma de dos funciones pares es par}
\label{sec:org349205b}
\begin{verbatim}
-- ---------------------------------------------------------------------
-- Demostrar que la suma de dos funciones pares es par.
-- ----------------------------------------------------------------------

-- Demostración en lenguaje natural
-- ================================

-- Supongamos que f y g son funciones pares. Tenemos que demostrar que
-- f+g es par; es decir, que
--    (∀ x ∈ ℝ) (f + g)(x) = (f + g)(-x)
-- Sea x ∈ ℝ. Entonces,
--    (f + g) x = f x + g x
--              = f (-x) + g x    [porque f es par]
--              = f (-x) + g (-x) [porque g es par]
--              = (f + g) (-x)

-- Demostraciones con Lean4
-- ========================

import Mathlib.Data.Real.Basic

variable (f g : ℝ → ℝ)

-- (esPar f) expresa que f es par.
def esPar (f : ℝ → ℝ) : Prop :=
  ∀ x, f x = f (-x)

-- 1ª demostración
-- ===============

example
  (h1 : esPar f)
  (h2 : esPar g)
  : esPar (f + g) :=
by
  intro x
  have h1 : f x = f (-x) := h1 x
  have h2 : g x = g (-x) := h2 x
  calc (f + g) x
       = f x + g x       := rfl
     _ = f (-x) + g x    := congrArg (. + g x) h1
     _ = f (-x) + g (-x) := congrArg (f (-x) + .) h2
     _ = (f + g) (-x)    := rfl

-- 2ª demostración
-- ===============

example
  (h1 : esPar f)
  (h2 : esPar g)
  : esPar (f + g) :=
by
  intro x
  calc (f + g) x
       = f x + g x       := rfl
     _ = f (-x) + g x    := congrArg (. + g x) (h1 x)
     _ = f (-x) + g (-x) := congrArg (f (-x) + .) (h2 x)
     _ = (f + g) (-x)    := rfl

-- 3ª demostración
-- ===============

example
  (h1 : esPar f)
  (h2 : esPar g)
  : esPar (f + g) :=
by
  intro x
  calc (f + g) x
       = f x + g x       := rfl
     _ = f (-x) + g (-x) := by rw [h1, h2]
     _ = (f + g) (-x)    := rfl
\end{verbatim}
Se puede interactuar con las pruebas anteriores en \href{https://lean.math.hhu.de/\#url=https://raw.githubusercontent.com/jaalonso/Calculemus2/main/src/Suma\_funciones\_pares.lean}{Lean 4 Web}

\section{El producto de dos funciones impares es par}
\label{sec:org3bc8faf}
\begin{verbatim}
-- ---------------------------------------------------------------------
-- Demostrar que el producto de dos funciones impares es par.
-- ----------------------------------------------------------------------

-- Demostración en lenguaje natural
-- ================================

-- Supongamos que f y g son funciones impares. Tenemos que demostrar que
-- f·g es par; es decir, que
--    (∀ x ∈ ℝ) (f·g)(x) = (f·g)(-x)
-- Sea x ∈ ℝ. Entonces,
--    (f·g) x = f(x)g(x)
--            = (-f(-x))g(x)      [porque f es impar]
--            = (-f(-x)(-g(-x))   [porque g es impar]
--            = f(-x)g(-x))
--            = (f·g)(-x)

-- Demostraciones con Lean4
-- ========================

import Mathlib.Data.Real.Basic

variable (f g : ℝ → ℝ)

-- (esPar f) expresa que f es par.
def esPar (f : ℝ → ℝ) : Prop :=
  ∀ x, f x = f (-x)

-- (esImpar f) expresa que f es impar.
def esImpar  (f : ℝ → ℝ) : Prop :=
  ∀ x, f x = - f (-x)

-- 1ª demostración
example
  (h1 : esImpar f)
  (h2 : esImpar g)
  : esPar (f * g) :=
by
  intro x
  have h1 : f x = -f (-x) := h1 x
  have h2 : g x = -g (-x) := h2 x
  calc (f * g) x
       = f x * g x             := rfl
     _ = (-f (-x)) * g x       := congrArg (. * g x) h1
     _ = (-f (-x)) * (-g (-x)) := congrArg ((-f (-x)) * .) h2
     _ = f (-x) * g (-x)       := neg_mul_neg (f (-x)) (g (-x))
     _ = (f * g) (-x)          := rfl

-- 2ª demostración
example
  (h1 : esImpar f)
  (h2 : esImpar g)
  : esPar (f * g) :=
by
  intro x
  calc (f * g) x
       = f x * g x             := rfl
     _ = (-f (-x)) * g x       := congrArg (. * g x) (h1 x)
     _ = (-f (-x)) * (-g (-x)) := congrArg ((-f (-x)) * .) (h2 x)
     _ = f (-x) * g (-x)       := neg_mul_neg (f (-x)) (g (-x))
     _ = (f * g) (-x)          := rfl

-- 3ª demostración
example
  (h1 : esImpar f)
  (h2 : esImpar g)
  : esPar (f * g) :=
by
  intro x
  calc (f * g) x
       = f x * g x         := rfl
     _ = -f (-x) * -g (-x) := by rw [h1, h2]
     _ = f (-x) * g (-x)   := by rw [neg_mul_neg]
     _ = (f * g) (-x)      := rfl

-- 4ª demostración
example
  (h1 : esImpar f)
  (h2 : esImpar g)
  : esPar (f * g) :=
by
  intro x
  calc (f * g) x
       = f x * g x       := rfl
     _ = f (-x) * g (-x) := by rw [h1, h2, neg_mul_neg]
     _ = (f * g) (-x)    := rfl

-- Lemas usados
-- ============

-- variable (a b : ℝ)
-- #check (neg_mul_neg a b : -a * -b = a * b)
\end{verbatim}
Se puede interactuar con las pruebas anteriores en \href{https://lean.math.hhu.de/\#url=https://raw.githubusercontent.com/jaalonso/Calculemus2/main/src/Producto\_de\_funciones\_impares.lean}{Lean 4 Web}

\section{El producto de una función par por una impar es impar}
\label{sec:org051d2d4}
\begin{verbatim}
-- ---------------------------------------------------------------------
-- Demostrar que el producto de una función par por una impar es impar.
-- ----------------------------------------------------------------------

-- Demostración en lenguaje natural
-- ================================

-- Supongamos que f es una función par y g lo es impar. Tenemos que
-- demostrar que f·g es imppar; es decir, que
--    (∀ x ∈ ℝ) (f·g)(x) = -(f·g)(-x)
-- Sea x ∈ ℝ. Entonces,
--    (f·g) x = f(x)g(x)
--            = f(-x)g(x)       [porque f es par]
--            = f(-x)(-g(-x))   [porque g es impar]
--            = -f(-x)g(-x))
--            = -(f·g)(-x)

-- Demostraciones con Lean4
-- ========================

import Mathlib.Data.Real.Basic

variable (f g : ℝ → ℝ)

-- (esPar f) expresa que f es par.
def esPar (f : ℝ → ℝ) : Prop :=
  ∀ x, f x = f (-x)

-- (esImpar f) expresa que f es impar.
def esImpar  (f : ℝ → ℝ) : Prop :=
  ∀ x, f x = - f (-x)

-- 1ª demostración
example
  (h1 : esPar f)
  (h2 : esImpar g)
  : esImpar (f * g) :=
by
  intro x
  have h1 : f x = f (-x) := h1 x
  have h2 : g x = -g (-x) := h2 x
  calc (f * g) x
       = f x * g x            := rfl
     _ = (f (-x)) * g x       := congrArg (. * g x) h1
     _ = (f (-x)) * (-g (-x)) := congrArg (f (-x) * .) h2
     _ = -(f (-x) * g (-x))   := mul_neg (f (-x)) (g (-x))
     _ = -(f * g) (-x)        := rfl

-- 2ª demostración
example
  (h1 : esPar f)
  (h2 : esImpar g)
  : esImpar (f * g) :=
by
  intro x
  calc (f * g) x
       = f x * g x          := rfl
    _  = f (-x) * -g (-x)   := by rw [h1, h2]
    _  = -(f (-x) * g (-x)) := by rw [mul_neg]
    _  = -(f * g) (-x)      := rfl

-- 3ª demostración
example
  (h1 : esPar f)
  (h2 : esImpar g)
  : esImpar (f * g) :=
by
  intro x
  calc (f * g) x
       = f x * g x          := rfl
     _ = -(f (-x) * g (-x)) := by rw [h1, h2, mul_neg]
     _ = -((f * g) (-x))    := rfl

-- Lemas usados
-- ===========

-- variable (a b : ℝ)
-- #check (mul_neg a b : a * -b = -(a * b))
\end{verbatim}
Se puede interactuar con las pruebas anteriores en \href{https://lean.math.hhu.de/\#url=https://raw.githubusercontent.com/jaalonso/Calculemus2/main/src/Producto\_funcion\_par\_e\_impar.lean}{Lean 4 Web}

\section{Si f es par y g es impar, entonces (f ∘ g) es par}
\label{sec:orga5ebfff}
\begin{verbatim}
-- ---------------------------------------------------------------------
-- Demostrar que si f es par y g es impar, entonces f ∘ g es par.
-- ----------------------------------------------------------------------

-- Demostración en lenguaje natural
-- ================================

-- Supongamos que f es una función par y g lo es impar. Tenemos que
-- demostrar que (f ∘ g) es par; es decir, que
--    (∀ x ∈ ℝ) (f ∘ g)(x) = (f ∘ g)(-x)
-- Sea x ∈ ℝ. Entonces,
--    (f ∘ g)(x) = f(g(x))
--               = f(-g(-x))    [porque g es impar]
--               = f(g(-x))     [porque f es par]
--               = (f ∘ g)(-x)

-- Demostraciones con Lean4
-- ========================

import Mathlib.Data.Real.Basic

variable (f g : ℝ → ℝ)

-- (esPar f) expresa que f es par.
def esPar (f : ℝ → ℝ) : Prop :=
  ∀ x, f x = f (-x)

-- (esImpar f) expresa que f es impar.
def esImpar  (f : ℝ → ℝ) : Prop :=
  ∀ x, f x = - f (-x)

-- 1ª demostración
example
  (h1 : esPar f)
  (h2 : esImpar g)
  : esPar (f ∘ g) :=
by
  intro x
  calc (f ∘ g) x
       = f (g x)      := rfl
    _  = f (-g (-x))  := congr_arg f (h2 x)
    _  = f (g (-x))   := (h1 (g (-x))).symm
    _  = (f ∘ g) (-x) := rfl

-- 2ª demostración
example
  (h1 : esPar f)
  (h2 : esImpar g)
  : esPar (f ∘ g) :=
by
  intro x
  calc (f ∘ g) x
       = f (g x)      := rfl
     _ = f (-g (-x))  := by rw [h2]
     _ = f (g (-x))   := by rw [← h1]
     _ = (f ∘ g) (-x) := rfl

-- 3ª demostración
example
  (h1 : esPar f)
  (h2 : esImpar g)
  : esPar (f ∘ g) :=
by
  intro x
  calc (f ∘ g) x
       = f (g x)      := rfl
     _ = f (g (-x))   := by rw [h2, ← h1]
\end{verbatim}
Se puede interactuar con las pruebas anteriores en \href{https://lean.math.hhu.de/\#url=https://raw.githubusercontent.com/jaalonso/Calculemus2/main/src/Composicion\_de\_par\_e\_impar.lean}{Lean 4 Web}

\section{Para cualquier conjunto s, s ⊆ s}
\label{sec:org8f31f5d}
\begin{verbatim}
-- ---------------------------------------------------------------------
-- Demostrar que para cualquier conjunto s, s ⊆ s.
-- ----------------------------------------------------------------------

-- Demostración en lenguaje natural
-- ================================

-- Tenemos que demostrar que
--    (∀ x) [x ∈ s → × ∈ s]
-- Sea x tal que
--    x ∈ s                                                          (1)
-- Entonces, por (1), se tiene que
--    x ∈ s
-- que es lo que teníamos que demostrar.

-- Demostraciones con Lean 4
-- =========================

import Mathlib.Tactic

variable {α : Type _}
variable (s : Set α)

-- 1ª demostración
example : s ⊆ s :=
by
  intro x xs
  exact xs

-- 2ª demostración
example : s ⊆ s :=
  fun (x : α) (xs : x ∈ s) ↦ xs

-- 3ª demostración
example : s ⊆ s :=
  fun _ xs ↦ xs

-- 4ª demostración
example : s ⊆ s :=
  -- by exact?
  rfl.subset

-- 5ª demostración
example : s ⊆ s :=
by rfl
\end{verbatim}
Se puede interactuar con las pruebas anteriores en \href{https://lean.math.hhu.de/\#url=https://raw.githubusercontent.com/jaalonso/Calculemus2/main/src/Propiedad\_reflexiva\_del\_subconjunto.lean}{Lean 4 Web}

\section{Las funciones f(x,y) = (x + y)² y g(x,y) = x² + 2xy + y² son iguales}
\label{sec:org619550d}
\begin{verbatim}
-- ---------------------------------------------------------------------
-- Demostrar que
--    (fun x y : ℝ ↦ (x + y)^2) = (fun x y : ℝ ↦ x^2 + 2*x*y + y^2)
-- ----------------------------------------------------------------------

import  Mathlib.Data.Real.Basic

-- 1ª demostración
-- ===============

example : (fun x y : ℝ ↦ (x + y)^2) = (fun x y : ℝ ↦ x^2 + 2*x*y + y^2) :=
by
  ext u v
  -- u v : ℝ
  -- ⊢ (u + v) ^ 2 = u ^ 2 + 2 * u * v + v ^ 2
  ring

-- Comentario: La táctica ext transforma las conclusiones de la forma
-- (fun x ↦ f x) = (fun x ↦ g x) en f x = g x.

-- 2ª demostración
-- ===============

example : (fun x y : ℝ ↦ (x + y)^2) = (fun x y : ℝ ↦ x^2 + 2*x*y + y^2) :=
by { ext ; ring }
\end{verbatim}
Se puede interactuar con las pruebas anteriores en \href{https://lean.math.hhu.de/\#url=https://raw.githubusercontent.com/jaalonso/Calculemus2/main/src/Demostracion\_por\_extensionalidad.lean}{Lean 4 Web}

\chapter{Teoría de conjuntos}
\label{sec:org8099504}

\section{Si r ⊆ s y s ⊆ t, entonces r ⊆ t}
\label{sec:org172c5d8}
\begin{verbatim}
-- ---------------------------------------------------------------------
-- Demostrar que si r ⊆ s y s ⊆ t, entonces r ⊆ t.
-- ----------------------------------------------------------------------

-- Demostración en lenguaje natural (LN)
-- =====================================

-- 1ª demostración en LN
-- ---------------------

-- Tenemos que demostrar que
--    (∀ x) [x ∈ r → x ∈ t]
-- Sea x tal que
--    x ∈ r.
-- Puesto que r ⊆ s, se tiene que
--    x ∈ s
-- y, puesto que s ⊆ t, se tiene que
--    x ∈ t
-- que es lo que teníamos que demostrar.

-- 2ª demostración en LN
-- ---------------------

-- Tenemos que demostrar que
--    (∀ x) [x ∈ r → x ∈ t]
-- Sea x tal que
--    x ∈ r
-- Tenemos que demostrar que
--    x ∈ t
-- que, puesto que s ⊆ t, se reduce a
--    x ∈ s
-- que, puesto que r ⊆ s, se redece a
--    x ∈ r
-- que es lo que hemos supuesto.

-- Demostraciones con Lean4
-- ========================

import Mathlib.Tactic

open Set

variable {α : Type _}
variable (r s t : Set α)

-- 1ª demostración
example
  (rs : r ⊆ s)
  (st : s ⊆ t)
  : r ⊆ t :=
by
  intros x xr
  -- xr : x ∈ r
  have xs : x ∈ s := rs xr
  show x ∈ t
  exact st xs

-- 2ª demostración
example
  (rs : r ⊆ s)
  (st : s ⊆ t)
  : r ⊆ t :=
by
  intros x xr
  -- x : α
  -- xr : x ∈ r
  apply st
  -- ⊢ x ∈ s
  apply rs
  -- ⊢ x ∈ r
  exact xr

-- 3ª demostración
example
  (rs : r ⊆ s)
  (st : s ⊆ t)
  : r ⊆ t :=
fun _ xr ↦ st (rs xr)

-- 4ª demostración
example
  (rs : r ⊆ s)
  (st : s ⊆ t)
  : r ⊆ t :=
-- by exact?
Subset.trans rs st

-- 5ª demostración
example
  (rs : r ⊆ s)
  (st : s ⊆ t)
  : r ⊆ t :=
by tauto

-- Lemas usados
-- ============

-- #check (Subset.trans : r ⊆ s → s ⊆ t → r ⊆ t)
\end{verbatim}
Se puede interactuar con las pruebas anteriores en \href{https://lean.math.hhu.de/\#url=https://raw.githubusercontent.com/jaalonso/Calculemus2/main/src/Propiedad\_transitiva\_del\_subconjunto.lean}{Lean 4 Web}

\section{Si a es una cota superior de s y a ≤ b, entonces b es una cota superior de s}
\label{sec:org60a401c}
\begin{verbatim}
-- ---------------------------------------------------------------------
-- Demostrar que si a es una cota superior de s y a ≤ b, entonces b es
-- una cota superior de s.
-- ----------------------------------------------------------------------

import Mathlib.Tactic

variable {α : Type _} [PartialOrder α]
variable (s : Set α)
variable (a b : α)

-- (CotaSupConj s a) afirma que a es una cota superior del conjunto s.
def CotaSupConj (s : Set α) (a : α) :=
  ∀ {x}, x ∈ s → x ≤ a

-- Demostración en lenguaje natural
-- ================================

-- Tenemos que demostrar que
--    (∀ x) [x ∈ s → x ≤ b]
-- Sea x tal que x ∈ s. Entonces,
--    x ≤ a   [porque a es una cota superior de s]
--      ≤ b
-- Por tanto, x ≤ b.

-- 1ª demostración
example
  (h1 : CotaSupConj s a)
  (h2 : a ≤ b)
  : CotaSupConj s b :=
by
  intro x (xs : x ∈ s)
  have h3 : x ≤ a := h1 xs
  show x ≤ b
  exact le_trans h3 h2

-- 2ª demostración
example
  (h1 : CotaSupConj s a)
  (h2 : a ≤ b)
  : CotaSupConj s b :=
by
  intro x (xs : x ∈ s)
  calc x ≤ a := h1 xs
       _ ≤ b := h2
-
-- Lemas usados
-- ============

-- variable (c : α)
-- #check (le_trans : a ≤ b → b ≤ c → a ≤ c)
\end{verbatim}
Se puede interactuar con las pruebas anteriores en \href{https://lean.math.hhu.de/\#url=https://raw.githubusercontent.com/jaalonso/Calculemus2/main/src/Cotas\_superiores\_de\_conjuntos.lean}{Lean 4 Web}

\section{Si s ⊆ t, entonces s ∩ u ⊆ t ∩ u}
\label{sec:org1a8d165}
\begin{verbatim}
-- ---------------------------------------------------------------------
-- Demostrar que si
--    s ⊆ t
-- entonces
--    s ∩ u ⊆ t ∩ u
-- ----------------------------------------------------------------------

-- Demostración en lenguaje natural
-- ================================

-- Sea x ∈ s ∩ u. Entonces, se tiene que
--   x ∈ s                                                           (1)
--   x ∈ u                                                           (2)
-- De (1) y s ⊆ t, se tiene que
--   x ∈ t                                                           (3)
-- De (3) y (2) se tiene que
--   x ∈ t ∩ u
-- que es lo que teníamos que demostrar.

-- Demostraciones con Lean4
-- ========================

import Mathlib.Data.Set.Basic
import Mathlib.Tactic

open Set

variable {α : Type}
variable (s t u : Set α)

-- 1ª demostración
-- ===============

example
  (h : s ⊆ t)
  : s ∩ u ⊆ t ∩ u :=
by
  rw [subset_def]
  -- ⊢ ∀ (x : α), x ∈ s ∩ u → x ∈ t ∩ u
  intros x h1
  -- x : α
  -- h1 : x ∈ s ∩ u
  -- ⊢ x ∈ t ∩ u
  rcases h1 with ⟨xs, xu⟩
  -- xs : x ∈ s
  -- xu : x ∈ u
  constructor
  . -- ⊢ x ∈ t
    rw [subset_def] at h
    -- h : ∀ (x : α), x ∈ s → x ∈ t
    apply h
    -- ⊢ x ∈ s
    exact xs
  . -- ⊢ x ∈ u
    exact xu

-- 2ª demostración
-- ===============

example
  (h : s ⊆ t)
  : s ∩ u ⊆ t ∩ u :=
by
  rw [subset_def]
  -- ⊢ ∀ (x : α), x ∈ s ∩ u → x ∈ t ∩ u
  rintro x ⟨xs, xu⟩
  -- x : α
  -- xs : x ∈ s
  -- xu : x ∈ u
  rw [subset_def] at h
  -- h : ∀ (x : α), x ∈ s → x ∈ t
  exact ⟨h x xs, xu⟩

-- 3ª demostración
-- ===============

example
  (h : s ⊆ t)
  : s ∩ u ⊆ t ∩ u :=
by
  simp only [subset_def]
  -- ⊢ ∀ (x : α), x ∈ s ∩ u → x ∈ t ∩ u
  rintro x ⟨xs, xu⟩
  -- x : α
  -- xs : x ∈ s
  -- xu : x ∈ u
  rw [subset_def] at h
  -- h : ∀ (x : α), x ∈ s → x ∈ t
  exact ⟨h _ xs, xu⟩

-- 4ª demostración
-- ===============

example
  (h : s ⊆ t)
  : s ∩ u ⊆ t ∩ u :=
by
  intros x xsu
  -- x : α
  -- xsu : x ∈ s ∩ u
  -- ⊢ x ∈ t ∩ u
  exact ⟨h xsu.1, xsu.2⟩

-- 5ª demostración
-- ===============

example
  (h : s ⊆ t)
  : s ∩ u ⊆ t ∩ u :=
by
  rintro x ⟨xs, xu⟩
  -- xs : x ∈ s
  -- xu : x ∈ u
  -- ⊢ x ∈ t ∩ u
  exact ⟨h xs, xu⟩

-- 6ª demostración
-- ===============

example
  (h : s ⊆ t)
  : s ∩ u ⊆ t ∩ u :=
  fun _ ⟨xs, xu⟩ ↦  ⟨h xs, xu⟩

-- 7ª demostración
-- ===============

example
  (h : s ⊆ t)
  : s ∩ u ⊆ t ∩ u :=
  inter_subset_inter_left u h

-- Lema usado
-- ==========

-- #check (inter_subset_inter_left u : s ⊆ t → s ∩ u ⊆ t ∩ u)
\end{verbatim}
Se puede interactuar con las pruebas anteriores en \href{https://lean.math.hhu.de/\#url=https://raw.githubusercontent.com/jaalonso/Calculemus2/main/src/Propiedad\_de\_monotonia\_de\_la\_interseccion.lean}{Lean 4 Web}

\section{s ∩ (t ∪ u) ⊆ (s ∩ t) ∪ (s ∩ u)}
\label{sec:org81afd4f}
\begin{verbatim}
-- ---------------------------------------------------------------------
-- Demostrar que
--    s ∩ (t ∪ u) ⊆ (s ∩ t) ∪ (s ∩ u)
-- ----------------------------------------------------------------------

-- Demostración en lenguaje natural
-- ================================

-- Sea x ∈ s ∩ (t ∪ u). Entonces se tiene que
--    x ∈ s                                                          (1)
--    x ∈ t ∪ u                                                      (2)
-- La relación (2) da lugar a dos casos.
--
-- Caso 1: Supongamos que x ∈ t. Entonces, por (1), x ∈ s ∩ t y, por
-- tanto, x ∈ (s ∩ t) ∪ (s ∩ u).
--
-- Caso 2: Supongamos que x ∈ u. Entonces, por (1), x ∈ s ∩ u y, por
-- tanto, x ∈ (s ∩ t) ∪ (s ∩ u).

-- Demostraciones con Lean4
-- ========================

import Mathlib.Data.Set.Basic
import Mathlib.Tactic

open Set

variable {α : Type}
variable (s t u : Set α)

-- 1ª demostración
-- ===============

example :
  s ∩ (t ∪ u) ⊆ (s ∩ t) ∪ (s ∩ u) :=
by
  intros x hx
  -- x : α
  -- hx : x ∈ s ∩ (t ∪ u)
  -- ⊢ x ∈ s ∩ t ∪ s ∩ u
  rcases hx with ⟨hxs, hxtu⟩
  -- hxs : x ∈ s
  -- hxtu : x ∈ t ∪ u
  rcases hxtu with (hxt | hxu)
  . -- hxt : x ∈ t
    left
    -- ⊢ x ∈ s ∩ t
    constructor
    . -- ⊢ x ∈ s
      exact hxs
    . -- hxt : x ∈ t
      exact hxt
  . -- hxu : x ∈ u
    right
    -- ⊢ x ∈ s ∩ u
    constructor
    . -- ⊢ x ∈ s
      exact hxs
    . -- ⊢ x ∈ u
      exact hxu

-- 2ª demostración
-- ===============

example :
  s ∩ (t ∪ u) ⊆ (s ∩ t) ∪ (s ∩ u) :=
by
  rintro x ⟨hxs, hxt | hxu⟩
  -- x : α
  -- hxs : x ∈ s
  -- ⊢ x ∈ s ∩ t ∪ s ∩ u
  . -- hxt : x ∈ t
    left
    -- ⊢ x ∈ s ∩ t
    exact ⟨hxs, hxt⟩
  . -- hxu : x ∈ u
    right
    -- ⊢ x ∈ s ∩ u
    exact ⟨hxs, hxu⟩

-- 3ª demostración
-- ===============

example :
  s ∩ (t ∪ u) ⊆ (s ∩ t) ∪ (s ∩ u) :=
by
  rintro x ⟨hxs, hxt | hxu⟩
  -- x : α
  -- hxs : x ∈ s
  -- ⊢ x ∈ s ∩ t ∪ s ∩ u
  . -- hxt : x ∈ t
    exact Or.inl ⟨hxs, hxt⟩
  . -- hxu : x ∈ u
    exact Or.inr ⟨hxs, hxu⟩

-- 4ª demostración
-- ===============

example :
  s ∩ (t ∪ u) ⊆ (s ∩ t) ∪ (s ∩ u) :=
by
  intro x hx
  -- x : α
  -- hx : x ∈ s ∩ (t ∪ u)
  -- ⊢ x ∈ s ∩ t ∪ s ∩ u
  aesop

-- 5ª demostración
-- ===============

example :
  s ∩ (t ∪ u) ⊆ (s ∩ t) ∪ (s ∩ u) :=
by rw [inter_union_distrib_left]
\end{verbatim}
Se puede interactuar con las pruebas anteriores en \href{https://lean.math.hhu.de/\#url=https://raw.githubusercontent.com/jaalonso/Calculemus2/main/src/Propiedad\_semidistributiva\_de\_la\_interseccion\_sobre\_la\_union.lean}{Lean 4 Web}

\section{(s $\backslash$ t) $\backslash$ u ⊆ s $\backslash$ (t ∪ u)}
\label{sec:org365f3d7}
\begin{verbatim}
-- ---------------------------------------------------------------------
-- Demostrar que
--    (s \ t) \ u ⊆ s \ (t ∪ u)
-- ----------------------------------------------------------------------

-- Demostración en lenguaje natural
-- ================================

-- Sea x ∈ (s \ t) \ u. Entonces, se tiene que
--    x ∈ s                                                      (1)
--    x ∉ t                                                      (2)
--    x ∉ u                                                      (3)
-- Tenemos que demostrar que
--    x ∈ s \ (t ∪ u)
-- pero, por (1), se reduce a
--    x ∉ t ∪ u
-- que se verifica por (2) y (3).

-- Demostraciones con Lean4
-- ========================

import Mathlib.Data.Set.Basic
open Set

variable {α : Type}
variable (s t u : Set α)

-- 1ª demostración
-- ===============

example : (s \ t) \ u ⊆ s \ (t ∪ u) :=
by
  intros x hx
  -- x : α
  -- hx : x ∈ (s \ t) \ u
  -- ⊢ x ∈ s \ (t ∪ u)
  rcases hx with ⟨hxst, hxnu⟩
  -- hxst : x ∈ s \ t
  -- hxnu : ¬x ∈ u
  rcases hxst with ⟨hxs, hxnt⟩
  -- hxs : x ∈ s
  -- hxnt : ¬x ∈ t
  constructor
  . -- ⊢ x ∈ s
    exact hxs
  . -- ⊢ ¬x ∈ t ∪ u
    by_contra hxtu
    -- hxtu : x ∈ t ∪ u
    -- ⊢ False
    rcases hxtu with (hxt | hxu)
    . -- hxt : x ∈ t
      apply hxnt
      -- ⊢ x ∈ t
      exact hxt
    . -- hxu : x ∈ u
      apply hxnu
      -- ⊢ x ∈ u
      exact hxu

-- 2ª demostración
-- ===============

example : (s \ t) \ u ⊆ s \ (t ∪ u) :=
by
  rintro x ⟨⟨hxs, hxnt⟩, hxnu⟩
  -- x : α
  -- hxnu : ¬x ∈ u
  -- hxs : x ∈ s
  -- hxnt : ¬x ∈ t
  -- ⊢ x ∈ s \ (t ∪ u)
  constructor
  . -- ⊢ x ∈ s
    exact hxs
  . -- ⊢ ¬x ∈ t ∪ u
    by_contra hxtu
    -- hxtu : x ∈ t ∪ u
    -- ⊢ False
    rcases hxtu with (hxt | hxu)
    . -- hxt : x ∈ t
      exact hxnt hxt
    . -- hxu : x ∈ u
      exact hxnu hxu

-- 3ª demostración
-- ===============

example : (s \ t) \ u ⊆ s \ (t ∪ u) :=
by
  rintro x ⟨⟨xs, xnt⟩, xnu⟩
  -- x : α
  -- xnu : ¬x ∈ u
  -- xs : x ∈ s
  -- xnt : ¬x ∈ t
  -- ⊢ x ∈ s \ (t ∪ u)
  use xs
  -- ⊢ ¬x ∈ t ∪ u
  rintro (xt | xu)
  . -- xt : x ∈ t
    -- ⊢ False
    contradiction
  . -- xu : x ∈ u
    -- ⊢ False
    contradiction

-- 4ª demostración
-- ===============

example : (s \ t) \ u ⊆ s \ (t ∪ u) :=
by
  rintro x ⟨⟨xs, xnt⟩, xnu⟩
  -- x : α
  -- xnu : ¬x ∈ u
  -- xs : x ∈ s
  -- xnt : ¬x ∈ t
  -- ⊢ x ∈ s \ (t ∪ u)
  use xs
  -- ⊢ ¬x ∈ t ∪ u
  rintro (xt | xu) <;> contradiction

-- 5ª demostración
-- ===============

example : (s \ t) \ u ⊆ s \ (t ∪ u) :=
by
  intro x xstu
  -- x : α
  -- xstu : x ∈ (s \ t) \ u
  -- ⊢ x ∈ s \ (t ∪ u)
  simp at *
  -- ⊢ x ∈ s ∧ ¬(x ∈ t ∨ x ∈ u)
  aesop

-- 6ª demostración
-- ===============

example : (s \ t) \ u ⊆ s \ (t ∪ u) :=
by
  intro x xstu
  -- x : α
  -- xstu : x ∈ (s \ t) \ u
  -- ⊢ x ∈ s \ (t ∪ u)
  aesop

-- 7ª demostración
-- ===============

example : (s \ t) \ u ⊆ s \ (t ∪ u) :=
by rw [diff_diff]

-- Lema usado
-- ==========

-- #check (diff_diff : (s \ t) \ u = s \ (t ∪ u))
\end{verbatim}
Se puede interactuar con las pruebas anteriores en \href{https://lean.math.hhu.de/\#url=https://raw.githubusercontent.com/jaalonso/Calculemus2/main/src/Diferencia\_de\_diferencia\_de\_conjuntos.lean}{Lean 4 Web}

\section{(s ∩ t) ∪ (s ∩ u) ⊆ s ∩ (t ∪ u)}
\label{sec:org203d1e3}
\begin{verbatim}
-- ---------------------------------------------------------------------
-- Demostrar que
--    (s ∩ t) ∪ (s ∩ u) ⊆ s ∩ (t ∪ u)
-- ----------------------------------------------------------------------

-- Demostración en lenguaje natural
-- ================================

-- Sea x ∈ (s ∩ t) ∪ (s ∩ u). Entonces son posibles dos casos.
--
-- 1º caso: Supongamos que x ∈ s ∩ t. Entonces, x ∈ s y x ∈ t (y, por
-- tanto, x ∈ t ∪ u). Luego, x ∈ s ∩ (t ∪ u).
--
-- 2º caso: Supongamos que x ∈ s ∩ u. Entonces, x ∈ s y x ∈ u (y, por
-- tanto, x ∈ t ∪ u). Luego, x ∈ s ∩ (t ∪ u).

-- Demostraciones con Lean4
-- ========================

import Mathlib.Data.Set.Basic
open Set

variable {α : Type}
variable (s t u : Set α)

-- 1ª demostración
-- ===============

example : (s ∩ t) ∪ (s ∩ u) ⊆ s ∩ (t ∪ u):=
by
  intros x hx
  -- x : α
  -- hx : x ∈ s ∩ t ∪ s ∩ u
  -- ⊢ x ∈ s ∩ (t ∪ u)
  rcases hx with (xst | xsu)
  . -- xst : x ∈ s ∩ t
    constructor
    . -- ⊢ x ∈ s
      exact xst.1
    . -- ⊢ x ∈ t ∪ u
      left
      -- ⊢ x ∈ t
      exact xst.2
  . -- xsu : x ∈ s ∩ u
    constructor
    . -- ⊢ x ∈ s
      exact xsu.1
    . -- ⊢ x ∈ t ∪ u
      right
      -- ⊢ x ∈ u
      exact xsu.2

-- 2ª demostración
-- ===============

example : (s ∩ t) ∪ (s ∩ u) ⊆ s ∩ (t ∪ u):=
by
  rintro x (⟨xs, xt⟩ | ⟨xs, xu⟩)
  . -- x : α
    -- xs : x ∈ s
    -- xt : x ∈ t
    -- ⊢ x ∈ s ∩ (t ∪ u)
    use xs
    -- ⊢ x ∈ t ∪ u
    left
    -- ⊢ x ∈ t
    exact xt
  . -- x : α
    -- xs : x ∈ s
    -- xu : x ∈ u
    -- ⊢ x ∈ s ∩ (t ∪ u)
    use xs
    -- ⊢ x ∈ t ∪ u
    right
    -- ⊢ x ∈ u
    exact xu

-- 3ª demostración
-- ===============

example : (s ∩ t) ∪ (s ∩ u) ⊆ s ∩ (t ∪ u):=
by rw [inter_distrib_left s t u]

-- 4ª demostración
-- ===============

example : (s ∩ t) ∪ (s ∩ u) ⊆ s ∩ (t ∪ u):=
by
  intros x hx
  -- x : α
  -- hx : x ∈ s ∩ t ∪ s ∩ u
  -- ⊢ x ∈ s ∩ (t ∪ u)
  aesop
\end{verbatim}
Se puede interactuar con las pruebas anteriores en \href{https://lean.math.hhu.de/\#url=https://raw.githubusercontent.com/jaalonso/Calculemus2/main/src/Propiedad\_semidistributiva\_de\_la\_interseccion\_sobre\_la\_union\_2.lean}{Lean 4 Web}

\section{s $\backslash$ (t ∪ u) ⊆ (s $\backslash$ t) $\backslash$ u}
\label{sec:org8083a24}
\begin{verbatim}
-- ---------------------------------------------------------------------
-- Demostrar que
--    s \ (t ∪ u) ⊆ (s \ t) \ u
-- ----------------------------------------------------------------------

-- Demostración en lenguaje natural
-- ================================

-- Sea x ∈ s \ (t ∪ u). Entonces,
--    x ∈ s                                                          (1)
--    x ∉ t ∪ u                                                      (2)
-- Tenemos que demostrar que x ∈ (s \ t) \ u; es decir, que se verifican
-- las relaciones
--    x ∈ s \ t                                                      (3)
--    x ∉ u                                                          (4)
-- Para demostrar (3) tenemos que demostrar las relaciones
--    x ∈ s                                                          (5)
--    x ∉ t                                                          (6)
-- La (5) se tiene por la (1). Para demostrar la (6), supongamos que
-- x ∈ t; entonces, x ∈ t ∪ u, en contracción con (2). Para demostrar la
-- (4), supongamos que x ∈ u; entonces, x ∈ t ∪ u, en contracción con
-- (2).

-- Demostraciones con Lean4
-- ========================

import Mathlib.Data.Set.Basic
open Set

variable {α : Type}
variable (s t u : Set α)

-- 1ª demostración
-- ===============

example : s \ (t ∪ u) ⊆ (s \ t) \ u :=
by
  intros x hx
  -- x : α
  -- hx : x ∈ s \ (t ∪ u)
  -- ⊢ x ∈ (s \ t) \ u
  constructor
  . -- ⊢ x ∈ s \ t
    constructor
    . -- ⊢ x ∈ s
      exact hx.1
    . -- ⊢ ¬x ∈ t
      intro xt
      -- xt : x ∈ t
      -- ⊢ False
      apply hx.2
      -- ⊢ x ∈ t ∪ u
      left
      -- ⊢ x ∈ t
      exact xt
  . -- ⊢ ¬x ∈ u
    intro xu
    -- xu : x ∈ u
    -- ⊢ False
    apply hx.2
    -- ⊢ x ∈ t ∪ u
    right
    -- ⊢ x ∈ u
    exact xu

-- 2ª demostración
-- ===============

example : s \ (t ∪ u) ⊆ (s \ t) \ u :=
by
  rintro x ⟨xs, xntu⟩
  -- x : α
  -- xs : x ∈ s
  -- xntu : ¬x ∈ t ∪ u
  -- ⊢ x ∈ (s \ t) \ u
  constructor
  . -- ⊢ x ∈ s \ t
    constructor
    . -- ⊢ x ∈ s
      exact xs
    . -- ¬x ∈ t
      intro xt
      -- xt : x ∈ t
      -- ⊢ False
      exact xntu (Or.inl xt)
  . -- ⊢ ¬x ∈ u
    intro xu
    -- xu : x ∈ u
    -- ⊢ False
    exact xntu (Or.inr xu)

-- 2ª demostración
-- ===============

example : s \ (t ∪ u) ⊆ (s \ t) \ u :=
  fun _ ⟨xs, xntu⟩ ↦ ⟨⟨xs, fun xt ↦ xntu (Or.inl xt)⟩,
                      fun xu ↦ xntu (Or.inr xu)⟩

-- 4ª demostración
-- ===============

example : s \ (t ∪ u) ⊆ (s \ t) \ u :=
by
  rintro x ⟨xs, xntu⟩
  -- x : α
  -- xs : x ∈ s
  -- xntu : ¬x ∈ t ∪ u
  -- ⊢ x ∈ (s \ t) \ u
  aesop

-- 5ª demostración
-- ===============

example : s \ (t ∪ u) ⊆ (s \ t) \ u :=
by intro ; aesop

-- 6ª demostración
-- ===============

example : s \ (t ∪ u) ⊆ (s \ t) \ u :=
by rw [diff_diff]

-- Lema usado
-- ==========

-- #check (diff_diff : (s \ t) \ u = s \ (t ∪ u))
\end{verbatim}
Se puede interactuar con las pruebas anteriores en \href{https://lean.math.hhu.de/\#url=https://raw.githubusercontent.com/jaalonso/Calculemus2/main/src/Diferencia\_de\_diferencia\_de\_conjuntos\_2.lean}{Lean 4 Web}

\section{s ∩ t = t ∩ s}
\label{sec:org78f1021}
\begin{verbatim}
-- ---------------------------------------------------------------------
-- Demostrar que
--    s ∩ t = t ∩ s
-- ----------------------------------------------------------------------

-- Demostración en lenguaje natural
-- ================================

-- Tenemos que demostrar que
--    (∀ x)[x ∈ s ∩ t ↔ x ∈ t ∩ s]
-- Demostratemos la equivalencia por la doble implicación.
--
-- Sea x ∈ s ∩ t. Entonces, se tiene
--    x ∈ s                                                          (1)
--    x ∈ t                                                          (2)
-- Luego x ∈ t ∩ s (por (2) y (1)).
--
-- La segunda implicación se demuestra análogamente.

-- Demostraciones con Lean4
-- ========================

import Mathlib.Data.Set.Basic
open Set

variable {α : Type}
variable (s t : Set α)

-- 1ª demostración
-- ===============

example : s ∩ t = t ∩ s :=
by
  ext x
  -- x : α
  -- ⊢ x ∈ s ∩ t ↔ x ∈ t ∩ s
  simp only [mem_inter_iff]
  -- ⊢ x ∈ s ∧ x ∈ t ↔ x ∈ t ∧ x ∈ s
  constructor
  . -- ⊢ x ∈ s ∧ x ∈ t → x ∈ t ∧ x ∈ s
    intro h
    -- h : x ∈ s ∧ x ∈ t
    -- ⊢ x ∈ t ∧ x ∈ s
    constructor
    . -- ⊢ x ∈ t
      exact h.2
    . -- ⊢ x ∈ s
      exact h.1
  . -- ⊢ x ∈ t ∧ x ∈ s → x ∈ s ∧ x ∈ t
    intro h
    -- h : x ∈ t ∧ x ∈ s
    -- ⊢ x ∈ s ∧ x ∈ t
    constructor
    . -- ⊢ x ∈ s
      exact h.2
    . -- ⊢ x ∈ t
      exact h.1

-- 2ª demostración
-- ===============

example : s ∩ t = t ∩ s :=
by
  ext
  -- x : α
  -- ⊢ x ∈ s ∩ t ↔ x ∈ t ∩ s
  simp only [mem_inter_iff]
  -- ⊢ x ∈ s ∧ x ∈ t ↔ x ∈ t ∧ x ∈ s
  exact ⟨fun h ↦ ⟨h.2, h.1⟩,
         fun h ↦ ⟨h.2, h.1⟩⟩

-- 3ª demostración
-- ===============

example : s ∩ t = t ∩ s :=
by
  ext
  -- x : α
  -- ⊢ x ∈ s ∩ t ↔ x ∈ t ∩ s
  exact ⟨fun h ↦ ⟨h.2, h.1⟩,
         fun h ↦ ⟨h.2, h.1⟩⟩

-- 4ª demostración
-- ===============

example : s ∩ t = t ∩ s :=
by
  ext x
  -- x : α
  -- ⊢ x ∈ s ∩ t ↔ x ∈ t ∩ s
  simp only [mem_inter_iff]
  -- ⊢ x ∈ s ∧ x ∈ t ↔ x ∈ t ∧ x ∈ s
  constructor
  . -- ⊢ x ∈ s ∧ x ∈ t → x ∈ t ∧ x ∈ s
    rintro ⟨xs, xt⟩
    -- xs : x ∈ s
    -- xt : x ∈ t
    -- ⊢ x ∈ t ∧ x ∈ s
    exact ⟨xt, xs⟩
  . -- ⊢ x ∈ t ∧ x ∈ s → x ∈ s ∧ x ∈ t
    rintro ⟨xt, xs⟩
    -- xt : x ∈ t
    -- xs : x ∈ s
    -- ⊢ x ∈ s ∧ x ∈ t
    exact ⟨xs, xt⟩

-- 5ª demostración
-- ===============

example : s ∩ t = t ∩ s :=
by
  ext x
  -- x : α
  -- ⊢ x ∈ s ∩ t ↔ x ∈ t ∩ s
  simp only [mem_inter_iff]
  -- ⊢ x ∈ s ∧ x ∈ t ↔ x ∈ t ∧ x ∈ s
  simp only [And.comm]

-- 6ª demostración
-- ===============

example : s ∩ t = t ∩ s :=
ext (fun _ ↦ And.comm)

-- 7ª demostración
-- ===============

example : s ∩ t = t ∩ s :=
by ext ; simp [And.comm]

-- 8ª demostración
-- ===============

example : s ∩ t = t ∩ s :=
inter_comm s t

-- Lemas usados
-- ============

-- variable (x : α)
-- variable (a b : Prop)
-- #check (And.comm : a ∧ b ↔ b ∧ a)
-- #check (inter_comm s t : s ∩ t = t ∩ s)
-- #check (mem_inter_iff x s t : x ∈ s ∩ t ↔ x ∈ s ∧ x ∈ t)
\end{verbatim}
Se puede interactuar con las pruebas anteriores en \href{https://lean.math.hhu.de/\#url=https://raw.githubusercontent.com/jaalonso/Calculemus2/main/src/Conmutatividad\_de\_la\_interseccion.lean}{Lean 4 Web}

\section{s ∩ (s ∪ t) = s}
\label{sec:orgd2600a0}
\begin{verbatim}
-- ---------------------------------------------------------------------
-- Demostrar que
--    s ∩ (s ∪ t) = s
-- ----------------------------------------------------------------------

-- Demostación en lenguaje natural
-- ===============================

-- Tenemos que demostrar que
--    (∀ x)[x ∈ s ∩ (s ∪ t) ↔ x ∈ s]
-- y lo haremos demostrando las dos implicaciones.
--
-- (⟹) Sea x ∈ s ∩ (s ∪ t). Entonces, x ∈ s.
--
-- (⟸) Sea x ∈ s. Entonces, x ∈ s ∪ t y, por tanto,
-- x ∈ s ∩ (s ∪ t).

-- Demostraciones con Lean4
-- ========================

import Mathlib.Data.Set.Basic
import Mathlib.Tactic
open Set

variable {α : Type}
variable (s t : Set α)

-- 1ª demostración
-- ===============

example : s ∩ (s ∪ t) = s :=
by
  ext x
  -- x : α
  -- ⊢ x ∈ s ∩ (s ∪ t) ↔ x ∈ s
  constructor
  . -- ⊢ x ∈ s ∩ (s ∪ t) → x ∈ s
    intros h
  -- h : x ∈ s ∩ (s ∪ t)
  -- ⊢ x ∈ s
    exact h.1
  . -- ⊢ x ∈ s → x ∈ s ∩ (s ∪ t)
    intro xs
    -- xs : x ∈ s
    -- ⊢ x ∈ s ∩ (s ∪ t)
    constructor
    . -- ⊢ x ∈ s
      exact xs
    . -- ⊢ x ∈ s ∪ t
      left
      -- ⊢ x ∈ s
      exact xs

-- 2ª demostración
-- ===============

example : s ∩ (s ∪ t) = s :=
by
  ext x
  -- x : α
  -- ⊢ x ∈ s ∩ (s ∪ t) ↔ x ∈ s
  constructor
  . -- ⊢ x ∈ s ∩ (s ∪ t) → x ∈ s
    intro h
    -- h : x ∈ s ∩ (s ∪ t)
    -- ⊢ x ∈ s
    exact h.1
  . -- ⊢ x ∈ s → x ∈ s ∩ (s ∪ t)
    intro xs
    -- xs : x ∈ s
    -- ⊢ x ∈ s ∩ (s ∪ t)
    constructor
    . -- ⊢ x ∈ s
      exact xs
    . -- ⊢ x ∈ s ∪ t
      exact (Or.inl xs)

-- 3ª demostración
-- ===============

example : s ∩ (s ∪ t) = s :=
by
  ext
  -- x : α
  -- ⊢ x ∈ s ∩ (s ∪ t) ↔ x ∈ s
  exact ⟨fun h ↦ h.1,
         fun xs ↦ ⟨xs, Or.inl xs⟩⟩

-- 4ª demostración
-- ===============

example : s ∩ (s ∪ t) = s :=
by
  ext
  -- x : α
  -- ⊢ x ∈ s ∩ (s ∪ t) ↔ x ∈ s
  exact ⟨And.left,
         fun xs ↦ ⟨xs, Or.inl xs⟩⟩

-- 5ª demostración
-- ===============

example : s ∩ (s ∪ t) = s :=
by
  ext x
  -- x : α
  -- ⊢ x ∈ s ∩ (s ∪ t) ↔ x ∈ s
  constructor
  . -- ⊢ x ∈ s ∩ (s ∪ t) → x ∈ s
    rintro ⟨xs, -⟩
    -- xs : x ∈ s
    -- ⊢ x ∈ s
    exact xs
  . -- ⊢ x ∈ s → x ∈ s ∩ (s ∪ t)
    intro xs
    -- xs : x ∈ s
    -- ⊢ x ∈ s ∩ (s ∪ t)
    use xs
    -- ⊢ x ∈ s ∪ t
    left
    -- ⊢ x ∈ s
    exact xs

-- 6ª demostración
-- ===============

example : s ∩ (s ∪ t) = s :=
by
  apply subset_antisymm
  . -- ⊢ s ∩ (s ∪ t) ⊆ s
    rintro x ⟨hxs, -⟩
    -- x : α
    -- hxs : x ∈ s
    -- ⊢ x ∈ s
    exact hxs
  . -- ⊢ s ⊆ s ∩ (s ∪ t)
    intros x hxs
    -- x : α
    -- hxs : x ∈ s
    -- ⊢ x ∈ s ∩ (s ∪ t)
    exact ⟨hxs, Or.inl hxs⟩

-- 7ª demostración
-- ===============

example : s ∩ (s ∪ t) = s :=
inf_sup_self

-- 8ª demostración
-- ===============

example : s ∩ (s ∪ t) = s :=
by aesop

-- Lemas usados
-- ============

-- variable (a b : Prop)
-- #check (And.left : a ∧ b → a)
-- #check (Or.inl : a → a ∨ b)
-- #check (inf_sup_self : s ∩ (s ∪ t) = s)
-- #check (subset_antisymm : s ⊆ t → t ⊆ s → s = t)
\end{verbatim}
Se puede interactuar con las pruebas anteriores en \href{https://lean.math.hhu.de/\#url=https://raw.githubusercontent.com/jaalonso/Calculemus2/main/src/Interseccion\_con\_su\_union.lean}{Lean 4 Web}

\section{s ∪ (s ∩ t) = s}
\label{sec:org499eaf1}
\begin{verbatim}
-- ---------------------------------------------------------------------
-- Demostrar que
--    s ∪ (s ∩ t) = s
-- ----------------------------------------------------------------------

-- Demostración en lenguaje natural
-- ================================

-- Tenemos que demostrar que
--    (∀ x)[x ∈ s ∪ (s ∩ t) ↔ x ∈ s]
-- y lo haremos demostrando las dos implicaciones.
--
-- (⟹) Sea x ∈ s ∪ (s ∩ t). Entonces, c ∈ s o x ∈ s ∩ t. En ambos casos,
-- x ∈ s.
--
-- (⟸) Sea x ∈ s. Entonces, x ∈ s ∩ t y, por tanto, x ∈ s ∪ (s ∩ t).

-- Demostraciones con Lean4
-- ========================

import Mathlib.Data.Set.Basic
open Set

variable {α : Type}
variable (s t : Set α)

-- 1ª demostración
-- ===============

example : s ∪ (s ∩ t) = s :=
by
  ext x
  -- x : α
  -- ⊢ x ∈ s ∪ (s ∩ t) ↔ x ∈ s
  constructor
  . -- ⊢ x ∈ s ∪ (s ∩ t) → x ∈ s
    intro hx
    -- hx : x ∈ s ∪ (s ∩ t)
    -- ⊢ x ∈ s
    rcases hx with (xs | xst)
    . -- xs : x ∈ s
      exact xs
    . -- xst : x ∈ s ∩ t
      exact xst.1
  . -- ⊢ x ∈ s → x ∈ s ∪ (s ∩ t)
    intro xs
    -- xs : x ∈ s
    -- ⊢ x ∈ s ∪ (s ∩ t)
    left
    -- ⊢ x ∈ s
    exact xs

-- 2ª demostración
-- ===============

example : s ∪ (s ∩ t) = s :=
by
  ext x
  -- x : α
  -- ⊢ x ∈ s ∪ s ∩ t ↔ x ∈ s
  exact ⟨fun hx ↦ Or.elim hx id And.left,
         fun xs ↦ Or.inl xs⟩

-- 3ª demostración
-- ===============

example : s ∪ (s ∩ t) = s :=
by
  ext x
  -- x : α
  -- ⊢ x ∈ s ∪ (s ∩ t) ↔ x ∈ s
  constructor
  . -- ⊢ x ∈ s ∪ (s ∩ t) → x ∈ s
    rintro (xs | ⟨xs, -⟩) <;>
    -- xs : x ∈ s
    -- ⊢ x ∈ s
    exact xs
  . -- ⊢ x ∈ s → x ∈ s ∪ (s ∩ t)
    intro xs
    -- xs : x ∈ s
    -- ⊢ x ∈ s ∪ s ∩ t
    left
    -- ⊢ x ∈ s
    exact xs

-- 4ª demostración
-- ===============

example : s ∪ (s ∩ t) = s :=
sup_inf_self

-- Lemas usados
-- ============

-- variable (a b c : Prop)
-- #check (And.left : a ∧ b → a)
-- #check (Or.elim : a ∨ b → (a → c) → (b → c) → c)
-- #check (sup_inf_self : s ∪ (s ∩ t) = s)
\end{verbatim}
Se puede interactuar con las pruebas anteriores en \href{https://lean.math.hhu.de/\#url=https://raw.githubusercontent.com/jaalonso/Calculemus2/main/src/Union\_con\_su\_interseccion.lean}{Lean 4 Web}

\section{(s $\backslash$ t) ∪ t = s ∪ t}
\label{sec:org515f227}
\begin{verbatim}
-- ---------------------------------------------------------------------
-- Demostrar que
--    (s \ t) ∪ t = s ∪ t
-- ----------------------------------------------------------------------

-- Demostración en lenguaje natural
-- ================================

-- Tenemos que demostrar que
--    (∀ x)[x ∈ (s \ t) ∪ t ↔ x ∈ s ∪ t]
-- y lo demostraremos por la siguiente cadena de equivalencias:
--    x ∈ (s \ t) ∪ t ↔ x ∈ (s \ t) ∨ (x ∈ t)
--                    ↔ (x ∈ s ∧ x ∉ t) ∨ x ∈ t
--                    ↔ (x ∈ s ∨ x ∈ t) ∧ (x ∉ t ∨ x ∈ t)
--                    ↔ x ∈ s ∨ x ∈ t
--                    ↔ x ∈ s ∪ t

-- Demostraciones con Lean4
-- ========================

import Mathlib.Data.Set.Basic
open Set

variable {α : Type}
variable (s t : Set α)

-- 1ª demostración
-- ===============

example : (s \ t) ∪ t = s ∪ t :=
by
  ext x
  -- x : α
  -- ⊢ x ∈ (s \ t) ∪ t ↔ x ∈ s ∪ t
  calc x ∈ (s \ t) ∪ t
       ↔ x ∈ s \ t ∨ x ∈ t                 := mem_union x (s \ t) t
     _ ↔ (x ∈ s ∧ x ∉ t) ∨ x ∈ t           := by simp only [mem_diff x]
     _ ↔ (x ∈ s ∨ x ∈ t) ∧ (x ∉ t ∨ x ∈ t) := and_or_right
     _ ↔ (x ∈ s ∨ x ∈ t) ∧ True            := by simp only [em' (x ∈ t)]
     _ ↔ x ∈ s ∨ x ∈ t                     := and_true_iff (x ∈ s ∨ x ∈ t)
     _ ↔ x ∈ s ∪ t                         := (mem_union x s t).symm

-- 2ª demostración
-- ===============

example : (s \ t) ∪ t = s ∪ t :=
by
  ext x
  -- x : α
  -- ⊢ x ∈ (s \ t) ∪ t ↔ x ∈ s ∪ t
  constructor
  . -- ⊢ x ∈ (s \ t) ∪ t → x ∈ s ∪ t
    intro hx
    -- hx : x ∈ (s \ t) ∪ t
    -- ⊢ x ∈ s ∪ t
    rcases hx with (xst | xt)
    . -- xst : x ∈ s \ t
      -- ⊢ x ∈ s ∪ t
      left
      -- ⊢ x ∈ s
      exact xst.1
    . -- xt : x ∈ t
      -- ⊢ x ∈ s ∪ t
      right
      -- ⊢ x ∈ t
      exact xt
  . -- ⊢ x ∈ s ∪ t → x ∈ (s \ t) ∪ t
    by_cases h : x ∈ t
    . -- h : x ∈ t
      intro _xst
      -- _xst : x ∈ s ∪ t
      right
      -- ⊢ x ∈ t
      exact h
    . -- ⊢ x ∈ s ∪ t → x ∈ (s \ t) ∪ t
      intro hx
      -- hx : x ∈ s ∪ t
      -- ⊢ x ∈ (s \ t) ∪ t
      rcases hx with (xs | xt)
      . -- xs : x ∈ s
        left
        -- ⊢ x ∈ s \ t
        constructor
        . -- ⊢ x ∈ s
          exact xs
        . -- ⊢ ¬x ∈ t
          exact h
      . -- xt : x ∈ t
        right
        -- ⊢ x ∈ t
        exact xt

-- 3ª demostración
-- ===============

example : (s \ t) ∪ t = s ∪ t :=
by
  ext x
  -- x : α
  -- ⊢ x ∈ (s \ t) ∪ t ↔ x ∈ s ∪ t
  constructor
  . -- ⊢ x ∈ (s \ t) ∪ t → x ∈ s ∪ t
    rintro (⟨xs, -⟩ | xt)
    . -- xs : x ∈ s
      -- ⊢ x ∈ s ∪ t
      left
      -- ⊢ x ∈ s
      exact xs
    . -- xt : x ∈ t
      -- ⊢ x ∈ s ∪ t
      right
      -- ⊢ x ∈ t
      exact xt
  . -- ⊢ x ∈ s ∪ t → x ∈ (s \ t) ∪ t
    by_cases h : x ∈ t
    . -- h : x ∈ t
      intro _xst
      -- _xst : x ∈ s ∪ t
      -- ⊢ x ∈ (s \ t) ∪ t
      right
      -- ⊢ x ∈ t
      exact h
    . -- ⊢ x ∈ s ∪ t → x ∈ (s \ t) ∪ t
      rintro (xs | xt)
      . -- xs : x ∈ s
        -- ⊢ x ∈ (s \ t) ∪ t
        left
        -- ⊢ x ∈ s \ t
        exact ⟨xs, h⟩
      . -- xt : x ∈ t
        -- ⊢ x ∈ (s \ t) ∪ t
        right
        -- ⊢ x ∈ t
        exact xt

-- 4ª demostración
-- ===============

example : (s \ t) ∪ t = s ∪ t :=
diff_union_self

-- 5ª demostración
-- ===============

example : (s \ t) ∪ t = s ∪ t :=
by
  ext
  -- x : α
  -- ⊢ x ∈ s \ t ∪ t ↔ x ∈ s ∪ t
  simp

-- 6ª demostración
-- ===============

example : (s \ t) ∪ t = s ∪ t :=
by simp

-- Lemas usados
-- ============

-- variable (a b c : Prop)
-- variable (x : α)
-- #check (and_or_right : (a ∧ b) ∨ c ↔ (a ∨ c) ∧ (b ∨ c))
-- #check (and_true_iff a : a ∧ True ↔ a)
-- #check (diff_union_self : (s \ t) ∪ t = s ∪ t)
-- #check (em' a : ¬a ∨ a)
-- #check (mem_diff x : x ∈ s \ t ↔ x ∈ s ∧ x ∉ t)
-- #check (mem_union x s t : x ∈ s ∪ t ↔ x ∈ s ∨ x ∈ t)
\end{verbatim}
Se puede interactuar con las pruebas anteriores en \href{https://lean.math.hhu.de/\#url=https://raw.githubusercontent.com/jaalonso/Calculemus2/main/src/Union\_con\_su\_diferencia.lean}{Lean 4 Web}

\section{(s $\backslash$ t) ∪ (t $\backslash$ s) = (s ∪ t) $\backslash$ (s ∩ t)}
\label{sec:orgbca52e7}
\begin{verbatim}
-- ---------------------------------------------------------------------
-- Demostrar que
--    (s \ t) ∪ (t \ s) = (s ∪ t) \ (s ∩ t)
-- ----------------------------------------------------------------------

-- Demostración en lenguaje natural
-- ================================

-- Tenemos que demostrar que, para todo x,
--    x ∈ (s \ t) ∪ (t \ s) ↔ x ∈ (s ∪ t) \ (s ∩ t)
-- Se demuestra mediante la siguiente cadena de equivalencias:
--    x ∈ (s \ t) ∪ (t \ s)
--    ↔ x ∈ (s \ t) ∨ x ∈ (t \ s)
--    ↔ (x ∈ s ∧ x ∉ t) ∨ x ∈ (t \ s)
--    ↔ (x ∈ s ∨ x ∈ (t \ s)) ∧ (x ∉ t ∨ x ∈ (t \ s))
--    ↔ (x ∈ s ∨ (x ∈ t ∧ x ∉ s)) ∧ (x ∉ t ∨ (x ∈ t ∧ x ∉ s))
--    ↔ ((x ∈ s ∨ x ∈ t) ∧ (x ∈ s ∨ x ∉ s)) ∧ ((x ∉ t ∨ x ∈ t) ∧ (x ∉ t ∨ x ∉ s))
--    ↔ ((x ∈ s ∨ x ∈ t) ∧ True) ∧ (True ∧ (x ∉ t ∨ x ∉ s))
--    ↔ (x ∈ s ∨ x ∈ t) ∧ (x ∉ t ∨ x ∉ s)
--    ↔ (x ∈ s ∪ t) ∧ (x ∉ t ∨ x ∉ s)
--    ↔ (x ∈ s ∪ t) ∧ (x ∉ s ∨ x ∉ t)
--    ↔ (x ∈ s ∪ t) ∧ ¬(x ∈ s ∧ x ∈ t)
--    ↔ (x ∈ s ∪ t) ∧ ¬(x ∈ s ∩ t)
--    ↔ x ∈ (s ∪ t) \ (s ∩ t)

-- Demostraciones con Lean4
-- ========================

import Mathlib.Data.Set.Basic
open Set

variable {α : Type}
variable (s t : Set α)

-- 1ª demostración
-- ===============

example : (s \ t) ∪ (t \ s) = (s ∪ t) \ (s ∩ t) :=
by
  ext x
  -- x : α
  -- ⊢ x ∈ (s \ t) ∪ (t \ s) ↔ x ∈ (s ∪ t) \ (s ∩ t)
  calc x ∈ (s \ t) ∪ (t \ s)
     ↔ x ∈ (s \ t) ∨ x ∈ (t \ s) :=
         by exact mem_union x (s \ t) (t \ s)
   _ ↔ (x ∈ s ∧ x ∉ t) ∨ x ∈ (t \ s) :=
         by simp only [mem_diff]
   _ ↔ (x ∈ s ∨ x ∈ (t \ s)) ∧ (x ∉ t ∨ x ∈ (t \ s)) :=
         by exact and_or_right
   _ ↔ (x ∈ s ∨ (x ∈ t ∧ x ∉ s)) ∧ (x ∉ t ∨ (x ∈ t ∧ x ∉ s)) :=
         by simp only [mem_diff]
   _ ↔ ((x ∈ s ∨ x ∈ t) ∧ (x ∈ s ∨ x ∉ s)) ∧
       ((x ∉ t ∨ x ∈ t) ∧ (x ∉ t ∨ x ∉ s)) :=
         by simp_all only [or_and_left]
   _ ↔ ((x ∈ s ∨ x ∈ t) ∧ True) ∧
       (True ∧ (x ∉ t ∨ x ∉ s)) :=
         by simp only [em (x ∈ s), em' (x ∈ t)]
   _ ↔ (x ∈ s ∨ x ∈ t) ∧ (x ∉ t ∨ x ∉ s) :=
         by simp only [and_true_iff (x ∈ s ∨ x ∈ t),
                       true_and_iff (¬x ∈ t ∨ ¬x ∈ s)]
   _ ↔ (x ∈ s ∪ t) ∧ (x ∉ t ∨ x ∉ s) :=
         by simp only [mem_union]
   _ ↔ (x ∈ s ∪ t) ∧ (x ∉ s ∨ x ∉ t) :=
         by simp only [or_comm]
   _ ↔ (x ∈ s ∪ t) ∧ ¬(x ∈ s ∧ x ∈ t) :=
         by simp only [not_and_or]
   _ ↔ (x ∈ s ∪ t) ∧ ¬(x ∈ s ∩ t) :=
         by simp only [mem_inter_iff]
   _ ↔ x ∈ (s ∪ t) \ (s ∩ t)     :=
         by simp only [mem_diff]

-- 2ª demostración
-- ===============

example : (s \ t) ∪ (t \ s) = (s ∪ t) \ (s ∩ t) :=
by
  ext x
  -- x : α
  -- ⊢ x ∈ (s \ t) ∪ (t \ s) ↔ x ∈ (s ∪ t) \ (s ∩ t)
  constructor
  . -- ⊢ x ∈ (s \ t) ∪ (t \ s) → x ∈ (s ∪ t) \ (s ∩ t)
    rintro (⟨xs, xnt⟩ | ⟨xt, xns⟩)
    . -- xs : x ∈ s
      -- xnt : ¬x ∈ t
      -- ⊢ x ∈ (s ∪ t) \ (s ∩ t)
      constructor
      . -- ⊢ x ∈ s ∪ t
        left
        -- ⊢ x ∈ s
        exact xs
      . -- ⊢ ¬x ∈ s ∩ t
        rintro ⟨-, xt⟩
        -- xt : x ∈ t
        -- ⊢ False
        exact xnt xt
    . -- xt : x ∈ t
      -- xns : ¬x ∈ s
      -- ⊢ x ∈ (s ∪ t) \ (s ∩ t)
      constructor
      . -- ⊢ x ∈ s ∪ t
        right
        -- ⊢ x ∈ t
        exact xt
      . -- ⊢ ¬x ∈ s ∩ t
        rintro ⟨xs, -⟩
        -- xs : x ∈ s
        -- ⊢ False
        exact xns xs
  . -- ⊢ x ∈ (s ∪ t) \ (s ∩ t) → x ∈ (s \ t) ∪ (t \ s)
    rintro ⟨xs | xt, nxst⟩
    . -- xs : x ∈ s
      -- ⊢ x ∈ (s \ t) ∪ (t \ s)
      left
      -- ⊢ x ∈ s \ t
      use xs
      -- ⊢ ¬x ∈ t
      intro xt
      -- xt : x ∈ t
      -- ⊢ False
      apply nxst
      -- ⊢ x ∈ s ∩ t
      constructor
      . -- ⊢ x ∈ s
        exact xs
      . -- ⊢ x ∈ t
        exact xt
    . -- nxst : ¬x ∈ s ∩ t
      -- xt : x ∈ t
      -- ⊢ x ∈ (s \ t) ∪ (t \ s)
      right
      -- ⊢ x ∈ t \ s
      use xt
      -- ⊢ ¬x ∈ s
      intro xs
      -- xs : x ∈ s
      -- ⊢ False
      apply nxst
      -- ⊢ x ∈ s ∩ t
      constructor
      . -- ⊢ x ∈ s
        exact xs
      . -- ⊢ x ∈ t
        exact xt

-- 3ª demostración
-- ===============

example : (s \ t) ∪ (t \ s) = (s ∪ t) \ (s ∩ t) :=
by
  ext x
  -- x : α
  -- ⊢ x ∈ (s \ t) ∪ (t \ s) ↔ x ∈ (s ∪ t) \ (s ∩ t)
  constructor
  . -- ⊢ x ∈ (s \ t) ∪ (t \ s) → x ∈ (s ∪ t) \ (s ∩ t)
    rintro (⟨xs, xnt⟩ | ⟨xt, xns⟩)
    . -- xt : x ∈ t
      -- xns : ¬x ∈ s
      -- ⊢ x ∈ (s ∪ t) \ (s ∩ t)
      aesop
    . -- xt : x ∈ t
      -- xns : ¬x ∈ s
      -- ⊢ x ∈ (s ∪ t) \ (s ∩ t)
      aesop
  . rintro ⟨xs | xt, nxst⟩
    . -- xs : x ∈ s
      -- ⊢ x ∈ (s \ t) ∪ (t \ s)
      aesop
    . -- nxst : ¬x ∈ s ∩ t
      -- xt : x ∈ t
      -- ⊢ x ∈ (s \ t) ∪ (t \ s)
      aesop

-- 4ª demostración
-- ===============

example : (s \ t) ∪ (t \ s) = (s ∪ t) \ (s ∩ t) :=
by
  ext x
  -- x : α
  -- ⊢ x ∈ (s \ t) ∪ (t \ s) ↔ x ∈ (s ∪ t) \ (s ∩ t)
  constructor
  . -- ⊢ x ∈ (s \ t) ∪ (t \ s) → x ∈ (s ∪ t) \ (s ∩ t)
    rintro (⟨xs, xnt⟩ | ⟨xt, xns⟩) <;> aesop
  . -- ⊢ x ∈ (s ∪ t) \ (s ∩ t) → x ∈ (s \ t) ∪ (t \ s)
    rintro ⟨xs | xt, nxst⟩ <;> aesop

-- 5ª demostración
-- ===============

example : (s \ t) ∪ (t \ s) = (s ∪ t) \ (s ∩ t) :=
by
  ext
  constructor
  . aesop
  . aesop

-- 6ª demostración
-- ===============

example : (s \ t) ∪ (t \ s) = (s ∪ t) \ (s ∩ t) :=
by
  ext
  constructor <;> aesop

-- 7ª demostración
-- ===============

example : (s \ t) ∪ (t \ s) = (s ∪ t) \ (s ∩ t) :=
by
  rw [ext_iff]
  -- ⊢ ∀ (x : α), x ∈ (s \ t) ∪ (t \ s) ↔ x ∈ (s ∪ t) \ (s ∩ t)
  intro
  -- x : α
  -- ⊢ x ∈ (s \ t) ∪ (t \ s) ↔ x ∈ (s ∪ t) \ (s ∩ t)
  rw [iff_def]
  -- ⊢ (x ∈ (s \ t) ∪ (t \ s) → x ∈ (s ∪ t) \ (s ∩ t)) ∧
  --   (x ∈ (s ∪ t) \ (s ∩ t) → x ∈ (s \ t) ∪ (t \ s))
  aesop

-- Lemas usados
-- ============

-- variable (x : α)
-- variable (a b c : Prop)
-- #check (mem_union x s t : x ∈ s ∪ t ↔ x ∈ s ∨ x ∈ t)
-- #check (mem_diff x : x ∈ s \ t ↔ x ∈ s ∧ ¬x ∈ t)
-- #check (and_or_right : (a ∧ b) ∨ c ↔ (a ∨ c) ∧ (b ∨ c))
-- #check (or_and_left : a ∨ (b ∧ c) ↔ (a ∨ b) ∧ (a ∨ c))
-- #check (em a : a ∨ ¬ a)
-- #check (em' a : ¬ a ∨ a)
-- #check (and_true_iff a : a ∧ True ↔ a)
-- #check (true_and_iff a : True ∧ a ↔ a)
-- #check (or_comm : a ∨ b ↔ b ∨ a)
-- #check (not_and_or : ¬(a ∧ b) ↔ ¬a ∨ ¬b)
-- #check (mem_inter_iff x s t : x ∈ s ∩ t ↔ x ∈ s ∧ x ∈ t)
-- #check (ext_iff : s = t ↔ ∀ (x : α), x ∈ s ↔ x ∈ t)
-- #check (iff_def : (a ↔ b) ↔ (a → b) ∧ (b → a))
\end{verbatim}
Se puede interactuar con las pruebas anteriores en \href{https://lean.math.hhu.de/\#url=https://raw.githubusercontent.com/jaalonso/Calculemus2/main/src/Diferencia\_de\_union\_e\_interseccion.lean}{Lean 4 Web}

\section{Pares ∪ Impares = Naturales}
\label{sec:orgc724441}
\begin{verbatim}
-- ---------------------------------------------------------------------
-- Los conjuntos de los números naturales, de los pares y de los impares
-- se definen por
--    def Naturales : Set ℕ := {n | True}
--    def Pares     : Set ℕ := {n | Even n}
--    def Impares   : Set ℕ := {n | ¬Even n}
--
-- Demostrar que
--    Pares ∪ Impares = Naturales
-- ----------------------------------------------------------------------

-- Demostración en lenguaje natural
-- ================================

-- Tenemos que demostrar que
--    {n | Even n} ∪ {n | ¬Even n} = {n | True}
-- es decir,
--    n ∈ {n | Even n} ∪ {n | ¬Even n} ↔ n ∈ {n | True}
-- que se reduce a
--    ⊢ Even n ∨ ¬Even n
-- que es una tautología.

-- Demostraciones con Lean4
-- ========================

import Mathlib.Data.Nat.Parity
open Set

def Naturales : Set ℕ := {n | True}
def Pares     : Set ℕ := {n | Even n}
def Impares   : Set ℕ := {n | ¬Even n}

-- 1ª demostración
-- ===============

example : Pares ∪ Impares = Naturales :=
by
  unfold Pares Impares Naturales
  -- ⊢ {n | Even n} ∪ {n | ¬Even n} = {n | True}
  ext n
  -- ⊢ n ∈ {n | Even n} ∪ {n | ¬Even n} ↔ n ∈ {n | True}
  simp
  -- ⊢ Even n ∨ ¬Even n
  exact em (Even n)

-- 2ª demostración
-- ===============

example : Pares ∪ Impares = Naturales :=
by
  unfold Pares Impares Naturales
  -- ⊢ {n | Even n} ∪ {n | ¬Even n} = {n | True}
  ext n
  -- ⊢ n ∈ {n | Even n} ∪ {n | ¬Even n} ↔ n ∈ {n | True}
  tauto
\end{verbatim}
Se puede interactuar con las pruebas anteriores en \href{https://lean.math.hhu.de/\#url=https://raw.githubusercontent.com/jaalonso/Calculemus2/main/src/Union\_de\_pares\_e\_impares.lean}{Lean 4 Web}

\section{Los primos mayores que 2 son impares}
\label{sec:orgcb1d767}
\begin{verbatim}
-- ---------------------------------------------------------------------
-- Los números primos, los mayores que 2 y los impares se definen por
--    def Primos      : Set ℕ := {n | Nat.Prime n}
--    def MayoresQue2 : Set ℕ := {n | n > 2}
--    def Impares     : Set ℕ := {n | ¬Even n}
--
-- Demostrar que
--    Primos ∩ MayoresQue2 ⊆ Impares
-- ----------------------------------------------------------------------

-- Demostraciones con Lean4
-- ========================

import Mathlib.Data.Nat.Parity
import Mathlib.Data.Nat.Prime
import Mathlib.Tactic

open Nat

def Primos      : Set ℕ := {n | Nat.Prime n}
def MayoresQue2 : Set ℕ := {n | n > 2}
def Impares     : Set ℕ := {n | ¬Even n}

-- 1ª demostración
-- ===============

example : Primos ∩ MayoresQue2 ⊆ Impares :=
by
  unfold Primos MayoresQue2 Impares
  -- ⊢ {n | Nat.Prime n} ∩ {n | n > 2} ⊆ {n | ¬Even n}
  intro n
  -- n : ℕ
  -- ⊢ n ∈ {n | Nat.Prime n} ∩ {n | n > 2} → n ∈ {n | ¬Even n}
  simp
  -- ⊢ Nat.Prime n → 2 < n → ¬Even n
  intro hn
  -- hn : Nat.Prime n
  -- ⊢ 2 < n → ¬Even n
  rcases Prime.eq_two_or_odd hn with (h | h)
  . -- h : n = 2
    rw [h]
    -- ⊢ 2 < 2 → ¬Even 2
    intro h1
    -- h1 : 2 < 2
    -- ⊢ ¬Even 2
    exfalso
    exact absurd h1 (lt_irrefl 2)
  . -- h : n % 2 = 1
    rw [even_iff]
    -- ⊢ 2 < n → ¬n % 2 = 0
    rw [h]
    -- ⊢ 2 < n → ¬1 = 0
    intro
    -- a : 2 < n
    -- ⊢ ¬1 = 0
    exact one_ne_zero

-- 2ª demostración
-- ===============

example : Primos ∩ MayoresQue2 ⊆ Impares :=
by
  unfold Primos MayoresQue2 Impares
  -- ⊢ {n | Nat.Prime n} ∩ {n | n > 2} ⊆ {n | ¬Even n}
  rintro n ⟨h1, h2⟩
  -- n : ℕ
  -- h1 : n ∈ {n | Nat.Prime n}
  -- h2 : n ∈ {n | n > 2}
  -- ⊢ n ∈ {n | ¬Even n}
  simp at *
  -- h1 : Nat.Prime n
  -- h2 : 2 < n
  -- ⊢ ¬Even n
  rcases Prime.eq_two_or_odd h1 with (h3 | h4)
  . -- h3 : n = 2
    rw [h3] at h2
    -- h2 : 2 < 2
    exfalso
    -- ⊢ False
    exact absurd h2 (lt_irrefl 2)
  . -- h4 : n % 2 = 1
    rw [even_iff]
    -- ⊢ ¬n % 2 = 0
    rw [h4]
    -- ⊢ ¬1 = 0
    exact one_ne_zero

-- 3ª demostración
-- ===============

example : Primos ∩ MayoresQue2 ⊆ Impares :=
by
  unfold Primos MayoresQue2 Impares
  -- ⊢ {n | Nat.Prime n} ∩ {n | n > 2} ⊆ {n | ¬Even n}
  rintro n ⟨h1, h2⟩
  -- n : ℕ
  -- h1 : n ∈ {n | Nat.Prime n}
  -- h2 : n ∈ {n | n > 2}
  -- ⊢ n ∈ {n | ¬Even n}
  simp at *
  -- h1 : Nat.Prime n
  -- h2 : 2 < n
  -- ⊢ ¬Even n
  rcases Prime.eq_two_or_odd h1 with (h3 | h4)
  . -- h3 : n = 2
    rw [h3] at h2
    -- h2 : 2 < 2
    linarith
  . -- h4 : n % 2 = 1
    rw [even_iff]
    -- ⊢ ¬n % 2 = 0
    linarith

-- Lemas usados
-- ============

-- variable (p n : ℕ)
-- variable (a b : Prop)
-- #check (Prime.eq_two_or_odd : Nat.Prime p → p = 2 ∨ p % 2 = 1)
-- #check (absurd : a → ¬a → b)
-- #check (even_iff : Even n ↔ n % 2 = 0)
-- #check (lt_irrefl n : ¬n < n)
-- #check (one_ne_zero : 1 ≠ 0)
\end{verbatim}
Se puede interactuar con las pruebas anteriores en \href{https://lean.math.hhu.de/\#url=https://raw.githubusercontent.com/jaalonso/Calculemus2/main/src/Interseccion\_de\_los\_primos\_y\_los\_mayores\_que\_dos.lean}{Lean 4 Web}

\section{s ∩ (⋃ i, A i) = ⋃ i, (A i ∩ s)}
\label{sec:org017e06e}
\begin{verbatim}
-- ---------------------------------------------------------------------
-- Demostrar que
--    s ∩ (⋃ i, A i) = ⋃ i, (A i ∩ s)
-- ----------------------------------------------------------------------

-- Demostración en lenguaje natural
-- ================================

-- Tenemos que demostrar que para cada x, se verifica que
--    x ∈ s ∩ ⋃ (i : ℕ), A i ↔ x ∈ ⋃ (i : ℕ), A i ∩ s
-- Lo demostramos mediante la siguiente cadena de equivalencias
--    x ∈ s ∩ ⋃ (i : ℕ), A i ↔ x ∈ s ∧ x ∈ ⋃ (i : ℕ), A i
--                           ↔ x ∈ s ∧ (∃ i : ℕ, x ∈ A i)
--                           ↔ ∃ i : ℕ, x ∈ s ∧ x ∈ A i
--                           ↔ ∃ i : ℕ, x ∈ A i ∧ x ∈ s
--                           ↔ ∃ i : ℕ, x ∈ A i ∩ s
--                           ↔ x ∈ ⋃ (i : ℕ), A i ∩ s

-- Demostraciones con Lean4
-- ========================

import Mathlib.Data.Set.Basic
import Mathlib.Data.Set.Lattice
import Mathlib.Tactic

open Set

variable {α : Type}
variable (s : Set α)
variable (A : ℕ → Set α)

-- 1ª demostración
-- ===============

example : s ∩ (⋃ i, A i) = ⋃ i, (A i ∩ s) :=
by
  ext x
  -- x : α
  -- ⊢ x ∈ s ∩ ⋃ (i : ℕ), A i ↔ x ∈ ⋃ (i : ℕ), A i ∩ s
  calc x ∈ s ∩ ⋃ (i : ℕ), A i
     ↔ x ∈ s ∧ x ∈ ⋃ (i : ℕ), A i :=
         by simp only [mem_inter_iff]
   _ ↔ x ∈ s ∧ (∃ i : ℕ, x ∈ A i) :=
         by simp only [mem_iUnion]
   _ ↔ ∃ i : ℕ, x ∈ s ∧ x ∈ A i :=
         by simp only [exists_and_left]
   _ ↔ ∃ i : ℕ, x ∈ A i ∧ x ∈ s :=
         by simp only [and_comm]
   _ ↔ ∃ i : ℕ, x ∈ A i ∩ s :=
         by simp only [mem_inter_iff]
   _ ↔ x ∈ ⋃ (i : ℕ), A i ∩ s :=
         by simp only [mem_iUnion]

-- 2ª demostración
-- ===============

example : s ∩ (⋃ i, A i) = ⋃ i, (A i ∩ s) :=
by
  ext x
  -- x : α
  -- ⊢ x ∈ s ∩ ⋃ (i : ℕ), A i ↔ x ∈ ⋃ (i : ℕ), A i ∩ s
  constructor
  . -- ⊢ x ∈ s ∩ ⋃ (i : ℕ), A i → x ∈ ⋃ (i : ℕ), A i ∩ s
    intro h
    -- h : x ∈ s ∩ ⋃ (i : ℕ), A i
    -- ⊢ x ∈ ⋃ (i : ℕ), A i ∩ s
    rw [mem_iUnion]
    -- ⊢ ∃ i, x ∈ A i ∩ s
    rcases h with ⟨xs, xUAi⟩
    -- xs : x ∈ s
    -- xUAi : x ∈ ⋃ (i : ℕ), A i
    rw [mem_iUnion] at xUAi
    -- xUAi : ∃ i, x ∈ A i
    rcases xUAi with ⟨i, xAi⟩
    -- i : ℕ
    -- xAi : x ∈ A i
    use i
    -- ⊢ x ∈ A i ∩ s
    constructor
    . -- ⊢ x ∈ A i
      exact xAi
    . -- ⊢ x ∈ s
      exact xs
  . -- ⊢ x ∈ ⋃ (i : ℕ), A i ∩ s → x ∈ s ∩ ⋃ (i : ℕ), A i
    intro h
    -- h : x ∈ ⋃ (i : ℕ), A i ∩ s
    -- ⊢ x ∈ s ∩ ⋃ (i : ℕ), A i
    rw [mem_iUnion] at h
    -- h : ∃ i, x ∈ A i ∩ s
    rcases h with ⟨i, hi⟩
    -- i : ℕ
    -- hi : x ∈ A i ∩ s
    rcases hi with ⟨xAi, xs⟩
    -- xAi : x ∈ A i
    -- xs : x ∈ s
    constructor
    . -- ⊢ x ∈ s
      exact xs
    . -- ⊢ x ∈ ⋃ (i : ℕ), A i
      rw [mem_iUnion]
      -- ⊢ ∃ i, x ∈ A i
      use i
      -- ⊢ x ∈ A i
      exact xAi

-- 3ª demostración
-- ===============

example : s ∩ (⋃ i, A i) = ⋃ i, (A i ∩ s) :=
by
  ext x
  -- x : α
  -- ⊢ x ∈ s ∩ ⋃ (i : ℕ), A i ↔ x ∈ ⋃ (i : ℕ), A i ∩ s
  simp
  -- ⊢ (x ∈ s ∧ ∃ i, x ∈ A i) ↔ (∃ i, x ∈ A i) ∧ x ∈ s
  constructor
  . -- ⊢ (x ∈ s ∧ ∃ i, x ∈ A i) → (∃ i, x ∈ A i) ∧ x ∈ s
    rintro ⟨xs, ⟨i, xAi⟩⟩
    -- xs : x ∈ s
    -- i : ℕ
    -- xAi : x ∈ A i
    -- ⊢ (∃ i, x ∈ A i) ∧ x ∈ s
    exact ⟨⟨i, xAi⟩, xs⟩
  . -- ⊢ (∃ i, x ∈ A i) ∧ x ∈ s → x ∈ s ∧ ∃ i, x ∈ A i
    rintro ⟨⟨i, xAi⟩, xs⟩
    -- xs : x ∈ s
    -- i : ℕ
    -- xAi : x ∈ A i
    -- ⊢ x ∈ s ∧ ∃ i, x ∈ A i
    exact ⟨xs, ⟨i, xAi⟩⟩

-- 3ª demostración
-- ===============

example : s ∩ (⋃ i, A i) = ⋃ i, (A i ∩ s) :=
by
  ext x
  -- x : α
  -- ⊢ x ∈ s ∩ ⋃ (i : ℕ), A i ↔ x ∈ ⋃ (i : ℕ), A i ∩ s
  aesop

-- 4ª demostración
-- ===============

example : s ∩ (⋃ i, A i) = ⋃ i, (A i ∩ s) :=
by ext; aesop

-- Lemas usados
-- ============

-- variable (x : α)
-- variable (t : Set α)
-- variable (a b : Prop)
-- variable (p : α → Prop)
-- #check (mem_iUnion : x ∈ ⋃ i, A i ↔ ∃ i, x ∈ A i)
-- #check (mem_inter_iff x s t : x ∈ s ∩ t ↔ x ∈ s ∧ x ∈ t)
-- #check (exists_and_left : (∃ (x : α), b ∧ p x) ↔ b ∧ ∃ (x : α), p x)
-- #check (and_comm : a ∧ b ↔ b ∧ a)
\end{verbatim}
Se puede interactuar con las pruebas anteriores en \href{https://lean.math.hhu.de/\#url=https://raw.githubusercontent.com/jaalonso/Calculemus2/main/src/Distributiva\_de\_la\_interseccion\_respecto\_de\_la\_union\_general.lean}{Lean 4 Web}

\section{(⋂ i, A i ∩ B i) = (⋂ i, A i) ∩ (⋂ i, B i)}
\label{sec:orgad452f6}
\begin{verbatim}
-- ---------------------------------------------------------------------
-- Demostrar que
--    (⋂ i, A i ∩ B i) = (⋂ i, A i) ∩ (⋂ i, B i)
-- ----------------------------------------------------------------------

-- Demostración en lenguaje natural
-- ================================

-- Tenemos que demostrar que para x se verifica
--    x ∈ ⋂ i, (A i ∩ B i) ↔ x ∈ (⋂ i, A i) ∩ (⋂ i, B i)
-- Lo demostramos mediante la siguiente cadena de equivalencias
--    x ∈ ⋂ i, (A i ∩ B i) ↔ (∀ i)[x ∈ A i ∩ B i]
--                         ↔ (∀ i)[x ∈ A i ∧ x ∈ B i]
--                         ↔ (∀ i)[x ∈ A i] ∧ (∀ i)[x ∈ B i]
--                         ↔ x ∈ (⋂ i, A i) ∧ x ∈ (⋂ i, B i)
--                         ↔ x ∈ (⋂ i, A i) ∩ (⋂ i, B i)

-- Demostraciones con Lean4
-- ========================

import Mathlib.Data.Set.Basic
import Mathlib.Tactic

open Set

variable {α : Type}
variable (A B : ℕ → Set α)

-- 1ª demostración
-- ===============

example : (⋂ i, A i ∩ B i) = (⋂ i, A i) ∩ (⋂ i, B i) :=
by
  ext x
  -- x : α
  -- ⊢ x ∈ ⋂ (i : ℕ), A i ∩ B i ↔ x ∈ (⋂ (i : ℕ), A i) ∩ ⋂ (i : ℕ), B i
  calc x ∈ ⋂ i, A i ∩ B i
     ↔ ∀ i, x ∈ A i ∩ B i :=
         by exact mem_iInter
   _ ↔ ∀ i, x ∈ A i ∧ x ∈ B i :=
         by simp only [mem_inter_iff]
   _ ↔ (∀ i, x ∈ A i) ∧ (∀ i, x ∈ B i) :=
         by exact forall_and
   _ ↔ x ∈ (⋂ i, A i) ∧ x ∈ (⋂ i, B i) :=
         by simp only [mem_iInter]
   _ ↔ x ∈ (⋂ i, A i) ∩ ⋂ i, B i :=
         by simp only [mem_inter_iff]

-- 2ª demostración
-- ===============

example : (⋂ i, A i ∩ B i) = (⋂ i, A i) ∩ (⋂ i, B i) :=
by
  ext x
  -- x : α
  -- ⊢ x ∈ ⋂ (i : ℕ), A i ∩ B i ↔ x ∈ (⋂ (i : ℕ), A i) ∩ ⋂ (i : ℕ), B i
  simp only [mem_inter_iff, mem_iInter]
  -- ⊢ (∀ (i : ℕ), x ∈ A i ∧ x ∈ B i) ↔ (∀ (i : ℕ), x ∈ A i) ∧ ∀ (i : ℕ), x ∈ B i
  constructor
  . -- ⊢ (∀ (i : ℕ), x ∈ A i ∧ x ∈ B i) → (∀ (i : ℕ), x ∈ A i) ∧ ∀ (i : ℕ), x ∈ B i
    intro h
    -- h : ∀ (i : ℕ), x ∈ A i ∧ x ∈ B i
    -- ⊢ (∀ (i : ℕ), x ∈ A i) ∧ ∀ (i : ℕ), x ∈ B i
    constructor
    . -- ⊢ ∀ (i : ℕ), x ∈ A i
      intro i
      -- i : ℕ
      -- ⊢ x ∈ A i
      exact (h i).1
    . -- ⊢ ∀ (i : ℕ), x ∈ B i
      intro i
      -- i : ℕ
      -- ⊢ x ∈ B i
      exact (h i).2
  . -- ⊢ ((∀ (i : ℕ), x ∈ A i) ∧ ∀ (i : ℕ), x ∈ B i) → ∀ (i : ℕ), x ∈ A i ∧ x ∈ B i
    intros h i
    -- h : (∀ (i : ℕ), x ∈ A i) ∧ ∀ (i : ℕ), x ∈ B i
    -- i : ℕ
    -- ⊢ x ∈ A i ∧ x ∈ B i
    rcases h with ⟨h1, h2⟩
    -- h1 : ∀ (i : ℕ), x ∈ A i
    -- h2 : ∀ (i : ℕ), x ∈ B i
    constructor
    . -- ⊢ x ∈ A i
      exact h1 i
    . -- ⊢ x ∈ B i
      exact h2 i

-- 3ª demostración
-- ===============

example : (⋂ i, A i ∩ B i) = (⋂ i, A i) ∩ (⋂ i, B i) :=
by
  ext x
  -- x : α
  -- ⊢ x ∈ ⋂ (i : ℕ), A i ∩ B i ↔ x ∈ (⋂ (i : ℕ), A i) ∩ ⋂ (i : ℕ), B i
  simp only [mem_inter_iff, mem_iInter]
  -- ⊢ (∀ (i : ℕ), x ∈ A i ∧ x ∈ B i) ↔ (∀ (i : ℕ), x ∈ A i) ∧ ∀ (i : ℕ), x ∈ B i
  exact ⟨fun h ↦ ⟨fun i ↦ (h i).1, fun i ↦ (h i).2⟩,
         fun ⟨h1, h2⟩ i ↦ ⟨h1 i, h2 i⟩⟩

-- 4ª demostración
-- ===============

example : (⋂ i, A i ∩ B i) = (⋂ i, A i) ∩ (⋂ i, B i) :=
by
  ext
  -- x : α
  -- ⊢ x ∈ ⋂ (i : ℕ), A i ∩ B i ↔ x ∈ (⋂ (i : ℕ), A i) ∩ ⋂ (i : ℕ), B i
  simp only [mem_inter_iff, mem_iInter]
  -- ⊢ (∀ (i : ℕ), x ∈ A i ∧ x ∈ B i) ↔ (∀ (i : ℕ), x ∈ A i) ∧ ∀ (i : ℕ), x ∈ B i
  aesop

-- Lemas usados
-- ============

-- variable (x : α)
-- variable (a b : Set α)
-- variable (ι : Sort v)
-- variable (s : ι → Set α)
-- variable (p q : α → Prop)
-- #check (forall_and : (∀ (x : α), p x ∧ q x) ↔ (∀ (x : α), p x) ∧ ∀ (x : α), q x)
-- #check (mem_iInter : x ∈ ⋂ (i : ι), s i ↔ ∀ (i : ι), x ∈ s i)
-- #check (mem_inter_iff x a b : x ∈ a ∩ b ↔ x ∈ a ∧ x ∈ b)
\end{verbatim}
Se puede interactuar con las pruebas anteriores en \href{https://lean.math.hhu.de/\#url=https://raw.githubusercontent.com/jaalonso/Calculemus2/main/src/Interseccion\_de\_intersecciones.lean}{Lean 4 Web}

\section{s ∪ (⋂ i, A i) = ⋂ i, (A i ∪ s)}
\label{sec:orga7ba7ed}
\begin{verbatim}
-- ---------------------------------------------------------------------
-- Demostrar que
--    s ∪ (⋂ i, A i) = ⋂ i, (A i ∪ s)
-- ----------------------------------------------------------------------

-- Demostración en lenguaje natural
-- ================================

-- Tenemos que demostrar que para todo x,
--    x ∈ s ∪ ⋂ i, A i ↔ x ∈ ⋂ i, A i ∪ s
-- Lo haremos mediante la siguiente cadena de equivalencias
--    x ∈ s ∪ ⋂ i, A i ↔ x ∈ s ∨ x ∈ ⋂ i, A i
--                     ↔ x ∈ s ∨ ∀ i, x ∈ A i
--                     ↔ ∀ i, x ∈ s ∨ x ∈ A i
--                     ↔ ∀ i, x ∈ A i ∨ x ∈ s
--                     ↔ ∀ i, x ∈ A i ∪ s
--                     ↔ x ∈ ⋂ i, A i ∪ s

-- Demostraciones con Lean4
-- ========================

import Mathlib.Data.Set.Basic
import Mathlib.Tactic

open Set

variable {α : Type}
variable (s : Set α)
variable (A : ℕ → Set α)

-- 1ª demostración
-- ===============

example : s ∪ (⋂ i, A i) = ⋂ i, (A i ∪ s) :=
by
  ext x
  -- x : α
  -- ⊢ x ∈ s ∪ ⋂ (i : ℕ), A i ↔ x ∈ ⋂ (i : ℕ), A i ∪ s
  calc x ∈ s ∪ ⋂ i, A i
     ↔ x ∈ s ∨ x ∈ ⋂ i, A i :=
         by simp only [mem_union]
   _ ↔ x ∈ s ∨ ∀ i, x ∈ A i :=
         by simp only [mem_iInter]
   _ ↔ ∀ i, x ∈ s ∨ x ∈ A i :=
         by simp only [forall_or_left]
   _ ↔ ∀ i, x ∈ A i ∨ x ∈ s  :=
         by simp only [or_comm]
   _ ↔ ∀ i, x ∈ A i ∪ s  :=
         by simp only [mem_union]
   _ ↔ x ∈ ⋂ i, A i ∪ s :=
         by simp only [mem_iInter]

-- 2ª demostración
-- ===============

example : s ∪ (⋂ i, A i) = ⋂ i, (A i ∪ s) :=
by
  ext x
  -- x : α
  -- ⊢ x ∈ s ∪ ⋂ (i : ℕ), A i ↔ x ∈ ⋂ (i : ℕ), A i ∪ s
  simp only [mem_union, mem_iInter]
  -- ⊢ (x ∈ s ∨ ∀ (i : ℕ), x ∈ A i) ↔ ∀ (i : ℕ), x ∈ A i ∨ x ∈ s
  constructor
  . -- ⊢ (x ∈ s ∨ ∀ (i : ℕ), x ∈ A i) → ∀ (i : ℕ), x ∈ A i ∨ x ∈ s
    intros h i
    -- h : x ∈ s ∨ ∀ (i : ℕ), x ∈ A i
    -- i : ℕ
    -- ⊢ x ∈ A i ∨ x ∈ s
    rcases h with (xs | xAi)
    . -- xs : x ∈ s
      right
      -- ⊢ x ∈ s
      exact xs
    . -- xAi : ∀ (i : ℕ), x ∈ A i
      left
      -- ⊢ x ∈ A i
      exact xAi i
  . -- ⊢ (∀ (i : ℕ), x ∈ A i ∨ x ∈ s) → x ∈ s ∨ ∀ (i : ℕ), x ∈ A i
    intro h
    -- h : ∀ (i : ℕ), x ∈ A i ∨ x ∈ s
    -- ⊢ x ∈ s ∨ ∀ (i : ℕ), x ∈ A i
    by_cases cxs : x ∈ s
    . -- cxs : x ∈ s
      left
      -- ⊢ x ∈ s
      exact cxs
    . -- cns : ¬x ∈ s
      right
      -- ⊢ ∀ (i : ℕ), x ∈ A i
      intro i
      -- i : ℕ
      -- ⊢ x ∈ A i
      rcases h i with (xAi | xs)
      . -- ⊢ x ∈ A i
        exact xAi
      . -- xs : x ∈ s
        exact absurd xs cxs

-- 3ª demostración
-- ===============

example : s ∪ (⋂ i, A i) = ⋂ i, (A i ∪ s) :=
by
  ext x
  -- x : α
  -- ⊢ x ∈ s ∪ ⋂ (i : ℕ), A i ↔ x ∈ ⋂ (i : ℕ), A i ∪ s
  simp only [mem_union, mem_iInter]
  -- ⊢ (x ∈ s ∨ ∀ (i : ℕ), x ∈ A i) ↔ ∀ (i : ℕ), x ∈ A i ∨ x ∈ s
  constructor
  . -- ⊢ (x ∈ s ∨ ∀ (i : ℕ), x ∈ A i) → ∀ (i : ℕ), x ∈ A i ∨ x ∈ s
    rintro (xs | xI) i
    . -- xs : x ∈ s
      -- i : ℕ
      -- ⊢ x ∈ A i ∨ x ∈ s
      right
      -- ⊢ x ∈ s
      exact xs
    . -- xI : ∀ (i : ℕ), x ∈ A i
      -- i : ℕ
      -- ⊢ x ∈ A i ∨ x ∈ s
      left
      -- ⊢ x ∈ A i
      exact xI i
  . -- ⊢ (∀ (i : ℕ), x ∈ A i ∨ x ∈ s) → x ∈ s ∨ ∀ (i : ℕ), x ∈ A i
    intro h
    -- h : ∀ (i : ℕ), x ∈ A i ∨ x ∈ s
    -- ⊢ x ∈ s ∨ ∀ (i : ℕ), x ∈ A i
    by_cases cxs : x ∈ s
    . -- cxs : x ∈ s
      left
      -- ⊢ x ∈ s
      exact cxs
    . -- cxs : ¬x ∈ s
      right
      -- ⊢ ∀ (i : ℕ), x ∈ A i
      intro i
      -- i : ℕ
      -- ⊢ x ∈ A i
      cases h i
      . -- h : x ∈ A i
        assumption
      . -- h : x ∈ s
        contradiction

-- Lemas usados
-- ============

-- variable (x : α)
-- variable (s t : Set α)
-- variable (a b q : Prop)
-- variable (p : ℕ → Prop)
-- #check (absurd : a → ¬a → b)
-- #check (forall_or_left : (∀ x, q ∨ p x) ↔ q ∨ ∀ x, p x)
-- #check (mem_iInter : x ∈ ⋂ i, A i ↔ ∀ i, x ∈ A i)
-- #check (mem_union x a b : x ∈ s ∪ t ↔ x ∈ s ∨ x ∈ t)
-- #check (or_comm : a ∨ b ↔ b ∨ a)
\end{verbatim}
Se puede interactuar con las pruebas anteriores en \href{https://lean.math.hhu.de/\#url=https://raw.githubusercontent.com/jaalonso/Calculemus2/main/src/Union\_con\_interseccion\_general.lean}{Lean 4 Web}.

\section{f⁻¹[u ∩ v] = f⁻¹[u] ∩ f⁻¹[v]}
\label{sec:orgb01baac}
\begin{verbatim}
-- ---------------------------------------------------------------------
-- En Lean, la imagen inversa de un conjunto s (de elementos de tipo β)
-- por la función f (de tipo α → β) es el conjunto `f ⁻¹' s` de
-- elementos x (de tipo α) tales que `f x ∈ s`.
--
-- Demostrar que
--    f ⁻¹' (u ∩ v) = f ⁻¹' u ∩ f ⁻¹' v
-- ----------------------------------------------------------------------

-- Demostración en lenguaje natural
-- ================================

-- Tenemos que demostrar que, para todo x,
--    x ∈ f⁻¹[u ∩ v] ↔ x ∈ f⁻¹[u] ∩ f⁻¹[v]
-- Lo haremos mediante la siguiente cadena de equivalencias
--    x ∈ f⁻¹[u ∩ v] ↔ f x ∈ u ∩ v
--                   ↔ f x ∈ u ∧ f x ∈ v
--                   ↔ x ∈ f⁻¹[u] ∧ x ∈ f⁻¹[v]
--                   ↔ x ∈ f⁻¹[u] ∩ f⁻¹[v]

-- Demostraciones con Lean4
-- ========================

import Mathlib.Data.Set.Function

variable {α β : Type _}
variable (f : α → β)
variable (u v : Set β)

open Set

-- 1ª demostración
-- ===============

example : f ⁻¹' (u ∩ v) = f ⁻¹' u ∩ f ⁻¹' v :=
by
  ext x
  -- x : α
  -- ⊢ x ∈ f ⁻¹' (u ∩ v) ↔ x ∈ f ⁻¹' u ∩ f ⁻¹' v
  calc x ∈ f ⁻¹' (u ∩ v)
     ↔ f x ∈ u ∩ v :=
         by simp only [mem_preimage]
   _ ↔ f x ∈ u ∧ f x ∈ v :=
         by simp only [mem_inter_iff]
   _ ↔ x ∈ f ⁻¹' u ∧ x ∈ f ⁻¹' v :=
         by simp only [mem_preimage]
   _ ↔ x ∈ f ⁻¹' u ∩ f ⁻¹' v :=
         by simp only [mem_inter_iff]

-- 2ª demostración
-- ===============

example : f ⁻¹' (u ∩ v) = f ⁻¹' u ∩ f ⁻¹' v :=
by
  ext x
  -- x : α
  -- ⊢ x ∈ f ⁻¹' (u ∩ v) ↔ x ∈ f ⁻¹' u ∩ f ⁻¹' v
  constructor
  . -- ⊢ x ∈ f ⁻¹' (u ∩ v) → x ∈ f ⁻¹' u ∩ f ⁻¹' v
    intro h
    -- h : x ∈ f ⁻¹' (u ∩ v)
    -- ⊢ x ∈ f ⁻¹' u ∩ f ⁻¹' v
    constructor
    . -- ⊢ x ∈ f ⁻¹' u
      apply mem_preimage.mpr
      -- ⊢ f x ∈ u
      rw [mem_preimage] at h
      -- h : f x ∈ u ∩ v
      exact mem_of_mem_inter_left h
    . -- ⊢ x ∈ f ⁻¹' v
      apply mem_preimage.mpr
      -- ⊢ f x ∈ v
      rw [mem_preimage] at h
      -- h : f x ∈ u ∩ v
      exact mem_of_mem_inter_right h
  . -- ⊢ x ∈ f ⁻¹' u ∩ f ⁻¹' v → x ∈ f ⁻¹' (u ∩ v)
    intro h
    -- h : x ∈ f ⁻¹' u ∩ f ⁻¹' v
    -- ⊢ x ∈ f ⁻¹' (u ∩ v)
    apply mem_preimage.mpr
    -- ⊢ f x ∈ u ∩ v
    constructor
    . -- ⊢ f x ∈ u
      apply mem_preimage.mp
      -- ⊢ x ∈ f ⁻¹' u
      exact mem_of_mem_inter_left h
    . -- ⊢ f x ∈ v
      apply mem_preimage.mp
      -- ⊢ x ∈ f ⁻¹' v
      exact mem_of_mem_inter_right h

-- 3ª demostración
-- ===============

example : f ⁻¹' (u ∩ v) = f ⁻¹' u ∩ f ⁻¹' v :=
by
  ext x
  -- x : α
  -- ⊢ x ∈ f ⁻¹' (u ∩ v) ↔ x ∈ f ⁻¹' u ∩ f ⁻¹' v
  constructor
  . -- ⊢ x ∈ f ⁻¹' (u ∩ v) → x ∈ f ⁻¹' u ∩ f ⁻¹' v
    intro h
    -- h : x ∈ f ⁻¹' (u ∩ v)
    -- ⊢ x ∈ f ⁻¹' u ∩ f ⁻¹' v
    constructor
    . -- ⊢ x ∈ f ⁻¹' u
      simp at *
      -- h : f x ∈ u ∧ f x ∈ v
      -- ⊢ f x ∈ u
      exact h.1
    . -- ⊢ x ∈ f ⁻¹' v
      simp at *
      -- h : f x ∈ u ∧ f x ∈ v
      -- ⊢ f x ∈ v
      exact h.2
  . -- ⊢ x ∈ f ⁻¹' u ∩ f ⁻¹' v → x ∈ f ⁻¹' (u ∩ v)
    intro h
    -- h : x ∈ f ⁻¹' u ∩ f ⁻¹' v
    -- ⊢ x ∈ f ⁻¹' (u ∩ v)
    simp at *
    -- h : f x ∈ u ∧ f x ∈ v
    -- ⊢ f x ∈ u ∧ f x ∈ v
    exact h

-- 4ª demostración
-- ===============

example : f ⁻¹' (u ∩ v) = f ⁻¹' u ∩ f ⁻¹' v :=
by aesop

-- 5ª demostración
-- ===============

example : f ⁻¹' (u ∩ v) = f ⁻¹' u ∩ f ⁻¹' v :=
preimage_inter

-- 6ª demostración
-- ===============

example : f ⁻¹' (u ∩ v) = f ⁻¹' u ∩ f ⁻¹' v :=
rfl

-- Lemas usados
-- ============

-- variable (x : α)
-- variable (s t : Set α)
-- #check (mem_of_mem_inter_left : x ∈ s ∩ t → x ∈ s)
-- #check (mem_of_mem_inter_right : x ∈ s ∩ t → x ∈ t)
-- #check (mem_preimage : x ∈ f ⁻¹' u ↔ f x ∈ u)
-- #check (preimage_inter : f ⁻¹' (u ∩ v) = f ⁻¹' u ∩ f ⁻¹' v)
\end{verbatim}
Se puede interactuar con las pruebas anteriores en \href{https://lean.math.hhu.de/\#url=https://raw.githubusercontent.com/jaalonso/Calculemus2/main/src/Imagen\_inversa\_de\_la\_interseccion.lean}{Lean 4 Web}.

\section{f[s ∪ t] = f[s] ∪ f[t]}
\label{sec:orgb7338e9}
\begin{verbatim}
-- ---------------------------------------------------------------------
-- En Lean, la imagen de un conjunto s por una función f se representa
-- por `f '' s`; es decir,
--    f '' s = {y | ∃ x, x ∈ s ∧ f x = y}
--
-- Demostrar que
--    f '' (s ∪ t) = f '' s ∪ f '' t
-- ----------------------------------------------------------------------

-- Demostración en lenguaje natural
-- ================================

-- Tenemos que demostrar, para todo y, que
--    y ∈ f[s ∪ t] ↔ y ∈ f[s] ∪ f[t]
-- Lo haremos mediante la siguiente cadena de equivalencias
--    y ∈ f[s ∪ t] ↔ (∃x)(x ∈ s ∪ t ∧ f x = y)
--                 ↔ (∃x)((x ∈ s ∨ x ∈ t) ∧ f x = y)
--                 ↔ (∃x)((x ∈ s ∧ f x = y) ∨ (x ∈ t ∧ f x = y))
--                 ↔ (∃x)(x ∈ s ∧ f x = y) ∨ (∃x)(x ∈ t ∧ f x = y)
--                 ↔ y ∈ f[s] ∨ y ∈ f[t]
--                 ↔ y ∈ f[s] ∪ f[t]

-- Demostraciones con Lean4
-- ========================

import Mathlib.Data.Set.Function

variable {α β : Type _}
variable (f : α → β)
variable (s t : Set α)

open Set

-- 1ª demostración
-- ===============

example : f '' (s ∪ t) = f '' s ∪ f '' t :=
by
  ext y
  -- y : β
  -- ⊢ y ∈ f '' (s ∪ t) ↔ y ∈ f '' s ∪ f '' t
  calc y ∈ f '' (s ∪ t)
     ↔ ∃ x, x ∈ s ∪ t ∧ f x = y :=
         by simp only [mem_image]
   _ ↔ ∃ x, (x ∈ s ∨ x ∈ t) ∧ f x = y :=
         by simp only [mem_union]
   _ ↔ ∃ x, (x ∈ s ∧ f x = y) ∨ (x ∈ t ∧ f x = y) :=
         by simp only [or_and_right]
   _ ↔ (∃ x, x ∈ s ∧ f x = y) ∨ (∃ x, x ∈ t ∧ f x = y) :=
         by simp only [exists_or]
   _ ↔ y ∈ f '' s ∨ y ∈ f '' t :=
         by simp only [mem_image]
   _ ↔ y ∈ f '' s ∪ f '' t :=
         by simp only [mem_union]

-- 2ª demostración
-- ===============

example : f '' (s ∪ t) = f '' s ∪ f '' t :=
by
  ext y
  -- y : β
  -- ⊢ y ∈ f '' (s ∪ t) ↔ y ∈ f '' s ∪ f '' t
  constructor
  . -- ⊢ y ∈ f '' (s ∪ t) → y ∈ f '' s ∪ f '' t
    intro h
    -- h : y ∈ f '' (s ∪ t)
    -- ⊢ y ∈ f '' s ∪ f '' t
    rw [mem_image] at h
    -- h : ∃ x, x ∈ s ∪ t ∧ f x = y
    rcases h with ⟨x, hx⟩
    -- x : α
    -- hx : x ∈ s ∪ t ∧ f x = y
    rcases hx with ⟨xst, fxy⟩
    -- xst : x ∈ s ∪ t
    -- fxy : f x = y
    rw [←fxy]
    -- ⊢ f x ∈ f '' s ∪ f '' t
    rw [mem_union] at xst
    -- xst : x ∈ s ∨ x ∈ t
    rcases xst with (xs | xt)
    . -- xs : x ∈ s
      apply mem_union_left
      -- ⊢ f x ∈ f '' s
      apply mem_image_of_mem
      -- ⊢ x ∈ s
      exact xs
    . -- xt : x ∈ t
      apply mem_union_right
      -- ⊢ f x ∈ f '' t
      apply mem_image_of_mem
      -- ⊢ x ∈ t
      exact xt
  . -- ⊢ y ∈ f '' s ∪ f '' t → y ∈ f '' (s ∪ t)
    intro h
    -- h : y ∈ f '' s ∪ f '' t
    -- ⊢ y ∈ f '' (s ∪ t)
    rw [mem_union] at h
    -- h : y ∈ f '' s ∨ y ∈ f '' t
    rcases h with (yfs | yft)
    . -- yfs : y ∈ f '' s
      rw [mem_image]
      -- ⊢ ∃ x, x ∈ s ∪ t ∧ f x = y
      rw [mem_image] at yfs
      -- yfs : ∃ x, x ∈ s ∧ f x = y
      rcases yfs with ⟨x, hx⟩
      -- x : α
      -- hx : x ∈ s ∧ f x = y
      rcases hx with ⟨xs, fxy⟩
      -- xs : x ∈ s
      -- fxy : f x = y
      use x
      -- ⊢ x ∈ s ∪ t ∧ f x = y
      constructor
      . -- ⊢ x ∈ s ∪ t
        apply mem_union_left
        -- ⊢ x ∈ s
        exact xs
      . -- ⊢ f x = y
        exact fxy
    . -- yft : y ∈ f '' t
      rw [mem_image]
      -- ⊢ ∃ x, x ∈ s ∪ t ∧ f x = y
      rw [mem_image] at yft
      -- yft : ∃ x, x ∈ t ∧ f x = y
      rcases yft with ⟨x, hx⟩
      -- x : α
      -- hx : x ∈ t ∧ f x = y
      rcases hx with ⟨xt, fxy⟩
      -- xt : x ∈ t
      -- fxy : f x = y
      use x
      -- ⊢ x ∈ s ∪ t ∧ f x = y
      constructor
      . -- ⊢ x ∈ s ∪ t
        apply mem_union_right
        -- ⊢ x ∈ t
        exact xt
      . -- ⊢ f x = y
        exact fxy

-- 3ª demostración
-- ===============

example : f '' (s ∪ t) = f '' s ∪ f '' t :=
by
  ext y
  -- y : β
  -- ⊢ y ∈ f '' (s ∪ t) ↔ y ∈ f '' s ∪ f '' t
  constructor
  . -- ⊢ y ∈ f '' (s ∪ t) → y ∈ f '' s ∪ f '' t
    rintro ⟨x, xst, rfl⟩
    -- x : α
    -- xst : x ∈ s ∪ t
    -- ⊢ f x ∈ f '' s ∪ f '' t
    rcases xst with (xs | xt)
    . -- xs : x ∈ s
      left
      -- ⊢ f x ∈ f '' s
      exact mem_image_of_mem f xs
    . -- xt : x ∈ t
      right
      -- ⊢ f x ∈ f '' t
      exact mem_image_of_mem f xt
  . -- ⊢ y ∈ f '' s ∪ f '' t → y ∈ f '' (s ∪ t)
    rintro (yfs | yft)
    . -- yfs : y ∈ f '' s
      rcases yfs with ⟨x, xs, rfl⟩
      -- x : α
      -- xs : x ∈ s
      -- ⊢ f x ∈ f '' (s ∪ t)
      apply mem_image_of_mem
      -- ⊢ x ∈ s ∪ t
      left
      -- ⊢ x ∈ s
      exact xs
    . -- yft : y ∈ f '' t
      rcases yft with ⟨x, xt, rfl⟩
      -- x : α
      -- xs : x ∈ s
      -- ⊢ f x ∈ f '' (s ∪ t)
      apply mem_image_of_mem
      -- ⊢ x ∈ s ∪ t
      right
      -- ⊢ x ∈ t
      exact xt

-- 4ª demostración
-- ===============

example : f '' (s ∪ t) = f '' s ∪ f '' t :=
by
  ext y
  -- y : β
  -- ⊢ y ∈ f '' (s ∪ t) ↔ y ∈ f '' s ∪ f '' t
  constructor
  . -- ⊢ y ∈ f '' (s ∪ t) → y ∈ f '' s ∪ f '' t
    rintro ⟨x, xst, rfl⟩
    -- x : α
    -- xst : x ∈ s ∪ t
    -- ⊢ f x ∈ f '' s ∪ f '' t
    rcases xst with (xs | xt)
    . -- xs : x ∈ s
      left
      -- ⊢ f x ∈ f '' s
      use x, xs
    . -- xt : x ∈ t
      right
      -- ⊢ f x ∈ f '' t
      use x, xt
  . rintro (yfs | yft)
    . -- yfs : y ∈ f '' s
      rcases yfs with ⟨x, xs, rfl⟩
      -- x : α
      -- xs : x ∈ s
      -- ⊢ f x ∈ f '' (s ∪ t)
      use x, Or.inl xs
    . -- yft : y ∈ f '' t
      rcases yft with ⟨x, xt, rfl⟩
      -- x : α
      -- xt : x ∈ t
      -- ⊢ f x ∈ f '' (s ∪ t)
      use x, Or.inr xt

-- 5ª demostración
-- ===============

example : f '' (s ∪ t) = f '' s ∪ f '' t :=
by
  ext y
  -- y : β
  -- ⊢ y ∈ f '' (s ∪ t) ↔ y ∈ f '' s ∪ f '' t
  constructor
  . -- ⊢ y ∈ f '' (s ∪ t) → y ∈ f '' s ∪ f '' t
    rintro ⟨x, xs | xt, rfl⟩
    . -- x : α
      -- xs : x ∈ s
      -- ⊢ f x ∈ f '' s ∪ f '' t
      left
      -- ⊢ f x ∈ f '' s
      use x, xs
    . -- x : α
      -- xt : x ∈ t
      -- ⊢ f x ∈ f '' s ∪ f '' t
      right
      -- ⊢ f x ∈ f '' t
      use x, xt
  . -- ⊢ y ∈ f '' s ∪ f '' t → y ∈ f '' (s ∪ t)
    rintro (⟨x, xs, rfl⟩ | ⟨x, xt, rfl⟩)
    . -- x : α
      -- xs : x ∈ s
      -- ⊢ f x ∈ f '' (s ∪ t)
      use x, Or.inl xs
    . -- x : α
      -- xt : x ∈ t
      -- ⊢ f x ∈ f '' (s ∪ t)
      use x, Or.inr xt

-- 6ª demostración
-- ===============

example : f '' (s ∪ t) = f '' s ∪ f '' t :=
by
  ext y
  -- y : β
  -- ⊢ y ∈ f '' (s ∪ t) ↔ y ∈ f '' s ∪ f '' t
  constructor
  . -- ⊢ y ∈ f '' (s ∪ t) → y ∈ f '' s ∪ f '' t
    aesop
  . -- ⊢ y ∈ f '' s ∪ f '' t → y ∈ f '' (s ∪ t)
    aesop

-- 7ª demostración
-- ===============

example : f '' (s ∪ t) = f '' s ∪ f '' t :=
by
  ext y
  constructor <;> aesop

-- 8ª demostración
-- ===============

example : f '' (s ∪ t) = f '' s ∪ f '' t :=
by
  ext y
  -- y : β
  -- ⊢ y ∈ f '' (s ∪ t) ↔ y ∈ f '' s ∪ f '' t
  rw [iff_def]
  -- ⊢ (y ∈ f '' (s ∪ t) → y ∈ f '' s ∪ f '' t) ∧ (y ∈ f '' s ∪ f '' t → y ∈ f '' (s ∪ t))
  aesop

-- 9ª demostración
-- ===============

example : f '' (s ∪ t) = f '' s ∪ f '' t :=
image_union f s t

-- Lemas usados
-- ============

-- variable (x : α)
-- variable (y : β)
-- variable (a b c : Prop)
-- variable (p q : α → Prop)
-- #check (Or.inl : a → a ∨ b)
-- #check (Or.inr : b → a ∨ b)
-- #check (exists_or : (∃ x, p x ∨ q x) ↔ (∃ x, p x) ∨ ∃ x, q x)
-- #check (iff_def : (a ↔ b) ↔ (a → b) ∧ (b → a))
-- #check (image_union f s t : f '' (s ∪ t) = f '' s ∪ f '' t)
-- #check (mem_image f s y : (y ∈ f '' s ↔ ∃ (x : α), x ∈ s ∧ f x = y))
-- #check (mem_image_of_mem  f : x ∈ s → f x ∈ f '' s)
-- #check (mem_union x s t : x ∈ s ∪ t ↔ x ∈ s ∨ x ∈ t)
-- #check (mem_union_left t : x ∈ s → x ∈ s ∪ t)
-- #check (mem_union_right s : x ∈ t → x ∈ s ∪ t)
-- #check (or_and_right : (a ∨ b) ∧ c ↔ a ∧ c ∨ b ∧ c)
\end{verbatim}
Se puede interactuar con las pruebas anteriores en \href{https://lean.math.hhu.de/\#url=https://raw.githubusercontent.com/jaalonso/Calculemus2/main/src/Imagen\_de\_la\_union.lean}{Lean 4 Web}.

\section{s ⊆ f⁻¹[f[s]​]}
\label{sec:org773b2fd}
\begin{verbatim}
-- ---------------------------------------------------------------------
-- Demostrar que si s es un subconjunto del dominio de la función f,
-- entonces s está contenido en la imagen inversa de la imagen de s por
-- f; es decir,
--    s ⊆ f⁻¹[f[s]]
-- ----------------------------------------------------------------------

-- Demostración en lenguaje natural
-- ================================

-- Se demuestra mediante la siguiente cadena de implicaciones
--    x ∈ s ⟹ f(x) ∈ f[s]
--          ⟹ x ∈ f⁻¹[f[s]]

-- Demostraciones con Lean4
-- ========================

import Mathlib.Data.Set.Function

open Set

variable {α β : Type _}
variable (f : α → β)
variable (s : Set α)

-- 1ª demostración
-- ===============

example : s ⊆ f ⁻¹' (f '' s) :=
by
  intros x xs
  -- x : α
  -- xs : x ∈ s
  -- ⊢ x ∈ f ⁻¹' (f '' s)
  have h1 : f x ∈ f '' s := mem_image_of_mem f xs
  show x ∈ f ⁻¹' (f '' s)
  exact mem_preimage.mp h1

-- 2ª demostración
-- ===============

example : s ⊆ f ⁻¹' (f '' s) :=
by
  intros x xs
  -- x : α
  -- xs : x ∈ s
  -- ⊢ x ∈ f ⁻¹' (f '' s)
  apply mem_preimage.mpr
  -- ⊢ f x ∈ f '' s
  apply mem_image_of_mem
  -- ⊢ x ∈ s
  exact xs

-- 3ª demostración
-- ===============

example : s ⊆ f ⁻¹' (f '' s) :=
by
  intros x xs
  -- x : α
  -- xs : x ∈ s
  -- ⊢ x ∈ f ⁻¹' (f '' s)
  apply mem_image_of_mem
  -- ⊢ x ∈ s
  exact xs

-- 4ª demostración
-- ===============

example : s ⊆ f ⁻¹' (f '' s) :=
fun _ ↦ mem_image_of_mem f

-- 5ª demostración
-- ===============

example : s ⊆ f ⁻¹' (f '' s) :=
by
  intros x xs
  -- x : α
  -- xs : x ∈ s
  -- ⊢ x ∈ f ⁻¹' (f '' s)
  show f x ∈ f '' s
  use x, xs

-- 6ª demostración
-- ===============

example : s ⊆ f ⁻¹' (f '' s) :=
by
  intros x xs
  -- x : α
  -- xs : x ∈ s
  -- ⊢ x ∈ f ⁻¹' (f '' s)
  use x, xs

-- 7ª demostración
-- ===============

example : s ⊆ f ⁻¹' (f '' s) :=
subset_preimage_image f s

-- Lemas usados
-- ============

-- variable (x : α)
-- variable (t : Set β)
-- #check (mem_preimage : x ∈ f ⁻¹' t ↔ f x ∈ t)
-- #check (mem_image_of_mem f : x ∈ s → f x ∈ f '' s)
-- #check (subset_preimage_image f s : s ⊆ f ⁻¹' (f '' s))
\end{verbatim}
Se puede interactuar con las pruebas anteriores en \href{https://lean.math.hhu.de/\#url=https://raw.githubusercontent.com/jaalonso/Calculemus2/main/src/Imagen\_inversa\_de\_la\_imagen.lean}{Lean 4 Web}.

\section{f[s] ⊆ u ↔ s ⊆ f⁻¹[u]}
\label{sec:org8563344}
\begin{verbatim}
-- ---------------------------------------------------------------------
-- Demostrar que
--    f[s] ⊆ u ↔ s ⊆ f⁻¹[u]
-- ----------------------------------------------------------------------

-- Demostración en lenguaje natural
-- ================================

-- Los demostraremos probando las dos implicaciones.
--
-- (⟹) Supongamos que
--    f[s] ⊆ u                                                       (1)
-- y tenemos que demostrar que
--    s ⊆ f⁻¹[u]
-- Se prueba mediante las siguientes implicaciones
--    x ∈ s ⟹ f(x) ∈ f[s]
--          ⟹ f(x) ∈ u       [por (1)]
--          ⟹ x ∈ f⁻¹[u]
--
-- (⟸) Supongamos que
--    s ⊆ f⁻¹[u]                                                     (2)
-- y tenemos que demostrar que
--    f[s] ⊆ u
-- Para ello, sea y ∈ f[s]. Entonces, existe un
--    x ∈ s                                                          (3)
-- tal que
--    y = f(x)                                                       (4)
-- Entonces,
--    x ∈ f⁻¹[u]    [por (2) y (3)]
--    ⟹ f(x) ∈ u
--    ⟹ y ∈ u     [por (4)]

-- Demostraciones con Lean4
-- ========================

import Mathlib.Data.Set.Function

open Set

variable {α β : Type _}
variable (f : α → β)
variable (s : Set α)
variable (u : Set β)

-- 1ª demostración
-- ===============

example : f '' s ⊆ u ↔ s ⊆ f ⁻¹' u :=
calc f '' s ⊆ u
   ↔ ∀ y, y ∈ f '' s → y ∈ u :=
       by simp only [subset_def]
 _ ↔ ∀ y, (∃ x, x ∈ s ∧ f x = y) → y ∈ u :=
       by simp only [mem_image]
 _ ↔ ∀ x, x ∈ s → f x ∈ u := by
       constructor
       . -- (∀ y, (∃ x, x ∈ s ∧ f x = y) → y ∈ u) → (∀ x, x ∈ s → f x ∈ u)
         intro h x xs
         -- h : ∀ (y : β), (∃ x, x ∈ s ∧ f x = y) → y ∈ u
         -- x : α
         -- xs : x ∈ s
         -- ⊢ f x ∈ u
         exact h (f x) (by use x, xs)
       . -- (∀ x, x ∈ s → f x ∈ u) → (∀ y, (∃ x, x ∈ s ∧ f x = y) → y ∈ u)
         intro h y hy
         -- h : ∀ (x : α), x ∈ s → f x ∈ u
         -- y : β
         -- hy : ∃ x, x ∈ s ∧ f x = y
         -- ⊢ y ∈ u
         obtain ⟨x, hx⟩ := hy
         -- x : α
         -- hx : x ∈ s ∧ f x = y
         have h1 : y = f x := hx.2.symm
         have h2 : f x ∈ u := h x hx.1
         show y ∈ u
         exact mem_of_eq_of_mem h1 h2
 _ ↔ ∀ x, x ∈ s → x ∈ f ⁻¹' u :=
       by simp only [mem_preimage]
 _ ↔ s ⊆ f ⁻¹' u :=
       by simp only [subset_def]

-- 2ª demostración
-- ===============

example : f '' s ⊆ u ↔ s ⊆ f ⁻¹' u :=
calc f '' s ⊆ u
   ↔ ∀ y, y ∈ f '' s → y ∈ u :=
       by simp only [subset_def]
 _ ↔ ∀ y, (∃ x, x ∈ s ∧ f x = y) → y ∈ u :=
       by simp only [mem_image]
 _ ↔ ∀ x, x ∈ s → f x ∈ u := by
       constructor
       . -- (∀ y, (∃ x, x ∈ s ∧ f x = y) → y ∈ u) → (∀ x, x ∈ s → f x ∈ u)
         intro h x xs
         -- h : ∀ (y : β), (∃ x, x ∈ s ∧ f x = y) → y ∈ u
         -- x : α
         -- xs : x ∈ s
         -- ⊢ f x ∈ u
         apply h (f x)
         -- ⊢ ∃ x_1, x_1 ∈ s ∧ f x_1 = f x
         use x, xs
       . -- (∀ x, x ∈ s → f x ∈ u) → (∀ y, (∃ x, x ∈ s ∧ f x = y) → y ∈ u)
         intro h y hy
         -- h : ∀ (x : α), x ∈ s → f x ∈ u
         -- y : β
         -- hy : ∃ x, x ∈ s ∧ f x = y
         -- ⊢ y ∈ u
         obtain ⟨x, hx⟩ := hy
         -- x : α
         -- hx : x ∈ s ∧ f x = y
         rw [←hx.2]
         -- ⊢ f x ∈ u
         apply h x
         -- ⊢ x ∈ s
         exact hx.1
 _ ↔ ∀ x, x ∈ s → x ∈ f ⁻¹' u :=
       by simp only [mem_preimage]
 _ ↔ s ⊆ f ⁻¹' u :=
       by simp only [subset_def]

-- 3ª demostración
-- ===============

example : f '' s ⊆ u ↔ s ⊆ f ⁻¹' u :=
by
  constructor
  . -- ⊢ f '' s ⊆ u → s ⊆ f ⁻¹' u
    intros h x xs
    -- h : f '' s ⊆ u
    -- x : α
    -- xs : x ∈ s
    -- ⊢ x ∈ f ⁻¹' u
    apply mem_preimage.mpr
    -- ⊢ f x ∈ u
    apply h
    -- ⊢ f x ∈ f '' s
    apply mem_image_of_mem
    -- ⊢ x ∈ s
    exact xs
  . -- ⊢ s ⊆ f ⁻¹' u → f '' s ⊆ u
    intros h y hy
    -- h : s ⊆ f ⁻¹' u
    -- y : β
    -- hy : y ∈ f '' s
    -- ⊢ y ∈ u
    rcases hy with ⟨x, xs, fxy⟩
    -- x : α
    -- xs : x ∈ s
    -- fxy : f x = y
    rw [←fxy]
    -- ⊢ f x ∈ u
    exact h xs

-- 4ª demostración
-- ===============

example : f '' s ⊆ u ↔ s ⊆ f ⁻¹' u :=
by
  constructor
  . -- ⊢ f '' s ⊆ u → s ⊆ f ⁻¹' u
    intros h x xs
    -- h : f '' s ⊆ u
    -- x : α
    -- xs : x ∈ s
    -- ⊢ x ∈ f ⁻¹' u
    apply h
    -- ⊢ f x ∈ f '' s
    apply mem_image_of_mem
    -- ⊢ x ∈ s
    exact xs
  . -- ⊢ s ⊆ f ⁻¹' u → f '' s ⊆ u
    rintro h y ⟨x, xs, rfl⟩
    -- h : s ⊆ f ⁻¹' u
    -- x : α
    -- xs : x ∈ s
    -- ⊢ f x ∈ u
    exact h xs

-- 5ª demostración
-- ===============

example : f '' s ⊆ u ↔ s ⊆ f ⁻¹' u :=
image_subset_iff

-- 4ª demostración
-- ===============

example : f '' s ⊆ u ↔ s ⊆ f ⁻¹' u :=
by simp

-- Lemas usados
-- ============

-- variable (x y : α)
-- #check (image_subset_iff : f '' s ⊆ u ↔ s ⊆ f ⁻¹' u)
-- #check (mem_image_of_mem f : x ∈ s → f x ∈ f '' s)
-- #check (mem_of_eq_of_mem : x = y → y ∈ s → x ∈ s)
-- #check (mem_preimage : x ∈ f ⁻¹' u ↔ f x ∈ u)
\end{verbatim}
Se puede interactuar con las pruebas anteriores en \href{https://lean.math.hhu.de/\#url=https://raw.githubusercontent.com/jaalonso/Calculemus2/main/src/Subconjunto\_de\_la\_imagen\_inversa.lean}{Lean 4 Web}.

\section{La función (x ↦ x + c) es inyectiva}
\label{sec:orgee41794}
\begin{verbatim}
-- ---------------------------------------------------------------------
-- Demostrar que, para todo c la función
--    f(x) = x + c
-- es inyectiva
-- ----------------------------------------------------------------------

-- Demostración en lenguaje natural
-- ================================

-- Se usará el lema
--    (∀ a, b, c) [a + b = c + b → a = c]                            (L1)
-- Hay que demostrar que
--    (∀ x₁ x₂) [f(x₁) = f(x₂) → x₁ = x₂]
-- Sean x₁, x₂ tales que f(x₁) = f(x₂). Entonces,
--    x₁ + c = x₂ + c
-- y, por L1, x₁ = x₂.

-- Demostraciones con Lean4
-- ========================

import Mathlib.Data.Real.Basic
open Function

variable {c : ℝ}

-- 1ª demostración
example : Injective ((. + c)) :=
by
  intro (x1 : ℝ) (x2 : ℝ) (h1 : x1 + c = x2 + c)
  show x1 = x2
  exact add_right_cancel h1

-- 2ª demostración
example : Injective ((. + c)) :=
by
  intro x1 x2 h1
  show x1 = x2
  exact add_right_cancel h1

-- 3ª demostración
example : Injective ((. + c)) :=
  fun _ _ h ↦ add_right_cancel h

-- Lemas usados
-- ============

-- variable {a b : ℝ}
-- #check (add_right_cancel : a + b = c + b → a = c)
\end{verbatim}
Se puede interactuar con las pruebas anteriores en \href{https://lean.math.hhu.de/\#url=https://raw.githubusercontent.com/jaalonso/Calculemus2/main/src/Suma\_constante\_es\_inyectiva.lean}{Lean 4 Web}

\section{Si c ≠ 0, entonces la función (x ↦ cx) es inyectiva}
\label{sec:orga4b635b}
\begin{verbatim}
-- ---------------------------------------------------------------------
-- Ejercicio 3. Demostrar que para todo c distinto de cero la función
--    f(x) = c * x
-- es inyectiva
-- ----------------------------------------------------------------------

-- Demostración en lenguaje natural
-- ================================

-- Se usará el lema
--    (∀ a, b, c) [a ≠ 0 → (a * b = a * c ↔ b = c))]             (L1)
-- Hay que demostrar que
--    (∀ x₁, x₂) [f(x₁) = f(x₂) → x₁ = x₂]
-- Sean x₁, x₂ tales que f(x₁) = f(x₂). Entonces,
--    cx₁ = cx₂
-- y, por L1 y puesto que c ≠ 0, se tiene que
--    x₁ = x₂.

-- Demostraciones con Lean4
-- ========================

import Mathlib.Data.Real.Basic
open Function
variable {c : ℝ}

-- 1ª demostración
example
  (h : c ≠ 0)
  : Injective ((c * .)) :=
by
  intro (x1 : ℝ) (x2 : ℝ) (h1 : c * x1 = c * x2)
  show x1 = x2
  exact (mul_right_inj' h).mp h1

-- 2ª demostración
example
  (h : c ≠ 0)
  : Injective ((c * .)) :=
fun _ _ h1 ↦ mul_left_cancel₀ h h1

-- Lemas usados
-- ============

-- variable (a b : ℝ)
-- #check (mul_right_inj' : a ≠ 0 → (a * b = a * c ↔ b = c))
-- #check (mul_left_cancel₀ : a ≠ 0 → a * b = a * c → b = c)
\end{verbatim}
Se puede interactuar con las pruebas anteriores en \href{https://lean.math.hhu.de/\#url=https://raw.githubusercontent.com/jaalonso/Calculemus2/main/src/Producto\_constante\_no\_nula\_es\_inyectiva.lean}{Lean 4 Web}

\section{La composición de funciones inyectivas es inyectiva}
\label{sec:org9b29f7d}
\begin{verbatim}
-- Demostraciones en lenguaje natural (LN)
-- =======================================

-- 1ª demostración en LN
-- =====================

-- Tenemos que demostrar que
--    (∀ x, y) [(g ∘ f)(x) = (g ∘ f)(y) → x = y]
-- Sean x, y tales que
--    (g ∘ f)(x) = (g ∘ f)(y)
-- Entonces, por la definición de la composición,
--    g(f(x)) = g(f(y))
-- y, ser g inyectiva,
--    f(x) = f(y)
-- y, ser f inyectiva,
--    x = y

-- 2ª demostración en LN
-- =====================

-- Tenemos que demostrar que
--    (∀ x, y) [(g ∘ f)(x) = (g ∘ f)(y) → x = y]
-- Sean x, y tales que
--    (g ∘ f)(x) = (g ∘ f)(y)                                        (1)
-- y tenemos que demostrar que
--    x = y                                                          (2)
-- El objetivo (2), usando que f es inyectiva, se reduce a
--    f(x) = f(y)
-- que, usando que g es inyectiva, se reduce a
--    g(f(x)) = g(f(y))
-- que, por la definición de la composición, coincide con (1).

-- Demostraciones con Lean4
-- ========================

import Mathlib.Tactic

open Function

variable {α : Type _} {β : Type _} {γ : Type _}
variable {f : α → β} {g : β → γ}

-- 1ª demostración (basada en la 1ª en LN)
example
  (hg : Injective g)
  (hf : Injective f) :
  Injective (g ∘ f) :=
by
  intro (x : α) (y : α) (h1: (g ∘ f) x = (g ∘ f) y)
  have h2: g (f x) = g (f y) := h1
  have h3: f x = f y := hg h2
  show x = y
  exact hf h3

-- 2ª demostración
example
  (hg : Injective g)
  (hf : Injective f) :
  Injective (g ∘ f) :=
by
  intro (x : α) (y : α) (h1: (g ∘ f) x = (g ∘ f) y)
  have h2: f x = f y := hg h1
  show x = y
  exact hf h2

-- 3ª demostración
example
  (hg : Injective g)
  (hf : Injective f) :
  Injective (g ∘ f) :=
by
  intro x y h
  exact hf (hg h)

-- 4ª demostración
example
  (hg : Injective g)
  (hf : Injective f) :
  Injective (g ∘ f) :=
fun _ _ h ↦ hf (hg h)

-- 5ª demostración (basada en la 2ª en LN)
example
  (hg : Injective g)
  (hf : Injective f) :
  Injective (g ∘ f) :=
by
  intros x y h
  -- x y : α
  -- h : (g ∘ f) x = (g ∘ f) y
  apply hf
  -- ⊢ f x = f y
  apply hg
  -- ⊢ g (f x) = g (f y)
  apply h

-- 6ª demostración
example
  (hg : Injective g)
  (hf : Injective f) :
  Injective (g ∘ f) :=
-- by exact?
Injective.comp hg hf

-- 7ª demostración
example
  (hg : Injective g)
  (hf : Injective f) :
  Injective (g ∘ f) :=
by tauto

-- Lemas usados
-- ============

-- #check (Injective.comp : Injective g → Injective f → Injective (g ∘ f))
\end{verbatim}
Se puede interactuar con las pruebas anteriores en \href{https://lean.math.hhu.de/\#url=https://raw.githubusercontent.com/jaalonso/Calculemus2/main/src/Composicion\_de\_funciones\_inyectivas.lean}{Lean 4 Web}

\section{La función (x ↦ x + c) es suprayectiva}
\label{sec:org4dba4aa}
\begin{verbatim}
-- ---------------------------------------------------------------------
-- Demostrar que para todo número real c, la función
--    f(x) = x + c
-- es suprayectiva.
-- ----------------------------------------------------------------------

-- Demostración en lenguaje natural
-- ================================

-- Tenemos que demostrar que
--    (∀ x ∈ ℝ)(∃ y ∈ ℝ)[y+c = x]
-- Sea x ∈ ℝ. Entonces, y = x-c ∈ ℝ y
--    y + c = (x - c) + c
--          = x

-- Demostraciones con Lean4
-- ========================

import Mathlib.Data.Real.Basic

variable {c : ℝ}

open Function

-- 1ª demostración
example : Surjective (fun x ↦ x + c) :=
by
  intro x
  -- x : ℝ
  -- ⊢ ∃ a, (fun x => x + c) a = x
  use x - c
  -- ⊢ (fun x => x + c) (x - c) = x
  dsimp
  -- ⊢ (x - c) + c = x
  exact sub_add_cancel x c

-- 2ª demostración
example : Surjective (fun x ↦ x + c) :=
by
  intro x
  -- x : ℝ
  -- ⊢ ∃ a, (fun x => x + c) a = x
  use x - c
  -- ⊢ (fun x => x + c) (x - c) = x
  change (x - c) + c = x
  -- ⊢ (x - c) + c = x
  exact sub_add_cancel x c

-- 3ª demostración
example : Surjective (fun x ↦ x + c) :=
by
  intro x
  -- x : ℝ
  -- ⊢ ∃ a, (fun x => x + c) a = x
  use x - c
  -- ⊢ (fun x => x + c) (x - c) = x
  exact sub_add_cancel x c

-- 4ª demostración
example : Surjective (fun x ↦ x + c) :=
fun x ↦ ⟨x - c, sub_add_cancel x c⟩

-- 5ª demostración
example : Surjective (fun x ↦ x + c) :=
fun x ↦ ⟨x - c, by ring⟩

-- 6ª demostración
example : Surjective (fun x ↦ x + c) :=
add_right_surjective c

-- Lemas usados
-- ============

-- variable (a b : ℝ)
-- #check (sub_add_cancel a b : (a - b) + b = a)
-- #check (add_right_surjective c : Surjective (fun x ↦ x + c))
\end{verbatim}
Se puede interactuar con las pruebas anteriores en \href{https://lean.math.hhu.de/\#url=https://raw.githubusercontent.com/jaalonso/Calculemus2/main/src/Suma\_constante\_es\_suprayectiva.lean}{Lean 4 Web}

\section{Si c ≠ 0, entonces la función (x ↦ cx) es suprayectiva}
\label{sec:orgdf02c0f}
\begin{verbatim}
-- ---------------------------------------------------------------------
-- Demostrar que si c es un número real no nulo, entonces la función
--    f(x) = c * x
-- es suprayectiva.
-- ----------------------------------------------------------------------

-- Demostración en lenguaje natural
-- ================================

-- Hay que demostrar que
--    (∀ x ∈ ℝ)(∃ y ∈ ℝ)[cy = x]
-- Sea x ∈ ℝ. Entonces, y = x/c ∈ R y
--    cy = c(x/c)
--       = y

-- Demostraciones con Lean4
-- ========================

import Mathlib.Data.Real.Basic
variable {c : ℝ}
open Function

-- 1ª demostración
example
  (h : c ≠ 0)
  : Surjective (fun x ↦ c * x) :=
by
  intro x
  -- x : ℝ
  -- ⊢ ∃ a, (fun x => c * x) a = x
  use (x / c)
  -- ⊢ (fun x => c * x) (x / c) = x
  dsimp
  -- ⊢ c * (x / c) = x
  rw [mul_comm]
  -- ⊢ (x / c) * c = x
  exact div_mul_cancel x h

-- 2ª demostración
example
  (h : c ≠ 0)
  : Surjective (fun x ↦ c * x) :=
by
  intro x
  -- x : ℝ
  -- ⊢ ∃ a, (fun x => c * x) a = x
  use (x / c)
  -- ⊢ (fun x => c * x) (x / c) = x
  exact mul_div_cancel' x h

-- 3ª demostración
example
  (h : c ≠ 0)
  : Surjective (fun x ↦ c * x) :=
fun x ↦ ⟨x / c, mul_div_cancel' x h⟩

-- 4ª demostración
example
  (h : c ≠ 0)
  : Surjective (fun x ↦ c * x) :=
mul_left_surjective₀ h

-- Lemas usados
-- ============

-- variable (a b : ℝ)
-- #check (div_mul_cancel a : b ≠ 0 → (a / b) * b = a)
-- #check (mul_comm a b : a * b = b * a)
-- #check (mul_div_cancel' a : b ≠ 0 → b * (a / b) = a)
-- #check (mul_left_surjective₀ : c ≠ 0 → Surjective (fun x ↦ c * x))
\end{verbatim}
Se puede interactuar con las pruebas anteriores en \href{https://lean.math.hhu.de/\#url=https://raw.githubusercontent.com/jaalonso/Calculemus2/main/src/Producto\_por\_no\_nula\_es\_suprayectiva.lean}{Lean 4 Web}

\section{Si c ≠ 0, entonces la función (x ↦ cx + d) es suprayectiva}
\label{sec:org5f080fe}
\begin{verbatim}
-- ---------------------------------------------------------------------
-- Demostrar que si c es un número real no nulo, entonces la función
--    f(x) = c * x + d
-- es suprayectiva.
-- ----------------------------------------------------------------------

-- Demostración en lenguaje natural
-- ================================

-- Hay que demostrar que
--    (∀x ∈ ℝ)(∃y ∈ ℝ)[cy+d = x]
-- Sea x ∈ ℝ. Entonces, y = (x-d)/c ∈ R y
--    cy = c((x-d)/c)+d
--       = (x-d)+d
--       = x

-- Demostraciones con Lean4
-- ========================

import Mathlib.Data.Real.Basic
variable {c d : ℝ}
open Function

-- 1ª demostración
-- ===============

example
  (h : c ≠ 0)
  : Surjective (fun x ↦ c * x + d) :=
by
  intro x
  -- x : ℝ
  -- ⊢ ∃ a, (fun x => c * x + d) a = x
  use ((x - d) / c)
  -- ⊢ (fun x => c * x + d) ((x - d) / c) = x
  dsimp
  -- ⊢ c * ((x - d) / c) + d = x
  show c * ((x - d) / c) + d = x
  calc c * ((x - d) / c) + d
         = (x - d) + d := congrArg (. + d) (mul_div_cancel' (x - d) h)
       _ = x           := sub_add_cancel x d

-- 2ª demostración
-- ===============

example
  (h : c ≠ 0)
  : Surjective (fun x ↦ c * x + d) :=
by
  intro x
  -- x : ℝ
  -- ⊢ ∃ a, (fun x => c * x + d) a = x
  use ((x - d) / c)
  -- ⊢ (fun x => c * x + d) ((x - d) / c) = x
  dsimp
  -- ⊢ c * ((x - d) / c) + d = x
  simp [mul_div_cancel', h]

-- 3ª demostración
-- ===============

example
  (h : c ≠ 0)
  : Surjective (fun x ↦ c * x + d) :=
by
  intro x
  -- x : ℝ
  -- ⊢ ∃ a, (fun x => c * x + d) a = x
  use ((x - d) / c)
  -- ⊢ (fun x => c * x + d) ((x - d) / c) = x
  simp [mul_div_cancel', h]

-- 4ª demostración
-- ===============

example
  (h : c ≠ 0)
  : Surjective (fun x ↦ c * x + d) :=
fun x ↦ ⟨(x - d) / c, by simp [mul_div_cancel', h]⟩

-- Lemas usados
-- ============

-- variable (a b : ℝ)
-- #check (mul_div_cancel' a : b ≠ 0 → b * (a / b) = a)
-- #check (sub_add_cancel a b : a - b + b = a)
\end{verbatim}
Se puede interactuar con las pruebas anteriores en \href{https://lean.math.hhu.de/\#url=https://raw.githubusercontent.com/jaalonso/Calculemus2/main/src/Producto\_por\_no\_nula\_y\_suma\_es\_suprayectiva.lean}{Lean 4 Web}

\section{Si f: ℝ → ℝ es suprayectiva, entonces ∃x ∈ ℝ tal que f(x)² = 9}
\label{sec:org9735233}
\begin{verbatim}
-- ---------------------------------------------------------------------
-- Demostrar que si f es una función suprayectiva de ℝ en ℝ,
-- entonces existe un x tal que (f x)^2 = 9.
-- ----------------------------------------------------------------------

-- Demostración en lenguaje natural
-- ================================

-- Al ser f suprayectiva, existe un y tal que f(y) = 3. Por tanto,
-- f(y)² = 9.

-- Demostración con Lean9
-- ======================

import Mathlib.Data.Real.Basic

open Function

example
  {f : ℝ → ℝ}
  (h : Surjective f)
  : ∃ x, (f x)^2 = 9 :=
by
  cases' h 3 with y hy
  -- y : ℝ
  -- hy : f y = 3
  use y
  -- ⊢ f y ^ 2 = 9
  rw [hy]
  -- ⊢ 3 ^ 2 = 9
  norm_num
\end{verbatim}
Se puede interactuar con las pruebas anteriores en \href{https://lean.math.hhu.de/\#url=https://raw.githubusercontent.com/jaalonso/Calculemus2/main/src/Propiedad\_de\_suprayectivas.lean}{Lean 4 Web}

\section{La composición de funciones suprayectivas es suprayectiva}
\label{sec:orged452c4}
\begin{verbatim}
-- ---------------------------------------------------------------------
-- Demostrar que la composición de funciones suprayectivas es
-- suprayectiva.
-- ----------------------------------------------------------------------

-- Demostración en lenguaje natural
-- ================================

-- Supongamos que f : A → B y g : B → C son suprayectivas. Tenemos que
-- demostrar que
--     (∀z ∈ C)(∃x ∈ A)[g(f(x)) = z]
-- Sea z ∈ C. Por ser g suprayectiva, existe un y ∈ B tal que
--     g(y) = z                                                      (1)
-- Por ser f suprayectiva, existe un x ∈ A tal que
--     f(x) = y                                                      (2)
-- Por tanto,
--     g(f(x)) = g(y)   [por (2)]
--             = z      [por (1)]

-- Demostraciones con lean4
-- ========================

import Mathlib.Tactic
open Function
variable {α : Type _} {β : Type _} {γ : Type _}
variable {f : α → β} {g : β → γ}

-- 1ª demostración
example
  (hg : Surjective g)
  (hf : Surjective f)
  : Surjective (g ∘ f) :=
by
  intro z
  -- z : γ
  -- ⊢ ∃ a, (g ∘ f) a = z
  cases' hg z with y hy
  -- y : β
  -- hy : g y = z
  cases' hf y with x hx
  -- x : α
  -- hx : f x = y
  use x
  -- ⊢ (g ∘ f) x = z
  dsimp
  -- ⊢ g (f x) = z
  rw [hx]
  -- ⊢ g y = z
  exact hy

-- 2ª demostración
example
  (hg : Surjective g)
  (hf : Surjective f)
  : Surjective (g ∘ f) :=
by
  intro z
  -- z : γ
  -- ⊢ ∃ a, (g ∘ f) a = z
  cases' hg z with y hy
  -- y : β
  -- hy : g y = z
  cases' hf y with x hx
  -- x : α
  -- hx : f x = y
  use x
  -- ⊢ (g ∘ f) x = z
  dsimp
  -- ⊢ g (f x) = z
  rw [hx, hy]

-- 3ª demostración
example
  (hg : Surjective g)
  (hf : Surjective f)
  : Surjective (g ∘ f) :=
by
  intro z
  -- z : γ
  -- ⊢ ∃ a, (g ∘ f) a = z
  cases' hg z with y hy
  -- y : β
  -- hy : g y = z
  cases' hf y with x hx
  -- x : α
  -- hx : f x = y
  exact ⟨x, by dsimp ; rw [hx, hy]⟩

-- 4ª demostración
example
  (hg : Surjective g)
  (hf : Surjective f)
  : Surjective (g ∘ f) :=
by
  intro z
  -- z : γ
  -- ⊢ ∃ a, (g ∘ f) a = z
  rcases hg z with ⟨y, hy : g y = z⟩
  rcases hf y with ⟨x, hx : f x = y⟩
  exact ⟨x, by dsimp ; rw [hx, hy]⟩

-- 5ª demostración
example
  (hg : Surjective g)
  (hf : Surjective f)
  : Surjective (g ∘ f) :=
Surjective.comp hg hf

-- Lemas usados
-- ============

-- #check (Surjective.comp : Surjective g → Surjective f → Surjective (g ∘ f))
\end{verbatim}
Se puede interactuar con las pruebas anteriores en \href{https://lean.math.hhu.de/\#url=https://raw.githubusercontent.com/jaalonso/Calculemus2/main/src/Composicion\_de\_suprayectivas.lean}{Lean 4 Web}

\section{Si f es inyectiva, entonces f⁻¹[f[s]​] ⊆ s}
\label{sec:orgf1280eb}
\begin{verbatim}
-- ---------------------------------------------------------------------
-- Demostrar que si f es inyectiva, entonces
--    f⁻¹[f[s]] ⊆ s
-- ----------------------------------------------------------------------

-- Demostración en lenguaje natural
-- ================================

-- Sea x tal que
--    x ∈ f⁻¹[f[s]]
-- Entonces,
--    f(x) ∈ f[s]
-- y, por tanto, existe un
--    y ∈ s                                                          (1)
-- tal que
--    f(y) = f(x)                                                    (2)
-- De (2), puesto que f es inyectiva, se tiene que
--    y = x                                                          (3)
-- Finalmente, de (3) y (1), se tiene que
--    x ∈ s
-- que es lo que teníamos que demostrar.

-- Demostraciones con Lean4
-- ========================

import Mathlib.Data.Set.Function

open Set Function

variable {α β : Type _}
variable (f : α → β)
variable (s : Set α)

-- 1ª demostración
-- ===============

example
  (h : Injective f)
  : f ⁻¹' (f '' s) ⊆ s :=
by
  intros x hx
  -- x : α
  -- hx : x ∈ f ⁻¹' (f '' s)
  -- ⊢ x ∈ s
  have h1 : f x ∈ f '' s := mem_preimage.mp hx
  have h2 : ∃ y, y ∈ s ∧ f y = f x := (mem_image f s (f x)).mp h1
  obtain ⟨y, hy : y ∈ s ∧ f y = f x⟩ := h2
  obtain ⟨ys : y ∈ s, fyx : f y = f x⟩ := hy
  have h3 : y = x := h fyx
  show x ∈ s
  exact h3 ▸ ys

-- 2ª demostración
-- ===============

example
  (h : Injective f)
  : f ⁻¹' (f '' s) ⊆ s :=
by
  intros x hx
  -- x : α
  -- hx : x ∈ f ⁻¹' (f '' s)
  -- ⊢ x ∈ s
  rw [mem_preimage] at hx
  -- hx : f x ∈ f '' s
  rw [mem_image] at hx
  -- hx : ∃ x_1, x_1 ∈ s ∧ f x_1 = f x
  rcases hx with ⟨y, hy⟩
  -- y : α
  -- hy : y ∈ s ∧ f y = f x
  rcases hy with ⟨ys, fyx⟩
  -- ys : y ∈ s
  -- fyx : f y = f x
  unfold Injective at h
  -- h : ∀ ⦃a₁ a₂ : α⦄, f a₁ = f a₂ → a₁ = a₂
  have h1 : y = x := h fyx
  rw [←h1]
  -- ⊢ y ∈ s
  exact ys

-- 3ª demostración
-- ===============

example
  (h : Injective f)
  : f ⁻¹' (f '' s) ⊆ s :=
by
  intros x hx
  -- x : α
  -- hx : x ∈ f ⁻¹' (f '' s)
  -- ⊢ x ∈ s
  rw [mem_preimage] at hx
  -- hx : f x ∈ f '' s
  rcases hx with ⟨y, ys, fyx⟩
  -- y : α
  -- ys : y ∈ s
  -- fyx : f y = f x
  rw [←h fyx]
  -- ⊢ y ∈ s
  exact ys

-- 4ª demostración
-- ===============

example
  (h : Injective f)
  : f ⁻¹' (f '' s) ⊆ s :=
by
  rintro x ⟨y, ys, hy⟩
  -- x y : α
  -- ys : y ∈ s
  -- hy : f y = f x
  -- ⊢ x ∈ s
  rw [←h hy]
  -- ⊢ y ∈ s
  exact ys

-- Lemas usados
-- ============

-- variable (x : α)
-- variable (y : β)
-- variable (t : Set β)
-- #check (mem_image f s y : y ∈ f '' s ↔ ∃ (x : α), x ∈ s ∧ f x = y)
-- #check (mem_preimage : x ∈ f ⁻¹' t ↔ f x ∈ t)
\end{verbatim}
Se puede interactuar con las pruebas anteriores en \href{https://lean.math.hhu.de/\#url=https://raw.githubusercontent.com/jaalonso/Calculemus2/main/src/Imagen\_inversa\_de\_la\_imagen\_de\_aplicaciones\_inyectivas.lean}{Lean 4 Web}.

\section{f[f⁻¹[u]​] ⊆ u}
\label{sec:org89f9ae1}
\begin{verbatim}
-- ---------------------------------------------------------------------
-- Demostrar que
--    f[f⁻¹[u]] ⊆ u
-- ----------------------------------------------------------------------

-- Demostración en lenguaje natural
-- ================================

-- Sea y ∈ f[f⁻¹[u]]. Entonces existe un x tal que
--    x ∈ f⁻¹[u]                                                     (1)
--    f(x) = y                                                       (2)
-- Por (1),
--    f(x) ∈ u
-- y, por (2),
--    y ∈ u

-- Demostraciones con Lean4
-- ========================

import Mathlib.Data.Set.Function
open Set

variable {α β : Type _}
variable (f : α → β)
variable (u : Set β)

-- 1ª demostración
-- ===============

example : f '' (f⁻¹' u) ⊆ u :=
by
  intros y h
  -- y : β
  -- h : y ∈ f '' (f ⁻¹' u)
  -- ⊢ y ∈ u
  obtain ⟨x : α, h1 : x ∈ f ⁻¹' u ∧ f x = y⟩ := h
  obtain ⟨hx : x ∈ f ⁻¹' u, fxy : f x = y⟩ := h1
  have h2 : f x ∈ u := mem_preimage.mp hx
  show y ∈ u
  exact fxy ▸ h2

-- 2ª demostración
-- ===============

example : f '' (f⁻¹' u) ⊆ u :=
by
  intros y h
  -- y : β
  -- h : y ∈ f '' (f ⁻¹' u)
  -- ⊢ y ∈ u
  rcases h with ⟨x, h2⟩
  -- x : α
  -- h2 : x ∈ f ⁻¹' u ∧ f x = y
  rcases h2 with ⟨hx, fxy⟩
  -- hx : x ∈ f ⁻¹' u
  -- fxy : f x = y
  rw [←fxy]
  -- ⊢ f x ∈ u
  exact hx

-- 3ª demostración
-- ===============

example : f '' (f⁻¹' u) ⊆ u :=
by
  intros y h
  -- y : β
  -- h : y ∈ f '' (f ⁻¹' u)
  -- ⊢ y ∈ u
  rcases h with ⟨x, hx, fxy⟩
  -- x : α
  -- hx : x ∈ f ⁻¹' u
  -- fxy : f x = y
  rw [←fxy]
  -- ⊢ f x ∈ u
  exact hx

-- 4ª demostración
-- ===============

example : f '' (f⁻¹' u) ⊆ u :=
by
  rintro y ⟨x, hx, fxy⟩
  -- y : β
  -- x : α
  -- hx : x ∈ f ⁻¹' u
  -- fxy : f x = y
  -- ⊢ y ∈ u
  rw [←fxy]
  -- ⊢ f x ∈ u
  exact hx

-- 5ª demostración
-- ===============

example : f '' (f⁻¹' u) ⊆ u :=
by
  rintro y ⟨x, hx, rfl⟩
  -- x : α
  -- hx : x ∈ f ⁻¹' u
  -- ⊢ f x ∈ u
  exact hx

-- 6ª demostración
-- ===============

example : f '' (f⁻¹' u) ⊆ u :=
image_preimage_subset f u

-- Lemas usados
-- ============

-- #check (image_preimage_subset f u : f '' (f⁻¹' u) ⊆ u)
\end{verbatim}
Se puede interactuar con las pruebas anteriores en \href{https://lean.math.hhu.de/\#url=https://raw.githubusercontent.com/jaalonso/Calculemus2/main/src/Imagen\_de\_la\_imagen\_inversa.lean}{Lean 4 Web}.

\section{Si f es suprayectiva, entonces u ⊆ f[f⁻¹[u]​]}
\label{sec:orga37f8c6}
\begin{verbatim}
-- ---------------------------------------------------------------------
-- Demostrar que si f es suprayectiva, entonces
--    u ⊆ f '' (f⁻¹' u)
-- ----------------------------------------------------------------------

-- Demostración en lenguaje natural
-- ================================

-- Sea y ∈ u. Por ser f suprayectiva, exite un x tal que
--    f(x) = y                                                       (1)
-- Por tanto, x ∈ f⁻¹[u] y
--    f(x) ∈ f[f⁻¹[u]]                                               (2)
-- Finalmente, por (1) y (2),
--    y ∈ f[f⁻¹[u]]

-- Demostraciones con Lean4
-- ========================

import Mathlib.Data.Set.Function
open Set Function
variable {α β : Type _}
variable (f : α → β)
variable (u : Set β)

-- 1ª demostración
-- ===============

example
  (h : Surjective f)
  : u ⊆ f '' (f⁻¹' u) :=
by
  intros y yu
  -- y : β
  -- yu : y ∈ u
  -- ⊢ y ∈ f '' (f ⁻¹' u)
  rcases h y with ⟨x, fxy⟩
  -- x : α
  -- fxy : f x = y
  use x
  -- ⊢ x ∈ f ⁻¹' u ∧ f x = y
  constructor
  { -- ⊢ x ∈ f ⁻¹' u
    apply mem_preimage.mpr
    -- ⊢ f x ∈ u
    rw [fxy]
    -- ⊢ y ∈ u
    exact yu }
  { -- ⊢ f x = y
    exact fxy }

-- 2ª demostración
-- ===============

example
  (h : Surjective f)
  : u ⊆ f '' (f⁻¹' u) :=
by
  intros y yu
  -- y : β
  -- yu : y ∈ u
  -- ⊢ y ∈ f '' (f ⁻¹' u)
  rcases h y with ⟨x, fxy⟩
  -- x : α
  -- fxy : f x = y
  -- ⊢ y ∈ f '' (f ⁻¹' u)
  use x
  -- ⊢ x ∈ f ⁻¹' u ∧ f x = y
  constructor
  { show f x ∈ u
    rw [fxy]
    -- ⊢ y ∈ u
    exact yu }
  { show f x = y
    exact fxy }

-- 3ª demostración
-- ===============

example
  (h : Surjective f)
  : u ⊆ f '' (f⁻¹' u) :=
by
  intros y yu
  -- y : β
  -- yu : y ∈ u
  -- ⊢ y ∈ f '' (f ⁻¹' u)
  rcases h y with ⟨x, fxy⟩
  -- x : α
  -- fxy : f x = y
  aesop

-- Lemas usados
-- ============

-- variable (x : α)
-- #check (mem_preimage : x ∈ f ⁻¹' u ↔ f x ∈ u)
\end{verbatim}
Se puede interactuar con las pruebas anteriores en \href{https://lean.math.hhu.de/\#url=https://raw.githubusercontent.com/jaalonso/Calculemus2/main/src/Imagen\_de\_imagen\_inversa\_de\_aplicaciones\_suprayectivas.lean}{Lean 4 Web}.

\section{Si s ⊆ t, entonces f[s] ⊆ f[t]}
\label{sec:org0b877a4}
\begin{verbatim}
-- ---------------------------------------------------------------------
-- Demostrar que si s ⊆ t, entonces
--    f '' s ⊆ f '' t
-- ----------------------------------------------------------------------

-- Demostración en lenguaje natural
-- ================================

-- Sea y ∈ f[s]. Entonces, existe un x tal que
--    x ∈ s                                                          (1)
--    f(x) = y                                                       (2)
-- Por (1) y la hipótesis,
--    x ∈ t                                                          (3)
-- Por (3),
--    f(x) ∈ f[t]                                                    (4)
-- y, por (2) y (4),
--    y ∈ f[t]

-- Demostraciones con Lean4
-- ========================

import Mathlib.Data.Set.Function
import Mathlib.Tactic

open Set

variable {α β : Type _}
variable (f : α → β)
variable (s t : Set α)

-- 1ª demostración
-- ===============

example
  (h : s ⊆ t)
  : f '' s ⊆ f '' t :=
by
  intros y hy
  -- y : β
  -- hy : y ∈ f '' s
  -- ⊢ y ∈ f '' t
  rw [mem_image] at hy
  -- hy : ∃ x, x ∈ s ∧ f x = y
  rcases hy with ⟨x, hx⟩
  -- x : α
  -- hx : x ∈ s ∧ f x = y
  rcases hx with ⟨xs, fxy⟩
  -- xs : x ∈ s
  -- fxy : f x = y
  use x
  -- ⊢ x ∈ t ∧ f x = y
  constructor
  . -- ⊢ x ∈ t
    exact h xs
  . -- ⊢ f x = y
    exact fxy

-- 2ª demostración
-- ===============

example
  (h : s ⊆ t)
  : f '' s ⊆ f '' t :=
by
  intros y hy
  -- y : β
  -- hy : y ∈ f '' s
  -- ⊢ y ∈ f '' t
  rcases hy with ⟨x, xs, fxy⟩
  -- x : α
  -- xs : x ∈ s
  -- fxy : f x = y
  use x
  -- ⊢ x ∈ t ∧ f x = y
  exact ⟨h xs, fxy⟩

-- 3ª demostración
-- ===============

example
  (h : s ⊆ t)
  : f '' s ⊆ f '' t :=
image_subset f h

-- Lemas usados
-- ============

-- variable (y : β)
-- #check (mem_image f s y : y ∈ f '' s ↔ ∃ x, x ∈ s ∧ f x = y)
-- #check (image_subset f : s ⊆ t → f '' s ⊆ f '' t)
\end{verbatim}
Se puede interactuar con las pruebas anteriores en \href{https://lean.math.hhu.de/\#url=https://raw.githubusercontent.com/jaalonso/Calculemus2/main/src/Monotonia\_de\_la\_imagen\_de\_conjuntos.lean}{Lean 4 Web}.

\section{Si u ⊆ v, entonces f⁻¹[u] ⊆ f⁻¹[v]}
\label{sec:org67a1b97}
\begin{verbatim}
-- ---------------------------------------------------------------------
-- Demostrar que si u ⊆ v, entonces
--    f ⁻¹' u ⊆ f ⁻¹' v
-- ----------------------------------------------------------------------

-- Demostración en lenguaje natural
-- ================================

-- Por la siguiente cadena de implicaciones:
--    x ∈ f⁻¹[u] ⟹ f(x) ∈ u
--               ⟹ f(x) ∈ v      [porque u ⊆ v]
--               ⟹ x ∈ f⁻¹[v]

-- Demostraciones con Lean4
-- ========================

import Mathlib.Data.Set.Function
open Set

variable {α β : Type _}
variable (f : α → β)
variable (u v : Set β)

-- 1ª demostración
-- ===============

example
  (h : u ⊆ v)
  : f ⁻¹' u ⊆ f ⁻¹' v :=
by
  intros x hx
  -- x : α
  -- hx : x ∈ f ⁻¹' u
  -- ⊢ x ∈ f ⁻¹' v
  have h1 : f x ∈ u := mem_preimage.mp hx
  have h2 : f x ∈ v := h h1
  show x ∈ f ⁻¹' v
  exact mem_preimage.mpr h2

-- 2ª demostración
-- ===============

example
  (h : u ⊆ v)
  : f ⁻¹' u ⊆ f ⁻¹' v :=
by
  intros x hx
  -- x : α
  -- hx : x ∈ f ⁻¹' u
  -- ⊢ x ∈ f ⁻¹' v
  apply mem_preimage.mpr
  -- ⊢ f x ∈ v
  apply h
  -- ⊢ f x ∈ u
  apply mem_preimage.mp
  -- ⊢ x ∈ f ⁻¹' u
  exact hx

-- 3ª demostración
-- ===============

example
  (h : u ⊆ v)
  : f ⁻¹' u ⊆ f ⁻¹' v :=
by
  intros x hx
  -- x : α
  -- hx : x ∈ f ⁻¹' u
  -- ⊢ x ∈ f ⁻¹' v
  apply h
  -- ⊢ f x ∈ u
  exact hx

-- 4ª demostración
-- ===============

example
  (h : u ⊆ v)
  : f ⁻¹' u ⊆ f ⁻¹' v :=
by
  intros x hx
  -- x : α
  -- hx : x ∈ f ⁻¹' u
  -- ⊢ x ∈ f ⁻¹' v
  exact h hx

-- 5ª demostración
-- ===============

example
  (h : u ⊆ v)
  : f ⁻¹' u ⊆ f ⁻¹' v :=
fun _ hx ↦ h hx

-- 6ª demostración
-- ===============

example
  (h : u ⊆ v)
  : f ⁻¹' u ⊆ f ⁻¹' v :=
by intro x; apply h

-- 7ª demostración
-- ===============

example
  (h : u ⊆ v)
  : f ⁻¹' u ⊆ f ⁻¹' v :=
preimage_mono h

-- 8ª demostración
-- ===============

example
  (h : u ⊆ v)
  : f ⁻¹' u ⊆ f ⁻¹' v :=
by tauto

-- Lemas usados
-- ============

-- variable (a : α)
-- #check (mem_preimage : a ∈ f ⁻¹' u ↔ f a ∈ u)
-- #check (preimage_mono : u ⊆ v → f ⁻¹' u ⊆ f ⁻¹' v)
\end{verbatim}
Se puede interactuar con las pruebas anteriores en \href{https://lean.math.hhu.de/\#url=https://raw.githubusercontent.com/jaalonso/Calculemus2/main/src/Monotonia\_de\_la\_imagen\_inversa.lean}{Lean 4 Web}.

\section{f⁻¹[A ∪ B] = f⁻¹[A] ∪ f⁻¹[B]}
\label{sec:orgf6872ca}
\begin{verbatim}
-- ---------------------------------------------------------------------
-- Demostrar que
--    f ⁻¹' (A ∪ B) = f ⁻¹' A ∪ f ⁻¹' B
-- ----------------------------------------------------------------------

-- Demostración en lenguaje natural
-- ================================

-- Tenemos que demostrar que, para todo x,
--    x ∈ f⁻¹[A ∪ B] ↔ x ∈ f⁻¹[A] ∪ f⁻¹[B]
-- Lo haremos demostrando las dos implicaciones.
--
-- (⟹) Supongamos que x ∈ f⁻¹[A ∪ B]. Entonces, f(x) ∈ A ∪ B.
-- Distinguimos dos casos:
--
-- Caso 1: Supongamos que f(x) ∈ A. Entonces, x ∈ f⁻¹[A] y, por tanto,
-- x ∈ f⁻¹[A] ∪ f⁻¹[B].
--
-- Caso 2: Supongamos que f(x) ∈ B. Entonces, x ∈ f⁻¹[B] y, por tanto,
-- x ∈ f⁻¹[A] ∪ f⁻¹[B].
--
-- (⟸) Supongamos que x ∈ f⁻¹[A] ∪ f⁻¹[B]. Distinguimos dos casos.
--
-- Caso 1: Supongamos que x ∈ f⁻¹[A]. Entonces, f(x) ∈ A y, por tanto,
-- f(x) ∈ A ∪ B. Luego, x ∈ f⁻¹[A ∪ B].
--
-- Caso 2: Supongamos que x ∈ f⁻¹[B]. Entonces, f(x) ∈ B y, por tanto,
-- f(x) ∈ A ∪ B. Luego, x ∈ f⁻¹[A ∪ B].

-- Demostraciones con Lean4
-- ========================

import Mathlib.Data.Set.Function

open Set

variable {α β : Type _}
variable (f : α → β)
variable (A B : Set β)

-- 1ª demostración
-- ===============

example : f ⁻¹' (A ∪ B) = f ⁻¹' A ∪ f ⁻¹' B :=
by
  ext x
  -- x : α
  -- ⊢ x ∈ f ⁻¹' (A ∪ B) ↔ x ∈ f ⁻¹' A ∪ f ⁻¹' B
  constructor
  . -- ⊢ x ∈ f ⁻¹' (A ∪ B) → x ∈ f ⁻¹' A ∪ f ⁻¹' B
    intro h
    -- h : x ∈ f ⁻¹' (A ∪ B)
    -- ⊢ x ∈ f ⁻¹' A ∪ f ⁻¹' B
    rw [mem_preimage] at h
    -- h : f x ∈ A ∪ B
    rcases h with fxA | fxB
    . -- fxA : f x ∈ A
      left
      -- ⊢ x ∈ f ⁻¹' A
      apply mem_preimage.mpr
      -- ⊢ f x ∈ A
      exact fxA
    . -- fxB : f x ∈ B
      right
      -- ⊢ x ∈ f ⁻¹' B
      apply mem_preimage.mpr
      -- ⊢ f x ∈ B
      exact fxB
  . -- ⊢ x ∈ f ⁻¹' A ∪ f ⁻¹' B → x ∈ f ⁻¹' (A ∪ B)
    intro h
    -- h : x ∈ f ⁻¹' A ∪ f ⁻¹' B
    -- ⊢ x ∈ f ⁻¹' (A ∪ B)
    rw [mem_preimage]
    -- ⊢ f x ∈ A ∪ B
    rcases h with xfA | xfB
    . -- xfA : x ∈ f ⁻¹' A
      rw [mem_preimage] at xfA
      -- xfA : f x ∈ A
      left
      -- ⊢ f x ∈ A
      exact xfA
    . -- xfB : x ∈ f ⁻¹' B
      rw [mem_preimage] at xfB
      -- xfB : f x ∈ B
      right
      -- ⊢ f x ∈ B
      exact xfB

-- 2ª demostración
-- ===============

example : f ⁻¹' (A ∪ B) = f ⁻¹' A ∪ f ⁻¹' B :=
by
  ext x
  -- x : α
  -- ⊢ x ∈ f ⁻¹' (A ∪ B) ↔ x ∈ f ⁻¹' A ∪ f ⁻¹' B
  constructor
  . -- ⊢ x ∈ f ⁻¹' (A ∪ B) → x ∈ f ⁻¹' A ∪ f ⁻¹' B
    intros h
    -- h : x ∈ f ⁻¹' (A ∪ B)
    -- ⊢ x ∈ f ⁻¹' A ∪ f ⁻¹' B
    rcases h with fxA | fxB
    . -- fxA : f x ∈ A
      left
      -- ⊢ x ∈ f ⁻¹' A
      exact fxA
    . -- fxB : f x ∈ B
      right
      -- ⊢ x ∈ f ⁻¹' B
      exact fxB
  . -- ⊢ x ∈ f ⁻¹' A ∪ f ⁻¹' B → x ∈ f ⁻¹' (A ∪ B)
    intro h
    -- h : x ∈ f ⁻¹' A ∪ f ⁻¹' B
    -- ⊢ x ∈ f ⁻¹' (A ∪ B)
    rcases h with xfA | xfB
    . -- xfA : x ∈ f ⁻¹' A
      left
      -- ⊢ f x ∈ A
      exact xfA
    . -- xfB : x ∈ f ⁻¹' B
      right
      -- ⊢ f x ∈ B
      exact xfB

-- 3ª demostración
-- ===============

example : f ⁻¹' (A ∪ B) = f ⁻¹' A ∪ f ⁻¹' B :=
by
  ext x
  -- x : α
  -- ⊢ x ∈ f ⁻¹' (A ∪ B) ↔ x ∈ f ⁻¹' A ∪ f ⁻¹' B
  constructor
  . -- ⊢ x ∈ f ⁻¹' (A ∪ B) → x ∈ f ⁻¹' A ∪ f ⁻¹' B
    rintro (fxA | fxB)
    . -- fxA : f x ∈ A
      -- ⊢ x ∈ f ⁻¹' A ∪ f ⁻¹' B
      exact Or.inl fxA
    . -- fxB : f x ∈ B
      -- ⊢ x ∈ f ⁻¹' A ∪ f ⁻¹' B
      exact Or.inr fxB
  . -- ⊢ x ∈ f ⁻¹' A ∪ f ⁻¹' B → x ∈ f ⁻¹' (A ∪ B)
    rintro (xfA | xfB)
    . -- xfA : x ∈ f ⁻¹' A
      -- ⊢ x ∈ f ⁻¹' (A ∪ B)
      exact Or.inl xfA
    . -- xfB : x ∈ f ⁻¹' B
      -- ⊢ x ∈ f ⁻¹' (A ∪ B)
      exact Or.inr xfB

-- 4ª demostración
-- ===============

example : f ⁻¹' (A ∪ B) = f ⁻¹' A ∪ f ⁻¹' B :=
by
  ext x
  -- x : α
  -- ⊢ x ∈ f ⁻¹' (A ∪ B) ↔ x ∈ f ⁻¹' A ∪ f ⁻¹' B
  constructor
  . -- ⊢ x ∈ f ⁻¹' (A ∪ B) → x ∈ f ⁻¹' A ∪ f ⁻¹' B
    aesop
  . -- ⊢ x ∈ f ⁻¹' A ∪ f ⁻¹' B → x ∈ f ⁻¹' (A ∪ B)
    aesop

-- 5ª demostración
-- ===============

example : f ⁻¹' (A ∪ B) = f ⁻¹' A ∪ f ⁻¹' B :=
by
  ext x
  -- x : α
  -- ⊢ x ∈ f ⁻¹' (A ∪ B) ↔ x ∈ f ⁻¹' A ∪ f ⁻¹' B
  aesop

-- 6ª demostración
-- ===============

example : f ⁻¹' (A ∪ B) = f ⁻¹' A ∪ f ⁻¹' B :=
by ext ; aesop

-- 7ª demostración
-- ===============

example : f ⁻¹' (A ∪ B) = f ⁻¹' A ∪ f ⁻¹' B :=
by ext ; rfl

-- 8ª demostración
-- ===============

example : f ⁻¹' (A ∪ B) = f ⁻¹' A ∪ f ⁻¹' B :=
rfl

-- 9ª demostración
-- ===============

example : f ⁻¹' (A ∪ B) = f ⁻¹' A ∪ f ⁻¹' B :=
preimage_union

-- 10ª demostración
-- ===============

example : f ⁻¹' (A ∪ B) = f ⁻¹' A ∪ f ⁻¹' B :=
by simp

-- Lemas usados
-- ============

-- variable (x : α)
-- variable (p q : Prop)
-- #check (Or.inl: p → p ∨ q)
-- #check (Or.inr: q → p ∨ q)
-- #check (mem_preimage : x ∈ f ⁻¹' A ↔ f x ∈ A)
-- #check (preimage_union : f ⁻¹' (A ∪ B) = f ⁻¹' A ∪ f ⁻¹' B)
\end{verbatim}
Se puede interactuar con las pruebas anteriores en \href{https://lean.math.hhu.de/\#url=https://raw.githubusercontent.com/jaalonso/Calculemus2/main/src/Imagen\_inversa\_de\_la\_union.lean}{Lean 4 Web}.

\section{f[s ∩ t] ⊆ f[s] ∩ f[t]}
\label{sec:orgc95ef2a}
\begin{verbatim}
-- ---------------------------------------------------------------------
-- Demostrar que
--    f '' (s ∩ t) ⊆ f '' s ∩ f '' t
-- ----------------------------------------------------------------------

-- Demostración en lenguaje natural
-- ================================

-- Sea tal que
--    y ∈ f[s ∩ t]
-- Por tanto, existe un x tal que
--   x ∈ s ∩ t                                                       (1)
--   f(x) = y                                                        (2)
-- Por (1), se tiene que
--   x ∈ s                                                           (3)
--   x ∈ t                                                           (4)
-- Por (2) y (3), se tiene
--   y ∈ f[s]                                                        (5)
-- Por (2) y (4), se tiene
--   y ∈ f[t]                                                        (6)
-- Por (5) y (6), se tiene
--   y ∈ f[s] ∩ f[t]

-- Demostraciones con Lean4
-- ========================

import Mathlib.Data.Set.Function
import Mathlib.Tactic

open Set

variable {α β : Type _}
variable (f : α → β)
variable (s t : Set α)

-- 1ª demostración
-- ===============

example : f '' (s ∩ t) ⊆ f '' s ∩ f '' t :=
by
  intros y hy
  -- y : β
  -- hy : y ∈ f '' (s ∩ t)
  -- ⊢ y ∈ f '' s ∩ f '' t
  rcases hy with ⟨x, hx⟩
  -- x : α
  -- hx : x ∈ s ∩ t ∧ f x = y
  rcases hx with ⟨xst, fxy⟩
  -- xst : x ∈ s ∩ t
  -- fxy : f x = y
  constructor
  . -- ⊢ y ∈ f '' s
    use x
    -- ⊢ x ∈ s ∧ f x = y
    constructor
    . -- ⊢ x ∈ s
      exact xst.1
    . -- ⊢ f x = y
      exact fxy
  . -- ⊢ y ∈ f '' t
    use x
    -- ⊢ x ∈ t ∧ f x = y
    constructor
    . -- ⊢ x ∈ t
      exact xst.2
    . -- ⊢ f x = y
      exact fxy

-- 2ª demostración
-- ===============

example : f '' (s ∩ t) ⊆ f '' s ∩ f '' t :=
by
  intros y hy
  -- y : β
  -- hy : y ∈ f '' (s ∩ t)
  -- ⊢ y ∈ f '' s ∩ f '' t
  rcases hy with ⟨x, ⟨xs, xt⟩, fxy⟩
  -- x : α
  -- fxy : f x = y
  -- xs : x ∈ s
  -- xt : x ∈ t
  constructor
  . -- ⊢ y ∈ f '' s
    use x
    -- ⊢ x ∈ s ∧ f x = y
    exact ⟨xs, fxy⟩
  . -- ⊢ y ∈ f '' t
    use x
    -- ⊢ x ∈ t ∧ f x = y
    exact ⟨xt, fxy⟩

-- 3ª demostración
-- ===============

example : f '' (s ∩ t) ⊆ f '' s ∩ f '' t :=
image_inter_subset f s t

-- 4ª demostración
-- ===============

example : f '' (s ∩ t) ⊆ f '' s ∩ f '' t :=
by intro ; aesop

-- Lemas usados
-- ============

-- #check (image_inter_subset f s t : f '' (s ∩ t) ⊆ f '' s ∩ f '' t)
\end{verbatim}
Se puede interactuar con las pruebas anteriores en \href{https://lean.math.hhu.de/\#url=https://raw.githubusercontent.com/jaalonso/Calculemus2/main/src/Imagen\_de\_la\_interseccion.lean}{Lean 4 Web}.

\section{Si f es inyectiva, entonces f[s] ∩ f[t] ⊆ f[s ∩ t]}
\label{sec:org3aba91a}
\begin{verbatim}
-- ---------------------------------------------------------------------
-- Demostrar que si f es inyectiva, entonces
--    f '' s ∩ f '' t ⊆ f '' (s ∩ t)
-- ----------------------------------------------------------------------

-- Demostración en lenguaje natural
-- ================================

-- Sea y ∈ f[s] ∩ f[t]. Entonces, existen x₁ y x₂ tales que
--    x₁ ∈ s                                                         (1)
--    f(x₁) = y                                                      (2)
--    x₂ ∈ t                                                         (3)
--    f(x₂) = y                                                      (4)
-- De (2) y (4) se tiene que
--    f(x₁) = f(x₂)
-- y, por ser f inyectiva, se tiene que
--    x₁ = x₂
-- y, por (1), se tiene que
--    x₂ ∈ t
-- y, por (3), se tiene que
--    x₂ ∈ s ∩ t
-- Por tanto,
--    f(x₂) ∈
-- y, por (4),
--    y ∈ f[s ∩ t]

-- Demostraciones con Lean4
-- ========================

import Mathlib.Data.Set.Function

open Set Function

variable {α β : Type _}
variable (f : α → β)
variable (s t : Set α)

-- 1ª demostración
-- ===============

example
  (h : Injective f)
  : f '' s ∩ f '' t ⊆ f '' (s ∩ t) :=
by
  intros y hy
  -- y : β
  -- hy : y ∈ f '' s ∩ f '' t
  -- ⊢ y ∈ f '' (s ∩ t)
  rcases hy with ⟨hy1, hy2⟩
  -- hy1 : y ∈ f '' s
  -- hy2 : y ∈ f '' t
  rcases hy1 with ⟨x1, hx1⟩
  -- x1 : α
  -- hx1 : x1 ∈ s ∧ f x1 = y
  rcases hx1 with ⟨x1s, fx1y⟩
  -- x1s : x1 ∈ s
  -- fx1y : f x1 = y
  rcases hy2 with ⟨x2, hx2⟩
  -- x2 : α
  -- hx2 : x2 ∈ t ∧ f x2 = y
  rcases hx2 with ⟨x2t, fx2y⟩
  -- x2t : x2 ∈ t
  -- fx2y : f x2 = y
  have h1 : f x1 = f x2 := Eq.trans fx1y fx2y.symm
  have h2 : x1 = x2 := h (congrArg f (h h1))
  have h3 : x2 ∈ s := by rwa [h2] at x1s
  have h4 : x2 ∈ s ∩ t := by exact ⟨h3, x2t⟩
  have h5 : f x2 ∈ f '' (s ∩ t) := mem_image_of_mem f h4
  show y ∈ f '' (s ∩ t)
  rwa [fx2y] at h5

-- 2ª demostración
-- ===============

example
  (h : Injective f)
  : f '' s ∩ f '' t ⊆ f '' (s ∩ t) :=
by
  intros y hy
  -- y : β
  -- hy : y ∈ f '' s ∩ f '' t
  -- ⊢ y ∈ f '' (s ∩ t)
  rcases hy  with ⟨hy1, hy2⟩
  -- hy1 : y ∈ f '' s
  -- hy2 : y ∈ f '' t
  rcases hy1 with ⟨x1, hx1⟩
  -- x1 : α
  -- hx1 : x1 ∈ s ∧ f x1 = y
  rcases hx1 with ⟨x1s, fx1y⟩
  -- x1s : x1 ∈ s
  -- fx1y : f x1 = y
  rcases hy2 with ⟨x2, hx2⟩
  -- x2 : α
  -- hx2 : x2 ∈ t ∧ f x2 = y
  rcases hx2 with ⟨x2t, fx2y⟩
  -- x2t : x2 ∈ t
  -- fx2y : f x2 = y
  use x1
  -- ⊢ x1 ∈ s ∩ t ∧ f x1 = y
  constructor
  . -- ⊢ x1 ∈ s ∩ t
    constructor
    . -- ⊢ x1 ∈ s
      exact x1s
    . -- ⊢ x1 ∈ t
      convert x2t
      -- ⊢ x1 = x2
      apply h
      -- ⊢ f x1 = f x2
      rw [← fx2y] at fx1y
      -- fx1y : f x1 = f x2
      exact fx1y
  . -- ⊢ f x1 = y
    exact fx1y

-- 3ª demostración
-- ===============

example
  (h : Injective f)
  : f '' s ∩ f '' t ⊆ f '' (s ∩ t) :=
by
  rintro y ⟨⟨x1, x1s, fx1y⟩, ⟨x2, x2t, fx2y⟩⟩
  -- y : β
  -- x1 : α
  -- x1s : x1 ∈ s
  -- fx1y : f x1 = y
  -- x2 : α
  -- x2t : x2 ∈ t
  -- fx2y : f x2 = y
  -- ⊢ y ∈ f '' (s ∩ t)
  use x1
  -- ⊢ x1 ∈ s ∩ t ∧ f x1 = y
  constructor
  . -- ⊢ x1 ∈ s ∩ t
    constructor
    . -- ⊢ x1 ∈ s
      exact x1s
    . -- ⊢ x1 ∈ t
      convert x2t
      -- ⊢ x1 = x2
      apply h
      -- ⊢ f x1 = f x2
      rw [← fx2y] at fx1y
      -- fx1y : f x1 = f x2
      exact fx1y
  . -- ⊢ f x1 = y
    exact fx1y
\end{verbatim}
Se puede interactuar con las pruebas anteriores en \href{https://lean.math.hhu.de/\#url=https://raw.githubusercontent.com/jaalonso/Calculemus2/main/src/Imagen\_de\_la\_interseccion\_de\_aplicaciones\_inyectivas.lean}{Lean 4 Web}.

\section{f[s] $\backslash$ f[t] ⊆ f[s $\backslash$ t]}
\label{sec:org2fbe89c}
\begin{verbatim}
-- ---------------------------------------------------------------------
-- Demostrar que
--    f '' s \ f '' t ⊆ f '' (s \ t)
-- ----------------------------------------------------------------------

-- Demostración en lenguaje natural
-- ================================

-- Sea y ∈ f[s] \ f[t]. Entonces,
--    y ∈ f[s]                                                       (1)
--    y ∉ f[t]                                                       (2)
-- Por (1), existe un x tal que
--    x ∈ s                                                          (3)
--    f(x) = y                                                       (4)
-- Por tanto, para demostrar que y ∈ f[s \ t], basta probar que
-- x ∉ t. Para ello, supongamos que x ∈ t. Entonces, por (4),
-- y ∈ f[t] en contradicción con (2).

-- Demostraciones con Lean4
-- ========================

import Mathlib.Data.Set.Function
import Mathlib.Tactic

open Set

variable {α β : Type _}
variable (f : α → β)
variable (s t : Set α)

-- 1ª demostración
-- ===============

example : f '' s \ f '' t ⊆ f '' (s \ t) :=
by
  intros y hy
  -- y : β
  -- hy : y ∈ f '' s \ f '' t
  -- ⊢ y ∈ f '' (s \ t)
  rcases hy with ⟨yfs, ynft⟩
  -- yfs : y ∈ f '' s
  -- ynft : ¬y ∈ f '' t
  rcases yfs with ⟨x, hx⟩
  -- x : α
  -- hx : x ∈ s ∧ f x = y
  rcases hx with ⟨xs, fxy⟩
  -- xs : x ∈ s
  -- fxy : f x = y
  have h1 : x ∉ t := by
    intro xt
    -- xt : x ∈ t
    -- ⊢ False
    have h2 : f x ∈ f '' t := mem_image_of_mem f xt
    have h3 : y ∈ f '' t := by rwa [fxy] at h2
    show False
    exact ynft h3
  have h4 : x ∈ s \ t := mem_diff_of_mem xs h1
  have h5 : f x ∈ f '' (s \ t) := mem_image_of_mem f h4
  show y ∈ f '' (s \ t)
  rwa [fxy] at h5

-- 2ª demostración
-- ===============

example : f '' s \ f '' t ⊆ f '' (s \ t) :=
by
  intros y hy
  -- y : β
  -- hy : y ∈ f '' s \ f '' t
  -- ⊢ y ∈ f '' (s \ t)
  rcases hy with ⟨yfs, ynft⟩
  -- yfs : y ∈ f '' s
  -- ynft : ¬y ∈ f '' t
  rcases yfs with ⟨x, hx⟩
  -- x : α
  -- hx : x ∈ s ∧ f x = y
  rcases hx with ⟨xs, fxy⟩
  -- xs : x ∈ s
  -- fxy : f x = y
  use x
  -- ⊢ x ∈ s \ t ∧ f x = y
  constructor
  . -- ⊢ x ∈ s \ t
    constructor
    . -- ⊢ x ∈ s
      exact xs
    . -- ⊢ ¬x ∈ t
      intro xt
      -- xt : x ∈ t
      -- ⊢ False
      apply ynft
      -- ⊢ y ∈ f '' t
      rw [←fxy]
      -- ⊢ f x ∈ f '' t
      apply mem_image_of_mem
      -- ⊢ x ∈ t
      exact xt
  . -- ⊢ f x = y
    exact fxy

-- 3ª demostración
-- ===============

example : f '' s \ f '' t ⊆ f '' (s \ t) :=
by
  rintro y ⟨⟨x, xs, fxy⟩, ynft⟩
  -- y : β
  -- ynft : ¬y ∈ f '' t
  -- x : α
  -- xs : x ∈ s
  -- fxy : f x = y
  -- ⊢ y ∈ f '' (s \ t)
  use x
  -- ⊢ x ∈ s \ t ∧ f x = y
  aesop

-- 4ª demostración
-- ===============

example : f '' s \ f '' t ⊆ f '' (s \ t) :=
fun y ⟨⟨x, xs, fxy⟩, ynft⟩ ↦ ⟨x, by aesop⟩

-- 5ª demostración
-- ===============

example : f '' s \ f '' t ⊆ f '' (s \ t) :=
subset_image_diff f s t

-- Lemmas usados
-- =============

-- variable (x : α)
-- #check (mem_image_of_mem f : x  ∈ s → f x ∈ f '' s)
-- #check (subset_image_diff f s t : f '' s \ f '' t ⊆ f '' (s \ t))
\end{verbatim}
Se puede interactuar con las pruebas anteriores en \href{https://lean.math.hhu.de/\#url=https://raw.githubusercontent.com/jaalonso/Calculemus2/main/src/Imagen\_de\_la\_diferencia\_de\_conjuntos.lean}{Lean 4 Web}.

\section{f[s] ∩ v = f[s ∩ f⁻¹[v]​]}
\label{sec:org26b64d8}
\begin{verbatim}
-- ---------------------------------------------------------------------
-- Demostrar que
--    (f '' s) ∩ t = f '' (s ∩ f ⁻¹' t)
-- ----------------------------------------------------------------------

-- Demostración en lenguaje natural
-- ================================

-- Tenemos que demostrar que, para toda y,
--    y ∈ f[s] ∩ t ↔ y ∈ f[s ∩ f⁻¹[t]]
-- Lo haremos probando las dos implicaciones.
--
-- (⟹) Supongamos que y ∈ f[s] ∩ t. Entonces, se tiene que
--    y ∈ f[s]                                                       (1)
--    y ∈ t                                                          (2)
-- Por (1), existe un x tal que
--    x ∈ s                                                          (3)
--    f(x) = y                                                       (4)
-- Por (2) y (4),
--    f(x) ∈ t
-- y, por tanto,
--    x ∈ f⁻¹[t]
-- que, junto con (3), da
--    x ∈ s ∩ f⁻¹[t]
-- y, por tanto,
--    f(x) ∈ f[s ∩ f⁻¹[t]]
-- que, junto con (4), da
--    y ∈ f[s ∩ f⁻¹[t]]
--
-- (⟸) Supongamos que y ∈ f[s ∩ f⁻¹[t]]. Entonces, existe un x tal que
--    x ∈ s ∩ f⁻¹[t]                                                 (5)
--    f(x) = y                                                       (6)
-- Por (1), se tiene que
--    x ∈ s                                                          (7)
--    x ∈ f⁻¹[t]                                                     (8)
-- Por (7) se tiene que
--    f(x) ∈ f[s]
-- y, junto con (6), se tiene que
--    y ∈ f[s]                                                       (9)
-- Por (8), se tiene que
--    f(x) ∈ t
-- y, junto con (6), se tiene que
--    y ∈ t                                                         (10)
-- Por (9) y (19), se tiene que
--    y ∈ f[s] ∩ t

-- Demostraciones con Lean4
-- ========================

import Mathlib.Data.Set.Function
import Mathlib.Tactic

open Set

variable {α β : Type _}
variable (f : α → β)
variable (s : Set α)
variable (t : Set β)

-- 1ª demostración
-- ===============

example : (f '' s) ∩ t = f '' (s ∩ f ⁻¹' t) :=
by
  ext y
  -- y : β
  -- ⊢ y ∈ f '' s ∩ t ↔ y ∈ f '' (s ∩ f ⁻¹' t)
  have h1 : y ∈ f '' s ∩ t → y ∈ f '' (s ∩ f ⁻¹' t) := by
    intro hy
    -- hy : y ∈ f '' s ∩ t
    -- ⊢ y ∈ f '' (s ∩ f ⁻¹' t)
    have h1a : y ∈ f '' s := hy.1
    obtain ⟨x : α, hx: x ∈ s ∧ f x = y⟩ := h1a
    have h1b : x ∈ s := hx.1
    have h1c : f x = y := hx.2
    have h1d : y ∈ t := hy.2
    have h1e : f x ∈ t := by rwa [←h1c] at h1d
    have h1f : x ∈ s ∩ f ⁻¹' t := mem_inter h1b h1e
    have h1g : f x ∈ f '' (s ∩ f ⁻¹' t) := mem_image_of_mem f h1f
    show y ∈ f '' (s ∩ f ⁻¹' t)
    rwa [h1c] at h1g
  have h2 : y ∈ f '' (s ∩ f ⁻¹' t) → y ∈ f '' s ∩ t :=  by
    intro hy
    -- hy : y ∈ f '' (s ∩ f ⁻¹' t)
    -- ⊢ y ∈ f '' s ∩ t
    obtain ⟨x : α, hx : x ∈ s ∩ f ⁻¹' t ∧ f x = y⟩ := hy
    have h2a : x ∈ s := hx.1.1
    have h2b : f x ∈ f '' s := mem_image_of_mem f h2a
    have h2c : y ∈ f '' s := by rwa [hx.2] at h2b
    have h2d : x ∈ f ⁻¹' t := hx.1.2
    have h2e : f x ∈ t := mem_preimage.mp h2d
    have h2f : y ∈ t := by rwa [hx.2] at h2e
    show y ∈ f '' s ∩ t
    exact mem_inter h2c h2f
  show y ∈ f '' s ∩ t ↔ y ∈ f '' (s ∩ f ⁻¹' t)
  exact ⟨h1, h2⟩

-- 2ª demostración
-- ===============

example : (f '' s) ∩ t = f '' (s ∩ f ⁻¹' t) :=
by
  ext y
  -- y : β
  -- ⊢ y ∈ f '' s ∩ t ↔ y ∈ f '' (s ∩ f ⁻¹' t)
  constructor
  . -- ⊢ y ∈ f '' s ∩ t → y ∈ f '' (s ∩ f ⁻¹' t)
    intro hy
    -- hy : y ∈ f '' s ∩ t
    -- ⊢ y ∈ f '' (s ∩ f ⁻¹' t)
    cases' hy with hyfs yt
    -- hyfs : y ∈ f '' s
    -- yt : y ∈ t
    cases' hyfs with x hx
    -- x : α
    -- hx : x ∈ s ∧ f x = y
    cases' hx with xs fxy
    -- xs : x ∈ s
    -- fxy : f x = y
    use x
    -- ⊢ x ∈ s ∩ f ⁻¹' t ∧ f x = y
    constructor
    . -- ⊢ x ∈ s ∩ f ⁻¹' t
      constructor
      . -- ⊢ x ∈ s
        exact xs
      . -- ⊢ x ∈ f ⁻¹' t
        rw [mem_preimage]
        -- ⊢ f x ∈ t
        rw [fxy]
        -- ⊢ y ∈ t
        exact yt
    . -- ⊢ f x = y
      exact fxy
  . -- ⊢ y ∈ f '' (s ∩ f ⁻¹' t) → y ∈ f '' s ∩ t
    intro hy
    -- hy : y ∈ f '' (s ∩ f ⁻¹' t)
    -- ⊢ y ∈ f '' s ∩ t
    cases' hy with x hx
    -- x : α
    -- hx : x ∈ s ∩ f ⁻¹' t ∧ f x = y
    constructor
    . -- ⊢ y ∈ f '' s
      use x
      -- ⊢ x ∈ s ∧ f x = y
      constructor
      . -- ⊢ x ∈ s
        exact hx.1.1
      . -- ⊢ f x = y
        exact hx.2
    . -- ⊢ y ∈ t
      cases' hx with hx1 fxy
      -- hx1 : x ∈ s ∩ f ⁻¹' t
      -- fxy : f x = y
      rw [←fxy]
      -- ⊢ f x ∈ t
      rw [←mem_preimage]
      -- ⊢ x ∈ f ⁻¹' t
      exact hx1.2

-- 3ª demostración
-- ===============

example : (f '' s) ∩ t = f '' (s ∩ f ⁻¹' t) :=
by
  ext y
  -- y : β
  -- ⊢ y ∈ f '' s ∩ t ↔ y ∈ f '' (s ∩ f ⁻¹' t)
  constructor
  . -- ⊢ y ∈ f '' s ∩ t → y ∈ f '' (s ∩ f ⁻¹' t)
    rintro ⟨⟨x, xs, fxy⟩, yt⟩
    -- yt : y ∈ t
    -- x : α
    -- xs : x ∈ s
    -- fxy : f x = y
    -- ⊢ y ∈ f '' (s ∩ f ⁻¹' t)
    use x
    -- ⊢ x ∈ s ∩ f ⁻¹' t ∧ f x = y
    constructor
    . -- ⊢ x ∈ s ∩ f ⁻¹' t
      constructor
      . -- ⊢ x ∈ s
        exact xs
      . -- ⊢ x ∈ f ⁻¹' t
        rw [mem_preimage]
        -- ⊢ f x ∈ t
        rw [fxy]
        -- ⊢ y ∈ t
        exact yt
    . -- ⊢ f x = y
      exact fxy
  . -- ⊢ y ∈ f '' (s ∩ f ⁻¹' t) → y ∈ f '' s ∩ t
    rintro ⟨x, ⟨xs, xt⟩, fxy⟩
    -- x : α
    -- fxy : f x = y
    -- xs : x ∈ s
    -- xt : x ∈ f ⁻¹' t
    -- ⊢ y ∈ f '' s ∩ t
    constructor
    . -- ⊢ y ∈ f '' s
      use x, xs
      -- ⊢ f x = y
      exact fxy
    . -- ⊢ y ∈ t
      rw [←fxy]
      -- ⊢ f x ∈ t
      rw [←mem_preimage]
      -- ⊢ x ∈ f ⁻¹' t
      exact xt

-- 4ª demostración
-- ===============

example : (f '' s) ∩ t = f '' (s ∩ f ⁻¹' t) :=
by
  ext y
  -- y : β
  -- ⊢ y ∈ f '' s ∩ t ↔ y ∈ f '' (s ∩ f ⁻¹' t)
  constructor
  . -- ⊢ y ∈ f '' s ∩ t → y ∈ f '' (s ∩ f ⁻¹' t)
    rintro ⟨⟨x, xs, fxy⟩, yt⟩
    -- yt : y ∈ t
    -- x : α
    -- xs : x ∈ s
    -- fxy : f x = y
    -- ⊢ y ∈ f '' (s ∩ f ⁻¹' t)
    aesop
  . -- ⊢ y ∈ f '' (s ∩ f ⁻¹' t) → y ∈ f '' s ∩ t
    rintro ⟨x, ⟨xs, xt⟩, fxy⟩
    -- x : α
    -- fxy : f x = y
    -- xs : x ∈ s
    -- xt : x ∈ f ⁻¹' t
    -- ⊢ y ∈ f '' s ∩ t
    aesop

-- 5ª demostración
-- ===============

example : (f '' s) ∩ t = f '' (s ∩ f ⁻¹' t) :=
by ext ; constructor <;> aesop

-- 6ª demostración
-- ===============

example : (f '' s) ∩ t = f '' (s ∩ f ⁻¹' t) :=
(image_inter_preimage f s t).symm

-- Lemas usados
-- ============

-- variable (x : α)
-- variable (v : Set α)
-- #check (image_inter_preimage f s t : f '' (s ∩ f ⁻¹' t) = f '' s ∩ t)
-- #check (mem_image_of_mem f : x ∈ s → f x ∈ f '' s)
-- #check (mem_inter : x ∈ s → x ∈ v → x ∈ s ∩ v)
-- #check (mem_preimage : x ∈ f ⁻¹' t ↔ f x ∈ t)
\end{verbatim}
Se puede interactuar con las pruebas anteriores en \href{https://lean.math.hhu.de/\#url=https://raw.githubusercontent.com/jaalonso/Calculemus2/main/src/Interseccion\_con\_la\_imagen.lean}{Lean 4 Web}.

\section{Unión con la imagen}
\label{sec:org4a1bb67}
\begin{verbatim}
-- ---------------------------------------------------------------------
-- Demostrar que
--    f '' (s ∪ f ⁻¹' v) ⊆ f '' s ∪ v
-- ----------------------------------------------------------------------

-- Demostración en lenguaje natural
-- ================================

-- Sea y ∈ f[s ∪ f⁻¹[v]]. Entonces, existe un x tal que
--    x ∈ s ∪ f⁻¹[v]                                                 (1)
--    f(x) = y                                                       (2)
-- De (1), se tiene que x ∈ s ó x ∈ f⁻¹[v]. Vamos a demostrar en ambos
-- casos que
--    y ∈ f[s] ∪ v
--
-- Caso 1: Supongamos que x ∈ s. Entonces,
--    f(x) ∈ f[s]
-- y, por (2), se tiene que
--    y ∈ f[s]
-- Por tanto,
--    y ∈ f[s] ∪ v
--
-- Caso 2: Supongamos que x ∈ f⁻¹[v]. Entonces,
--    f(x) ∈ v
-- y, por (2), se tiene que
--    y ∈ v
-- Por tanto,
--    y ∈ f[s] ∪ v

-- Demostraciones con Lean4
-- ========================

import Mathlib.Data.Set.Function
import Mathlib.Tactic

open Set

variable (α β : Type _)
variable (f : α → β)
variable (s : Set α)
variable (v : Set β)

-- 1ª demostración
-- ===============

example : f '' (s ∪ f ⁻¹' v) ⊆ f '' s ∪ v :=
by
  intros y hy
  obtain ⟨x : α, hx : x ∈ s ∪ f ⁻¹' v ∧ f x = y⟩ := hy
  obtain ⟨hx1 : x ∈ s ∪ f ⁻¹' v, fxy : f x = y⟩ := hx
  cases' hx1 with xs xv
  . -- xs : x ∈ s
    have h1 : f x ∈ f '' s := mem_image_of_mem f xs
    have h2 : y ∈ f '' s := by rwa [fxy] at h1
    show y ∈ f '' s ∪ v
    exact mem_union_left v h2
  . -- xv : x ∈ f ⁻¹' v
    have h3 : f x ∈ v := mem_preimage.mp xv
    have h4 : y ∈ v := by rwa [fxy] at h3
    show y ∈ f '' s ∪ v
    exact mem_union_right (f '' s) h4

-- 1ª demostración
-- ===============

example : f '' (s ∪ f ⁻¹' v) ⊆ f '' s ∪ v :=
by
  intros y hy
  obtain ⟨x : α, hx : x ∈ s ∪ f ⁻¹' v ∧ f x = y⟩ := hy
  obtain ⟨hx1 : x ∈ s ∪ f ⁻¹' v, fxy : f x = y⟩ := hx
  cases' hx1 with xs xv
  . -- xs : x ∈ s
    left
    -- ⊢ y ∈ f '' s
    use x
    -- ⊢ x ∈ s ∧ f x = y
    constructor
    . -- ⊢ x ∈ s
      exact xs
    . -- ⊢ f x = y
      exact fxy
  . -- ⊢ y ∈ f '' s ∪ v
    right
    -- ⊢ y ∈ v
    rw [←fxy]
    -- ⊢ f x ∈ v
    exact xv

-- 2ª demostración
-- ===============

example : f '' (s ∪ f ⁻¹' v) ⊆ f '' s ∪ v :=
by
  rintro y ⟨x, xs | xv, fxy⟩
  -- y : β
  -- x : α
  . -- xs : x ∈ s
    -- ⊢ y ∈ f '' s ∪ v
    left
    -- ⊢ y ∈ f '' s
    use x, xs
    -- ⊢ f x = y
    exact fxy
  . -- xv : x ∈ f ⁻¹' v
    -- ⊢ y ∈ f '' s ∪ v
    right
    -- ⊢ y ∈ v
    rw [←fxy]
    -- ⊢ f x ∈ v
    exact xv

-- 3ª demostración
-- ===============

example : f '' (s ∪ f ⁻¹' v) ⊆ f '' s ∪ v :=
by
  rintro y ⟨x, xs | xv, fxy⟩ <;>
  aesop

-- Lemas usados
-- ============

-- variable (x : α)
-- variable (t : Set α)
-- #check (mem_image_of_mem f : x ∈ s → f x ∈ f '' s)
-- #check (mem_preimage : x ∈ f ⁻¹' v ↔ f x ∈ v)
-- #check (mem_union_left t : x ∈ s → x ∈ s ∪ t)
-- #check (mem_union_right s : x ∈ t → x ∈ s ∪ t)
\end{verbatim}
Se puede interactuar con las pruebas anteriores en \href{https://lean.math.hhu.de/\#url=https://raw.githubusercontent.com/jaalonso/Calculemus2/main/src/Union\_con\_la\_imagen.lean}{Lean 4 Web}.

\section{Intersección con la imagen inversa}
\label{sec:orgadb2658}
\begin{verbatim}
-- ---------------------------------------------------------------------
-- Demostrar que
--    (f '' s) ∩ v = f '' (s ∩ f ⁻¹' v)
-- ----------------------------------------------------------------------

-- Demostración en lenguaje natural
-- ================================

-- Tenemmos que demostrar que, para todo y,
--    y ∈ f[s] ∩ v ↔ y ∈ f[s ∩ f⁻¹[v]]
-- Lo haremos demostrando las sod implicaciones.
--
-- (⟹) Supongamos que y ∈ f[s] ∩ v. Entonces,
--    y ∈ f[s]                                                       (1)
--    y ∈ v                                                          (2)
-- Por (1), existe un x tal que
--    x ∈ s                                                          (3)
--    f(x) = y                                                       (4)
-- De (2) y (4), se tiene que
--    f(x) ∈ v
-- y, por tanto,
--    x ∈ f⁻¹[v]                                                     (5)
-- De (3) y (5), se tiene que
--    x ∈ s ∩ f⁻¹[v]
-- Por tanto,
--    f(x) ∈ f[s ∩ f⁻¹[v]]
-- y, por (4),
--    y ∈ f[s ∩ f⁻¹[v]]
--
-- (⟸) Supongamos que y ∈ f[s ∩ f⁻¹[v]]. Entonces, existe un x tal que
--    x ∈ s ∩ f⁻¹[v]                                                 (6)
--    f(x) = y                                                       (7)
-- Por (6), se tiene que
--    x ∈ s                                                          (8)
--    x ∈ f⁻¹[v]                                                     (9)
-- Por (8), se tiene que
--    f(x) ∈ f[s]
-- y, por (7),
--    y ∈ f[s]                                                      (10)
-- Por (9),
--    f(x) ∈ v
-- y, por (7),
--    y ∈ v                                                         (11)
-- Por (10) y (11),
--    y ∈ f[s] ∩ v

-- Demostraciones con Lean4
-- ========================

import Mathlib.Data.Set.Function
import Mathlib.Tactic

open Set

variable {α β : Type _}
variable (f : α → β)
variable (s : Set α)
variable (v : Set β)

-- 1ª demostración
-- ===============

example : (f '' s) ∩ v = f '' (s ∩ f ⁻¹' v) :=
by
  ext y
  -- y : β
  -- ⊢ y ∈ f '' s ∩ v ↔ y ∈ f '' (s ∩ f ⁻¹' v)
  constructor
  . -- ⊢ y ∈ f '' s ∩ v → y ∈ f '' (s ∩ f ⁻¹' v)
    intro hy
    -- hy : y ∈ f '' s ∩ v
    -- ⊢ y ∈ f '' (s ∩ f ⁻¹' v)
    cases' hy with hyfs yv
    -- hyfs : y ∈ f '' s
    -- yv : y ∈ v
    cases' hyfs with x hx
    -- x : α
    -- hx : x ∈ s ∧ f x = y
    cases' hx with xs fxy
    -- xs : x ∈ s
    -- fxy : f x = y
    have h1 : f x ∈ v := by rwa [←fxy] at yv
    have h3 : x ∈ s ∩ f ⁻¹' v := mem_inter xs h1
    have h4 : f x ∈ f '' (s ∩ f ⁻¹' v) := mem_image_of_mem f h3
    show y ∈ f '' (s ∩ f ⁻¹' v)
    rwa [fxy] at h4
  . -- ⊢ y ∈ f '' (s ∩ f ⁻¹' v) → y ∈ f '' s ∩ v
    intro hy
    -- hy : y ∈ f '' (s ∩ f ⁻¹' v)
    -- ⊢ y ∈ f '' s ∩ v
    cases' hy with x hx
    -- x : α
    -- hx : x ∈ s ∩ f ⁻¹' v ∧ f x = y
    cases' hx with hx1 fxy
    -- hx1 : x ∈ s ∩ f ⁻¹' v
    -- fxy : f x = y
    cases' hx1 with xs xfv
    -- xs : x ∈ s
    -- xfv : x ∈ f ⁻¹' v
    have h5 : f x ∈ f '' s := mem_image_of_mem f xs
    have h6 : y ∈ f '' s := by rwa [fxy] at h5
    have h7 : f x ∈ v := mem_preimage.mp xfv
    have h8 : y ∈ v := by rwa [fxy] at h7
    show y ∈ f '' s ∩ v
    exact mem_inter h6 h8

-- 2ª demostración
-- ===============

example : (f '' s) ∩ v = f '' (s ∩ f ⁻¹' v) :=
by
  ext y
  -- y : β
  -- ⊢ y ∈ f '' s ∩ v ↔ y ∈ f '' (s ∩ f ⁻¹' v)
  constructor
  . -- ⊢ y ∈ f '' s ∩ v → y ∈ f '' (s ∩ f ⁻¹' v)
    intro hy
    -- hy : y ∈ f '' s ∩ v
    -- ⊢ y ∈ f '' (s ∩ f ⁻¹' v)
    cases' hy with hyfs yv
    -- hyfs : y ∈ f '' s
    -- yv : y ∈ v
    cases' hyfs with x hx
    -- x : α
    -- hx : x ∈ s ∧ f x = y
    cases' hx with xs fxy
    -- xs : x ∈ s
    -- fxy : f x = y
    use x
    -- ⊢ x ∈ s ∩ f ⁻¹' v ∧ f x = y
    constructor
    . -- ⊢ x ∈ s ∩ f ⁻¹' v
      constructor
      . -- ⊢ x ∈ s
        exact xs
      . -- ⊢ x ∈ f ⁻¹' v
        rw [mem_preimage]
        -- ⊢ f x ∈ v
        rw [fxy]
        -- ⊢ y ∈ v
        exact yv
    . -- ⊢ f x = y
      exact fxy
  . -- ⊢ y ∈ f '' (s ∩ f ⁻¹' v) → y ∈ f '' s ∩ v
    intro hy
    -- hy : y ∈ f '' (s ∩ f ⁻¹' v)
    -- ⊢ y ∈ f '' s ∩ v
    cases' hy with x hx
    -- x : α
    -- hx : x ∈ s ∩ f ⁻¹' v ∧ f x = y
    constructor
    . -- ⊢ y ∈ f '' s
      use x
      -- ⊢ x ∈ s ∧ f x = y
      constructor
      . -- ⊢ x ∈ s
        exact hx.1.1
      . -- ⊢ f x = y
        exact hx.2
    . -- ⊢ y ∈ v
      cases' hx with hx1 fxy
      -- hx1 : x ∈ s ∩ f ⁻¹' v
      -- fxy : f x = y
      rw [←fxy]
      -- ⊢ f x ∈ v
      rw [←mem_preimage]
      -- ⊢ x ∈ f ⁻¹' v
      exact hx1.2

-- 3ª demostración
-- ===============

example : (f '' s) ∩ v = f '' (s ∩ f ⁻¹' v) :=
by
  ext y
  -- y : β
  -- ⊢ y ∈ f '' s ∩ v ↔ y ∈ f '' (s ∩ f ⁻¹' v)
  constructor
  . -- ⊢ y ∈ f '' s ∩ v → y ∈ f '' (s ∩ f ⁻¹' v)
    rintro ⟨⟨x, xs, fxy⟩, yv⟩
    -- yv : y ∈ v
    -- x : α
    -- xs : x ∈ s
    -- fxy : f x = y
    -- ⊢ y ∈ f '' (s ∩ f ⁻¹' v)
    use x
    -- ⊢ x ∈ s ∩ f ⁻¹' v ∧ f x = y
    constructor
    . -- ⊢ x ∈ s ∩ f ⁻¹' v
      constructor
      . -- ⊢ x ∈ s
        exact xs
      . -- ⊢ x ∈ f ⁻¹' v
        rw [mem_preimage]
        -- ⊢ f x ∈ v
        rw [fxy]
        -- ⊢ y ∈ v
        exact yv
    . exact fxy
  . rintro ⟨x, ⟨xs, xv⟩, fxy⟩
    -- x : α
    -- fxy : f x = y
    -- xs : x ∈ s
    -- xv : x ∈ f ⁻¹' v
    -- ⊢ y ∈ f '' s ∩ v
    constructor
    . -- ⊢ y ∈ f '' s
      use x, xs
      -- ⊢ f x = y
      exact fxy
    . -- ⊢ y ∈ v
      rw [←fxy]
      -- ⊢ f x ∈ v
      rw [←mem_preimage]
      -- ⊢ x ∈ f ⁻¹' v
      exact xv

-- 4ª demostración
-- ===============

example : (f '' s) ∩ v = f '' (s ∩ f ⁻¹' v) :=
by
  ext y
  -- y : β
  -- ⊢ y ∈ f '' s ∩ v ↔ y ∈ f '' (s ∩ f ⁻¹' v)
  constructor
  . -- ⊢ y ∈ f '' s ∩ v → y ∈ f '' (s ∩ f ⁻¹' v)
    rintro ⟨⟨x, xs, fxy⟩, yv⟩
    -- yv : y ∈ v
    -- x : α
    -- xs : x ∈ s
    -- fxy : f x = y
    -- ⊢ y ∈ f '' (s ∩ f ⁻¹' v)
    aesop
  . -- ⊢ y ∈ f '' (s ∩ f ⁻¹' v) → y ∈ f '' s ∩ v
    rintro ⟨x, ⟨xs, xv⟩, fxy⟩
    -- x : α
    -- fxy : f x = y
    -- xs : x ∈ s
    -- xv : x ∈ f ⁻¹' v
    -- ⊢ y ∈ f '' s ∩ v
    aesop

-- 5ª demostración
-- ===============

example : (f '' s) ∩ v = f '' (s ∩ f ⁻¹' v) :=
by ext ; constructor <;> aesop

-- 6ª demostración
-- ===============

example : (f '' s) ∩ v = f '' (s ∩ f ⁻¹' v) :=
(image_inter_preimage f s v).symm

-- Lemas usados
-- ============

-- variable (x : α)
-- variable (a b : Set α)
-- #check (image_inter_preimage f s v : f '' (s ∩ f ⁻¹' v) = f '' s ∩ v)
-- #check (mem_image_of_mem  f : x ∈ a → f x ∈ f '' a)
-- #check (mem_inter : x ∈ a → x ∈ b → x ∈ a ∩ b)
-- #check (mem_preimage : x ∈ f ⁻¹' v ↔ f x ∈ v)
\end{verbatim}
Se puede interactuar con las pruebas anteriores en \href{https://lean.math.hhu.de/\#url=https://raw.githubusercontent.com/jaalonso/Calculemus2/main/src/Interseccion\_con\_la\_imagen\_inversa.lean}{Lean 4 Web}.

\section{Unión con la imagen inversa}
\label{sec:orga40a463}
\begin{verbatim}
-- ---------------------------------------------------------------------
-- Demostrar que
--    s ∪ f⁻¹[v] ⊆ f⁻¹[f[s] ∪ v]
-- ----------------------------------------------------------------------

-- Demostración en lenguaje natural
-- ================================

-- Sea x ∈ s ∪ f⁻¹[v]. Entonces, se puede dar dos casos.
--
-- Caso 1: Supongamos que x ∈ s. Entonces, se tiene
--    f(x) ∈ f[s]
--    f(x) ∈ f[s] ∪ v
--    x ∈ f⁻¹[f[s] ∪ v]
--
-- Caso 2: Supongamos que x ∈ f⁻¹[v]. Entonces, se tiene
--    f(x) ∈ v
--    f(x) ∈ f[s] ∪ v
--    x ∈ f⁻¹[f[s] ∪ v]

-- Demostraciones con Lean4
-- ========================

import Mathlib.Data.Set.Function

open Set

variable {α β : Type _}
variable (f : α → β)
variable (s : Set α)
variable (v : Set β)

-- 1ª demostración
-- ===============

example : s ∪ f ⁻¹' v ⊆ f ⁻¹' (f '' s ∪ v) :=
by
  intros x hx
  -- x : α
  -- hx : x ∈ s ∪ f ⁻¹' v
  -- ⊢ x ∈ f ⁻¹' (f '' s ∪ v)
  cases' hx with xs xv
  . -- xs : x ∈ s
    have h1 : f x ∈ f '' s := mem_image_of_mem f xs
    have h2 : f x ∈ f '' s ∪ v := mem_union_left v h1
    show x ∈ f ⁻¹' (f '' s ∪ v)
    exact mem_preimage.mpr h2
  . -- xv : x ∈ f ⁻¹' v
    have h3 : f x ∈ v := mem_preimage.mp xv
    have h4 : f x ∈ f '' s ∪ v := mem_union_right (f '' s) h3
    show x ∈ f ⁻¹' (f '' s ∪ v)
    exact mem_preimage.mpr h4

-- 2ª demostración
-- ===============

example : s ∪ f ⁻¹' v ⊆ f ⁻¹' (f '' s ∪ v) :=
by
  intros x hx
  -- x : α
  -- hx : x ∈ s ∪ f ⁻¹' v
  -- ⊢ x ∈ f ⁻¹' (f '' s ∪ v)
  rw [mem_preimage]
  -- ⊢ f x ∈ f '' s ∪ v
  cases' hx with xs xv
  . -- xs : x ∈ s
    apply mem_union_left
    -- ⊢ f x ∈ f '' s
    apply mem_image_of_mem
    -- ⊢ x ∈ s
    exact xs
  . -- xv : x ∈ f ⁻¹' v
    apply mem_union_right
    -- ⊢ f x ∈ v
    rw [←mem_preimage]
    -- ⊢ x ∈ f ⁻¹' v
    exact xv

-- 3ª demostración
-- ===============

example : s ∪ f ⁻¹' v ⊆ f ⁻¹' (f '' s ∪ v) :=
by
  intros x hx
  -- x : α
  -- hx : x ∈ s ∪ f ⁻¹' v
  -- ⊢ x ∈ f ⁻¹' (f '' s ∪ v)
  cases' hx with xs xv
  . -- xs : x ∈ s
    rw [mem_preimage]
    -- ⊢ f x ∈ f '' s ∪ v
    apply mem_union_left
    -- ⊢ f x ∈ f '' s
    apply mem_image_of_mem
    -- ⊢ x ∈ s
    exact xs
  . -- ⊢ x ∈ f ⁻¹' (f '' s ∪ v)
    rw [mem_preimage]
    -- ⊢ f x ∈ f '' s ∪ v
    apply mem_union_right
    -- ⊢ f x ∈ v
    exact xv

-- 4ª demostración
-- ===============

example : s ∪ f ⁻¹' v ⊆ f ⁻¹' (f '' s ∪ v) :=
by
  rintro x (xs | xv)
  -- x : α
  -- ⊢ x ∈ f ⁻¹' (f '' s ∪ v)
  . -- xs : x ∈ s
    left
    -- ⊢ f x ∈ f '' s
    exact mem_image_of_mem f xs
  . -- xv : x ∈ f ⁻¹' v
    right
    -- ⊢ f x ∈ v
    exact xv

-- 5ª demostración
-- ===============

example : s ∪ f ⁻¹' v ⊆ f ⁻¹' (f '' s ∪ v) :=
by
  rintro x (xs | xv)
  -- x : α
  -- ⊢ x ∈ f ⁻¹' (f '' s ∪ v)
  . -- xs : x ∈ s
    exact Or.inl (mem_image_of_mem f xs)
  . -- xv : x ∈ f ⁻¹' v
    exact Or.inr xv

-- 5ª demostración
-- ===============

example : s ∪ f ⁻¹' v ⊆ f ⁻¹' (f '' s ∪ v) :=
by
  intros x h
  -- x : α
  -- h : x ∈ s ∪ f ⁻¹' v
  -- ⊢ x ∈ f ⁻¹' (f '' s ∪ v)
  exact Or.elim h (fun xs ↦ Or.inl (mem_image_of_mem f xs)) Or.inr

-- 6ª demostración
-- ===============

example : s ∪ f ⁻¹' v ⊆ f ⁻¹' (f '' s ∪ v) :=
fun _ h ↦ Or.elim h (fun xs ↦ Or.inl (mem_image_of_mem f xs)) Or.inr

-- 7ª demostración
-- ===============

example : s ∪ f ⁻¹' v ⊆ f ⁻¹' (f '' s ∪ v) :=
union_preimage_subset s v f

-- Lemas usados
-- ============

-- variable (x : α)
-- variable (t : Set α)
-- variable (a b c : Prop)
-- #check (Or.elim : a ∨ b → (a → c) → (b → c) → c)
-- #check (Or.inl : a → a ∨ b)
-- #check (Or.inr : b → a ∨ b)
-- #check (mem_image_of_mem f : x  ∈ s → f x ∈ f '' s)
-- #check (mem_preimage : x ∈ f ⁻¹' v ↔ f x ∈ v)
-- #check (mem_union_left t : x ∈ s → x ∈ s ∪ t)
-- #check (mem_union_right s : x ∈ t → x ∈ s ∪ t)
-- #check (union_preimage_subset s v f : s ∪ f ⁻¹' v ⊆ f ⁻¹' (f '' s ∪ v))
\end{verbatim}
Se puede interactuar con las pruebas anteriores en \href{https://lean.math.hhu.de/\#url=https://raw.githubusercontent.com/jaalonso/Calculemus2/main/src/Union\_con\_la\_imagen\_inversa.lean}{Lean 4 Web}.

\section{Imagen de la unión general}
\label{sec:orgf60140c}
\begin{verbatim}
-- ----------------------------------------------------------------------
-- Demostrar que
--    f[⋃ᵢAᵢ] = ⋃ᵢf[Aᵢ]
-- ----------------------------------------------------------------------

-- Demostración en lenguaje natural
-- ================================

-- Tenemos que demostrar que, para todo y,
--    y ∈ f[⋃ᵢAᵢ] ↔ y ∈ ⋃ᵢf[Aᵢ]
-- Lo haremos demostrando las dos implicaciones.
--
-- (⟹) Supongamos que y ∈ f[⋃ᵢAᵢ]. Entonces, existe un x tal que
--    x ∈ ⋃ᵢAᵢ                                                       (1)
--    f(x) = y                                                       (2)
-- Por (1), existe un i tal que
--    i ∈ ℕ                                                          (3)
--    x ∈ Aᵢ                                                         (4)
-- Por (4),
--    f(x) ∈ f[Aᵢ]
-- Por (3),
--    f(x) ∈ ⋃ᵢf[Aᵢ]
-- y, por (2),
--    y ∈ ⋃ᵢf[Aᵢ]
--
-- (⟸) Supongamos que y ∈ ⋃ᵢf[Aᵢ]. Entonces, existe un i tal que
--    i ∈ ℕ                                                          (5)
--    y ∈ f[Aᵢ]                                                      (6)
-- Por (6), existe un x tal que
--    x ∈ Aᵢ                                                         (7)
--    f(x) = y                                                       (8)
-- Por (5) y (7),
--    x ∈ ⋃ᵢAᵢ
-- Luego,
--    f(x) ∈ f[⋃ᵢAᵢ]
-- y, por (8),
--    y ∈ f[⋃ᵢAᵢ]

-- Demostraciones con Lean4
-- ========================

import Mathlib.Data.Set.Basic
import Mathlib.Tactic

open Set

variable {α β I : Type _}
variable (f : α → β)
variable (A : ℕ → Set α)

-- 1ª demostración
-- ===============

example : f '' (⋃ i, A i) = ⋃ i, f '' A i :=
by
  ext y
  -- y : β
  -- ⊢ y ∈ f '' ⋃ (i : ℕ), A i ↔ y ∈ ⋃ (i : ℕ), f '' A i
  constructor
  . -- ⊢ y ∈ f '' ⋃ (i : ℕ), A i → y ∈ ⋃ (i : ℕ), f '' A i
    intro hy
    -- hy : y ∈ f '' ⋃ (i : ℕ), A i
    -- ⊢ y ∈ ⋃ (i : ℕ), f '' A i
    have h1 : ∃ x, x ∈ ⋃ i, A i ∧ f x = y := (mem_image f (⋃ i, A i) y).mp hy
    obtain ⟨x, hx : x ∈ ⋃ i, A i ∧ f x = y⟩ := h1
    have xUA : x ∈ ⋃ i, A i := hx.1
    have fxy : f x = y := hx.2
    have xUA : ∃ i, x ∈ A i := mem_iUnion.mp xUA
    obtain ⟨i, xAi : x ∈ A i⟩ := xUA
    have h2 : f x ∈ f '' A i := mem_image_of_mem f xAi
    have h3 : f x ∈ ⋃ i, f '' A i := mem_iUnion_of_mem i h2
    show y ∈ ⋃ i, f '' A i
    rwa [fxy] at h3
  . -- ⊢ y ∈ ⋃ (i : ℕ), f '' A i → y ∈ f '' ⋃ (i : ℕ), A i
    intro hy
    -- hy : y ∈ ⋃ (i : ℕ), f '' A i
    -- ⊢ y ∈ f '' ⋃ (i : ℕ), A i
    have h4 : ∃ i, y ∈ f '' A i := mem_iUnion.mp hy
    obtain ⟨i, h5 : y ∈ f '' A i⟩ := h4
    have h6 : ∃ x, x ∈ A i ∧ f x = y := (mem_image f (A i) y).mp h5
    obtain ⟨x, h7 : x ∈ A i ∧ f x = y⟩ := h6
    have h8 : x ∈ A i := h7.1
    have h9 : x ∈ ⋃ i, A i := mem_iUnion_of_mem i h8
    have h10 : f x ∈ f '' (⋃ i, A i) := mem_image_of_mem f h9
    show y ∈ f '' (⋃ i, A i)
    rwa [h7.2] at h10

-- 2ª demostración
-- ===============

example : f '' (⋃ i, A i) = ⋃ i, f '' A i :=
by
  ext y
  -- y : β
  -- ⊢ y ∈ f '' ⋃ (i : ℕ), A i ↔ y ∈ ⋃ (i : ℕ), f '' A i
  constructor
  . -- ⊢ y ∈ f '' ⋃ (i : ℕ), A i → y ∈ ⋃ (i : ℕ), f '' A i
    intro hy
    -- hy : y ∈ f '' ⋃ (i : ℕ), A i
    -- ⊢ y ∈ ⋃ (i : ℕ), f '' A i
    rw [mem_image] at hy
    -- hy : ∃ x, x ∈ ⋃ (i : ℕ), A i ∧ f x = y
    cases' hy with x hx
    -- x : α
    -- hx : x ∈ ⋃ (i : ℕ), A i ∧ f x = y
    cases' hx with xUA fxy
    -- xUA : x ∈ ⋃ (i : ℕ), A i
    -- fxy : f x = y
    rw [mem_iUnion] at xUA
    -- xUA : ∃ i, x ∈ A i
    cases' xUA with i xAi
    -- i : ℕ
    -- xAi : x ∈ A i
    rw [mem_iUnion]
    -- ⊢ ∃ i, y ∈ f '' A i
    use i
    -- ⊢ y ∈ f '' A i
    rw [←fxy]
    -- ⊢ f x ∈ f '' A i
    apply mem_image_of_mem
    -- ⊢ x ∈ A i
    exact xAi
  . -- ⊢ y ∈ ⋃ (i : ℕ), f '' A i → y ∈ f '' ⋃ (i : ℕ), A i
    intro hy
    -- hy : y ∈ ⋃ (i : ℕ), f '' A i
    -- ⊢ y ∈ f '' ⋃ (i : ℕ), A i
    rw [mem_iUnion] at hy
    -- hy : ∃ i, y ∈ f '' A i
    cases' hy with i yAi
    -- i : ℕ
    -- yAi : y ∈ f '' A i
    cases' yAi with x hx
    -- x : α
    -- hx : x ∈ A i ∧ f x = y
    cases' hx with xAi fxy
    -- xAi : x ∈ A i
    -- fxy : f x = y
    rw [←fxy]
    -- ⊢ f x ∈ f '' ⋃ (i : ℕ), A i
    apply mem_image_of_mem
    -- ⊢ x ∈ ⋃ (i : ℕ), A i
    rw [mem_iUnion]
    -- ⊢ ∃ i, x ∈ A i
    use i
    -- ⊢ x ∈ A i
    exact xAi

-- 3ª demostración
-- ===============

example : f '' (⋃ i, A i) = ⋃ i, f '' A i :=
by
  ext y
  -- y : β
  -- ⊢ y ∈ f '' ⋃ (i : ℕ), A i ↔ y ∈ ⋃ (i : ℕ), f '' A i
  simp
  -- ⊢ (∃ x, (∃ i, x ∈ A i) ∧ f x = y) ↔ ∃ i x, x ∈ A i ∧ f x = y
  constructor
  . -- ⊢ (∃ x, (∃ i, x ∈ A i) ∧ f x = y) → ∃ i x, x ∈ A i ∧ f x = y
    rintro ⟨x, ⟨i, xAi⟩, fxy⟩
    -- x : α
    -- fxy : f x = y
    -- i : ℕ
    -- xAi : x ∈ A i
    -- ⊢ ∃ i x, x ∈ A i ∧ f x = y
    use i, x, xAi
    -- ⊢ f x = y
    exact fxy
  . -- ⊢ (∃ i x, x ∈ A i ∧ f x = y) → ∃ x, (∃ i, x ∈ A i) ∧ f x = y
    rintro ⟨i, x, xAi, fxy⟩
    -- i : ℕ
    -- x : α
    -- xAi : x ∈ A i
    -- fxy : f x = y
    -- ⊢ ∃ x, (∃ i, x ∈ A i) ∧ f x = y
    exact ⟨x, ⟨i, xAi⟩, fxy⟩

-- 4ª demostración
-- ===============

example : f '' (⋃ i, A i) = ⋃ i, f '' A i :=
image_iUnion

-- Lemas usados
-- ============

-- variable (x : α)
-- variable (y : β)
-- variable (s : Set α)
-- variable (i : ℕ)
-- #check (image_iUnion : f '' ⋃ i, A i = ⋃ i, f '' A i)
-- #check (mem_iUnion : x ∈ ⋃ i, A i ↔ ∃ i, x ∈ A i)
-- #check (mem_iUnion_of_mem i : x ∈ A i → x ∈ ⋃ i, A i)
-- #check (mem_image f s y : (y ∈ f '' s ↔ ∃ x, x ∈ s ∧ f x = y))
-- #check (mem_image_of_mem f : x  ∈ s → f x ∈ f '' s)
\end{verbatim}
Se puede interactuar con las pruebas anteriores en \href{https://lean.math.hhu.de/\#url=https://raw.githubusercontent.com/jaalonso/Calculemus2/main/src/Imagen\_de\_la\_union\_general.lean}{Lean 4 Web}.

\section{Imagen de la intersección general}
\label{sec:orga448b98}
\begin{verbatim}
-- ---------------------------------------------------------------------
-- Demostrar que
--    f[⋂ᵢ Aᵢ] ⊆ ⋂ᵢ f[Aᵢ]
-- ----------------------------------------------------------------------

-- Demostración en lenguaje natural
-- ================================

-- Sea y tal que
--    y ∈ f[⋂ᵢ Aᵢ]                                                   (1)
-- Tenemos que demostrar que y ∈ ⋂ᵢ f[Aᵢ]. Para ello, sea i ∈ I, tenemos
-- que demostrar que y ∈ f[Aᵢ].
--
-- Por (1), existe un x tal que
--    x ∈ ⋂ᵢ Aᵢ                                                      (2)
--    f(x) = y                                                       (3)
-- Por (2),
--    x ∈ Aᵢ
-- y, por tanto,
--    f(x) ∈ f[Aᵢ]
-- que, junto con (3), da que
--    y ∈ f[Aᵢ]

-- Demostraciones con Lean4
-- ========================

import Mathlib.Data.Set.Basic
import Mathlib.Tactic

open Set

variable {α β I : Type _}
variable (f : α → β)
variable (A : I → Set α)

-- 1ª demostración
-- ===============

example : f '' (⋂ i, A i) ⊆ ⋂ i, f '' A i :=
by
  intros y h
  -- y : β
  -- h : y ∈ f '' ⋂ (i : I), A i
  -- ⊢ y ∈ ⋂ (i : I), f '' A i
  have h1 : ∃ x, x ∈ ⋂ i, A i ∧ f x = y := (mem_image f (⋂ i, A i) y).mp h
  obtain ⟨x, hx : x ∈ ⋂ i, A i ∧ f x = y⟩ := h1
  have h2 : x ∈ ⋂ i, A i := hx.1
  have h3 : f x = y := hx.2
  have h4 : ∀ i, y ∈ f '' A i := by
    intro i
    have h4a : x ∈ A i := mem_iInter.mp h2 i
    have h4b : f x ∈ f '' A i := mem_image_of_mem f h4a
    show y ∈ f '' A i
    rwa [h3] at h4b
  show y ∈ ⋂ i, f '' A i
  exact mem_iInter.mpr h4

-- 1ª demostración
-- ===============

example : f '' (⋂ i, A i) ⊆ ⋂ i, f '' A i :=
by
  intros y h
  -- y : β
  -- h : y ∈ f '' ⋂ (i : I), A i
  -- ⊢ y ∈ ⋂ (i : I), f '' A i
  apply mem_iInter_of_mem
  -- ⊢ ∀ (i : I), y ∈ f '' A i
  intro i
  -- i : I
  -- ⊢ y ∈ f '' A i
  cases' h with x hx
  -- x : α
  -- hx : x ∈ ⋂ (i : I), A i ∧ f x = y
  cases' hx with xIA fxy
  -- xIA : x ∈ ⋂ (i : I), A i
  -- fxy : f x = y
  rw [←fxy]
  -- ⊢ f x ∈ f '' A i
  apply mem_image_of_mem
  -- ⊢ x ∈ A i
  exact mem_iInter.mp xIA i

-- 2ª demostración
-- ===============

example : f '' (⋂ i, A i) ⊆ ⋂ i, f '' A i :=
by
  intros y h
  -- y : β
  -- h : y ∈ f '' ⋂ (i : I), A i
  -- ⊢ y ∈ ⋂ (i : I), f '' A i
  apply mem_iInter_of_mem
  -- ⊢ ∀ (i : I), y ∈ f '' A i
  intro i
  -- i : I
  -- ⊢ y ∈ f '' A i
  rcases h with ⟨x, xIA, rfl⟩
  -- x : α
  -- xIA : x ∈ ⋂ (i : I), A i
  -- ⊢ f x ∈ f '' A i
  exact mem_image_of_mem f (mem_iInter.mp xIA i)

-- 3ª demostración
-- ===============

example : f '' (⋂ i, A i) ⊆ ⋂ i, f '' A i :=
by
  intro y
  -- y : β
  -- ⊢ y ∈ f '' ⋂ (i : I), A i → y ∈ ⋂ (i : I), f '' A i
  simp
  -- ⊢ ∀ (x : α), (∀ (i : I), x ∈ A i) → f x = y → ∀ (i : I), ∃ x, x ∈ A i ∧ f x = y
  intros x xIA fxy i
  -- x : α
  -- xIA : ∀ (i : I), x ∈ A i
  -- fxy : f x = y
  -- i : I
  -- ⊢ ∃ x, x ∈ A i ∧ f x = y
  use x, xIA i
  -- ⊢ f x = y
  exact fxy

-- 4ª demostración
-- ===============

example : f '' (⋂ i, A i) ⊆ ⋂ i, f '' A i :=
image_iInter_subset A f

-- Lemas usados
-- ============

-- variable (x : α)
-- variable (s : Set α)
-- #check (image_iInter_subset A f : f '' ⋂ i, A i ⊆ ⋂ i, f '' A i)
-- #check (mem_iInter : x ∈ ⋂ i, A i ↔ ∀ i, x ∈ A i)
-- #check (mem_iInter_of_mem : (∀ i, x ∈ A i) → x ∈ ⋂ i, A i)
-- #check (mem_image_of_mem f : x ∈ s → f x ∈ f '' s)
\end{verbatim}
Se puede interactuar con las pruebas anteriores en \href{https://lean.math.hhu.de/\#url=https://raw.githubusercontent.com/jaalonso/Calculemus2/main/src/Imagen\_de\_la\_interseccion\_general.lean}{Lean 4 Web}.

\chapter{Lógica}
\label{sec:org60dd932}

\section{Si ¬(∃x)P(x), entonces (∀x)¬P(x)}
\label{sec:orgd8e564c}
\begin{verbatim}
-- ---------------------------------------------------------------------
-- Demostrar que si ¬(∃x)P(x), entonces (∀x)¬P(x).
-- ----------------------------------------------------------------------

-- Demostración en lenguaje natural
-- ================================

-- Sea y un elemento cualquiera. Tenemos que demostrar ¬P(y). Para ello,
-- supongamos que P(y). Entonces, (∃x)P(x) que es una contradicción con
-- la hipótesis,

-- Demostraciones con Lean4
-- ========================

import Mathlib.Tactic
variable {α : Type _}
variable (P : α → Prop)

-- 1ª demostración
-- ===============

example
  (h : ¬ ∃ x, P x)
  : ∀ x, ¬ P x :=
by
  intros y h1
  -- y : α
  -- h1 : P x
  -- ⊢ False
  apply h
  -- ⊢ ∃ x, P x
  existsi y
  -- ⊢ P y
  exact h1

-- 2ª demostración
-- ===============

example
  (h : ¬ ∃ x, P x)
  : ∀ x, ¬ P x :=
by
  intros y h1
  -- y : α
  -- h1 : P x
  -- ⊢ False
  apply h
  -- ⊢ ∃ x, P x
  use y
  -- ⊢ P y
  exact h1

-- 3ª demostración
-- ===============

example
  (h : ¬ ∃ x, P x)
  : ∀ x, ¬ P x :=
by
  intros y h1
  -- y : α
  -- h1 : P x
  -- ⊢ False
  apply h
  -- ⊢ ∃ x, P x
  exact ⟨y, h1⟩

-- 4ª demostración
-- ===============

example
  (h : ¬ ∃ x, P x)
  : ∀ x, ¬ P x :=
by
  intros y h1
  -- y : α
  -- h1 : P x
  -- ⊢ False
  exact h ⟨y, h1⟩

-- 5ª demostración
-- ===============

example
  (h : ¬ ∃ x, P x)
  : ∀ x, ¬ P x :=
fun y h1 ↦ h ⟨y, h1⟩

-- 6ª demostración
-- ===============

example
  (h : ¬ ∃ x, P x)
  : ∀ x, ¬ P x :=
by
  push_neg at h
  exact h

-- 7ª demostración
-- ===============

example
  (h : ¬ ∃ x, P x)
  : ∀ x, ¬ P x :=
not_exists.mp h

-- 8ª demostración
-- ===============

example
  (h : ¬ ∃ x, P x)
  : ∀ x, ¬ P x :=
by aesop

-- Lemas usados
-- ============

-- #check (not_exists : (¬∃ x, P x) ↔ ∀ (x : α), ¬P x)
\end{verbatim}
Se puede interactuar con las pruebas anteriores en \href{https://lean.math.hhu.de/\#url=https://raw.githubusercontent.com/jaalonso/Calculemus2/main/src/Para\_todo\_no\_de\_no\_existe.lean}{Lean 4 Web}

\section{Si (∀x)¬P(x), entonces ¬(∃x)P(x)}
\label{sec:orgbb3a7da}
\begin{verbatim}
-- ---------------------------------------------------------------------
-- Demostrar que si (∀x)¬P(x), entonces ¬(∃x)P(x).
-- ----------------------------------------------------------------------

-- Demostración en lenguaje natural
-- ================================

-- Supongamos que (∃x)P(x). Sea y tal que P(y). Puesto que (∀x)¬P(x), se
-- tiene que ¬P(y) que es una contradicción con P(y).

-- Demostraciones con Lean4
-- ========================

import Mathlib.Tactic
variable {α : Type _}
variable (P : α → Prop)

-- 1ª demostración
-- ===============

example
  (h : ∀ x, ¬ P x)
  : ¬ ∃ x, P x :=
by
  intro h1
  -- h1 : ∃ x, P x
  -- ⊢ False
  rcases h1 with ⟨y, hy : P y⟩
  have h2 : ¬P y := h y
  exact h2 hy

-- 2ª demostración
-- ===============

example
  (h : ∀ x, ¬ P x)
  : ¬ ∃ x, P x :=
by
  intro h1
  -- h1 : ∃ x, P x
  -- ⊢ False
  rcases h1 with ⟨y, hy : P y⟩
  exact (h y) hy

-- 3ª demostración
-- ===============

example
  (h : ∀ x, ¬ P x)
  : ¬ ∃ x, P x :=
by
  rintro ⟨y, hy : P y⟩
  exact (h y) hy

-- 4ª demostración
-- ===============

example
  (h : ∀ x, ¬ P x)
  : ¬ ∃ x, P x :=
fun ⟨y, hy⟩ ↦ (h y) hy

-- 5ª demostración
-- ===============

example
  (h : ∀ x, ¬ P x)
  : ¬ ∃ x, P x :=
not_exists_of_forall_not h

-- 6ª demostración
-- ===============

example
  (h : ∀ x, ¬ P x)
  : ¬ ∃ x, P x :=
by aesop

-- Lemas usados
-- ============

-- variable (q : Prop)
-- #check (not_exists_of_forall_not : (∀ x, P x → q) → (∃ x, P x) → q)
\end{verbatim}
Se puede interactuar con las pruebas anteriores en \href{https://lean.math.hhu.de/\#url=https://raw.githubusercontent.com/jaalonso/Calculemus2/main/src/No\_existe\_de\_para\_todo\_no.lean}{Lean 4 Web}

\section{Si ¬(∀x)P(x), entonces (∃x)¬P(x)}
\label{sec:orgb05a080}
\begin{verbatim}
-- ---------------------------------------------------------------------
-- Demostrar que si ¬(∀x)P(x), entonces (∃x)¬P(x).
-- ----------------------------------------------------------------------

-- Demostración en lenguaje natural
-- ================================

-- Por reducción al absurdo, supongamos que ¬(∃x)¬P(x). Para obtener una
-- contradicción, demostraremos la negación de la hipótesis; es decir,
-- que (∀x)P(x). Para ello, sea y un elemento cualquiera y tenemos que
-- demostrar P(y). De nuevo, lo haremos por reducción al absurdo: Para
-- ello, supongamos que ¬P(y). Entonces, se tiene que (∃x)¬P(x) en
-- contradicción con nuestro primer supuesto de ¬(∃x)¬P(x).

-- Demostraciones con Lean4
-- ========================

import Mathlib.Tactic
variable {α : Type _}
variable (P : α → Prop)

-- 1ª demostración
-- ===============

example
  (h : ¬ ∀ x, P x)
  : ∃ x, ¬ P x :=
by
  by_contra h1
  -- h1 : ¬∃ x, ¬P x
  -- ⊢ False
  apply h
  -- ⊢ ∀ (x : α), P x
  intro y
  -- y : α
  -- ⊢ P y
  show P y
  by_contra h2
  -- h2 : ¬P y
  -- ⊢ False
  exact h1 ⟨y, h2⟩

-- 2ª demostración
-- ===============

example
  (h : ¬ ∀ x, P x)
  : ∃ x, ¬ P x :=
not_forall.mp h

-- 3ª demostración
-- ===============

example
  (h : ¬ ∀ x, P x)
  : ∃ x, ¬ P x :=
by aesop

-- Lemas usados
-- ============

-- #check (not_forall : (¬∀ x, P x) ↔ ∃ x, ¬P x)
\end{verbatim}
Se puede interactuar con las pruebas anteriores en \href{https://lean.math.hhu.de/\#url=https://raw.githubusercontent.com/jaalonso/Calculemus2/main/src/Existe\_no\_de\_no\_para\_todo.lean}{Lean 4 Web}

\section{Si (∃x)¬P(x), entonces ¬(∀x)P(x)}
\label{sec:orgbc03b4e}
\begin{verbatim}
-- ---------------------------------------------------------------------
-- Demostrar que si (∃x)¬P(x), entonces ¬(∀x)P(x).
-- ----------------------------------------------------------------------

-- Demostración en lenguaje natural
-- ================================

-- Supongamos que (∀x)P(x) y tenemos que demostrar una
-- contradicción. Por hipótesis, (∃x)¬P(x). Sea y tal que
-- ¬P(y). Entonces, como (∀x)P(x), se tiene que P(y) que es una
-- contradicción con ¬P(y).

-- Demostraciones con Lean4
-- ========================

import Mathlib.Tactic
variable {α : Type _}
variable (P : α → Prop)

-- 1ª demostración
-- ===============

example
  (h : ∃ x, ¬ P x)
  : ¬ ∀ x, P x :=
by
  intro h1
  -- h1 : ∀ (x : α), P x
  -- ⊢ False
  cases' h with y hy
  -- y : α
  -- hy : ¬P y
  apply hy
  -- ⊢ P y
  exact (h1 y)

-- 2ª demostración
-- ===============

example
  (h : ∃ x, ¬ P x)
  : ¬ ∀ x, P x :=
by
  intro h1
  -- h1 : ∀ (x : α), P x
  -- ⊢ False
  rcases h with ⟨y, hy : ¬P y⟩
  apply hy
  -- ⊢ P y
  exact (h1 y)

-- 3ª demostración
-- ===============

example
  (h : ∃ x, ¬ P x)
  : ¬ ∀ x, P x :=
by
  intro h1
  -- h1 : ∀ (x : α), P x
  -- ⊢ False
  rcases h with ⟨y, hy : ¬P y⟩
  exact hy (h1 y)

-- 4ª demostración
-- ===============

example
  (h : ∃ x, ¬ P x)
  : ¬ ∀ x, P x :=
not_forall.mpr h

-- 5ª demostración
-- ===============

example
  (h : ∃ x, ¬ P x)
  : ¬ ∀ x, P x :=
not_forall_of_exists_not h

-- 6ª demostración
-- ===============

example
  (h : ∃ x, ¬ P x)
  : ¬ ∀ x, P x :=
by aesop

-- Lemas usados
-- ============

-- #check (not_forall : (¬∀ x, P x) ↔ ∃ x, ¬P x)
-- #check (not_forall_of_exists_not : (∃ x, ¬P x) → ¬∀ x, P x)
\end{verbatim}
Se puede interactuar con las pruebas anteriores en \href{https://lean.math.hhu.de/\#url=https://raw.githubusercontent.com/jaalonso/Calculemus2/main/src/No\_para\_todo\_de\_existe\_no.lean}{Lean 4 Web}

\section{¬¬P → P}
\label{sec:org4c76c71}
\begin{verbatim}
-- ---------------------------------------------------------------------
-- Demostrar que ¬¬P → P.
-- ----------------------------------------------------------------------

-- Demostración en lenguaje natural
-- ================================

-- Por reducción al absurdo. Supongamos ¬P. Entonces, tenemos una
-- contradicción con la hipótesis (¬¬P).

-- Demostraciones con Lean4
-- ========================

import Mathlib.Tactic
variable (P : Prop)

-- 1ª demostración
-- ===============

example
  (h : ¬¬P)
  : P :=
by
  by_contra h1
  -- h1 : ¬P
  -- ⊢ False
  exact (h h1)

-- 2ª demostración
-- ===============

example
  (h : ¬¬P)
  : P :=
by_contra (fun h1 ↦ h h1)

-- 3ª demostración
-- ===============

example
  (h : ¬¬P)
  : P :=
-- not_not.mp h
of_not_not h

-- 4ª demostración
-- ===============

example
  (h : ¬¬P)
  : P :=
by tauto

-- Lemas usados
-- ============

-- #check (of_not_not : ¬¬P → P)
\end{verbatim}
Se puede interactuar con las pruebas anteriores en \href{https://lean.math.hhu.de/\#url=https://raw.githubusercontent.com/jaalonso/Calculemus2/main/src/}{Lean 4 Web}

\section{P → ¬¬P}
\label{sec:org1c41050}
\begin{verbatim}
-- ---------------------------------------------------------------------
-- Demostrar que P → ¬¬P.
-- ----------------------------------------------------------------------

-- Demostración en lenguaje natural
-- ================================

-- Supongamos ¬P. Entonces, tenemos una contradicción con la hipótesis
-- (P).

-- Demostraciones con Lean4
-- ========================

import Mathlib.Tactic
variable (P : Prop)

-- 1ª demostración
-- ===============

example
  (h : P)
  : ¬¬P :=
by
  intro h1
  -- h1 : ¬P
  -- ⊢ False
  exact (h1 h)

-- 2ª demostración
-- ===============

example
  (h : P)
  : ¬¬P :=
fun h1 ↦ h1 h

-- 3ª demostración
-- ===============

example
  (h : P)
  : ¬¬P :=
not_not_intro h

-- 4ª demostración
-- ===============

example
  (h : P)
  : ¬ ¬ P :=
by tauto

-- Lemas usados
-- ============

-- #check (not_not_intro : P → ¬¬P)
\end{verbatim}
Se puede interactuar con las pruebas anteriores en \href{https://lean.math.hhu.de/\#url=https://raw.githubusercontent.com/jaalonso/Calculemus2/main/src/Introduccion\_doble\_negacion.lean}{Lean 4 Web}

\section{(P → Q) ↔ ¬P ∨ Q}
\label{sec:orgb51b84d}
\begin{verbatim}
-- ---------------------------------------------------------------------
-- Demostrar que
--    (P → Q) ↔ ¬P ∨ Q
-- ----------------------------------------------------------------------

-- Demostración en lenguaje natural
-- ================================

-- Demostraremos cada una de las implicaciones.
--
-- (==>) Supongamos que P → Q. Distinguimos dos subcasos según el valor de
-- P.
--
-- Primer subcaso: suponemos P. Entonces. tenemos Q (por P → Q) y. por
-- tanto, ¬P ∨ Q.
--
-- Segundo subcaso: suponemos ¬P. Entonces. tenemos ¬P ∨ Q.
--
-- (<==) Supongamos que ¬P ∨ Q y P y tenemos que demostrar
-- Q. Distinguimos dos subcasos según ¬P ∨ Q.
--
-- Primer subcaso: Suponemos ¬P. Entonces tenemos una contradicción con
-- P.
--
-- Segundo subcaso: Suponemos Q, que es lo que tenemos que demostrar.

-- Demostraciones con Lean4
-- ========================

import Mathlib.Tactic
variable (P Q : Prop)

-- 1ª demostración
-- ===============

example
  : (P → Q) ↔ ¬P ∨ Q :=
by
  constructor
  . -- ⊢ (P → Q) → ¬P ∨ Q
    intro h1
    -- h1 : P → Q
    -- ⊢ ¬P ∨ Q
    by_cases h2 : P
    . -- h2 : P
      right
      -- ⊢ Q
      apply h1
      -- ⊢ P
      exact h2
    . -- h2 : ¬P
      left
      -- ⊢ ¬P
      exact h2
  . -- ⊢ ¬P ∨ Q → P → Q
    intros h3 h4
    -- h3 : ¬P ∨ Q
    -- h4 : P
    -- ⊢ Q
    rcases h3 with h3a | h3b
    . -- h : ¬P
      exact absurd h4 h3a
    . -- h : Q
      exact h3b
  done

-- 2ª demostración
-- ===============

example
  : (P → Q) ↔ ¬P ∨ Q :=
by
  constructor
  . -- ⊢ (P → Q) → ¬P ∨ Q
    intro h1
    -- h1 : P → Q
    -- ⊢ ¬P ∨ Q
    by_cases h2: P
    . -- h2 : P
      right
      -- ⊢ Q
      exact h1 h2
    . -- h2 : ¬P
      left
      -- ⊢ ¬P
      exact h2
  . -- ⊢ ¬P ∨ Q → P → Q
    intros h3 h4
    -- h3 : ¬P ∨ Q
    -- h4 : P
    -- ⊢ Q
    cases h3
    . -- h : ¬P
      contradiction
    . -- h : Q
      assumption
  done

-- 3ª demostración
-- ===============

example
  (P Q : Prop)
  : (P → Q) ↔ ¬P ∨ Q :=
imp_iff_not_or

-- 4ª demostración
-- ===============

example
  (P Q : Prop)
  : (P → Q) ↔ ¬P ∨ Q :=
by tauto
\end{verbatim}
Se puede interactuar con las pruebas anteriores en \href{https://lean.math.hhu.de/\#url=https://raw.githubusercontent.com/jaalonso/Calculemus2/main/src/Implicacion\_mediante\_disyuncion\_y\_negacion.lean}{Lean 4 Web}

\chapter{Límites de sucesiones}
\label{sec:orgfc9c8b6}

\section{La sucesión constante sₙ = c converge a c}
\label{sec:orgb58767d}
\begin{verbatim}
-- ---------------------------------------------------------------------
-- Demostrar que, para todo a ∈ ℝ, la sucesión constante
--    s(n) = a
-- converge a a.
-- ----------------------------------------------------------------------

-- Demostración en lenguaje natural
-- ================================

-- Tenemos que demostrar que para cada ε ∈ ℝ tal que ε > 0, existe un
-- N ∈ ℕ, tal que (∀n ∈ ℕ)[n ≥ N → |s(n) - a| < ε]. Basta tomar N como
-- 0, ya que para todo n ≥ N se tiene
--    |s(n) - a| = |a - a|
--               = |0|
--               = 0
--               < ε

-- Demostraciones con Lean4
-- ========================

import Mathlib.Data.Real.Basic

def limite (s : ℕ → ℝ) (a : ℝ) :=
  ∀ ε > 0, ∃ N, ∀ n ≥ N, |s n - a| < ε

-- 1ª demostración
-- ===============

example : limite (fun _ : ℕ ↦ c) c :=
by
  intros ε hε
  -- ε : ℝ
  -- hε : ε > 0
  -- ⊢ ∃ N, ∀ (n : ℕ), n ≥ N → |(fun _ => c) n - c| < ε
  use 0
  -- ⊢ ∀ (n : ℕ), n ≥ 0 → |(fun _ => c) n - c| < ε
  intros n _hn
  -- n : ℕ
  -- hn : n ≥ 0
  -- ⊢ |(fun _ => c) n - c| < ε
  show |(fun _ => c) n - c| < ε
  calc |(fun _ => c) n - c| = |c - c| := by dsimp
                          _ = |0|     := by {congr ; exact sub_self c}
                          _ = 0       := abs_zero
                          _ < ε       := hε

-- 2ª demostración
-- ===============

example : limite (fun _ : ℕ ↦ c) c :=
by
  intros ε hε
  -- ε : ℝ
  -- hε : ε > 0
  -- ⊢ ∃ N, ∀ (n : ℕ), n ≥ N → |(fun _ => c) n - c| < ε
  use 0
  -- ⊢ ∀ (n : ℕ), n ≥ 0 → |(fun _ => c) n - c| < ε
  intros n _hn
  -- n : ℕ
  -- hn : n ≥ 0
  -- ⊢ |(fun _ => c) n - c| < ε
  show |(fun _ => c) n - c| < ε
  calc |(fun _ => c) n - c| = 0       := by simp
                          _ < ε       := hε

-- 3ª demostración
-- ===============

example : limite (fun _ : ℕ ↦ c) c :=
by
  intros ε hε
  -- ε : ℝ
  -- hε : ε > 0
  -- ⊢ ∃ N, ∀ (n : ℕ), n ≥ N → |(fun _ => c) n - c| < ε
  aesop

-- 4ª demostración
-- ===============

example : limite (fun _ : ℕ ↦ c) c :=
by
  intros ε hε
  -- ε : ℝ
  -- hε : ε > 0
  -- ⊢ ∃ N, ∀ (n : ℕ), n ≥ N → |(fun _ => c) n - c| < ε
  aesop

-- 5ª demostración
-- ===============

example : limite (fun _ : ℕ ↦ c) c :=
  fun ε hε ↦ by aesop

-- Lemas usados
-- ============

-- #check (sub_self a : a - a = 0)
\end{verbatim}
Se puede interactuar con las pruebas anteriores en \href{https://lean.math.hhu.de/\#url=https://raw.githubusercontent.com/jaalonso/Calculemus2/main/src/Convergencia\_de\_la\_sucesion\_constante.lean}{Lean 4 Web}

\section{Si la sucesión s converge a b y la t a c, entonces s+t converge a b+c}
\label{sec:org0dba1f1}
\begin{verbatim}
-- ---------------------------------------------------------------------
-- Demostrar que si la sucesión u converge a a y la v a b, entonces u+v
-- converge a a+b
-- ----------------------------------------------------------------------

-- Demostración en lenguaje natural
-- ================================

-- En la demostración usaremos los siguientes lemas
--    (∀ a ∈ ℝ)[a > 0 → a / 2 > 0]                                   (L1)
--    (∀ a, b, c ∈ ℝ)[max(a, b) ≤ c → a ≤ c]                         (L2)
--    (∀ a, b, c ∈ ℝ)[max(a, b) ≤ c → b ≤ c]                         (L3)
--    (∀ a, b ∈ ℝ)[|a + b| ≤ |a| + |b|]                              (L4)
--    (∀ a ∈ ℝ)[a / 2 + a / 2 = a]                                   (L5)
--
-- Tenemos que probar que para todo ε ∈ ℝ, si
--    ε > 0                                                          (1)
-- entonces
--    (∃N ∈ ℕ)(∀n ∈ ℕ)[n ≥ N → |(u + v)(n) - (a + b)| < ε]           (2)
--
-- Por (1) y el lema L1, se tiene que
--    ε/2 > 0                                                        (3)
-- Por (3) y porque el límite de u es a, se tiene que
--    (∃N ∈ ℕ)(∀n ∈ ℕ)[n ≥ N → |u(n) - a| < ε/2]
-- Sea N₁ ∈ ℕ tal que
--    (∀n ∈ ℕ)[n ≥ N₁ → |u(n) - a| < ε/2]                            (4)
-- Por (3) y porque el límite de v es b, se tiene que
--    (∃N ∈ ℕ)(∀n ∈ ℕ)[n ≥ N → |v(n) - b| < ε/2]
-- Sea N₂ ∈ ℕ tal que
--    (∀n ∈ ℕ)[n ≥ N₂ → |v(n) - b| < ε/2]                            (5)
-- Sea N = max(N₁, N₂). Veamos que verifica la condición (1). Para ello,
-- sea n ∈ ℕ tal que n ≥ N. Entonces, n ≥ N₁ (por L2) y n ≥ N₂ (por
-- L3). Por tanto, por las propiedades (4) y (5) se tiene que
--    |u(n) - a| < ε/2                                               (6)
--    |v(n) - b| < ε/2                                               (7)
-- Finalmente,
--    |(u + v)(n) - (a + b)| = |(u(n) + v(n)) - (a + b)|
--                           = |(u(n) - a) + (v(n) - b)|
--                           ≤ |u(n) - a| + |v(n) - b|      [por L4]
--                           < ε / 2 + ε / 2                [por (6) y (7)
--                           = ε                            [por L5]

-- Demostraciones con Lean4
-- ========================

import Mathlib.Data.Real.Basic
variable {s t : ℕ → ℝ} {a b c : ℝ}

def limite (s : ℕ → ℝ) (a : ℝ) :=
  ∀ ε > 0, ∃ N, ∀ n ≥ N, |s n - a| < ε

-- 1ª demostración
-- ===============

example
  (hu : limite u a)
  (hv : limite v b)
  : limite (u + v) (a + b) :=
by
  intros ε hε
  -- ε : ℝ
  -- hε : ε > 0
  -- ⊢ ∃ N, ∀ (n : ℕ), n ≥ N → |(u + v) n - (a + b)| < ε
  have hε2 : 0 < ε / 2 := half_pos hε
  cases' hu (ε / 2) hε2 with Nu hNu
  -- Nu : ℕ
  -- hNu : ∀ (n : ℕ), n ≥ Nu → |u n - a| < ε / 2
  cases' hv (ε / 2) hε2 with Nv hNv
  -- Nv : ℕ
  -- hNv : ∀ (n : ℕ), n ≥ Nv → |v n - b| < ε / 2
  clear hu hv hε2 hε
  let N := max Nu Nv
  use N
  -- ⊢ ∀ (n : ℕ), n ≥ N → |(s + t) n - (a + b)| < ε
  intros n hn
  -- n : ℕ
  -- hn : n ≥ N
  have nNu : n ≥ Nu := le_of_max_le_left hn
  specialize hNu n nNu
  -- hNu : |u n - a| < ε / 2
  have nNv : n ≥ Nv := le_of_max_le_right hn
  specialize hNv n nNv
  -- hNv : |v n - b| < ε / 2
  clear hn nNu nNv
  calc |(u + v) n - (a + b)|
       = |(u n + v n) - (a + b)|  := rfl
     _ = |(u n - a) + (v n - b)|  := by { congr; ring }
     _ ≤ |u n - a| + |v n - b|    := by apply abs_add
     _ < ε / 2 + ε / 2            := by linarith [hNu, hNv]
     _ = ε                        := by apply add_halves

-- 2ª demostración
-- ===============

example
  (hu : limite u a)
  (hv : limite v b)
  : limite (u + v) (a + b) :=
by
  intros ε hε
  cases' hu (ε/2) (by linarith) with Nu hNu
  cases' hv (ε/2) (by linarith) with Nv hNv
  use max Nu Nv
  intros n hn
  have hn₁ : n ≥ Nu := le_of_max_le_left hn
  specialize hNu n hn₁
  have hn₂ : n ≥ Nv := le_of_max_le_right hn
  specialize hNv n hn₂
  calc |(u + v) n - (a + b)|
       = |(u n + v n) - (a + b)|  := by rfl
     _ = |(u n - a) + (v n -  b)| := by {congr; ring}
     _ ≤ |u n - a| + |v n -  b|   := by apply abs_add
     _ < ε / 2 + ε / 2            := by linarith
     _ = ε                        := by apply add_halves

-- 3ª demostración
-- ===============

lemma max_ge_iff
  {α : Type _}
  [LinearOrder α]
  {p q r : α}
  : r ≥ max p q  ↔ r ≥ p ∧ r ≥ q :=
max_le_iff

example
  (hu : limite u a)
  (hv : limite v b)
  : limite (u + v) (a + b) :=
by
  intros ε hε
  cases' hu (ε/2) (by linarith) with Nu hNu
  cases' hv (ε/2) (by linarith) with Nv hNv
  use max Nu Nv
  intros n hn
  cases' max_ge_iff.mp hn with hn₁ hn₂
  have cota₁ : |u n - a| < ε/2 := hNu n hn₁
  have cota₂ : |v n - b| < ε/2 := hNv n hn₂
  calc |(u + v) n - (a + b)|
       = |(u n + v n) - (a + b)| := by rfl
     _ = |(u n - a) + (v n - b)| := by { congr; ring }
     _ ≤ |u n - a| + |v n - b|   := by apply abs_add
     _ < ε                       := by linarith

-- 4ª demostración
-- ===============

example
  (hu : limite u a)
  (hv : limite v b)
  : limite (u + v) (a + b) :=
by
  intros ε hε
  cases' hu (ε/2) (by linarith) with Nu hNu
  cases' hv (ε/2) (by linarith) with Nv hNv
  use max Nu Nv
  intros n hn
  cases' max_ge_iff.mp hn with hn₁ hn₂
  calc |(u + v) n - (a + b)|
       = |u n + v n - (a + b)|   := by rfl
     _ = |(u n - a) + (v n - b)| := by { congr; ring }
     _ ≤ |u n - a| + |v n - b|   := by apply abs_add
     _ < ε/2 + ε/2               := add_lt_add (hNu n hn₁) (hNv n hn₂)
     _ = ε                       := by simp

-- 5ª demostración
-- ===============

example
  (hu : limite u a)
  (hv : limite v b)
  : limite (u + v) (a + b) :=
by
  intros ε hε
  cases' hu (ε/2) (by linarith) with Nu hNu
  cases' hv (ε/2) (by linarith) with Nv hNv
  use max Nu Nv
  intros n hn
  rw [max_ge_iff] at hn
  calc |(u + v) n - (a + b)|
       = |u n + v n - (a + b)|   := by rfl
     _ = |(u n - a) + (v n - b)| := by { congr; ring }
     _ ≤ |u n - a| + |v n - b|   := by apply abs_add
     _ < ε                       := by linarith [hNu n (by linarith), hNv n (by linarith)]

-- 6ª demostración
-- ===============

example
  (hu : limite u a)
  (hv : limite v b)
  : limite (u + v) (a + b) :=
by
  intros ε Hε
  cases' hu (ε/2) (by linarith) with L HL
  cases' hv (ε/2) (by linarith) with M HM
  set N := max L M with _hN
  use N
  have HLN : N ≥ L := le_max_left _ _
  have HMN : N ≥ M := le_max_right _ _
  intros n Hn
  have H3 : |u n - a| < ε/2 := HL n (by linarith)
  have H4 : |v n - b| < ε/2 := HM n (by linarith)
  calc |(u + v) n - (a + b)|
       = |(u n + v n) - (a + b)|   := by rfl
     _ = |(u n - a) + (v n - b)|   := by {congr; ring }
     _ ≤ |(u n - a)| + |(v n - b)| := by apply abs_add
     _ < ε/2 + ε/2                 := by linarith
     _ = ε                         := by ring

-- Lemas usados
-- ============

-- variable (d : ℝ)
-- #check (abs_add a b : |a + b| ≤ |a| + |b|)
-- #check (add_halves a : a / 2 + a / 2 = a)
-- #check (add_lt_add : a < b → c < d → a + c < b + d)
-- #check (half_pos : a > 0 → a / 2 > 0)
-- #check (le_max_left a b : a ≤ max a b)
-- #check (le_max_right a b : b ≤ max a b)
-- #check (le_of_max_le_left : max a b ≤ c → a ≤ c)
-- #check (le_of_max_le_right : max a b ≤ c → b ≤ c)
-- #check (max_le_iff : max a b ≤ c ↔ a ≤ c ∧ b ≤ c)
\end{verbatim}
Se puede interactuar con las pruebas anteriores en \href{https://lean.math.hhu.de/\#url=https://raw.githubusercontent.com/jaalonso/Calculemus2/main/src/Convergencia\_de\_la\_suma.lean}{Lean 4 Web}

\section{Unicidad del límite de las sucesiones convergentes}
\label{sec:org540f6d9}
\begin{verbatim}
-- ---------------------------------------------------------------------
-- En Lean, una sucesión u₀, u₁, u₂, ... se puede representar mediante
-- una función (u : ℕ → ℝ) de forma que u(n) es uₙ.
--
-- Se define que a es el límite de la sucesión u, por
--    def limite : (ℕ → ℝ) → ℝ → Prop :=
--    λ u a, ∀ ε > 0, ∃ N, ∀ n ≥ N, |u n - a| < ε
-- donde se usa la notación |x| para el valor absoluto de x
--    notation `|`x`|` := abs x
--
-- Demostrar que cada sucesión tiene como máximo un límite.
-- ---------------------------------------------------------------------

import data.real.basic

variables {u : ℕ → ℝ}
variables {a b : ℝ}

notation `|`x`|` := abs x

def limite : (ℕ → ℝ) → ℝ → Prop :=
λ u c, ∀ ε > 0, ∃ N, ∀ n ≥ N, |u n - c| < ε

-- 1ª demostración
-- ===============

lemma aux
  (ha : limite u a)
  (hb : limite u b)
  : b ≤ a :=
begin
  by_contra h,
  set ε := b - a with hε,
  cases ha (ε/2) (by linarith) with A hA,
  cases hb (ε/2) (by linarith) with B hB,
  set N := max A B with hN,
  have hAN : A ≤ N := le_max_left A B,
  have hBN : B ≤ N := le_max_right A B,
  specialize hA N hAN,
  specialize hB N hBN,
  rw abs_lt at hA hB,
  linarith,
end

example
  (ha : limite u a)
  (hb : limite u b)
  : a = b :=
le_antisymm (aux hb ha) (aux ha hb)

-- 2ª demostración
-- ===============

example
  (ha : limite u a)
  (hb : limite u b)
  : a = b :=
begin
  by_contra h,
  wlog hab : a < b,
  { have : a < b ∨ a = b ∨ b < a := lt_trichotomy a b,
    tauto },
  set ε := b - a with hε,
  specialize ha (ε/2),
  have hε2 : ε/2 > 0 := by linarith,
  specialize ha hε2,
  cases ha with A hA,
  cases hb (ε/2) (by linarith) with B hB,
  set N := max A B with hN,
  have hAN : A ≤ N := le_max_left A B,
  have hBN : B ≤ N := le_max_right A B,
  specialize hA N hAN,
  specialize hB N hBN,
  rw abs_lt at hA hB,
  linarith,
end

-- 3ª demostración
-- ===============

example
  (ha : limite u a)
  (hb : limite u b)
  : a = b :=
begin
  by_contra h,
  wlog hab : a < b,
  { have : a < b ∨ a = b ∨ b < a := lt_trichotomy a b,
    tauto },
  set ε := b - a with hε,
  cases ha (ε/2) (by linarith) with A hA,
  cases hb (ε/2) (by linarith) with B hB,
  set N := max A B with hN,
  have hAN : A ≤ N := le_max_left A B,
  have hBN : B ≤ N := le_max_right A B,
  specialize hA N hAN,
  specialize hB N hBN,
  rw abs_lt at hA hB,
  linarith,
end
\end{verbatim}
Se puede interactuar con las pruebas anteriores en \href{https://lean.math.hhu.de/\#url=https://raw.githubusercontent.com/jaalonso/Calculemus2/main/src/Unicidad\_del\_limite\_de\_las\_sucesiones\_convergentes.lean}{Lean 4 Web}

\section{Si el límite de la sucesión uₙ es a y c ∈ ℝ, entonces el límite de uₙ+c es a+c}
\label{sec:orgbc0bda2}
\begin{verbatim}
-- ---------------------------------------------------------------------
-- En Lean, una sucesión u₀, u₁, u₂, ... se puede representar mediante
-- una función (u : ℕ → ℝ) de forma que u(n) es uₙ.
--
-- Se define que a es el límite de la sucesión u, por
--    def limite : (ℕ → ℝ) → ℝ → Prop :=
--      fun u c ↦ ∀ ε > 0, ∃ N, ∀ n ≥ N, |u n - c| < ε
--
-- Demostrar que si el límite de la sucesión uₙ es a y c ∈ ℝ, entonces
-- el límite de uₙ+c es a+c.
-- ---------------------------------------------------------------------

-- Demostración en lenguaje natural
-- ================================

-- Sea ε ∈ ℝ tal que ε > 0. Tenemos que demostrar que
--    (∃ N)(∀ n ≥ N)[|(u(n) + c) - (a + c)| < ε]                     (1)
-- Puesto que el límite de la sucesión u(i) es a, existe un k tal que
--    (∀ n ≥ k)[|u(n) - a| < ε]                                      (2)
-- Veamos que con k se verifica (1); es decir, que
--    (∀ n ≥ k)[|(u(n) + c) - (a + c)| < ε]
-- Sea n ≥ k. Entonces, por (2),
--    |u(n) - a| < ε                                                 (3)
-- y, por consiguiente,
--    |(u(n) + c) - (a + c)| = |u(n) - a|
--                           < ε            [por (3)]

-- Demostraciones con Lean4
-- ========================

import Mathlib.Data.Real.Basic
import Mathlib.Tactic
variable {u : ℕ → ℝ}
variable {a c : ℝ}

def limite : (ℕ → ℝ) → ℝ → Prop :=
  fun u c ↦ ∀ ε > 0, ∃ N, ∀ n ≥ N, |u n - c| < ε

-- 1ª demostración
-- ===============

example
  (h : limite u a)
  : limite (fun i ↦ u i + c) (a + c) :=
by
  intros ε hε
  -- ε : ℝ
  -- hε : ε > 0
  -- ⊢ ∃ N, ∀ (n : ℕ), n ≥ N → |(fun i => u i + c) n - (a + c)| < ε
  dsimp
  -- ⊢ ∃ N, ∀ (n : ℕ), n ≥ N → |u n + c - (a + c)| < ε
  cases' h ε hε with k hk
  -- k : ℕ
  -- hk : ∀ (n : ℕ), n ≥ k → |u n - a| < ε
  use k
  -- ⊢ ∀ (n : ℕ), n ≥ k → |u n + c - (a + c)| < ε
  intros n hn
  -- n : ℕ
  -- hn : n ≥ k
  calc |u n + c - (a + c)|
       = |u n - a|         := by norm_num
     _ < ε                 := hk n hn

-- 2ª demostración
-- ===============

example
  (h : limite u a)
  : limite (fun i ↦ u i + c) (a + c) :=
by
  intros ε hε
  -- ε : ℝ
  -- hε : ε > 0
  -- ⊢ ∃ N, ∀ (n : ℕ), n ≥ N → |(fun i => u i + c) n - (a + c)| < ε
  dsimp
  -- ⊢ ∃ N, ∀ (n : ℕ), n ≥ N → |u n + c - (a + c)| < ε
  cases' h ε hε with k hk
  -- k : ℕ
  -- hk : ∀ (n : ℕ), n ≥ k → |u n - a| < ε
  use k
  -- ⊢ ∀ (n : ℕ), n ≥ k → |u n + c - (a + c)| < ε
  intros n hn
  -- n : ℕ
  -- hn : n ≥ k
  -- ⊢ |u n + c - (a + c)| < ε
  convert hk n hn using 2
  -- ⊢ u n + c - (a + c) = u n - a
  ring

-- 3ª demostración
-- ===============

example
  (h : limite u a)
  : limite (fun i ↦ u i + c) (a + c) :=
by
  intros ε hε
  dsimp
  convert h ε hε using 6
  ring

-- 4ª demostración
-- ===============

example
  (h : limite u a)
  : limite (fun i ↦ u i + c) (a + c) :=
  fun ε hε ↦ (by convert h ε hε using 6; ring)
\end{verbatim}
Se puede interactuar con las pruebas anteriores en \href{https://lean.math.hhu.de/\#url=https://raw.githubusercontent.com/jaalonso/Calculemus2/main/src/Limite\_cuando\_se\_suma\_una\_constante.lean}{Lean 4 Web}

\section{Si el límite de la sucesión uₙ es a y c ∈ ℝ, entonces el límite de cuₙ es ca}
\label{sec:orgdffbeff}
\begin{verbatim}
-- ---------------------------------------------------------------------
-- En Lean, una sucesión u₀, u₁, u₂, ... se puede representar mediante
-- una función (u : ℕ → ℝ) de forma que u(n) es uₙ.
--
-- Se define que a es el límite de la sucesión u, por
--    def limite : (ℕ → ℝ) → ℝ → Prop :=
--      fun u c ↦ ∀ ε > 0, ∃ N, ∀ n ≥ N, |u n - c| < ε
--
-- Demostrar que que si el límite de uₙ es a, entonces el de
-- cuₙ es ca.
-- ---------------------------------------------------------------------

-- Demostración en lenguaje natural
-- ================================

-- Sea ε ∈ ℝ tal que ε > 0. Tenemos que demostrar que
--    (∃ N ∈ ℕ)(∀ n ≥ N)[|cuₙ - ca| < ε]                             (1)
-- Distinguiremos dos casos según sea c = 0 o no.
--
-- Primer caso: Supongamos que c = 0. Entonces, (1) se reduce a
--    (∃ N ∈ ℕ)(∀ n ≥ N)[|0·uₙ - 0·a| < ε]
-- es decir,
--    (∃ N ∈ ℕ)(∀ n ≥ N)[0 < ε]
-- que se verifica para cualquier número N, ya que ε > 0.
--
-- Segundo caso: Supongamos que c ≠ 0. Entonces, ε/|c| > 0 y, puesto que
-- el límite de uₙ es a, existe un k ∈ ℕ tal que
--    (∀ n ≥ k)[|uₙ - a| < ε/|c|]                                    (2)
-- Veamos que con k se cumple (1). En efecto, sea n ≥ k. Entonces,
--    |cuₙ - ca| = |c(uₙ - a)|
--               = |c||u n - a|
--               < |c|(ε/|c|)     [por (2)]
--               = ε

-- Demostraciones con Lean4
-- ========================

import Mathlib.Data.Real.Basic
import Mathlib.Tactic

variable (u v : ℕ → ℝ)
variable (a c : ℝ)

def limite : (ℕ → ℝ) → ℝ → Prop :=
  fun u c ↦ ∀ ε > 0, ∃ N, ∀ n ≥ N, |u n - c| < ε

-- 1ª demostración
-- ===============

example
  (h : limite u a)
  : limite (fun n ↦ c * (u n)) (c * a) :=
by
  by_cases hc : c = 0
  . -- hc : c = 0
    subst hc
    -- ⊢ limite (fun n => 0 * u n) (0 * a)
    intros ε hε
    -- ε : ℝ
    -- hε : ε > 0
    -- ⊢ ∃ N, ∀ (n : ℕ), n ≥ N → |(fun n => 0 * u n) n - 0 * a| < ε
    aesop
  . -- hc : ¬c = 0
    intros ε hε
    -- ε : ℝ
    -- hε : ε > 0
    -- ⊢ ∃ N, ∀ (n : ℕ), n ≥ N → |(fun n => c * u n) n - c * a| < ε
    have hc' : 0 < |c| := abs_pos.mpr hc
    have hεc : 0 < ε / |c| := div_pos hε hc'
    specialize h (ε/|c|) hεc
    -- h : ∃ N, ∀ (n : ℕ), n ≥ N → |u n - a| < ε / |c|
    cases' h with N hN
    -- N : ℕ
    -- hN : ∀ (n : ℕ), n ≥ N → |u n - a| < ε / |c|
    use N
    -- ⊢ ∀ (n : ℕ), n ≥ N → |(fun n => c * u n) n - c * a| < ε
    intros n hn
    -- n : ℕ
    -- hn : n ≥ N
    -- ⊢ |(fun n => c * u n) n - c * a| < ε
    specialize hN n hn
    -- hN : |u n - a| < ε / |c|
    dsimp only
    calc |c * u n - c * a|
         = |c * (u n - a)| := congr_arg abs (mul_sub c (u n) a).symm
       _ = |c| * |u n - a| := abs_mul c  (u n - a)
       _ < |c| * (ε / |c|) := (mul_lt_mul_left hc').mpr hN
       _ = ε               := mul_div_cancel' ε (ne_of_gt hc')

-- 2ª demostración
-- ===============

example
  (h : limite u a)
  : limite (fun n ↦ c * (u n)) (c * a) :=
by
  by_cases hc : c = 0
  . -- hc : c = 0
    subst hc
    -- ⊢ limite (fun n => 0 * u n) (0 * a)
    intros ε hε
    -- ε : ℝ
    -- hε : ε > 0
    -- ⊢ ∃ N, ∀ (n : ℕ), n ≥ N → |(fun n => 0 * u n) n - 0 * a| < ε
    aesop
  . -- hc : ¬c = 0
    intros ε hε
    -- ε : ℝ
    -- hε : ε > 0
    -- ⊢ ∃ N, ∀ (n : ℕ), n ≥ N → |(fun n => c * u n) n - c * a| < ε
    have hc' : 0 < |c| := abs_pos.mpr hc
    have hεc : 0 < ε / |c| := div_pos hε hc'
    specialize h (ε/|c|) hεc
    -- h : ∃ N, ∀ (n : ℕ), n ≥ N → |u n - a| < ε / |c|
    cases' h with N hN
    -- N : ℕ
    -- hN : ∀ (n : ℕ), n ≥ N → |u n - a| < ε / |c|
    use N
    -- ⊢ ∀ (n : ℕ), n ≥ N → |(fun n => c * u n) n - c * a| < ε
    intros n hn
    -- n : ℕ
    -- hn : n ≥ N
    -- ⊢ |(fun n => c * u n) n - c * a| < ε
    specialize hN n hn
    -- hN : |u n - a| < ε / |c|
    dsimp only
    -- ⊢ |c * u n - c * a| < ε
    rw [← mul_sub]
    -- ⊢ |c * (u n - a)| < ε
    rw [abs_mul]
    -- ⊢ |c| * |u n - a| < ε
    rw [← lt_div_iff' hc']
    -- ⊢ |u n - a| < ε / |c|
    exact hN

-- 3ª demostración
-- ===============

example
  (h : limite u a)
  : limite (fun n ↦ c * (u n)) (c * a) :=
by
  by_cases hc : c = 0
  . subst hc
    intros ε hε
    aesop
  . intros ε hε
    have hc' : 0 < |c| := by aesop
    have hεc : 0 < ε / |c| := div_pos hε hc'
    cases' h (ε/|c|) hεc with N hN
    use N
    intros n hn
    specialize hN n hn
    dsimp only
    rw [← mul_sub, abs_mul, ← lt_div_iff' hc']
    exact hN

-- Lemas usados
-- ============

-- variable (b c : ℝ)
-- #check (abs_mul a b : |a * b| = |a| * |b|)
-- #check (abs_pos.mpr : a ≠ 0 → 0 < |a|)
-- #check (div_pos : 0 < a → 0 < b → 0 < a / b)
-- #check (lt_div_iff' : 0 < c → (a < b / c ↔ c * a < b))
-- #check (mul_div_cancel' a : b ≠ 0 → b * (a / b) = a)
-- #check (mul_lt_mul_left : 0 < a → (a * b < a * c ↔ b < c))
-- #check (mul_sub a b c : a * (b - c) = a * b - a * c)
\end{verbatim}
Se puede interactuar con las pruebas anteriores en \href{https://lean.math.hhu.de/\#url=https://raw.githubusercontent.com/jaalonso/Calculemus2/main/src/Limite\_multiplicado\_por\_una\_constante.lean}{Lean 4 Web}

\section{El límite de u es a syss el de u-a es 0}
\label{sec:orgdeba2c6}
\begin{verbatim}
-- ---------------------------------------------------------------------
-- En Lean, una sucesión u₀, u₁, u₂, ... se puede representar mediante
-- una función (u : ℕ → ℝ) de forma que u(n) es uₙ.
--
-- Se define que a es el límite de la sucesión u, por
--    def limite : (ℕ → ℝ) → ℝ → Prop :=
--     fun u c ↦ ∀ ε > 0, ∃ N, ∀ n ≥ N, |u n - c| < ε
--
-- Demostrar que el límite de uₙ es a si y solo si el de uₙ-a es 0.
-- ---------------------------------------------------------------------

-- Demostración en lenguaje natural
-- ================================

-- Se prueba por la siguiente cadena de equivalencias
--    limite u a ↔ (∀ε>0)(∃N)(∀n≥N)[|u(n) - a| < ε]
--               ↔ (∀ε>0)(∃N)(∀n≥N)[|(u(n) - a) - 0| < ε]
--               ↔ limite (fun n ↦ u(n) - a) 0

-- Demostraciones con Lean4
-- ========================

import Mathlib.Data.Real.Basic
import Mathlib.Tactic

variable  {u : ℕ → ℝ}
variable {a c x : ℝ}

def limite : (ℕ → ℝ) → ℝ → Prop :=
 fun u c ↦ ∀ ε > 0, ∃ N, ∀ n ≥ N, |u n - c| < ε

-- 1ª demostración
-- ===============

example
  : limite u a ↔ limite (fun i ↦ u i - a) 0 :=
by
  rw [iff_eq_eq]
  calc limite u a
       = ∀ ε > 0, ∃ N, ∀ n ≥ N, |u n - a| < ε       := rfl
     _ = ∀ ε > 0, ∃ N, ∀ n ≥ N, |(u n - a) - 0| < ε := by simp
     _ = limite (fun i ↦ u i - a) 0                 := rfl

-- 2ª demostración
-- ===============

example
  : limite u a ↔ limite (fun i ↦ u i - a) 0 :=
by
  constructor
  . -- ⊢ limite u a → limite (fun i => u i - a) 0
    intros h ε hε
    -- h : limite u a
    -- ε : ℝ
    -- hε : ε > 0
    -- ⊢ ∃ N, ∀ (n : ℕ), n ≥ N → |(fun i => u i - a) n - 0| < ε
    convert h ε hε using 2
    -- x : ℕ
    -- ⊢ (∀ (n : ℕ), n ≥ x → |(fun i => u i - a) n - 0| < ε) ↔ ∀ (n : ℕ), n ≥ x → |u n - a| < ε
    norm_num
  . -- ⊢ limite (fun i => u i - a) 0 → limite u a
    intros h ε hε
    -- h : limite (fun i => u i - a) 0
    -- ε : ℝ
    -- hε : ε > 0
    -- ⊢ ∃ N, ∀ (n : ℕ), n ≥ N → |u n - a| < ε
    convert h ε hε using 2
    -- x : ℕ
    -- ⊢ (∀ (n : ℕ), n ≥ x → |u n - a| < ε) ↔ ∀ (n : ℕ), n ≥ x → |(fun i => u i - a) n - 0| < ε
    norm_num

-- 3ª demostración
-- ===============

example
  : limite u a ↔ limite (fun i ↦ u i - a) 0 :=
by
  constructor <;>
  { intros h ε hε
    convert h ε hε using 2
    norm_num }

-- 4ª demostración
-- ===============

lemma limite_con_suma
  (c : ℝ)
  (h : limite u a)
  : limite (fun i ↦ u i + c) (a + c) :=
  fun ε hε ↦ (by convert h ε hε using 2; norm_num)

lemma CNS_limite_con_suma
  (c : ℝ)
  : limite u a ↔ limite (fun i ↦ u i + c) (a + c) :=
by
  constructor
  . -- ⊢ limite u a → limite (fun i => u i + c) (a + c)
    apply limite_con_suma
  . -- ⊢ limite (fun i => u i + c) (a + c) → limite u a
    intro h
    -- h : limite (fun i => u i + c) (a + c)
    -- ⊢ limite u a
    convert limite_con_suma (-c) h using 2
    . -- ⊢ u x = u x + c + -c
      simp
    . -- ⊢ a = a + c + -c
      simp

example
  (u : ℕ → ℝ)
  (a : ℝ)
  : limite u a ↔ limite (fun i ↦ u i - a) 0 :=
by
  convert CNS_limite_con_suma (-a) using 2
  -- ⊢ 0 = a + -a
  simp

-- Lemas usados
-- ============

-- variable (p q : Prop)
-- #check (iff_eq_eq : (p ↔ q) = (p = q))
\end{verbatim}
Se puede interactuar con las pruebas anteriores en \href{https://lean.math.hhu.de/\#url=https://raw.githubusercontent.com/jaalonso/Calculemus2/main/src/El\_limite\_de\_u\_es\_a\_syss\_el\_de\_u-a\_es\_0.lean}{Lean 4 Web}

\section{Si uₙ y vₙ convergen a 0, entonces uₙvₙ converge a 0}
\label{sec:org125c468}
\begin{verbatim}
-- ---------------------------------------------------------------------
-- En Lean, una sucesión u₀, u₁, u₂, ... se puede representar mediante
-- una función (u : ℕ → ℝ) de forma que u(n) es uₙ.
--
-- Se define que a es el límite de la sucesión u, por
--    def limite : (ℕ → ℝ) → ℝ → Prop :=
--      fun u c ↦ ∀ ε > 0, ∃ N, ∀ n ≥ N, |u n - c| < ε
--
-- Demostrar que si las sucesiones u(n) y v(n) convergen a cero,
-- entonces u(n)·v(n) también converge a cero.
-- ---------------------------------------------------------------------

-- Demostración en lenguaje natural
-- ================================

-- Sea ε ∈ ℝ tal que ε > 0. Tenemos ue demostrar que
--    (∃N ∈ ℕ)(∀n ≥ N)[|(uv)(n) - 0| < ε]                         (1)
-- Puesto que el límite de uₙ es 0, existe un U ∈ ℕ tal que
--    (∀n ≥ U)[|u(n) - 0| < ε]                                       (2)
-- y, puesto que el límite de vₙ es 0, existe un V ∈ ℕ tal que
--    (∀n ≥ V)[|v(n) - 0| < 1]                                       (3)
-- Entonces, N = max(U, V) cumple (1). En efecto, sea n ≥ N. Entonces,
-- n ≥ U y n ≥ V y, aplicando (2) y (3), se tiene
--    |u(n) - 0| < ε                                                 (4)
--    |v(n) - 0| < 1                                                 (5)
-- Por tanto,
--    |(u·v)(n) - 0| = |u(n)·v(n)|
--                   = |u(n)|·|v n|
--                   < ε·1             [por (4) y (5)]
--                   = ε

-- Demostraciones con Lean4
-- ========================

import Mathlib.Data.Real.Basic
import Mathlib.Tactic

variable {u v : ℕ → ℝ}

def limite : (ℕ → ℝ) → ℝ → Prop :=
  fun u c ↦ ∀ ε > 0, ∃ N, ∀ n ≥ N, |u n - c| < ε

-- 1ª demostración
-- ===============

example
  (hu : limite u 0)
  (hv : limite v 0)
  : limite (u * v) 0 :=
by
  intros ε hε
  -- ε : ℝ
  -- hε : ε > 0
  -- ⊢ ∃ N, ∀ (n : ℕ), n ≥ N → |(u * v) n - 0| < ε
  cases' hu ε hε with U hU
  -- U : ℕ
  -- hU : ∀ (n : ℕ), n ≥ U → |u n - 0| < ε
  cases' hv 1 zero_lt_one with V hV
  -- V : ℕ
  -- hV : ∀ (n : ℕ), n ≥ V → |v n - 0| < 1
  let N := max U V
  use N
  -- ⊢ ∀ (n : ℕ), n ≥ N → |(u * v) n - 0| < ε
  intros n hn
  -- n : ℕ
  -- hn : n ≥ N
  -- ⊢ |(u * v) n - 0| < ε
  specialize hU n (le_of_max_le_left hn)
  -- hU : |u n - 0| < ε
  specialize hV n (le_of_max_le_right hn)
  -- hV : |v n - 0| < 1
  rw [sub_zero] at *
  -- hU : |u n - 0| < ε
  -- hV : |v n - 0| < 1
  -- ⊢ |(u * v) n - 0| < ε
  calc |(u * v) n|
       = |u n * v n|   := rfl
     _ = |u n| * |v n| := abs_mul (u n) (v n)
     _ < ε * 1         := mul_lt_mul'' hU hV (abs_nonneg (u n)) (abs_nonneg (v n))
     _ = ε             := mul_one ε

-- 2ª demostración
-- ===============

example
  (hu : limite u 0)
  (hv : limite v 0)
  : limite (u * v) 0 :=
by
  intros ε hε
  -- ε : ℝ
  -- hε : ε > 0
  -- ⊢ ∃ N, ∀ (n : ℕ), n ≥ N → |(u * v) n - 0| < ε
  cases' hu ε hε with U hU
  -- U : ℕ
  -- hU : ∀ (n : ℕ), n ≥ U → |u n - 0| < ε
  cases' hv 1 (by linarith) with V hV
  -- V : ℕ
  -- hV : ∀ (n : ℕ), n ≥ V → |v n - 0| < 1
  let N := max U V
  use N
  -- ⊢ ∀ (n : ℕ), n ≥ N → |(u * v) n - 0| < ε
  intros n hn
  -- n : ℕ
  -- hn : n ≥ N
  -- ⊢ |(u * v) n - 0| < ε
  specialize hU n (le_of_max_le_left hn)
  -- hU : |u n - 0| < ε
  specialize hV n (le_of_max_le_right hn)
  -- hV : |v n - 0| < 1
  rw [sub_zero] at *
  -- hU : |u n| < ε
  -- hV : |v n| < 1
  -- ⊢ |(u * v) n| < ε
  calc |(u * v) n|
       = |u n * v n|   := rfl
     _ = |u n| * |v n| := abs_mul (u n) (v n)
     _ < ε * 1         := by { apply mul_lt_mul'' hU hV <;> simp [abs_nonneg] }
     _ = ε             := mul_one ε

-- 3ª demostración
-- ===============

example
  (hu : limite u 0)
  (hv : limite v 0)
  : limite (u * v) 0 :=
by
  intros ε hε
  -- ε : ℝ
  -- hε : ε > 0
  -- ⊢ ∃ N, ∀ (n : ℕ), n ≥ N → |(u * v) n - 0| < ε
  cases' hu ε hε with U hU
  -- U : ℕ
  -- hU : ∀ (n : ℕ), n ≥ U → |u n - 0| < ε
  cases' hv 1 (by linarith) with V hV
  -- V : ℕ
  -- hV : ∀ (n : ℕ), n ≥ V → |v n - 0| < 1
  let N := max U V
  use N
  -- ⊢ ∀ (n : ℕ), n ≥ N → |(u * v) n - 0| < ε
  intros n hn
  -- n : ℕ
  -- hn : n ≥ N
  -- ⊢ |(u * v) n - 0| < ε
  have hUN : U ≤ N := le_max_left U V
  have hVN : V ≤ N := le_max_right U V
  specialize hU n (by linarith)
  -- hU : |u n - 0| < ε
  specialize hV n (by linarith)
  -- hV : |v n - 0| < 1
  rw [sub_zero] at *
  -- hU : |u n| < ε
  -- hV : |v n| < 1
  -- ⊢ |(u * v) n| < ε
  rw [Pi.mul_apply]
  -- ⊢ |u n * v n| < ε
  rw [abs_mul]
  -- ⊢ |u n| * |v n| < ε
  convert mul_lt_mul'' hU hV _ _ using 2 <;> simp

-- Lemas usados
-- ============

-- variable (a b c d : ℝ)
-- variable (I : Type _)
-- variable (f : I → Type _)
-- #check (zero_lt_one : 0 < 1)
-- #check (le_of_max_le_left : max a b ≤ c → a ≤ c)
-- #check (le_of_max_le_right : max a b ≤ c → b ≤ c)
-- #check (sub_zero a : a - 0 = a)
-- #check (abs_mul a b : |a * b| = |a| * |b|)
-- #check (mul_lt_mul'' : a < c → b < d → 0 ≤ a → 0 ≤ b → a * b < c * d)
-- #check (abs_nonneg a : 0 ≤ |a|)
-- #check (mul_one a : a * 1 = a)
\end{verbatim}
Se puede interactuar con las pruebas anteriores en \href{https://lean.math.hhu.de/\#url=https://raw.githubusercontent.com/jaalonso/Calculemus2/main/src/Producto\_de\_sucesiones\_convergentes\_a\_cero.lean}{Lean 4 Web}

\section{Teorema del emparedado}
\label{sec:org0d29d0c}
\begin{verbatim}
-- ---------------------------------------------------------------------
-- En Lean, una sucesión u₀, u₁, u₂, ... se puede representar mediante
-- una función (u : ℕ → ℝ) de forma que u(n) es uₙ.
--
-- Se define que a es el límite de la sucesión u, por
--    def limite : (ℕ → ℝ) → ℝ → Prop :=
--      fun u c ↦ ∀ ε > 0, ∃ N, ∀ n ≥ N, |u n - c| ≤ ε
--
-- Demostrar que si para todo n, u(n) ≤ v(n) ≤ w(n) y u(n) tiene el
-- mismo límite que w(n), entonces v(n) también tiene dicho límite.
-- ---------------------------------------------------------------------

-- Demostración en lenguaje natural
-- ================================

-- Tenemos que demostrar que para cada ε > 0, existe un N ∈ ℕ tal que
--    (∀ n ≥ N)[|v(n) - a| ≤ ε]                                       (1)
--
-- Puesto que el límite de u es a, existe un U ∈ ℕ tal que
--    (∀ n ≥ U)[|u(n) - a| ≤ ε]                                       (2)
-- y, puesto que el límite de w es a, existe un W ∈ ℕ tal que
--    (∀ n ≥ W)[|w(n) - a| ≤ ε]                                       (3)
-- Sea N = máx(U, W). Veamos que se verifica (1). Para ello, sea
-- n ≥ N. Entonces, n ≥ U y n ≥ W. Por (2) y (3), se tiene que
--     |u(n) - a| ≤ ε                                                 (4)
--     |w(n) - a| ≤ ε                                                 (5)
-- Para demostrar que
--     |v(n) - a| ≤ ε
-- basta demostrar las siguientes desigualdades
--     -ε ≤ v(n) - a                                                  (6)
--      v(n) - a ≤ ε                                                  (7)
-- La demostración de (6) es
--    -ε ≤ u(n) - a    [por (4)]
--       ≤ v(n) - a    [por hipótesis]
-- La demostración de (7) es
--    v(n) - a ≤ w(n) - a    [por hipótesis]
--             ≤ ε           [por (5)]

-- Demostraciones con Lean4
-- ========================

import Mathlib.Data.Real.Basic

variable (u v w : ℕ → ℝ)
variable (a : ℝ)

def limite : (ℕ → ℝ) → ℝ → Prop :=
  fun u c ↦ ∀ ε > 0, ∃ N, ∀ n ≥ N, |u n - c| ≤ ε

-- Nota. En la demostración se usará el siguiente lema:
lemma max_ge_iff
  {p q r : ℕ}
  : r ≥ max p q ↔ r ≥ p ∧ r ≥ q :=
  max_le_iff

-- 1ª demostración
-- ===============

example
  (hu : limite u a)
  (hw : limite w a)
  (h1 : ∀ n, u n ≤ v n)
  (h2 : ∀ n, v n ≤ w n) :
  limite v a :=
by
  intros ε hε
  -- ε : ℝ
  -- hε : ε > 0
  -- ⊢ ∃ N, ∀ (n : ℕ), n ≥ N → |v n - a| ≤ ε
  rcases hu ε hε with ⟨U, hU⟩
  -- U : ℕ
  -- hU : ∀ (n : ℕ), n ≥ U → |u n - a| ≤ ε
  clear hu
  rcases hw ε hε with ⟨W, hW⟩
  -- W : ℕ
  -- hW : ∀ (n : ℕ), n ≥ W → |w n - a| ≤ ε
  clear hw hε
  use max U W
  intros n hn
  -- n : ℕ
  -- hn : n ≥ max U W
  -- ⊢ |v n - a| ≤ ε
  rw [max_ge_iff] at hn
  -- hn : n ≥ U ∧ n ≥ W
  specialize hU n hn.1
  -- hU : |u n - a| ≤ ε
  specialize hW n hn.2
  -- hW : |w n - a| ≤ ε
  specialize h1 n
  -- h1 : u n ≤ v n
  specialize h2 n
  -- h2 : v n ≤ w n
  clear hn
  rw [abs_le] at *
  -- ⊢ -ε ≤ v n - a ∧ v n - a ≤ ε
  constructor
  . -- ⊢ -ε ≤ v n - a
    calc -ε
         ≤ u n - a := hU.1
       _ ≤ v n - a := by linarith
  . -- ⊢ v n - a ≤ ε
    calc v n - a
         ≤ w n - a := by linarith
       _ ≤ ε       := hW.2

-- 2ª demostración
example
  (hu : limite u a)
  (hw : limite w a)
  (h1 : ∀ n, u n ≤ v n)
  (h2 : ∀ n, v n ≤ w n) :
  limite v a :=
by
  intros ε hε
  -- ε : ℝ
  -- hε : ε > 0
  -- ⊢ ∃ N, ∀ (n : ℕ), n ≥ N → |v n - a| ≤ ε
  rcases hu ε hε with ⟨U, hU⟩
  -- U : ℕ
  -- hU : ∀ (n : ℕ), n ≥ U → |u n - a| ≤ ε
  clear hu
  rcases hw ε hε with ⟨W, hW⟩
  -- W : ℕ
  -- hW : ∀ (n : ℕ), n ≥ W → |w n - a| ≤ ε
  clear hw hε
  use max U W
  intros n hn
  -- n : ℕ
  -- hn : n ≥ max U W
  rw [max_ge_iff] at hn
  -- hn : n ≥ U ∧ n ≥ W
  specialize hU n (by linarith)
  -- hU : |u n - a| ≤ ε
  specialize hW n (by linarith)
  -- hW : |w n - a| ≤ ε
  specialize h1 n
  -- h1 : u n ≤ v n
  specialize h2 n
  -- h2 : v n ≤ w n
  rw [abs_le] at *
  -- ⊢ -ε ≤ v n - a ∧ v n - a ≤ ε
  constructor
  . -- ⊢ -ε ≤ v n - a
    linarith
  . -- ⊢ v n - a ≤ ε
    linarith

-- 3ª demostración
example
  (hu : limite u a)
  (hw : limite w a)
  (h1 : ∀ n, u n ≤ v n)
  (h2 : ∀ n, v n ≤ w n) :
  limite v a :=
by
  intros ε hε
  -- ε : ℝ
  -- hε : ε > 0
  -- ⊢ ∃ N, ∀ (n : ℕ), n ≥ N → |v n - a| ≤ ε
  rcases hu ε hε with ⟨U, hU⟩
  -- U : ℕ
  -- hU : ∀ (n : ℕ), n ≥ U → |u n - a| ≤ ε
  clear hu
  rcases hw ε hε with ⟨W, hW⟩
  -- W : ℕ
  -- hW : ∀ (n : ℕ), n ≥ W → |w n - a| ≤ ε
  clear hw hε
  use max U W
  intros n hn
  -- n : ℕ
  -- hn : n ≥ max U W
  -- ⊢ |v n - a| ≤ ε
  rw [max_ge_iff] at hn
  -- hn : n ≥ U ∧ n ≥ W
  specialize hU n (by linarith)
  -- hU : |u n - a| ≤ ε
  specialize hW n (by linarith)
  -- hW : |w n - a| ≤ ε
  specialize h1 n
  -- h1 : u n ≤ v n
  specialize h2 n
  -- h2 : v n ≤ w n
  rw [abs_le] at *
  -- hU : -ε ≤ u n - a ∧ u n - a ≤ ε
  -- hW : -ε ≤ w n - a ∧ w n - a ≤ ε
  -- ⊢ -ε ≤ v n - a ∧ v n - a ≤ ε
  constructor <;> linarith
\end{verbatim}
Se puede interactuar con las pruebas anteriores en \href{https://lean.math.hhu.de/\#url=https://raw.githubusercontent.com/jaalonso/Calculemus2/main/src/Teorema\_del\_emparedado.lean}{Lean 4 Web}

\input Calculemus2_bibliografia

\chapter*{Lemas usados}
\addcontentsline{toc}{chapter}{Lemas usados}
\begin{verbatim}
import Mathlib.Algebra.Group.Basic
import Mathlib.Algebra.Order.Ring.Defs
import Mathlib.Algebra.Ring.Defs
import Mathlib.Analysis.SpecialFunctions.Log.Basic
import Mathlib.Data.Real.Basic
import Mathlib.Order.Lattice
import Mathlib.Topology.MetricSpace.Basic

-- Números naturales
-- =================

section naturales
variable (x y z k m n : ℕ)
open Nat
#check (_root_.dvd_antisymm : m ∣ n → n ∣ m → m = n)
#check (dvd_add : x ∣ y → x ∣ z → x ∣ y + z)
#check (dvd_factorial : 0 < k → k ≤ n → k ∣ n !)
#check (dvd_gcd : k ∣ m → k ∣ n → k ∣ gcd m n)
#check (dvd_mul_left x y : x ∣ y * x)
#check (dvd_mul_of_dvd_left : x ∣ y → ∀ (c : ℕ), x ∣ y * c)
#check (dvd_mul_of_dvd_right : x ∣ y → ∀ (c : ℕ), x ∣ c * y)
#check (dvd_mul_right x y : x ∣ x * y)
#check (dvd_trans : x ∣ y → y ∣ z → x ∣ z)
#check (Dvd.intro k : m * k = n → m ∣ n)
#check (factorial_pos n: n ! > 0)
#check (gcd_comm m n : gcd m n = gcd n m)
#check (gcd_dvd_left  m n: gcd m n ∣ m)
#check (gcd_dvd_right m n : gcd m n ∣ n)
#check (minFac_dvd n : minFac n ∣ n)
#check (minFac_pos n : 0 < minFac n)
#check (minFac_prime : n ≠ 1 → Nat.Prime (minFac n))
#check (Nat.dvd_add_iff_right : k ∣ m → (k ∣ n ↔ k ∣ m + n))
#check (Nat.dvd_one : n ∣ 1 ↔ n = 1)
#check (Nat.lt_add_of_pos_left : 0 < k → n < k + n)
#check (Nat.ne_of_gt : k < n → n ≠ k)
#check (Nat.Prime.not_dvd_one : Nat.Prime n → ¬n ∣ 1)
end naturales

-- Números reales
-- ==============

section reales
open Real
variable (a b c d x y : ℝ)
#check (Left.self_le_neg : x ≤ 0 → x ≤ -x)
#check (abs_add a b : |a + b| ≤ |a| + |b|)
#check (abs_le' : |a| ≤ b ↔ a ≤ b ∧ -a ≤ b)
#check (abs_lt: |a| < b ↔ -b < a ∧ a < b)
#check (abs_mul a b : |a * b| = |a| * |b|)
#check (abs_nonneg a : 0 ≤ |a|)
#check (abs_of_neg : x < 0 → |x| = -x)
#check (abs_of_nonneg : 0 ≤ x → |x| = x)
#check (abs_sub_abs_le_abs_sub a b : |a| - |b| ≤ |a - b|)
#check (add_le_add : a ≤ b → c ≤ d → a + c ≤ b + d)
#check (add_le_add_left : b ≤ c → ∀ (a : ℝ), a + b ≤ a + c)
#check (add_le_add_right : b ≤ c → ∀ (a : ℝ),  b + a ≤ c + a)
#check (add_lt_add_of_le_of_lt : a ≤ b → c < d → a + c < b + d)
#check (add_lt_add_of_lt_of_le : a < b → c ≤ d → a + c < b + d)
#check (add_lt_add_right : b < c → ∀ (a : ℝ), b + a < c + a)
#check (add_neg_le_iff_le_add : a - b ≤ c ↔ a ≤ c + b)
#check (add_pos : 0 < a → 0 < b → 0 < a + b)
#check (add_sub_cancel a b : a + b - b = a)
#check (div_mul_cancel a : b ≠ 0 → (a / b) * b = a)
#check (eq_neg_of_add_eq_zero_left : x + y = 0 → x = -y)
#check (eq_zero_or_eq_zero_of_mul_eq_zero : x * y = 0 → x = 0 ∨ y = 0)
#check (exp_le_exp : exp a ≤ exp b ↔ a ≤ b)
#check (exp_lt_exp : exp a < exp b ↔ a < b)
#check (exp_pos a : 0 < exp a)
#check (half_lt_self : 0 < a → a / 2 < a)
#check (half_pos : 0 < a → 0 < a / 2)
#check (le_abs_self x : x ≤ |x|)
#check (le_add_of_nonneg_right : 0 ≤ b → a ≤ a + b)
#check (le_antisymm : a ≤ b → b ≤ a → a = b)
#check (le_div_iff : 0 < c → (a ≤ b / c ↔ a * c ≤ b))
#check (le_max_left a b : a ≤ max a b)
#check (le_max_right a b : b ≤ max a b)
#check (le_min : c ≤ a → c ≤ b → c ≤ min a b)
#check (le_neg_self_iff : x ≤ -x ↔ x ≤ 0)
#check (le_of_eq : a = b → a ≤ b)
#check (le_of_lt : x < y → x ≤ y)
#check (le_of_not_ge : ¬x ≥ y → x ≤ y)
#check (le_of_not_gt : ¬a > b → a ≤ b)
#check (le_or_gt x y : x ≤ y ∨ x > y)
#check (le_refl a : a ≤ a)
#check (log_le_log' : 0 < a → a ≤ b → log a ≤ log b)
#check (lt_abs : x < |y| ↔ x < y ∨ x < -y)
#check (lt_asymm : a < b → ¬b < a)
#check (lt_iff_le_and_ne : a < b ↔ a ≤ b ∧ a ≠ b)
#check (lt_iff_le_not_le : a < b ↔ a ≤ b ∧ ¬b ≤ a)
#check (lt_irrefl a : ¬a < a)
#check (lt_neg : a < -b ↔ b < -a)
#check (lt_of_le_of_ne : a ≤ b → a ≠ b → a < b)
#check (lt_of_lt_of_le : a < b → b ≤ c → a < c)
#check (lt_of_le_of_lt : a ≤ b → b < c → a < c)
#check (lt_of_le_of_ne : a ≤ b → a ≠ b → a < b)
#check (lt_of_not_ge : ¬ a ≥ b → a < b)
#check (lt_of_not_le : ¬b ≤ a → a < b)
#check (lt_trans : a < b → b < c → a < c)
#check (lt_trichotomy a b : a < b ∨ a = b ∨ b < a)
#check (max_comm a b : max a b = max b a)
#check (max_le : a ≤ c → b ≤ c → max a b ≤ c)
#check (min_add_add_right a b c : min (a + c) (b + c) = min a b + c)
#check (min_assoc a b c : min (min a b) c = min a (min b c))
#check (min_comm a b : min a b = min b a)
#check (min_eq_left : a ≤ b → min a b = a)
#check (min_eq_right : b ≤ a → min a b = b)
#check (min_le_left a b : min a b ≤ a)
#check (min_le_right a b : min a b ≤ b)
#check (mul_comm a b : a * b = b * a)
#check (mul_div_cancel' a : b ≠ 0 → b * (a / b) = a)
#check (mul_le_mul : a ≤ b → c ≤ d → 0 ≤ c → 0 ≤ b → a * c ≤ b * d)
#check (mul_le_mul_right : 0 < a → (b * a ≤ c * a ↔ b ≤ c))
#check (mul_left_cancel₀ : a ≠ 0 → a * b = a * c → b = c)
#check (mul_lt_mul_left : 0 < a → (a * b < a * c ↔ b < c))
#check (mul_lt_mul_right : 0 < a → (b * a < c * a ↔ b < c))
#check (mul_neg a b : a * -b = -(a * b))
#check (mul_right_inj' :  a ≠ 0 → (a * b = a * c ↔ b = c))
#check (mul_sub a b c : a * (b - c) = a * b - a * c)
#check (mul_two a : a * 2 = a + a)
#check (ne_comm : a ≠ b ↔ b ≠ a)
#check (neg_add x y : -(x + y) = -x + -y)
#check (neg_add_self a : -a + a = 0)
#check (neg_le_abs_self x : -x ≤ |x|)
#check (neg_mul_neg a b : -a * -b = a * b)
#check (nonneg_of_mul_nonneg_left : 0 ≤ a * b → 0 < b → 0 ≤ a)
#check (not_lt_of_ge : a ≥ b → ¬ a < b)
#check (pow_eq_zero : ∀ {n : ℕ}, a ^ n = 0 → a = 0)
#check (pow_two a : a ^ 2 = a * a)
#check (pow_two_nonneg a : 0 ≤ a ^ 2)
#check (sq_eq_one_iff : x ^ 2 = 1 ↔ x = 1 ∨ x = -1)
#check (sq_eq_sq_iff_eq_or_eq_neg : x ^ 2 = y ^ 2 ↔ x = y ∨ x = -y)
#check (sq_nonneg a : 0 ≤ a ^ 2)
#check (sub_add_cancel a b : a - b + b = a)
#check (sub_eq_zero : x - y = 0 ↔ x = y)
#check (sub_le_sub_left : a ≤ b → ∀ (c : ℝ), c - b ≤ c - a)
#check (sub_le_sub_right : a ≤ b → ∀ (c : ℝ), a - c ≤ b - c)
#check (sub_sq a b : (a - b) ^ 2 = a ^ 2 - 2 * a * b + b ^ 2)
#check (two_mul a : 2 * a = a + a)
#check (two_mul_le_add_sq a b : 2 * a * b ≤ a ^ 2 + b ^ 2)
#check (zero_lt_one : 0 < 1)
#check (zero_lt_two : 0 < 2)
end reales

-- Anillos
-- =======

section anillos
variable {R : Type _} [Ring R]
variable (a b c : R)
#check (add_assoc a b c : (a + b) + c = a + (b + c))
#check (add_comm a b : a + b = b + a)
#check (add_eq_zero_iff_eq_neg : a + b = 0 ↔ a = -b)
#check (add_left_cancel : a + b = a + c → b = c)
#check (add_left_neg a : -a + a = 0)
#check (add_mul a b c : (a + b) * c = a * c + b * c)
#check (add_neg_cancel_right a b : (a + b) + -b = a)
#check (add_neg_self a : a + -a = 0)
#check (add_right_cancel : a + b = c + b → a = c)
#check (add_right_neg a : a + -a = 0)
#check (add_zero a : a + 0 = a)
#check (mul_add a b c : a * (b + c) = a * b + a * c)
#check (mul_zero a : a * 0 = 0)
#check (neg_add_cancel_left a b : -a + (a + b) = b)
#check (neg_eq_iff_add_eq_zero : -a = b ↔ a + b = 0)
#check (neg_eq_of_add_eq_zero_left : a + b = 0 → -b = a)
#check (neg_eq_of_add_eq_zero_right : a + b = 0 → -a = b)
#check (neg_neg a : -(-a) = a)
#check (neg_zero : -0 = 0)
#check (one_add_one_eq_two : (1 : R) + 1 = 2)
#check (sub_add_cancel a b : a - b + b = a)
#check (sub_eq_add_neg a b : a - b = a + -b)
#check (sub_mul a b c : (a - b) * c = a * c - b * c)
#check (sub_self a : a - a = 0)
#check (two_mul a : 2 * a = a + a)
#check (zero_add a : 0 + a = a)
#check (zero_mul a : 0 * a = 0)
end anillos

-- Grupos
-- ======

section grupos
variable {G : Type _} [Group G]
variable (a b c : G)
#check (inv_eq_of_mul_eq_one_right : a * b = 1 → a⁻¹ = b)
#check (mul_assoc a b c : (a * b) * c = a * (b * c))
#check (mul_inv_self a : a * a⁻¹ = 1)
#check (mul_inv_rev a b : (a * b)⁻¹ = b⁻¹ * a⁻¹)
#check (mul_left_inv a : a⁻¹  * a = 1)
#check (mul_one a : a * 1 = a)
#check (mul_right_inv a : a * a⁻¹ = 1)
#check (one_mul a : 1 * a = a)
end grupos

-- Retículos
-- =========

section reticulos
variable {α : Type _} [Lattice α]
variable (x y z : α)
#check (inf_assoc : (x ⊓ y) ⊓ z = x ⊓ (y ⊓ z))
#check (inf_comm : x ⊓ y = y ⊓ x)
#check (inf_le_left : x ⊓ y ≤ x)
#check (inf_le_of_left_le : x ≤ z → x ⊓ y ≤ z)
#check (inf_le_of_right_le : y ≤ z → x ⊓ y ≤ z)
#check (inf_le_right : x ⊓ y ≤ y)
#check (inf_sup_self : x ⊓ (x ⊔ y) = x)
#check (le_antisymm : x ≤ y → y ≤ x → x = y)
#check (le_inf : z ≤ x → z ≤ y → z ≤ x ⊓ y)
#check (le_rfl : x ≤ x)
#check (le_sup_left : x ≤ x ⊔ y)
#check (le_sup_of_le_left : z ≤ x → z ≤ x ⊔ y)
#check (le_sup_of_le_right : z ≤ y → z ≤ x ⊔ y)
#check (le_sup_right : y ≤ x ⊔ y)
#check (le_trans : x ≤ y → y ≤ z → x ≤ z)
#check (sup_assoc : (x ⊔ y) ⊔ z = x ⊔ (y ⊔ z))
#check (sup_comm : x ⊔ y = y ⊔ x)
#check (sup_inf_self : x ⊔ (x ⊓ y) = x)
#check (sup_le : x ≤ z → y ≤ z → x ⊔ y ≤ z)
end reticulos

-- AnillosOrdenados
-- ================

section AnillosOrdenados
variable {R : Type _} [StrictOrderedRing R]
variable (a b c : R)
#check (add_le_add_right : b ≤ c → ∀ (a : R),  b + a ≤ c + a)
#check (mul_le_mul_of_nonneg_left : b ≤ c → 0 ≤ a → a * b ≤ a * c)
#check (mul_le_mul_of_nonneg_right : a ≤ b → 0 ≤ c → a * c ≤ b * c)
#check (mul_nonneg : 0 ≤ a → 0 ≤ b → 0 ≤ a * b)
#check (sub_le_sub_right : a ≤ b → ∀ (c : R), a - c ≤ b - c)
#check (sub_nonneg : 0 ≤ a - b ↔ b ≤ a)
end AnillosOrdenados

-- Espacios métricos
-- =================

section EspacioMetrico
variable {X : Type _} [MetricSpace X]
variable (x y z : X)
#check (dist_comm x y : dist x y = dist y x)
#check (dist_nonneg : 0 ≤ dist x y)
#check (dist_self x : dist x x = 0)
#check (dist_triangle x y z : dist x z ≤ dist x y + dist y z)
end EspacioMetrico

-- Conjuntos
-- =========

section Conjuntos
open Set
variable {α : Type _}
variable (r s t : Set α)
#check (Subset.trans : r ⊆ s → s ⊆ t → r ⊆ t)
end Conjuntos

-- Órdenes parciales
-- =================

section OrdenParcial
variable {α : Type _} [PartialOrder α]
variable (a b c : α)
#check (irrefl a : ¬a < a)
#check (le_trans : a ≤ b → b ≤ c → a ≤ c)
#check (lt_trans : a < b → b < c → a < c)
#check (monotone_const : Monotone fun _ : ℝ ↦ c)
end OrdenParcial

-- Funciones
-- =========

section Funciones
open Function
variable {α : Type _} {β : Type _} {γ : Type _}
variable {f : α → β} {g : β → γ}
variable (c : ℝ)
#check (Injective.comp : Injective g → Injective f → Injective (g ∘ f))
#check (Surjective.comp : Surjective g → Surjective f → Surjective (g ∘ f))
#check (add_right_surjective c : Surjective (fun x ↦ x + c))
#check (mul_left_surjective₀ : c ≠ 0 → Surjective (fun x ↦ c * x))
end Funciones

-- Lógica
-- ======

section Logica
variable (p q : Prop)
variable {α : Type _}
variable (P : α → Prop)
#check (absurd : p → ¬p → q)
#check (forall_not_of_not_exists : (¬∃ x, P x) → ∀ x, ¬P x)
#check (not_exists : (¬∃ x, P x) ↔ ∀ (x : α), ¬P x)
#check (not_exists_of_forall_not : (∀ x, P x → q) → (∃ x, P x) → q)
#check (not_imp : ¬(p → q) ↔ p ∧ ¬q)
#check (not_forall : (¬∀ x, P x) ↔ ∃ x, ¬P x)
#check (not_forall_of_exists_not : (∃ x, ¬P x) → ¬∀ x, P x)
#check (not_not_intro : p → ¬¬p)
#check (of_not_not : ¬¬p → p)
#check (Or.inl : p → p ∨ q)
#check (Or.inr : q → p ∨ q)
end Logica
\end{verbatim}

\end{document}
