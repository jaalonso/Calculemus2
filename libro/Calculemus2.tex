\input Calculemus2_preambulo

\chapter{Introducción}
\label{sec:org256f211}

Este libro es una recopilación de los ejercicios de demostración con
Lean4 que se han ido publicando, desde el 10 de julio de 20023, en el
blog \href{https://www.glc.us.es}{Calculemus}.

La ordenación de los ejercicios es simplemente temporal según su fecha
de publicación en Calculemus y el orden de los ejercicios en Calculemus
responde a los que me voy encontrando en mis \href{https://github.com/jaalonso/Lecturas\_GLC}{lecturas}.

En cada ejercicio, se comienza proponiendo soluciones en lenguaje
natural y, a continuación, se exponen distintas demostraciones con Lean4
ordenadas desde las más detalladas a las más automáticas. Al final de
cada ejercicio hay un enlace para interactuar con sus soluciones en
\href{https://lean.math.hhu.de/}{Lean4 Web}.

Las soluciones del libro están en \href{https://github.com/jaalonso/Calculemus2}{este repositorio de GitHub}.

El libro se irá actualizando periódicamente con los nuevos ejercicios que se
proponen diariamente en \href{https://www.glc.us.es}{Calculemus}.

Este libro es una continuación de
\begin{itemize}
\item \href{https://raw.githubusercontent.com/jaalonso/DAO\_con\_Lean/master/DAO\_con\_Lean.pdf}{DAO (Demostración Asistida por Ordenador) con Lean} que es una
introducción a la demostración con Lean3 y
\item \href{https://raw.githubusercontent.com/jaalonso/Calculemus/master/Calculemus.pdf}{Calculemus (Vol. 1: Demostraciones con Isabelle/HOL y Lean3)} que es la
recopilación de la primera parte de los ejercicios del blog con
demostraciones en Isabelle/HOL y Lean3.
\end{itemize}

\chapter{Demostraciones de una propiedad de los números enteros}
\label{sec:org7ff1e2e}

\section{∀ m n ∈ ℕ, Even n → Even (m * n)}
\label{sec:org2c52ee1}
\begin{verbatim}
-- ---------------------------------------------------------------------
-- Demostrar que los productos de los números naturales por números
-- pares son pares.
-- ---------------------------------------------------------------------

-- Demostración en lenguaje natural
-- ================================

-- Si n es par, entonces (por la definición de `Even`) existe un k tal que
--    n = k + k         (1)
-- Por tanto,
--    mn = m(k + k)     (por (1))
--       = mk + mk      (por la propiedad distributiva)
-- Por consiguiente, mn es par.

-- Demostraciones en Lean4
-- =======================

import Mathlib.Data.Nat.Basic
import Mathlib.Data.Nat.Parity
import Mathlib.Tactic

open Nat

-- 1ª demostración
-- ===============

example : ∀ m n : ℕ, Even n → Even (m * n) := by
  rintro m n ⟨k, hk⟩
  use m * k
  rw [hk]
  ring

-- 2ª demostración
-- ===============

example : ∀ m n : ℕ, Even n → Even (m * n) := by
  rintro m n ⟨k, hk⟩
  use m * k
  rw [hk]
  rw [mul_add]

-- 3ª demostración
-- ===============

example : ∀ m n : ℕ, Even n → Even (m * n) := by
  rintro m n ⟨k, hk⟩
  use m * k
  rw [hk, mul_add]

-- 4ª demostración
-- ===============

example : ∀ m n : Nat, Even n → Even (m * n) := by
  rintro m n ⟨k, hk⟩; use m * k; rw [hk, mul_add]

-- 5ª demostración
-- ===============

example : ∀ m n : ℕ, Even n → Even (m * n) := by
  rintro m n ⟨k, hk⟩
  exact ⟨m * k, by rw [hk, mul_add]⟩

-- 6ª demostración
-- ===============

example : ∀ m n : Nat, Even n → Even (m * n) :=
fun m n ⟨k, hk⟩ ↦ ⟨m * k, by rw [hk, mul_add]⟩

-- 7ª demostración
-- ===============

example : ∀ m n : ℕ, Even n → Even (m * n) := by
  rintro m n ⟨k, hk⟩
  use m * k
  rw [hk]
  exact mul_add m k k

-- 8ª demostración
-- ===============

example : ∀ m n : ℕ, Even n → Even (m * n) := by
  intros m n hn
  unfold Even at *
  cases hn with
  | intro k hk =>
    use m * k
    rw [hk, mul_add]

-- 9ª demostración
-- ===============

example : ∀ m n : ℕ, Even n → Even (m * n) := by
  intros m n hn
  unfold Even at *
  cases hn with
  | intro k hk =>
    use m * k
    calc m * n
       = m * (k + k)   := by exact congrArg (HMul.hMul m) hk
     _ = m * k + m * k := by exact mul_add m k k

-- 10ª demostración
-- ================

example : ∀ m n : Nat, Even n → Even (m * n) := by
  intros; simp [*, parity_simps]

-- Lemas usados
-- ============

-- #check (mul_add : ∀ a b c : ℕ, a * (b + c) = a * b + a * c)
\end{verbatim}
Se puede interactuar con las pruebas anteriores en \href{https://lean.math.hhu.de/\#url=https://raw.githubusercontent.com/jaalonso/Calculemus2/main/src/El\_producto\_por\_un\_par\_es\_par.lean}{Lean 4 Web}.

\chapter{Propiedades elementales de los números reales}
\label{sec:org8eed77e}

\section{En ℝ, (ab)c = b(ac)}
\label{sec:org98194c9}
\begin{verbatim}
-- ---------------------------------------------------------------------
-- Demostrar que los números reales tienen la siguiente propiedad
--    (a * b) * c = b * (a * c)
-- ---------------------------------------------------------------------

-- Demostración en lenguaje natural
-- ================================

-- Por la siguiente cadena de igualdades
--    (ab)c = (ba)c   [por la conmutativa]
--          = b(ac)   [por la asociativa]

-- Demostraciones con Lean4
-- ========================

import Mathlib.Tactic
import Mathlib.Data.Real.Basic

-- 1ª demostración
example
  (a b c : ℝ)
  : (a * b) * c = b * (a * c) :=
calc
  (a * b) * c = (b * a) * c := by rw [mul_comm a b]
            _ = b * (a * c) := by rw [mul_assoc b a c]

-- 2ª demostración
example (a b c : ℝ) : (a * b) * c = b * (a * c) :=
by
  rw [mul_comm a b]
  rw [mul_assoc b a c]

-- 3ª demostración
example (a b c : ℝ) : (a * b) * c = b * (a * c) :=
by ring

-- Lemas usados
-- ============

-- #check (mul_comm : ∀ (a b : ℝ), a * b = b * a)
-- #check (mul_assoc : ∀ (a b c : ℝ), (a * b) * c = a * (b * c))
\end{verbatim}
Se puede interactuar con las pruebas anteriores en \href{https://lean.math.hhu.de/\#url=https://raw.githubusercontent.com/jaalonso/Calculemus2/main/src/Asociativa\_conmutativa\_de\_los\_reales.lean}{Lean 4 Web}.

\section{En ℝ, (cb)a = b(ac)}
\label{sec:orgb401259}
\begin{verbatim}
-- ---------------------------------------------------------------------
-- Demostrar que los números reales tienen la siguiente propiedad
--    (c * b) * a = b * (a * c)
-- ---------------------------------------------------------------------

-- Demostración en lenguaje natural
-- ================================

-- Por la siguiente cadena de igualdades:
--    (c * b) * a
--    = (b * c) * a    [por la conmutativa]
--    = b * (c * a)    [por la asociativa]
--    = b * (a * c)    [por la conmutativa]

-- Demostraciones con Lean4
-- ========================

import Mathlib.Tactic
import Mathlib.Data.Real.Basic

-- 1ª demostración
example
  (a b c : ℝ)
  : (c * b) * a = b * (a * c) :=
calc
  (c * b) * a
    = (b * c) * a := by rw [mul_comm c b]
  _ = b * (c * a) := by rw [mul_assoc]
  _ = b * (a * c) := by rw [mul_comm c a]

-- 2ª demostración
example
  (a b c : ℝ)
  : (c * b) * a = b * (a * c) :=
by
  rw [mul_comm c b]
  rw [mul_assoc]
  rw [mul_comm c a]

-- 3ª demostración
example
  (a b c : ℝ)
  : (c * b) * a = b * (a * c) :=
by ring

-- Lemas usados
-- ============

-- #check (mul_comm : ∀ (a b : ℝ), a * b = b * a)
-- #check (mul_assoc : ∀ (a b c : ℝ), (a * b) * c = a * (b * c))
\end{verbatim}
Se puede interactuar con las pruebas anteriores en \href{https://lean.math.hhu.de/\#url=https://raw.githubusercontent.com/jaalonso/Calculemus2/main/src/(cb)a\_eq\_b(ac).lean}{Lean 4 Web}.

\section{En ℝ, a(bc) = b(ac)}
\label{sec:org2c3b99f}
\begin{verbatim}
-- ---------------------------------------------------------------------
-- Demostrar que los números reales tienen la siguiente propiedad
--    a * (b * c) = b * (a * c)
-- ---------------------------------------------------------------------

-- Demostración en lenguaje natural
-- ================================

-- Por la siguiente cadena de igualdades:
--    a(bc)
--    = (ab)c    [por la asociativa]
--    = (ba)c    [por la conmutativa]
--    = b(ac)    [por la asociativa]

-- Demostraciones en Lean4
-- =======================

import Mathlib.Tactic
import Mathlib.Data.Real.Basic

-- 1ª demostración
example
  (a b c : ℝ) : a * (b * c) = b * (a * c) :=
calc
  a * (b * c)
    = (a * b) * c := by rw [←mul_assoc]
  _ = (b * a) * c := by rw [mul_comm a b]
  _ = b * (a * c) := by rw [mul_assoc]

-- 2ª demostración
example
  (a b c : ℝ) : a * (b * c) = b * (a * c) :=
by
  rw [←mul_assoc]
  rw [mul_comm a b]
  rw [mul_assoc]

-- 3ª demostración
example
  (a b c : ℝ) : a * (b * c) = b * (a * c) :=
by ring

-- Lemas usados
-- ============

-- #check (mul_comm : ∀ (a b : ℝ), a * b = b * a)
-- #check (mul_assoc : ∀ (a b c : ℝ), (a * b) * c = a * (b * c))
\end{verbatim}
Se puede interactuar con las pruebas anteriores en \href{https://lean.math.hhu.de/\#url=https://raw.githubusercontent.com/jaalonso/Calculemus2/main/src/a(bc)\_eq\_b(ac).lean}{Lean 4 Web}.

\section{En ℝ, si ab = cd y e = f, entonces a(be) = c(df)}
\label{sec:orgec7b0b4}
\begin{verbatim}
-- ---------------------------------------------------------------------
-- Demostrar que si a, b, c, d, e y f son números reales tales que
--    a * b = c * d y
--    e = f,
-- entonces
--    a * (b * e) = c * (d * f)
-- ---------------------------------------------------------------------

-- Demostración en leguaje natural
-- ===============================

-- Por la siguiente cadena de igualdades
--    a(be)
--    = a(bf)    [por la segunda hipótesis]
--    = (ab)f    [por la asociativa]
--    = (cd)f    [por la primera hipótesis]
--    = c(df)    [por la asociativa]

-- Demostraciones en Lean4
-- =======================

import Mathlib.Tactic
import Mathlib.Data.Real.Basic

-- 1ª demostración
example
  (a b c d e f : ℝ)
  (h1 : a * b = c * d)
  (h2 : e = f)
  : a * (b * e) = c * (d * f) :=
calc
  a * (b * e)
    = a * (b * f) := by rw [h2]
  _ = (a * b) * f := by rw [←mul_assoc]
  _ = (c * d) * f := by rw [h1]
  _ = c * (d * f) := by rw [mul_assoc]

-- 2ª demostración
example
  (a b c d e f : ℝ)
  (h1 : a * b = c * d)
  (h2 : e = f)
  : a * (b * e) = c * (d * f) :=
by
  rw [h2]
  rw [←mul_assoc]
  rw [h1]
  rw [mul_assoc]

-- 3ª demostración
example
  (a b c d e f : ℝ)
  (h1 : a * b = c * d)
  (h2 : e = f)
  : a * (b * e) = c * (d * f) :=
by
  simp [*, ←mul_assoc]

-- Lemas usados
-- ============

-- #check (mul_assoc : ∀ (a b c : ℝ), (a * b) * c = a * (b * c))
\end{verbatim}
Se puede interactuar con las pruebas anteriores en \href{https://lean.math.hhu.de/\#url=https://raw.githubusercontent.com/jaalonso/Calculemus2/main/src/a(be)\_eq\_c(df).lean}{Lean 4 Web}.

\section{En ℝ, si bc = ef, entonces ((ab)c)d = ((ae)f)d}
\label{sec:org69cef22}
\begin{verbatim}
-- ---------------------------------------------------------------------
-- Demostrar que si a, b, c, d, e y f son números reales tales que
--    b * c = e * f
-- entonces
--    ((a * b) * c) * d = ((a * e) * f) * d
-- ---------------------------------------------------------------------

-- Demostración en lenguaje natural
-- ================================

-- Por la siguiente cadena de igualdades
--    ((ab)c)d
--    = (a(bc))d    [por la asociativa]
--    = (a(ef))d    [por la hipótesis]
--    = ((ae)f)d    [por la asociativa]

-- Demostraciones con Lean4
-- ========================

import Mathlib.Data.Real.Basic
import Mathlib.Tactic

-- 1ª demostración
example
  (a b c d e f : ℝ)
  (h : b * c = e * f)
  : ((a * b) * c) * d = ((a * e) * f) * d :=
calc
  ((a * b) * c) * d
    = (a * (b * c)) * d := by rw [mul_assoc a]
  _ = (a * (e * f)) * d := by rw [h]
  _ = ((a * e) * f) * d := by rw [←mul_assoc a]

-- 2ª demostración
example
  (a b c d e f : ℝ)
  (h : b * c = e * f)
  : ((a * b) * c) * d = ((a * e) * f) * d :=
by
  rw [mul_assoc a]
  rw [h]
  rw [←mul_assoc a]

-- 3ª demostración
example
  (a b c d e f : ℝ)
  (h : b * c = e * f)
  : ((a * b) * c) * d = ((a * e) * f) * d :=
by
  rw [mul_assoc a, h, ←mul_assoc a]

-- Lemas usados
-- ============

-- #check (mul_assoc : ∀ (a b c : ℝ), (a * b) * c = a * (b * c))
\end{verbatim}
Se puede interactuar con las pruebas anteriores en \href{https://lean.math.hhu.de/\#url=https://raw.githubusercontent.com/jaalonso/Calculemus2/main/src/Si\_bc\_eq\_ef\_entonces\_((ab)c)d\_eq\_((ae)f)d.lean}{Lean 4 Web}.

\section{En ℝ, si c = ba-d y d = ab, entonces c = 0}
\label{sec:org765b143}
\begin{verbatim}
-- ---------------------------------------------------------------------
-- Demostrar que si a, b, c y d son números reales tales que
--    c = b * a - d
--    d = a * b
-- entonces
--    c = 0
-- ---------------------------------------------------------------------

-- Demostración en lenguaje natural
-- ================================

-- Por la siguiente cadena de igualdades
--    c = ba - d     [por la primera hipótesis]
--      = ab - d     [por la conmutativa]
--      = ab - ab    [por la segunda hipótesis]
--      = 0

-- Demostraciones en Lean4
-- =======================

import Mathlib.Data.Real.Basic
import Mathlib.Tactic

-- 1ª demostración
example
  (a b c d : ℝ)
  (h1 : c = b * a - d)
  (h2 : d = a * b)
  : c = 0 :=
calc
  c = b * a - d     := by rw [h1]
  _ = a * b - d     := by rw [mul_comm]
  _ = a * b - a * b := by rw [h2]
  _ = 0             := by rw [sub_self]

-- 2ª demostración
example
  (a b c d : ℝ)
  (h1 : c = b * a - d)
  (h2 : d = a * b)
  : c = 0 :=
by
  rw [h1]
  rw [mul_comm]
  rw [h2]
  rw [sub_self]

-- 3ª demostración
example
  (a b c d : ℝ)
  (h1 : c = b * a - d)
  (h2 : d = a * b)
  : c = 0 :=
by
  rw [h1, mul_comm, h2, sub_self]

-- Lemas usados
-- ============

-- #check (mul_comm : ∀ (a b : ℝ), a * b = b * a)
-- #check (sub_self : ∀ (a : ℝ), a - a = 0)
\end{verbatim}
Se puede interactuar con las pruebas anteriores en \href{https://lean.math.hhu.de/\#url=https://raw.githubusercontent.com/jaalonso/Calculemus2/main/src/Si\_c\_eq\_ba-d\_y\_d\_eq\_ab\_entonces\_c\_eq\_0.lean}{Lean 4 Web}.

\section{En ℝ, (a+b)(a+b) = aa+2ab+bb}
\label{sec:orga75aae8}
\begin{verbatim}
-- ---------------------------------------------------------------------
-- Demostrar que si a y b son números reales, entonces
--    (a + b) * (a + b) = a * a + 2 * (a * b) + b * b
-- ---------------------------------------------------------------------

-- Demostración en lenguaje natural
-- ================================

-- Por la siguiente cadena de igualdades
--    (a + b)(a + b)
--    = (a + b)a + (a + b)b    [por la distributiva]
--    = aa + ba + (a + b)b     [por la distributiva]
--    = aa + ba + (ab + bb)    [por la distributiva]
--    = aa + ba + ab + bb      [por la asociativa]
--    = aa + (ba + ab) + bb    [por la asociativa]
--    = aa + (ab + ab) + bb    [por la conmutativa]
--    = aa + 2(ab) + bb        [por def. de doble]

-- Demostraciones con Lean4
-- ========================

import Mathlib.Data.Real.Basic
import Mathlib.Tactic

variable (a b c : ℝ)

-- 1ª demostración
example :
  (a + b) * (a + b) = a * a + 2 * (a * b) + b * b :=
calc
  (a + b) * (a + b)
    = (a + b) * a + (a + b) * b       := by rw [mul_add]
  _ = a * a + b * a + (a + b) * b     := by rw [add_mul]
  _ = a * a + b * a + (a * b + b * b) := by rw [add_mul]
  _ = a * a + b * a + a * b + b * b   := by rw [←add_assoc]
  _ = a * a + (b * a + a * b) + b * b := by rw [add_assoc (a * a)]
  _ = a * a + (a * b + a * b) + b * b := by rw [mul_comm b a]
  _ = a * a + 2 * (a * b) + b * b     := by rw [←two_mul]

-- 2ª demostración
example :
  (a + b) * (a + b) = a * a + 2 * (a * b) + b * b :=
calc
  (a + b) * (a + b)
    = a * a + b * a + (a * b + b * b) := by rw [mul_add, add_mul, add_mul]
  _ = a * a + (b * a + a * b) + b * b := by rw [←add_assoc, add_assoc (a * a)]
  _ = a * a + 2 * (a * b) + b * b     := by rw [mul_comm b a, ←two_mul]

-- 3ª demostración
example :
  (a + b) * (a + b) = a * a + 2 * (a * b) + b * b :=
calc
  (a + b) * (a + b)
    = a * a + b * a + (a * b + b * b) := by ring
  _ = a * a + (b * a + a * b) + b * b := by ring
  _ = a * a + 2 * (a * b) + b * b     := by ring

-- 4ª demostración
example :
  (a + b) * (a + b) = a * a + 2 * (a * b) + b * b :=
by ring

-- 5ª demostración
example :
  (a + b) * (a + b) = a * a + 2 * (a * b) + b * b :=
by
  rw [mul_add]
  rw [add_mul]
  rw [add_mul]
  rw [←add_assoc]
  rw [add_assoc (a * a)]
  rw [mul_comm b a]
  rw [←two_mul]

-- 6ª demostración
example :
  (a + b) * (a + b) = a * a + 2 * (a * b) + b * b :=
by
  rw [mul_add, add_mul, add_mul]
  rw [←add_assoc, add_assoc (a * a)]
  rw [mul_comm b a, ←two_mul]

-- 7ª demostración
example :
  (a + b) * (a + b) = a * a + 2 * (a * b) + b * b :=
by linarith

-- Lemas usados
-- ============

-- #check (add_assoc : ∀ a b c : ℝ, (a + b) + c = a + (b + c))
-- #check (add_mul : ∀ a b c : ℝ, (a + b) * c = a * c + b * c)
-- #check (mul_add : ∀ a b c : ℝ, a * (b + c) = a * b + a * c)
-- #check (mul_comm : ∀ (a b : ℝ), a * b = b * a)
-- #check (two_mul : ∀ (a : ℝ), 2 * a = a + a)
\end{verbatim}
Se puede interactuar con las pruebas anteriores en \href{https://lean.math.hhu.de/\#url=https://raw.githubusercontent.com/jaalonso/Calculemus2/main/src/(a+b)(a+b)\_eq\_aa+2ab+bb.lean}{Lean 4 Web}.

\section{En ℝ, (a+b)(c+d) = ac+ad+bc+bd}
\label{sec:orgfe42cec}
\begin{verbatim}
-- ---------------------------------------------------------------------
-- Demostrar que si a, b, c y d son números reales, entonces
--    (a + b) * (c + d) = a * c + a * d + b * c + b * d
-- ---------------------------------------------------------------------

-- Demostración en lenguaje natural
-- ================================

-- Por la siguiente cadena de igualdades
--    (a + b)(c + d)
--    = a(c + d) + b(c + d)    [por la distributiva]
--    = ac + ad + b(c + d)     [por la distributiva]
--    = ac + ad + (bc + bd)    [por la distributiva]
--    = ac + ad + bc + bd      [por la asociativa]

-- Demostraciones con Lean4
-- ========================

import Mathlib.Data.Real.Basic
import Mathlib.Tactic

variable (a b c d : ℝ)

-- 1ª demostración
example
  : (a + b) * (c + d) = a * c + a * d + b * c + b * d :=
calc
  (a + b) * (c + d)
    = a * (c + d) + b * (c + d)       := by rw [add_mul]
  _ = a * c + a * d + b * (c + d)     := by rw [mul_add]
  _ = a * c + a * d + (b * c + b * d) := by rw [mul_add]
  _ = a * c + a * d + b * c + b * d   := by rw [←add_assoc]

-- 2ª demostración
example
  : (a + b) * (c + d) = a * c + a * d + b * c + b * d :=
calc
  (a + b) * (c + d)
    = a * (c + d) + b * (c + d)       := by ring
  _ = a * c + a * d + b * (c + d)     := by ring
  _ = a * c + a * d + (b * c + b * d) := by ring
  _ = a * c + a * d + b * c + b * d   := by ring

-- 3ª demostración
example : (a + b) * (c + d) = a * c + a * d + b * c + b * d :=
by ring

-- 4ª demostración
example
  : (a + b) * (c + d) = a * c + a * d + b * c + b * d :=
by
   rw [add_mul]
   rw [mul_add]
   rw [mul_add]
   rw [← add_assoc]

-- 5ª demostración
example : (a + b) * (c + d) = a * c + a * d + b * c + b * d :=
by rw [add_mul, mul_add, mul_add, ←add_assoc]

-- Lemas usados
-- ============

-- #check (add_mul : ∀ (a b c : ℝ), (a + b) * c = a * c + b * c)
-- #check (mul_add : ∀ (a b c : ℝ), a * (b + c) = a * b + a * c)
-- #check (add_assoc : ∀ (a b c : ℝ), (a + b) + c = a + (b + c))
\end{verbatim}
Se puede interactuar con las pruebas anteriores en \href{https://lean.math.hhu.de/\#url=https://raw.githubusercontent.com/jaalonso/Calculemus2/main/src/(a+b)(c+d)\_eq\_ac+ad+bc+bd.lean}{Lean 4 Web}.

\section{En ℝ, (a+b)(a-b) = a\^{}2-b\^{}2}
\label{sec:orgd0b6753}
\begin{verbatim}
-- ---------------------------------------------------------------------
-- Demostrar que si a y b son números reales, entonces
--    (a + b) * (a - b) = a^2 - b^2
-- ---------------------------------------------------------------------

-- Demostración en lenguaje natural
-- ================================

-- Por la siguiente cadena de igualdades:
--    (a + b)(a - b)
--    = a(a - b) + b(a - b)            [por la distributiva]
--    = (aa - ab) + b(a - b)           [por la distributiva]
--    = (a^2 - ab) + b(a - b)          [por def. de cuadrado]
--    = (a^2 - ab) + (ba - bb)         [por la distributiva]
--    = (a^2 - ab) + (ba - b^2)        [por def. de cuadrado]
--    = (a^2 + -(ab)) + (ba - b^2)     [por def. de resta]
--    = a^2 + (-(ab) + (ba - b^2))     [por la asociativa]
--    = a^2 + (-(ab) + (ba + -b^2))    [por def. de resta]
--    = a^2 + ((-(ab) + ba) + -b^2)    [por la asociativa]
--    = a^2 + ((-(ab) + ab) + -b^2)    [por la conmutativa]
--    = a^2 + (0 + -b^2)               [por def. de opuesto]
--    = (a^2 + 0) + -b^2               [por asociativa]
--    = a^2 + -b^2                     [por def. de cero]
--    = a^2 - b^2                      [por def. de resta]

-- Demostraciones con Lean4
-- ========================

import Mathlib.Data.Real.Basic
import Mathlib.Tactic

variable (a b : ℝ)

-- 1ª demostración
-- ===============

example : (a + b) * (a - b) = a^2 - b^2 :=
calc
  (a + b) * (a - b)
    = a * (a - b) + b * (a - b)         := by rw [add_mul]
  _ = (a * a - a * b) + b * (a - b)     := by rw [mul_sub]
  _ = (a^2 - a * b) + b * (a - b)       := by rw [← pow_two]
  _ = (a^2 - a * b) + (b * a - b * b)   := by rw [mul_sub]
  _ = (a^2 - a * b) + (b * a - b^2)     := by rw [← pow_two]
  _ = (a^2 + -(a * b)) + (b * a - b^2)  := by ring
  _ = a^2 + (-(a * b) + (b * a - b^2))  := by rw [add_assoc]
  _ = a^2 + (-(a * b) + (b * a + -b^2)) := by ring
  _ = a^2 + ((-(a * b) + b * a) + -b^2) := by rw [← add_assoc
                                              (-(a * b)) (b * a) (-b^2)]
  _ = a^2 + ((-(a * b) + a * b) + -b^2) := by rw [mul_comm]
  _ = a^2 + (0 + -b^2)                  := by rw [neg_add_self (a * b)]
  _ = (a^2 + 0) + -b^2                  := by rw [← add_assoc]
  _ = a^2 + -b^2                        := by rw [add_zero]
  _ = a^2 - b^2                         := by linarith

-- 2ª demostración
-- ===============

example : (a + b) * (a - b) = a^2 - b^2 :=
calc
  (a + b) * (a - b)
    = a * (a - b) + b * (a - b)         := by ring
  _ = (a * a - a * b) + b * (a - b)     := by ring
  _ = (a^2 - a * b) + b * (a - b)       := by ring
  _ = (a^2 - a * b) + (b * a - b * b)   := by ring
  _ = (a^2 - a * b) + (b * a - b^2)     := by ring
  _ = (a^2 + -(a * b)) + (b * a - b^2)  := by ring
  _ = a^2 + (-(a * b) + (b * a - b^2))  := by ring
  _ = a^2 + (-(a * b) + (b * a + -b^2)) := by ring
  _ = a^2 + ((-(a * b) + b * a) + -b^2) := by ring
  _ = a^2 + ((-(a * b) + a * b) + -b^2) := by ring
  _ = a^2 + (0 + -b^2)                  := by ring
  _ = (a^2 + 0) + -b^2                  := by ring
  _ = a^2 + -b^2                        := by ring
  _ = a^2 - b^2                         := by ring

-- 3ª demostración
-- ===============

example : (a + b) * (a - b) = a^2 - b^2 :=
by ring

-- 4ª demostración
-- ===============

-- El lema anterior es
lemma aux : (a + b) * (c + d) = a * c + a * d + b * c + b * d :=
by ring

-- La demostración es
example : (a + b) * (a - b) = a^2 - b^2 :=
by
  rw [sub_eq_add_neg]
  rw [aux]
  rw [mul_neg]
  rw [add_assoc (a * a)]
  rw [mul_comm b a]
  rw [neg_add_self]
  rw [add_zero]
  rw [← pow_two]
  rw [mul_neg]
  rw [← pow_two]
  rw [← sub_eq_add_neg]

-- Lemas usados
-- ============

-- #check (add_assoc : ∀ (a b c : ℝ), (a + b) + c = a + (b + c))
-- #check (add_zero : ∀ (a : ℝ), a + 0 = a)
-- #check (add_mul : ∀ (a b c : ℝ), (a + b) * c = a * c + b * c)
-- #check (mul_comm : ∀ (a b : ℝ), a * b = b * a)
-- #check (mul_neg : ∀ (a b : ℝ), a * -b = -(a * b))
-- #check (mul_sub : ∀ (a b c : ℝ), a * (b - c) = a * b - a * c)
-- #check (neg_add_self : ∀ (a : ℝ), -a + a = 0)
-- #check (pow_two : ∀ (a : ℝ), a ^ 2 = a * a)
-- #check (sub_eq_add_neg : ∀ (a b : ℝ), a - b = a + -b)
\end{verbatim}
Se puede interactuar con las pruebas anteriores en \href{https://lean.math.hhu.de/\#url=https://raw.githubusercontent.com/jaalonso/Calculemus2/main/src/(a+b)(a-b)\_eq\_aa-bb.lean}{Lean 4 Web}.

\section{En ℝ, si c = da+b y b = ad, entonces c = 2ad}
\label{sec:orge417e85}
\begin{verbatim}
-- ---------------------------------------------------------------------
-- Demostrar que si a, b, c y d son números reales tales que
--    c = d * a + b
--    b = a * d
-- entonces
--    c = 2 * a * d
-- ---------------------------------------------------------------------

-- Demostración en lenguaje natural
-- ================================

-- Por la siguiente cadena de igualdades
--    c = da + b     [por la primera hipótesis]
--      = da + ad    [por la segunda hipótesis]
--      = ad + ad    [por la conmutativa]
--      = 2(ad)      [por la def. de doble]
--      = 2ad        [por la asociativa]

-- Demostraciones con Lean4
-- ========================

import Mathlib.Data.Real.Basic
import Mathlib.Tactic

variable (a b c d : ℝ)

-- 1ª demostración
example
  (h1 : c = d * a + b)
  (h2 : b = a * d)
  : c = 2 * a * d :=
calc
  c = d * a + b     := by rw [h1]
  _ = d * a + a * d := by rw [h2]
  _ = a * d + a * d := by rw [mul_comm d a]
  _ = 2 * (a * d)   := by rw [← two_mul (a * d)]
  _ = 2 * a * d     := by rw [mul_assoc]

-- 2ª demostración
example
  (h1 : c = d * a + b)
  (h2 : b = a * d)
  : c = 2 * a * d :=
by
  rw [h2] at h1
  clear h2
  rw [mul_comm d a] at h1
  rw [← two_mul (a*d)] at h1
  rw [← mul_assoc 2 a d] at h1
  exact h1

-- 3ª demostración
example
  (h1 : c = d * a + b)
  (h2 : b = a * d)
  : c = 2 * a * d :=
by rw [h1, h2, mul_comm d a, ← two_mul (a * d), mul_assoc]

-- 4ª demostración
example
  (h1 : c = d * a + b)
  (h2 : b = a * d)
  : c = 2 * a * d :=
by
  rw [h1]
  rw [h2]
  ring

-- 5ª demostración
example
  (h1 : c = d * a + b)
  (h2 : b = a * d)
  : c = 2 * a * d :=
by
  rw [h1, h2]
  ring

-- 6ª demostración
example
  (h1 : c = d * a + b)
  (h2 : b = a * d)
  : c = 2 * a * d :=
by rw [h1, h2] ; ring

-- 7ª demostración
example
  (h1 : c = d * a + b)
  (h2 : b = a * d)
  : c = 2 * a * d :=
by linarith

-- Lemas usados
-- ============

-- #check (mul_assoc : ∀ (a b c : ℝ), (a * b) * c = a * (b * c))
-- #check (mul_comm : ∀ (a b : ℝ), a * b = b * a)
-- #check (two_mul : ∀ (a : ℝ), 2 * a = a + a)
\end{verbatim}
Se puede interactuar con las pruebas anteriores en \href{https://lean.math.hhu.de/\#url=https://raw.githubusercontent.com/jaalonso/Calculemus2/main/src/Si\_c\_eq\_da+b\_y\_b\_eq\_ad\_entonces\_c\_eq\_2ad.lean}{Lean 4 Web}.

\section{En ℝ, si a+b = c, entonces (a+b)(a+b) = ac+bc}
\label{sec:org61bbd84}
\begin{verbatim}
-- ---------------------------------------------------------------------
-- Demostrar que si a, b y c son números reales tales que
--    a + b = c,
-- entonces
--    (a + b) * (a + b) = a * c + b * c
-- ---------------------------------------------------------------------

-- Demostración en lenguaje natural
-- ================================

-- Por la siguiente cadena de igualdades
--    (a + b)(a + b)
--    = (a + b)c        [por la hipótesis]
--    = ac + bc         [por la distributiva]

-- Demostraciones con Lean4
-- ========================

import Mathlib.Data.Real.Basic
import Mathlib.Tactic

variable (a b c : ℝ)

-- 1ª demostración
example
  (h : a + b = c)
  : (a + b) * (a + b) = a * c + b * c :=
calc
  (a + b) * (a + b)
    = (a + b) * c   := by exact congrArg (HMul.hMul (a + b)) h
  _ = a * c + b * c := by rw [add_mul]

-- 2ª demostración
example
  (h : a + b = c)
  : (a + b) * (a + b) = a * c + b * c :=
by
  nth_rewrite 2 [h]
  rw [add_mul]

-- Lemas usados
-- ============

-- #check (add_mul : ∀ (a b c : ℝ), (a + b) * c = a * c + b * c)
\end{verbatim}
Se puede interactuar con las pruebas anteriores en \href{https://lean.math.hhu.de/\#url=https://raw.githubusercontent.com/jaalonso/Calculemus2/main/src/Sia+b\_eq\_c\_entonces\_(a+b)(a+b)\_eq\_ac+bc.lean}{Lean 4 Web}.

\section{Si x e y son sumas de dos cuadrados, entonces xy también lo es}
\label{sec:org37fa8b7}
\begin{verbatim}
-- ---------------------------------------------------------------------
-- Demostrar que si x e y son sumas de dos cuadrados, entonces xy
-- también lo es.
-- ----------------------------------------------------------------------

-- Demostración en lenguaje natural
-- ================================

-- Puesto que x e y se pueden escribir como la suma de dos cuadrados,
-- existen a, b , c y d tales que
--    x = a² + b²
--    y = c² + d²
-- Entonces,
--    xy = (ac - bd)² + (ad + bc)²
-- En efecto,
--    xy = (a² + b²)(c² + d²)
--       = a²c² + b²d² + a²d² + b²c²
--       = a²c² - 2acbd + b²d² + a²d² + 2adbc + b²c²
--       = (ac - bd)² + (ad + bc)²
-- Por tanto, xy es la suma de dos cuadrados.

-- Demostraciones con Lean4
-- ========================

import Mathlib.Tactic
variable {α : Type _} [CommRing α]
variable {x y : α}

-- (suma_de_cuadrados x) afirma que x se puede escribir como la suma
-- de dos cuadrados.
def suma_de_cuadrados (x : α) :=
  ∃ a b, x = a^2 + b^2

-- 1ª demostración
example
  (hx : suma_de_cuadrados x)
  (hy : suma_de_cuadrados y)
  : suma_de_cuadrados (x * y) :=
by
  rcases hx with ⟨a, b, xeq : x = a^2 + b^2⟩
  -- a b : α
  -- xeq : x = a ^ 2 + b ^ 2
  rcases hy with ⟨c, d, yeq : y = c^2 + d^2⟩
  -- c d : α
  -- yeq : y = c ^ 2 + d ^ 2
  have h1: x * y = (a*c - b*d)^2 + (a*d + b*c)^2 :=
    calc x * y
         = (a^2 + b^2) * (c^2 + d^2) :=
                by rw [xeq, yeq]
       _ = a^2*c^2 + b^2*d^2 + a^2*d^2 + b^2*c^2 :=
                by ring
       _ = a^2*c^2 - 2*a*c*b*d + b^2*d^2 + a^2*d^2 + 2*a*d*b*c + b^2*c^2 :=
                by ring
       _ = (a*c - b*d)^2 + (a*d + b*c)^2 :=
                by ring
  have h2 : ∃ f, x * y = (a*c - b*d)^2 + f^2 :=
    Exists.intro (a*d + b*c) h1
  have h3 : ∃ e f, x * y = e^2 + f^2 :=
    Exists.intro (a*c - b*d) h2
  show suma_de_cuadrados (x * y)
  exact h3

-- 2ª demostración
example
  (hx : suma_de_cuadrados x)
  (hy : suma_de_cuadrados y)
  : suma_de_cuadrados (x * y) :=
by
  rcases hx with ⟨a, b, xeq : x = a^2 + b^2⟩
  -- a b : α
  -- xeq : x = a ^ 2 + b ^ 2
  rcases hy with ⟨c, d, yeq : y = c^2 + d^2⟩
  -- c d : α
  -- yeq : y = c ^ 2 + d ^ 2
  have h1: x * y = (a*c - b*d)^2 + (a*d + b*c)^2 :=
    calc x * y
         = (a^2 + b^2) * (c^2 + d^2)     := by rw [xeq, yeq]
       _ = (a*c - b*d)^2 + (a*d + b*c)^2 := by ring
  have h2 : ∃ e f, x * y = e^2 + f^2 :=
    by tauto
  show suma_de_cuadrados (x * y)
  exact h2

-- 3ª demostración
example
  (hx : suma_de_cuadrados x)
  (hy : suma_de_cuadrados y)
  : suma_de_cuadrados (x * y) :=
by
  rcases hx with ⟨a, b, xeq⟩
  -- a b : α
  -- xeq : x = a ^ 2 + b ^ 2
  rcases hy with ⟨c, d, yeq⟩
  -- c d : α
  -- yeq : y = c ^ 2 + d ^ 2
  rw [xeq, yeq]
  -- ⊢ suma_de_cuadrados ((a ^ 2 + b ^ 2) * (c ^ 2 + d ^ 2))
  use a*c - b*d, a*d + b*c
  -- ⊢ (a ^ 2 + b ^ 2) * (c ^ 2 + d ^ 2)
  --   = (a * c - b * d) ^ 2 + (a * d + b * c) ^ 2
  ring

-- 4ª demostración
example
  (hx : suma_de_cuadrados x)
  (hy : suma_de_cuadrados y)
  : suma_de_cuadrados (x * y) :=
by
  rcases hx with ⟨a, b, rfl⟩
  -- ⊢ suma_de_cuadrados ((a ^ 2 + b ^ 2) * y)
  rcases hy with ⟨c, d, rfl⟩
  -- ⊢ suma_de_cuadrados ((a ^ 2 + b ^ 2) * (c ^ 2 + d ^ 2))
  use a*c - b*d, a*d + b*c
  -- ⊢ (a ^ 2 + b ^ 2) * (c ^ 2 + d ^ 2)
  --   = (a * c - b * d) ^ 2 + (a * d + b * c) ^ 2
  ring
\end{verbatim}
Se puede interactuar con las pruebas anteriores en \href{https://lean.math.hhu.de/\#url=https://raw.githubusercontent.com/jaalonso/Calculemus2/main/src/Producto\_de\_suma\_de\_cuadrados.lean}{Lean 4 Web}

\chapter{Propiedades elementales de los anillos}
\label{sec:org5587522}

\section{Si R es un anillo y a ∈ R, entonces a + 0 = a}
\label{sec:org5b272f5}
\begin{verbatim}
-- ---------------------------------------------------------------------
-- Demostrar en Lean4 que si R es un anillo, entonces
--    ∀ a : R, a + 0 = a
-- ---------------------------------------------------------------------

-- Demostración en lenguaje natural
-- ================================

-- Por la siguiente cadena de igualdades
--    a + 0 = 0 + a    [por la conmutativa de la suma]
--          = a        [por el axioma del cero por la izquierda]

import Mathlib.Algebra.Ring.Defs

variable {R : Type _} [Ring R]
variable (a : R)

-- Demostraciones con Lean4
-- ========================

-- 1ª demostración
example : a + 0 = a :=
calc a + 0
     = 0 + a := by rw [add_comm]
   _ = a     := by rw [zero_add]

-- 2ª demostración
example : a + 0 = a :=
by
  rw [add_comm]
  rw [zero_add]

-- 3ª demostración
example : a + 0 = a :=
by rw [add_comm, zero_add]

-- 4ª demostración
example : a + 0 = a :=
by exact add_zero a

-- 5ª demostración
example : a + 0 = a :=
  add_zero a

-- 5ª demostración
example : a + 0 = a :=
by simp

-- Lemas usados
-- ============

variable (a b : R)

-- #check (add_comm a b : a + b = b + a)
-- #check (zero_add a :  0 + a = a)
\end{verbatim}
Se puede interactuar con las pruebas anteriores en \href{https://lean.math.hhu.de/\#url=https://raw.githubusercontent.com/jaalonso/Calculemus2/main/src/Suma\_con\_cero.lean}{Lean 4 Web}.

\section{Si R es un anillo y a ∈ R, entonces a + -a = 0}
\label{sec:orgae73cb3}
\begin{verbatim}
-- ---------------------------------------------------------------------
-- Demostrar en Lean4 que si R es un anillo, entonces
--    ∀ a : R, a + -a = 0
-- ---------------------------------------------------------------------

-- Demostración en lenguaje natural
-- ================================

-- Por la siguiente cadena de igualdades
--    a + -a = -a + a    [por la conmutativa de la suma]
--           = 0         [por el axioma de inverso por la izquierda]

import Mathlib.Algebra.Ring.Defs

variable {R : Type _} [Ring R]
variable (a : R)

-- 1ª demostración
-- ===============

example : a + -a = 0 :=
calc a + -a = -a + a := by rw [add_comm]
          _ = 0      := by rw [add_left_neg]

-- 2ª demostración
-- ===============

example : a + -a = 0 :=
by
  rw [add_comm]
  rw [add_left_neg]

-- 3ª demostración
-- ===============

example : a + -a = 0 :=
by rw [add_comm, add_left_neg]

-- 4ª demostración
-- ===============

example : a + -a = 0 :=
by exact add_neg_self a

-- 5ª demostración
-- ===============

example : a + -a = 0 :=
  add_neg_self a

-- 6ª demostración
-- ===============

example : a + -a = 0 :=
by simp

-- Lemas usados
-- ============

-- variable (a b : R)
-- #check (add_comm a b : a + b = b + a)
-- #check (add_left_neg a : -a + a = 0)
-- #check (add_neg_self a : a + -a = 0)
\end{verbatim}
Se puede interactuar con las pruebas anteriores en \href{https://lean.math.hhu.de/\#url=https://raw.githubusercontent.com/jaalonso/Calculemus2/main/src/Suma\_con\_opuesto.lean}{Lean 4 Web}.

\section{Si R es un anillo y a, b ∈ R, entonces -a + (a + b) = b}
\label{sec:orgd864252}
\begin{verbatim}
-- ---------------------------------------------------------------------
-- Demostrar en Lean4 que si R es un anillo, entonces
--    ∀ a, b : R, -a + (a + b) = b
-- ---------------------------------------------------------------------

-- Demostración en lenguaje natural
-- ================================

-- Por la siguiente cadena de igualdades
--    -a + (a + b) = (-a + a) + b [por la asociativa]
--                 = 0 + b        [por inverso por la izquierda]
--                 = b            [por cero por la izquierda]

import Mathlib.Algebra.Ring.Defs

variable {R : Type _} [Ring R]
variable (a b : R)

-- Demostraciones con Lean4
-- ========================

-- 1ª demostración
example : -a + (a + b) = b :=
calc -a + (a + b) = (-a + a) + b := by rw [← add_assoc]
                _ = 0 + b        := by rw [add_left_neg]
                _ = b            := by rw [zero_add]

-- 2ª demostración
example : -a + (a + b) = b :=
by
  rw [←add_assoc]
  rw [add_left_neg]
  rw [zero_add]

-- 3ª demostración
example : -a + (a + b) = b :=
by rw [←add_assoc, add_left_neg, zero_add]

-- 4ª demostración
example : -a + (a + b) = b :=
by exact neg_add_cancel_left a b

-- 5ª demostración
example : -a + (a + b) = b :=
  neg_add_cancel_left a b

-- 6ª demostración
example : -a + (a + b) = b :=
by simp

-- Lemas usados
-- ============

-- variable (c : R)
-- #check (add_assoc a b c : (a + b) + c = a + (b + c))
-- #check (add_left_neg a : -a + a = 0)
-- #check (neg_add_cancel_left a b : -a + (a + b) = b)
-- #check (zero_add a :  0 + a = a)
\end{verbatim}
Se puede interactuar con las pruebas anteriores en \href{https://lean.math.hhu.de/\#url=https://raw.githubusercontent.com/jaalonso/Calculemus2/main/src/Opuesto\_se\_cancela\_con\_la\_suma\_por\_la\_izquierda.lean}{Lean 4 Web}.

\section{Si R es un anillo y a, b ∈ R, entonces (a + b) + -b = a}
\label{sec:orge702542}
\begin{verbatim}
-- ---------------------------------------------------------------------
-- Demostrar en Lean4 que si R es un anillo, entonces
--    ∀ a, b : R, (a + b) + -b = a
-- ---------------------------------------------------------------------

-- Demostración en lenguaje natural
-- ================================

-- Por la siguiente cadena de igualdades
--    (a + b) + -b = a + (b + -b)    [por la asociativa]
--               _ = a + 0           [por suma con opuesto]
--               _ = a               [por suma con cero]

-- Demostraciones con Lean4
-- ========================

import Mathlib.Algebra.Ring.Defs

variable {R : Type _} [Ring R]
variable (a b : R)

-- 1ª demostración
example : (a + b) + -b = a :=
calc
  (a + b) + -b = a + (b + -b) := by rw [add_assoc]
             _ = a + 0        := by rw [add_right_neg]
             _ = a            := by rw [add_zero]

-- 2ª demostración
example : (a + b) + -b = a :=
by
  rw [add_assoc]
  rw [add_right_neg]
  rw [add_zero]

-- 3ª demostración
example : (a + b) + -b = a :=
by rw [add_assoc, add_right_neg, add_zero]

-- 4ª demostración
example : (a + b) + -b = a :=
  add_neg_cancel_right a b

-- 5ª demostración
example : (a + b) + -b = a :=
  add_neg_cancel_right _ _

-- 6ª demostración
example : (a + b) + -b = a :=
by simp

-- Lemas usados
-- ============

-- variable (c : R)
-- #check (add_assoc a b c : (a + b) + c = a + (b + c))
-- #check (add_neg_cancel_right a b : (a + b) + -b = a)
-- #check (add_right_neg a : a + -a = 0)
-- #check (add_zero a :  a + 0 = a)
\end{verbatim}
Se puede interactuar con las pruebas anteriores en \href{https://lean.math.hhu.de/\#url=https://raw.githubusercontent.com/jaalonso/Calculemus2/main/src/Opuesto\_se\_cancela\_con\_la\_suma\_por\_la\_derecha.lean}{Lean 4 Web}.

\section{Si R es un anillo y a, b, c ∈ R tales que a+b=a+c, entonces b=c}
\label{sec:org75c83f8}
\begin{verbatim}
-- ---------------------------------------------------------------------
-- Demostrar que si R es un anillo y a, b, c ∈ R tales que
--    a + b = a + c
-- entonces
--    b = c
-- ----------------------------------------------------------------------

-- Demostraciones en lenguaje natural (LN)
-- ======================================

-- 1ª demostración en LN
-- =====================

-- Por la siguiente cadena de igualdades
--    b = 0 + b           [por suma con cero]
--      = (-a + a) + b    [por suma con opuesto]
--      = -a + (a + b)    [por asociativa]
--      = -a + (a + c)    [por hipótesis]
--      = (-a + a) + c    [por asociativa]
--      = 0 + c           [por suma con opuesto]
--      = c               [por suma con cero]

-- 2ª demostración en LN
-- =====================

-- Por la siguiente cadena de implicaciones
--    a + b = a + c
--    ==> -a + (a + b) = -a + (a + c)     [sumando -a]
--    ==>  (-a + a) + b = (-a + a) + c    [por la asociativa]
--    ==>  0 + b = 0 + b                  [suma con opuesto]
--    ==>  b = c                          [suma con cero]

-- 3ª demostración en LN
-- =====================

-- Por la siguiente cadena de igualdades
--    b = -a + (a + b)
--      = -a + (a + c)   [por la hipótesis]
--      = c

-- Demostraciones con Lean4
-- ========================

import Mathlib.Algebra.Ring.Defs
import Mathlib.Tactic

variable {R : Type _} [Ring R]
variable {a b c : R}

-- 1ª demostración
example
  (h : a + b = a + c)
  : b = c :=
calc
  b = 0 + b        := by rw [zero_add]
  _ = (-a + a) + b := by rw [add_left_neg]
  _ = -a + (a + b) := by rw [add_assoc]
  _ = -a + (a + c) := by rw [h]
  _ = (-a + a) + c := by rw [←add_assoc]
  _ = 0 + c        := by rw [add_left_neg]
  _ = c            := by rw [zero_add]

-- 2ª demostración
example
  (h : a + b = a + c)
  : b = c :=
by
  have h1 : -a + (a + b) = -a + (a + c) :=
    congrArg (HAdd.hAdd (-a)) h
  clear h
  rw [← add_assoc] at h1
  rw [add_left_neg] at h1
  rw [zero_add] at h1
  rw [← add_assoc] at h1
  rw [add_left_neg] at h1
  rw [zero_add] at h1
  exact h1

-- 3ª demostración
example
  (h : a + b = a + c)
  : b = c :=
calc
  b = -a + (a + b) := by rw [neg_add_cancel_left a b]
  _ = -a + (a + c) := by rw [h]
  _ = c            := by rw [neg_add_cancel_left]

-- 4ª demostración
example
  (h : a + b = a + c)
  : b = c :=
by
  rw [← neg_add_cancel_left a b]
  rw [h]
  rw [neg_add_cancel_left]

-- 5ª demostración
example
  (h : a + b = a + c)
  : b = c :=
by
  rw [← neg_add_cancel_left a b, h, neg_add_cancel_left]

-- 6ª demostración
example
  (h : a + b = a + c)
  : b = c :=
add_left_cancel h

-- Lemas usados
-- ============

-- #check (add_assoc a b c : (a + b) + c = a + (b + c))
-- #check (add_left_cancel : a + b = a + c → b = c)
-- #check (add_left_neg a : -a + a = 0)
-- #check (neg_add_cancel_left a b : -a + (a + b) = b)
-- #check (zero_add a :  0 + a = a)
\end{verbatim}
Se puede interactuar con las pruebas anteriores en \href{https://lean.math.hhu.de/\#url=https://raw.githubusercontent.com/jaalonso/Calculemus2/main/src/Cancelativa\_izquierda.lean}{Lean 4 Web}.

\section{Si R es un anillo y a, b, c ∈ R tales que a+b=c+b, entonces a=c}
\label{sec:orgb8076a9}
\begin{verbatim}
-- ---------------------------------------------------------------------
-- Demostrar que si R es un anillo y a, b, c ∈ R tales que
--    a + b = c + b
-- entonces
--    a = c
-- ----------------------------------------------------------------------

-- Demostraciones en lenguaje natural (LN)
-- =======================================

-- 1ª demostración en LN
-- =====================

-- Por la siguiente cadena de igualdades
--    a = a + 0           [por suma con cero]
--      = a + (b + -b)    [por suma con opuesto]
--      = (a + b) + -b    [por asociativa]
--      = (c + b) + -b    [por hipótesis]
--      = c + (b + -b)    [por asociativa]
--      = c + 0           [por suma con opuesto]
--      = c               [por suma con cero]

-- 2ª demostración en LN
-- =====================

-- Por la siguiente cadena de igualdades
--    a = (a + b) + -b
--      = (c + b) + -b    [por hipótesis]
--      = c

-- Demostraciones con Lean4
-- ========================

import Mathlib.Algebra.Ring.Defs
import Mathlib.Tactic

variable {R : Type _} [Ring R]
variable {a b c : R}

-- 1ª demostración con Lean4
-- =========================

example
  (h : a + b = c + b)
  : a = c :=
calc
  a = a + 0        := by rw [add_zero]
  _ = a + (b + -b) := by rw [add_right_neg]
  _ = (a + b) + -b := by rw [add_assoc]
  _ = (c + b) + -b := by rw [h]
  _ = c + (b + -b) := by rw [← add_assoc]
  _ = c + 0        := by rw [← add_right_neg]
  _ = c            := by rw [add_zero]

-- 2ª demostración con Lean4
-- =========================

example
  (h : a + b = c + b)
  : a = c :=
calc
  a = (a + b) + -b := (add_neg_cancel_right a b).symm
  _ = (c + b) + -b := by rw [h]
  _ = c            := add_neg_cancel_right c b

-- 3ª demostración con Lean4
-- =========================

example
  (h : a + b = c + b)
  : a = c :=
by
  rw [← add_neg_cancel_right a b]
  rw [h]
  rw [add_neg_cancel_right]

-- 4ª demostración con Lean4
-- =========================

example
  (h : a + b = c + b)
  : a = c :=
by
  rw [← add_neg_cancel_right a b, h, add_neg_cancel_right]

-- 5ª demostración con Lean4
-- =========================

example
  (h : a + b = c + b)
  : a = c :=
add_right_cancel h

-- Lemas usados
-- ============

-- #check (add_assoc a b c : (a + b) + c = a + (b + c))
-- #check (add_neg_cancel_right a b : (a + b) + -b = a)
-- #check (add_right_cancel : a + b = c + b → a = c)
-- #check (add_right_neg a : a + -a = 0)
-- #check (add_zero a :  a + 0 = a)
\end{verbatim}
Se puede interactuar con las pruebas anteriores en \href{https://lean.math.hhu.de/\#url=https://raw.githubusercontent.com/jaalonso/Calculemus2/main/src/Cancelativa\_derecha.lean}{Lean 4 Web}.

\section{Si R es un anillo y a ∈ R, entonces a.0 = 0}
\label{sec:org4666c00}
\begin{verbatim}
-- ---------------------------------------------------------------------
-- Demostrar que si R es un anillo y a ∈ R, entonces
--    a * 0 = 0
-- ----------------------------------------------------------------------

-- Demostración en lenguaje natural
-- ================================

-- Basta aplicar la propiedad cancelativa a
--    a.0 + a.0 = a.0 + 0
-- que se demuestra mediante la siguiente cadena de igualdades
--    a.0 + a.0 = a.(0 + 0)    [por la distributiva]
--              = a.0          [por suma con cero]
--              = a.0 + 0      [por suma con cero]

-- Demostraciones con Lean4
-- ========================

import Mathlib.Algebra.Ring.Defs
import Mathlib.Tactic

variable {R : Type _} [Ring R]
variable (a : R)

-- 1ª demostración
-- ===============

example : a * 0 = 0 :=
by
  have h : a * 0 + a * 0 = a * 0 + 0 :=
    calc a * 0 + a * 0 = a * (0 + 0) := by rw [mul_add a 0 0]
                     _ = a * 0       := by rw [add_zero 0]
                     _ = a * 0 + 0   := by rw [add_zero (a * 0)]
  rw [add_left_cancel h]

-- 2ª demostración
-- ===============

example : a * 0 = 0 :=
by
  have h : a * 0 + a * 0 = a * 0 + 0 :=
    calc a * 0 + a * 0 = a * (0 + 0) := by rw [← mul_add]
                     _ = a * 0       := by rw [add_zero]
                     _ = a * 0 + 0   := by rw [add_zero]
  rw [add_left_cancel h]

-- 3ª demostración
-- ===============

example : a * 0 = 0 :=
by
  have h : a * 0 + a * 0 = a * 0 + 0 :=
    by rw [← mul_add, add_zero, add_zero]
  rw [add_left_cancel h]

-- 4ª demostración
-- ===============

example : a * 0 = 0 :=
by
  have : a * 0 + a * 0 = a * 0 + 0 :=
    calc a * 0 + a * 0 = a * (0 + 0) := by simp
                     _ = a * 0       := by simp
                     _ = a * 0 + 0   := by simp
  simp

-- 5ª demostración
-- ===============

example : a * 0 = 0 :=
  mul_zero a

-- 6ª demostración
-- ===============

example : a * 0 = 0 :=
by simp

-- Lemas usados
-- ============

-- variable (b c : R)
-- #check (add_left_cancel : a + b = a + c → b = c)
-- #check (add_zero a :  a + 0 = a)
-- #check (mul_add a b c :  a * (b + c) = a * b + a * c)
-- #check (mul_zero a : a * 0 = 0)
\end{verbatim}
Se puede interactuar con las pruebas anteriores en \href{https://lean.math.hhu.de/\#url=https://raw.githubusercontent.com/jaalonso/Calculemus2/main/src/Multiplicacion\_por\_cero.lean}{Lean 4 Web}.

\section{Si R es un anillo y a ∈ R, entonces 0.a = 0}
\label{sec:org6bfec67}
\begin{verbatim}
-- ---------------------------------------------------------------------
-- Demostrar que si R es un anillo y a ∈ R, entonces
--    0 * a = 0
-- ----------------------------------------------------------------------

-- Demostración en lenguaje natural
-- ================================

-- Basta aplicar la propiedad cancelativa a
--    0.a + 0.a = 0.a + 0
-- que se demuestra mediante la siguiente cadena de igualdades
--    0.a + 0.a = (0 + 0).a    [por la distributiva]
--              = 0.a          [por suma con cero]
--              = 0.a + 0      [por suma con cero]

-- Demostraciones con Lean4
-- ========================

import Mathlib.Algebra.Ring.Defs
import Mathlib.Tactic

variable {R : Type _} [Ring R]
variable (a : R)

-- 1ª demostración
example : 0 * a = 0 :=
by
  have h : 0 * a + 0 * a = 0 * a + 0 :=
    calc 0 * a + 0 * a = (0 + 0) * a := by rw [add_mul]
                     _ = 0 * a       := by rw [add_zero]
                     _ = 0 * a + 0   := by rw [add_zero]
  rw [add_left_cancel h]

-- 2ª demostración
example : 0 * a = 0 :=
by
  have h : 0 * a + 0 * a = 0 * a + 0 :=
    by rw [←add_mul, add_zero, add_zero]
  rw [add_left_cancel h]

-- 3ª demostración
example : 0 * a = 0 :=
by
  have : 0 * a + 0 * a = 0 * a + 0 :=
    calc 0 * a + 0 * a = (0 + 0) * a := by simp
                     _ = 0 * a       := by simp
                     _ = 0 * a + 0   := by simp
  simp

-- 4ª demostración
example : 0 * a = 0 :=
by
  have : 0 * a + 0 * a = 0 * a + 0 := by simp
  simp

-- 5ª demostración
example : 0 * a = 0 :=
by simp

-- 6ª demostración
example : 0 * a = 0 :=
zero_mul a

-- Lemas usados
-- ============

-- variable (b c : R)
-- #check (add_mul a b c :  (a + b) * c = a * c + b * c)
-- #check (add_zero a :  a + 0 = a)
-- #check (zero_mul a : 0 * a = 0)
\end{verbatim}
Se puede interactuar con las pruebas anteriores en \href{https://lean.math.hhu.de/\#url=https://raw.githubusercontent.com/jaalonso/Calculemus2/main/src/Multiplicacion\_por\_cero\_izquierda.lean}{Lean 4 Web}.

\section{Si R es un anillo y a, b ∈ R tales que a+b=0, entonces -a=b}
\label{sec:orge5db5fe}
\begin{verbatim}
-- ---------------------------------------------------------------------
-- Demostrar que si es un anillo y a, b ∈ R tales que
--    a + b = 0
-- entonces
--    -a = b
-- ----------------------------------------------------------------------

-- Demostraciones en lenguaje natural (LN)
-- =======================================

-- 1ª demostración en LN
-- ---------------------

-- Por la siguiente cadena de igualdades
--    -a = -a + 0          [por suma cero]
--       = -a + (a + b)    [por hipótesis]
--       = b               [por cancelativa]

-- 2ª demostración en LN
-- ---------------------

-- Sumando -a a ambos lados de la hipótesis, se tiene
--    -a + (a + b) = -a + 0
-- El término de la izquierda se reduce a b (por la cancelativa) y el de
-- la derecha a -a (por la suma con cero). Por tanto, se tiene
--     b = -a
-- Por la simetría de la igualdad, se tiene
--     -a = b

-- Demostraciones con Lean 4
-- =========================

import Mathlib.Algebra.Ring.Defs
import Mathlib.Tactic

variable {R : Type _} [Ring R]
variable {a b : R}

-- 1ª demostración (basada en la 1º en LN)
example
  (h : a + b = 0)
  : -a = b :=
calc
  -a = -a + 0       := by rw [add_zero]
   _ = -a + (a + b) := by rw [h]
   _ = b            := by rw [neg_add_cancel_left]

-- 2ª demostración (basada en la 1º en LN)
example
  (h : a + b = 0)
  : -a = b :=
calc
  -a = -a + 0       := by simp
   _ = -a + (a + b) := by rw [h]
   _ = b            := by simp

-- 3ª demostración (basada en la 2º en LN)
example
  (h : a + b = 0)
  : -a = b :=
by
  have h1 : -a + (a + b) = -a + 0 := congrArg (HAdd.hAdd (-a)) h
  have h2 : -a + (a + b) = b := neg_add_cancel_left a b
  have h3 : -a + 0 = -a := add_zero (-a)
  rw [h2, h3] at h1
  exact h1.symm

-- 4ª demostración
example
  (h : a + b = 0)
  : -a = b :=
neg_eq_iff_add_eq_zero.mpr h

-- Lemas usados
-- ============

-- #check (add_zero a : a + 0 = a)
-- #check (neg_add_cancel_left a b : -a + (a + b) = b)
-- #check (neg_eq_iff_add_eq_zero : -a = b ↔ a + b = 0)
\end{verbatim}
Se puede interactuar con las pruebas anteriores en \href{https://lean.math.hhu.de/\#url=https://raw.githubusercontent.com/jaalonso/Calculemus2/main/src/Opuesto\_ig\_si\_suma\_ig\_cero.lean}{Lean 4 Web}.

\section{Si R es un anillo y a, b ∈ R tales que a+b=0, entonces a=-b}
\label{sec:org84706a1}
\begin{verbatim}
-- ---------------------------------------------------------------------
-- Demostrar que si R es un anillo y a, b ∈ R tales que
--    a + b = 0
-- entonces
--    a = -b
-- ----------------------------------------------------------------------

-- Demostraciones en lenguaje natural (LN)
-- =======================================

-- 1ª demostración en LN
-- ---------------------

-- Por la siguiente cadena de igualdades
--    a = (a + b) + -b    [por la concelativa]
--      = 0 + -b          [por la hipótesis]
--      = -b              [por la suma con cero]

-- 2ª demostración en LN
-- ---------------------

-- Sumando -a a ambos lados de la hipótesis, se tiene
--    (a + b) + -b = 0 + -b
-- El término de la izquierda se reduce a a (por la cancelativa) y el de
-- la derecha a -b (por la suma con cero). Por tanto, se tiene
--     a = -b

-- Demostraciones con Lean4
-- ========================

import Mathlib.Algebra.Ring.Defs
import Mathlib.Tactic

variable {R : Type _} [Ring R]
variable {a b : R}

-- 1ª demostración (basada en la 1ª en LN)
example
  (h : a + b = 0)
  : a = -b :=
calc
  a = (a + b) + -b := by rw [add_neg_cancel_right]
  _ = 0 + -b       := by rw [h]
  _ = -b           := by rw [zero_add]

-- 2ª demostración (basada en la 1ª en LN)
example
  (h : a + b = 0)
  : a = -b :=
calc
  a = (a + b) + -b := by simp
  _ = 0 + -b       := by rw [h]
  _ = -b           := by simp

-- 3ª demostración (basada en la 1ª en LN)
example
  (h : a + b = 0)
  : a = -b :=
by
  have h1 : (a + b) + -b = 0 + -b := by rw [h]
  have h2 : (a + b) + -b = a := add_neg_cancel_right a b
  have h3 : 0 + -b = -b := zero_add (-b)
  rwa [h2, h3] at h1

-- 4ª demostración
example
  (h : a + b = 0)
  : a = -b :=
add_eq_zero_iff_eq_neg.mp h

-- Lemas usados
-- ============

-- #check (add_eq_zero_iff_eq_neg : a + b = 0 ↔ a = -b)
-- #check (add_neg_cancel_right a b : (a + b) + -b = a)
-- #check (zero_add a : 0 + a = a)
\end{verbatim}
Se puede interactuar con las pruebas anteriores en \href{https://lean.math.hhu.de/\#url=https://raw.githubusercontent.com/jaalonso/Calculemus2/main/src/Ig\_opuesto\_si\_suma\_ig\_cero.lean}{Lean 4 Web}.

\section{Si R es un anillo, entonces -0 = 0}
\label{sec:org59d278b}
\begin{verbatim}
-- ---------------------------------------------------------------------
-- Demostrar que si R es un anillo, entonces
--    -0 = 0
-- ----------------------------------------------------------------------

-- Demostraciones en lenguaje natural (LN)
-- =======================================

-- 1ª demostración en LN
-- =====================

-- Por la suma con cero se tiene
--    0 + 0 = 0
-- Aplicándole la propiedad
--    ∀ a b ∈ R, a + b = 0 → -a = b
-- se obtiene
--    -0 = 0

-- 2ª demostración en LN
-- =====================

-- Puesto que
--    ∀ a b ∈ R, a + b = 0 → -a = b
-- basta demostrar que
--    0 + 0 = 0
-- que es cierta por la suma con cero.

-- Demostraciones con Lean4
-- ========================

import Mathlib.Algebra.Ring.Defs
import Mathlib.Tactic

variable {R : Type _} [Ring R]

-- 1ª demostración (basada en la 1ª en LN)
example : (-0 : R) = 0 :=
by
  have h1 : (0 : R) + 0 = 0 := add_zero 0
  show (-0 : R) = 0
  exact neg_eq_of_add_eq_zero_left h1

-- 2ª demostración (basada en la 2ª en LN)
example : (-0 : R) = 0 :=
by
  apply neg_eq_of_add_eq_zero_left
  rw [add_zero]

-- 3ª demostración
example : (-0 : R) = 0 :=
  neg_zero

-- 4ª demostración
example : (-0 : R) = 0 :=
by simp

-- Lemas usados
-- ============

-- variable (a b : R)
-- #check (add_zero a : a + 0 = a)
-- #check (neg_eq_of_add_eq_zero_left : a + b = 0 → -b = a)
-- #check (neg_zero : -0 = 0)
\end{verbatim}
Se puede interactuar con las pruebas anteriores en \href{https://lean.math.hhu.de/\#url=https://raw.githubusercontent.com/jaalonso/Calculemus2/main/src/Opuesto\_del\_cero.lean}{Lean 4 Web}.

\section{Si R es un anillo y a ∈ R, entonces -(-a) = a}
\label{sec:org0d31011}
\begin{verbatim}
-- ---------------------------------------------------------------------
-- Demostrar que si R es un anillo y a ∈ R, entonces
--     -(-a) = a
-- ----------------------------------------------------------------------

-- Demostración en lenguaje natural
-- ================================

-- Es consecuencia de las siguiente propiedades demostradas en
-- ejercicios anteriores:
--    ∀ a b ∈ R, a + b = 0 → -a = b
--    ∀ a ∈ R, -a + a = 0

-- Demostraciones con Lean4
-- ========================

import Mathlib.Algebra.Ring.Defs
import Mathlib.Tactic

variable {R : Type _} [Ring R]
variable {a : R}

-- 1ª demostración
example : -(-a) = a :=
by
  have h1 : -a + a = 0 := add_left_neg a
  show -(-a) = a
  exact neg_eq_of_add_eq_zero_right h1

-- 2ª demostración
example : -(-a) = a :=
by
  apply neg_eq_of_add_eq_zero_right
  rw [add_left_neg]

-- 3ª demostración
example : -(-a) = a :=
neg_neg a

-- 4ª demostración
example : -(-a) = a :=
by simp

-- Lemas usados
-- ============

-- variable (b : R)
-- #check (add_left_neg a : -a + a = 0)
-- #check (neg_eq_of_add_eq_zero_right : a + b = 0 → -a = b)
-- #check (neg_neg a : -(-a) = a)
\end{verbatim}
Se puede interactuar con las pruebas anteriores en \href{https://lean.math.hhu.de/\#url=https://raw.githubusercontent.com/jaalonso/Calculemus2/main/src/Opuesto\_del\_opuesto.lean}{Lean 4 Web}.

\section{Si R es un anillo y a, b ∈ R, entonces a - b = a + -b}
\label{sec:org9c76d7c}
\begin{verbatim}
-- ---------------------------------------------------------------------
-- Demostrar que si R es un anillo y a, b ∈ R, entonces
--    a - b = a + -b
-- ----------------------------------------------------------------------

-- Demostración en lenguaje natural
-- ================================

-- Por la definición de la resta.

-- Demostración en Lean4
-- =====================

import Mathlib.Algebra.Ring.Defs

variable {R : Type _} [Ring R]
variable (a b : R)

example : a - b = a + -b :=
-- by exact?
sub_eq_add_neg a b

-- Lemas usados
-- ============

-- #check (sub_eq_add_neg a b : a - b = a + -b)
\end{verbatim}
Se puede interactuar con las pruebas anteriores en \href{https://lean.math.hhu.de/\#url=https://raw.githubusercontent.com/jaalonso/Calculemus2/main/src/Resta\_igual\_suma\_opuesto.lean}{Lean 4 Web}.

\section{Si R es un anillo y a ∈ R, entonces a - a = 0}
\label{sec:org4a9e6fc}
\begin{verbatim}
-- ---------------------------------------------------------------------
-- Demostrar que si R es un anillo y a ∈ R, entonces
--     a - a = 0
-- ----------------------------------------------------------------------

-- Demostración en lenguaje natural
-- ================================

-- Por la siguiente cadena de igualdades:
--    a - a = a + -a    [por definición de resta]
--          = 0         [por suma con opuesto]

-- Demostraciones con Lean4
-- ========================

import Mathlib.Algebra.Ring.Defs
variable {R : Type _} [Ring R]
variable (a : R)

-- 1ª demostración
example : a - a = 0 :=
calc
  a - a = a + -a := by rw [sub_eq_add_neg a a]
      _ = 0      := by rw [add_right_neg]

-- 2ª demostración
example : a - a = 0 :=
sub_self a

-- 3ª demostración
example : a - a = 0 :=
by simp

-- Lemas usados
-- ============

-- #check (add_right_neg a : a + -a = 0)
-- #check (sub_eq_add_neg a b : a - b = a + -b)
-- #check (sub_self a : a - a = 0)
\end{verbatim}
Se puede interactuar con las pruebas anteriores en \href{https://lean.math.hhu.de/\#url=https://raw.githubusercontent.com/jaalonso/Calculemus2/main/src/Resta\_consigo\_mismo.lean}{Lean 4 Web}.

\section{En los anillos, 1 + 1 = 2}
\label{sec:orgb55e8f6}
\begin{verbatim}
-- ---------------------------------------------------------------------
-- Demostrar que en los anillos,
--    1 + 1 = 2
-- ----------------------------------------------------------------------

-- Demostración en lenguaje natural
-- ================================

-- Por cálculo.

-- Demostración con Lean4
-- ======================

import Mathlib.Algebra.Ring.Defs
import Mathlib.Tactic
variable {R : Type _} [Ring R]

-- Demostraciones con Lean4
-- ========================

-- 1ª demostración
example : 1 + 1 = (2 : R) :=
by norm_num

-- 2ª demostración
example : 1 + 1 = (2 : R) :=
one_add_one_eq_two

-- Lemas usados
-- ============

-- #check (one_add_one_eq_two : 1 + 1 = 2)
\end{verbatim}
Se puede interactuar con las pruebas anteriores en \href{https://lean.math.hhu.de/\#url=https://raw.githubusercontent.com/jaalonso/Calculemus2/main/src/Uno\_mas\_uno\_es\_dos.lean}{Lean 4 Web}.

\section{Si R es un anillo y a ∈ R, entonces 2a = a+a}
\label{sec:org0693eca}
\begin{verbatim}
-- ---------------------------------------------------------------------
-- Demostrar que si R es un anillo y a ∈ R, entonces
--    2 * a = a + a
-- ----------------------------------------------------------------------

-- Demostración en lenguaje natural
-- ================================

-- Por la siguiente cadena de igualdades
--    2·a = (1 + 1)·a    [por la definición de 2]
--        = 1·a + 1·a    [por la distributiva]
--        = a + a        [por producto con uno]

-- Demostraciones con Lean4
-- ========================

import Mathlib.Algebra.Ring.Defs

variable {R : Type _} [Ring R]
variable (a : R)

-- 1ª demostración
example : 2 * a = a + a :=
calc
  2 * a = (1 + 1) * a   := by rw [one_add_one_eq_two]
      _ = 1 * a + 1 * a := by rw [add_mul]
      _ = a + a         := by rw [one_mul]

-- 2ª demostración
example : 2 * a = a + a :=
by exact two_mul a

-- Lemas usados
-- ============

-- variable (b c : R)
-- #check (add_mul a b c : (a + b) * c = a * c + b * c)
-- #check (one_add_one_eq_two : (1 : R) + 1 = 2)
-- #check (one_mul a : 1 * a = a)
-- #check (two_mul a : 2 * a = a + a)
\end{verbatim}
Se puede interactuar con las pruebas anteriores en \href{https://lean.math.hhu.de/\#url=https://raw.githubusercontent.com/jaalonso/Calculemus2/main/src/Producto\_por\_dos.lean}{Lean 4 Web}.

\chapter{Propiedades elementales de los grupos}
\label{sec:org18e2dd7}

\section{Si G es un grupo y a ∈ G, entonces aa⁻¹ = 1}
\label{sec:org63dc332}
\begin{verbatim}
-- ---------------------------------------------------------------------
-- En Lean4, se declara que G es un grupo mediante la expresión
--    variable {G : Type _} [Group G]
--
-- Como consecuencia, se tiene los siguientes axiomas
--    mul_assoc :    ∀ a b c : G, a * b * c = a * (b * c)
--    one_mul :      ∀ a : G, 1 * a = a
--    mul_left_inv : ∀ a : G, a⁻¹ * a = 1
--
-- Demostrar que si G es un grupo y a ∈ G, entonces
--    a * a⁻¹ = 1
-- ----------------------------------------------------------------------

-- Demostración en lenguaje natural
-- ================================

-- Por la siguiente cadena de igualdades
--    a·a⁻¹ = 1·(a·a⁻¹)                 [por producto con uno]
--          = (1·a)·a⁻¹                 [por asociativa]
--          = (((a⁻¹)⁻¹·a⁻¹) ·a)·a⁻¹    [por producto con inverso]
--          = ((a⁻¹)⁻¹·(a⁻¹ ·a))·a⁻¹    [por asociativa]
--          = ((a⁻¹)⁻¹·1)·a⁻¹           [por producto con inverso]
--          = (a⁻¹)⁻¹·(1·a⁻¹)           [por asociativa]
--          = (a⁻¹)⁻¹·a⁻¹               [por producto con uno]
--          = 1                         [por producto con inverso]

-- Demostraciones con Lean4
-- ========================

import Mathlib.Algebra.Group.Defs

variable {G : Type _} [Group G]
variable (a b : G)

-- 1ª demostración
example : a * a⁻¹ = 1 :=
calc
  a * a⁻¹ = 1 * (a * a⁻¹)                := by rw [one_mul]
        _ = (1 * a) * a⁻¹                := by rw [mul_assoc]
        _ = (((a⁻¹)⁻¹ * a⁻¹)  * a) * a⁻¹ := by rw [mul_left_inv]
        _ = ((a⁻¹)⁻¹ * (a⁻¹  * a)) * a⁻¹ := by rw [← mul_assoc]
        _ = ((a⁻¹)⁻¹ * 1) * a⁻¹          := by rw [mul_left_inv]
        _ = (a⁻¹)⁻¹ * (1 * a⁻¹)          := by rw [mul_assoc]
        _ = (a⁻¹)⁻¹ * a⁻¹                := by rw [one_mul]
        _ = 1                            := by rw [mul_left_inv]

-- 2ª demostración
example : a * a⁻¹ = 1 :=
calc
  a * a⁻¹ = 1 * (a * a⁻¹)                := by simp
        _ = (1 * a) * a⁻¹                := by simp
        _ = (((a⁻¹)⁻¹ * a⁻¹)  * a) * a⁻¹ := by simp
        _ = ((a⁻¹)⁻¹ * (a⁻¹  * a)) * a⁻¹ := by simp
        _ = ((a⁻¹)⁻¹ * 1) * a⁻¹          := by simp
        _ = (a⁻¹)⁻¹ * (1 * a⁻¹)          := by simp
        _ = (a⁻¹)⁻¹ * a⁻¹                := by simp
        _ = 1                            := by simp

-- 3ª demostración
example : a * a⁻¹ = 1 :=
by simp

-- 4ª demostración
example : a * a⁻¹ = 1 :=
by exact mul_inv_self a

-- Lemas usados
-- ============

-- variable (c : G)
-- #check (mul_assoc a b c : (a * b) * c = a * (b * c))
-- #check (mul_inv_self a : a * a⁻¹ = 1)
-- #check (mul_left_inv a : a⁻¹  * a = 1)
-- #check (one_mul a : 1 * a = a)
\end{verbatim}
Se puede interactuar con las pruebas anteriores en \href{https://lean.math.hhu.de/\#url=https://raw.githubusercontent.com/jaalonso/Calculemus2/main/src/Producto\_por\_dos.lean}{Lean 4 Web}.

\section{Si G es un grupo y a ∈ G, entonces a·1 = a}
\label{sec:org79f7eb4}
\begin{verbatim}
-- ---------------------------------------------------------------------
-- Demostrar que si G es un grupo y a ∈ G, entonces
--    a * 1 = a
-- ----------------------------------------------------------------------

-- Demostración en lenguaje natural
-- ================================

-- Se tiene por la siguiente cadena de igualdades
--    a·1 = a·(a⁻¹·a)    [por producto con inverso]
--        = (a·a⁻¹)·a    [por asociativa]
--        = 1·a          [por producto con inverso]
--        = a            [por producto con uno]

-- Demostraciones con Lean4
-- ========================

import Mathlib.Algebra.Group.Defs

variable {G : Type _} [Group G]
variable (a b : G)

-- 1ª demostración
example : a * 1 = a :=
calc
  a * 1 = a * (a⁻¹ * a) := by rw [mul_left_inv]
      _ = (a * a⁻¹) * a := by rw [mul_assoc]
      _ = 1 * a         := by rw [mul_right_inv]
      _ = a             := by rw [one_mul]

-- 2ª demostración
example : a * 1 = a :=
calc
  a * 1 = a * (a⁻¹ * a) := by simp
      _ = (a * a⁻¹) * a := by simp
      _ = 1 * a         := by simp
      _ = a             := by simp

-- 3ª demostración
example : a * 1 = a :=
by simp

-- 4ª demostración
example : a * 1 = a :=
by exact mul_one a

-- Lemas usados
-- ============

-- variable (c : G)
-- #check (mul_left_inv a : a⁻¹  * a = 1)
-- #check (mul_assoc a b c : (a * b) * c = a * (b * c))
-- #check (mul_right_inv a : a * a⁻¹ = 1)
-- #check (one_mul a : 1 * a = a)
-- #check (mul_one a : a * 1 = a)
\end{verbatim}
Se puede interactuar con las pruebas anteriores en \href{https://lean.math.hhu.de/\#url=https://raw.githubusercontent.com/jaalonso/Calculemus2/main/src/Producto\_por\_uno.lean}{Lean 4 Web}.

\section{Si G es un grupo y a, b ∈ G tales que ab = 1 entonces a⁻¹ = b}
\label{sec:org9b82a06}
\begin{verbatim}
-- ---------------------------------------------------------------------
-- Demostrar que si G es un grupo y a, b ∈ G, tales que
--    a * b = 1
-- entonces
--    a⁻¹ = b
-- ----------------------------------------------------------------------

-- Demostración en lenguaje natural
-- ================================

-- Se tiene a partir de la siguiente cadena de igualdades
--    a⁻¹ = a⁻¹ * 1           [por producto por uno]
--        = a⁻¹ * (a * b)     [por hipótesis]
--        = (a⁻¹ * a) * b     [por asociativa]
--        = 1 * b             [por producto con inverso]
--        = b                 [por producto por uno]

-- Demostraciones con Lean4
-- ========================

import Mathlib.Algebra.Group.Defs

variable {G : Type _} [Group G]
variable (a b : G)

-- 1º demostración
example
  (h : a * b = 1)
  : a⁻¹ = b :=
calc
  a⁻¹ = a⁻¹ * 1       := by rw [mul_one]
    _ = a⁻¹ * (a * b) := by rw [h]
    _ = (a⁻¹ * a) * b := by rw [mul_assoc]
    _ = 1 * b         := by rw [mul_left_inv]
    _ = b             := by rw [one_mul]

-- 2º demostración
example
  (h : a * b = 1)
  : a⁻¹ = b :=
calc
  a⁻¹ = a⁻¹ * 1       := by simp
    _ = a⁻¹ * (a * b) := by simp [h]
    _ = (a⁻¹ * a) * b := by simp
    _ = 1 * b         := by simp
    _ = b             := by simp

-- 3º demostración
example
  (h : a * b = 1)
  : a⁻¹ = b :=
calc
  a⁻¹ =  a⁻¹ * (a * b) := by simp [h]
    _ =  b             := by simp

-- 4º demostración
example
  (h : a * b = 1)
  : a⁻¹ = b :=
by exact inv_eq_of_mul_eq_one_right h

-- Lemas usados
-- ============

-- variable (c : G)
-- #check (mul_assoc a b c : (a * b) * c = a * (b * c))
-- #check (mul_left_inv a : a⁻¹  * a = 1)
-- #check (mul_one a : a * 1 = a)
-- #check (one_mul a : 1 * a = a)
-- #check (inv_eq_of_mul_eq_one_right : a * b = 1 → a⁻¹ = b)
\end{verbatim}
Se puede interactuar con las pruebas anteriores en \href{https://lean.math.hhu.de/\#url=https://raw.githubusercontent.com/jaalonso/Calculemus2/main/src/CS\_de\_inverso.lean}{Lean 4 Web}.

\section{Si G es un grupo y a, b ∈ G, entonces (ab)⁻¹ = b⁻¹a⁻¹}
\label{sec:org15f4f52}
\begin{verbatim}
-- ---------------------------------------------------------------------
-- Demostrar que si G es un grupo y a, b ∈ G, entonces
--    (a * b)⁻¹ = b⁻¹ * a⁻¹
-- ----------------------------------------------------------------------

-- Demostración en lenguaje natural
-- ================================

-- Teniendo en cuenta la propiedad
--    ∀ a b ∈ R, ab = 1 → a⁻¹ = b,
-- basta demostrar que
--    (a·b)·(b⁻¹·a⁻¹) = 1.
-- La identidad anterior se demuestra mediante la siguiente cadena de
-- igualdades
--    (a·b)·(b⁻¹·a⁻¹) =  a·(b·(b⁻¹·a⁻¹))   [por la asociativa]
--                    =  a·((b·b⁻¹)·a⁻¹)   [por la asociativa]
--                    =  a·(1·a⁻¹)         [por producto con inverso]
--                    =  a·a⁻¹             [por producto con uno]
--                    =  1                 [por producto con inverso]

-- Demostraciones con Lean4
-- ========================

import Mathlib.Algebra.Group.Defs

variable {G : Type _} [Group G]
variable (a b : G)

lemma aux : (a * b) * (b⁻¹ * a⁻¹) = 1 :=
calc
  (a * b) * (b⁻¹ * a⁻¹)
    = a * (b * (b⁻¹ * a⁻¹)) := by rw [mul_assoc]
  _ = a * ((b * b⁻¹) * a⁻¹) := by rw [mul_assoc]
  _ = a * (1 * a⁻¹)         := by rw [mul_right_inv]
  _ = a * a⁻¹               := by rw [one_mul]
  _ = 1                     := by rw [mul_right_inv]

-- 1ª demostración
example : (a * b)⁻¹ = b⁻¹ * a⁻¹ :=
by
  have h1 : (a * b) * (b⁻¹ * a⁻¹) = 1 :=
    aux a b
  show (a * b)⁻¹ = b⁻¹ * a⁻¹
  exact inv_eq_of_mul_eq_one_right h1

-- 3ª demostración
example : (a * b)⁻¹ = b⁻¹ * a⁻¹ :=
by
  have h1 : (a * b) * (b⁻¹ * a⁻¹) = 1 :=
    aux a b
  show (a * b)⁻¹ = b⁻¹ * a⁻¹
  simp [h1]

-- 4ª demostración
example : (a * b)⁻¹ = b⁻¹ * a⁻¹ :=
by
  have h1 : (a * b) * (b⁻¹ * a⁻¹) = 1 :=
    aux a b
  simp [h1]

-- 5ª demostración
example : (a * b)⁻¹ = b⁻¹ * a⁻¹ :=
by
  apply inv_eq_of_mul_eq_one_right
  rw [aux]

-- 6ª demostración
example : (a * b)⁻¹ = b⁻¹ * a⁻¹ :=
by exact mul_inv_rev a b

-- 7ª demostración
example : (a * b)⁻¹ = b⁻¹ * a⁻¹ :=
by simp

-- Lemas usados
-- ============

-- variable (c : G)
-- #check (inv_eq_of_mul_eq_one_right : a * b = 1 → a⁻¹ = b)
-- #check (mul_assoc a b c : (a * b) * c = a * (b * c))
-- #check (mul_inv_rev a b : (a * b)⁻¹ = b⁻¹ * a⁻¹)
-- #check (mul_right_inv a : a * a⁻¹ = 1)
-- #check (one_mul a : 1 * a = a)
\end{verbatim}
Se puede interactuar con las pruebas anteriores en \href{https://lean.math.hhu.de/\#url=https://raw.githubusercontent.com/jaalonso/Calculemus2/main/src/Inverso\_del\_producto.lean}{Lean 4 Web}.

\chapter{Propiedades de orden en los números reales}
\label{sec:org5a97cd0}

\section{En ℝ, si a ≤ b, b < c, c ≤ d y d < e, entonces a < e}
\label{sec:org7998eb8}
\begin{verbatim}
-- ---------------------------------------------------------------------
-- Demostrar que si a, b, c, d y e son números reales tales que
--    a ≤ b,
--    b < c,
--    c ≤ d y
--    d < e,
-- entonces
--    a < e.
-- ----------------------------------------------------------------------

-- Demostraciones en lenguaje natural (LN)
-- =======================================

-- 1ª demostración en LN
-- =====================

-- Por la siguiente cadena de desigualdades
--    a ≤ b    [por h1]
--      < c    [por h2]
--      ≤ d    [por h3]
--      < e    [por h4]

-- 2ª demostración en LN
-- =====================

-- A partir de las hipótesis 1 (a ≤ b) y 2 (b < c) se tiene
--    a < c
-- que, junto la hipótesis 3 (c ≤ d) da
--    a < d
-- que, junto la hipótesis 4 (d < e) da
--    a < e.

-- 3ª demostración en LN
-- =====================

-- Para demostrar a < e, por la hipótesis 1 (a ≤ b) se reduce a probar
--    b < e
-- que, por la hipótesis 2 (b < c), se reduce a
--    c < e
-- que, por la hipótesis 3 (c ≤ d), se reduce a
--    d < e
-- que es cierto, por la hipótesis 4.

-- Demostraciones con Lean4
-- ========================

import Mathlib.Data.Real.Basic

variable (a b c d e : ℝ)

-- 1ª demostración
-- ===============

example
  (h1 : a ≤ b)
  (h2 : b < c)
  (h3 : c ≤ d)
  (h4 : d < e) :
  a < e :=
calc
  a ≤ b := h1
  _ < c := h2
  _ ≤ d := h3
  _ < e := h4

-- 2ª demostración
-- ===============

example
  (h1 : a ≤ b)
  (h2 : b < c)
  (h3 : c ≤ d)
  (h4 : d < e) :
  a < e :=
by
  have h5 : a < c := lt_of_le_of_lt h1 h2
  have h6 : a < d := lt_of_lt_of_le h5 h3
  show a < e
  exact lt_trans h6 h4

-- 3ª demostración
-- ===============

example
  (h1 : a ≤ b)
  (h2 : b < c)
  (h3 : c ≤ d)
  (h4 : d < e) :
  a < e :=
by
  apply lt_of_le_of_lt h1
  apply lt_trans h2
  apply lt_of_le_of_lt h3
  exact h4

-- El desarrollo de la prueba es
--
--    a b c d e : ℝ,
--    h1 : a ≤ b,
--    h2 : b < c,
--    h3 : c ≤ d,
--    h4 : d < e
--    ⊢ a < e
-- apply lt_of_le_of_lt h1,
--    ⊢ b < e
-- apply lt_trans h2,
--    ⊢ c < e
-- apply lt_of_le_of_lt h3,
--    ⊢ d < e
-- exact h4,
--    no goals

-- 4ª demostración
-- ===============

example
  (h1 : a ≤ b)
  (h2 : b < c)
  (h3 : c ≤ d)
  (h4 : d < e) :
  a < e :=
by linarith

-- Lemas usados
-- ============

-- #check (lt_of_le_of_lt : a ≤ b → b < c → a < c)
-- #check (lt_of_lt_of_le : a < b → b ≤ c → a < c)
-- #check (lt_trans : a < b → b < c → a < c)
\end{verbatim}
Se puede interactuar con las pruebas anteriores en \href{https://lean.math.hhu.de/\#url=https://raw.githubusercontent.com/jaalonso/Calculemus2/main/src/Cadena\_de\_desigualdades.lean}{Lean 4 Web}.

\section{En ℝ, si 2a ≤ 3b, 1 ≤ a y d = 2, entonces d + a ≤ 5b}
\label{sec:org955e90e}
\begin{verbatim}
-- ---------------------------------------------------------------------
-- Demostrar que si a, b y c son números reales tales que
--    2 * a ≤ 3 * b
--    1 ≤ a
--    c = 2
-- entonces
--    c + a ≤ 5 * b
-- ----------------------------------------------------------------------

-- Demostración en lenguaje natural
-- ================================

-- Por la siguiente cadena de desigualdades
--    c + a = 2 + a      [por la hipótesis 3 (c = 2)]
--          ≤ 2·a + a    [por la hipótesis 2 (1 ≤ a)]
--          = 3·a
--          ≤ 9/2·b      [por la hipótesis 1 (2·a ≤ 3·b)]
--          ≤ 5·b

-- Demostraciones con Lean4
-- ========================

import Mathlib.Data.Real.Basic

variable (a b c : ℝ)

-- 1ª demostración
example
  (h1 : 2 * a ≤ 3 * b)
  (h2 : 1 ≤ a)
  (h3 : c = 2)
  : c + a ≤ 5 * b :=
calc
  c + a = 2 + a     := by rw [h3]
      _ ≤ 2 * a + a := by linarith only [h2]
      _ = 3 * a     := by linarith only []
      _ ≤ 9/2 * b   := by linarith only [h1]
      _ ≤ 5 * b     := by linarith

-- 2ª demostración
example
  (h1 : 2 * a ≤ 3 * b)
  (h2 : 1 ≤ a)
  (h3 : c = 2)
  : c + a ≤ 5 * b :=
by linarith
\end{verbatim}
Se puede interactuar con las pruebas anteriores en \href{https://lean.math.hhu.de/\#url=https://raw.githubusercontent.com/jaalonso/Calculemus2/main/src/Inecuaciones.lean}{Lean 4 Web}.

\section{En ℝ, si 1 ≤ a y b ≤ d, entonces 2 + a + eᵇ ≤ 3a + eᵈ}
\label{sec:orgbd9ce09}
\begin{verbatim}
-- ---------------------------------------------------------------------
-- Sean a, b, y d números reales. Demostrar  que si
--    1 ≤ a
--    b ≤ d
-- entonces
--    2 + a + exp b ≤ 3 * a + exp d
-- ----------------------------------------------------------------------

-- Demostración en lenguaje natural
-- ================================

-- De la primera hipótesis (1 ≤ a), multiplicando por 2, se obtiene
--    2 ≤ 2a
-- y, sumando a ambos lados, se tiene
--    2 + a ≤ 3a             (1)
-- De la hipótesis 2 (b ≤ d) y de la monotonía de la función exponencial
-- se tiene
--    e^b ≤ e^d              (2)
-- Finalmente, de (1) y (2) se tiene
--    2 + a + e^b ≤ 3a + e^d

-- Demostraciones con Lean4
-- ========================

import Mathlib.Analysis.SpecialFunctions.Log.Basic

open Real

variable (a b d : ℝ)

-- 1ª demostración
example
  (h1 : 1 ≤ a)
  (h2 : b ≤ d)
  : 2 + a + exp b ≤ 3 * a + exp d :=
by
  have h3 : 2 + a ≤ 3 * a := calc
    2 + a = 2 * 1 + a := by linarith only []
        _ ≤ 2 * a + a := by linarith only [h1]
        _ ≤ 3 * a     := by linarith only []
  have h4 : exp b ≤ exp d := by
    linarith only [exp_le_exp.mpr h2]
  show 2 + a + exp b ≤ 3 * a + exp d
  exact add_le_add h3 h4

-- 2ª demostración
example
  (h1 : 1 ≤ a)
  (h2 : b ≤ d)
  : 2 + a + exp b ≤ 3 * a + exp d :=
calc
  2 + a + exp b
    ≤ 3 * a + exp b := by linarith only [h1]
  _ ≤ 3 * a + exp d := by linarith only [exp_le_exp.mpr h2]

-- 3ª demostración
example
  (h1 : 1 ≤ a)
  (h2 : b ≤ d)
  : 2 + a + exp b ≤ 3 * a + exp d :=
by linarith [exp_le_exp.mpr h2]

-- Lemas usados
-- ============

-- variable (c : ℝ)
-- #check (add_le_add : a ≤ b → c ≤ d → a + c ≤ b + d)
-- #check (exp_le_exp : exp a ≤ exp b ↔ a ≤ b)
\end{verbatim}
Se puede interactuar con las pruebas anteriores en \href{https://lean.math.hhu.de/\#url=https://raw.githubusercontent.com/jaalonso/Calculemus2/main/src/Inecuaciones\_con\_exponenciales.lean}{Lean 4 Web}.

\section{En ℝ, si a ≤ b y c < d, entonces a + eᶜ + f ≤ b + eᵈ + f}
\label{sec:org9917837}
\begin{verbatim}
-- ---------------------------------------------------------------------
-- Demostrar que si a, b, c, d y f son números reales tales que
--    a ≤ b
--    c < d
-- entonces
--    a + eᶜ + f ≤ b + eᵈ + f
-- ----------------------------------------------------------------------

-- Demostraciones en lenguaje natural (LN)
-- =======================================

-- 1ª demostración en LN
-- =====================

-- Aplicando a la hipótesis 3 (c < d) la monotonía de la exponencial, se
-- tiene
--    e^c < e^d
-- que, junto a la hipótesis 1 (a ≤ b) y la monotonía de la suma da
--    a + e^c < b + e^d
-- y, de nuevo por la monotonía de la suma, se tiene
--    a + e^c + f < b + e^d + f

-- 2ª demostración en LN
-- =====================

-- Tenemos que demostrar que
--    (a + e^c) + f < (b + e^d) + f
-- que, por la monotonía de la suma, se reduce a las siguientes dos
-- desigualdades:
--    a + e^c < b + e^d                                               (1)
--    f ≤ f                                                           (2)
--
-- La (1), de nuevo por la monotonía de la suma, se reduce a las
-- siguientes dos:
--    a ≤ b                                                         (1.1)
--    e^c < e^d                                                     (1.2)
--
-- La (1.1) se tiene por la hipótesis 1.
--
-- La (1.2) se tiene aplicando la monotonía de la exponencial a la
-- hipótesis 2.
--
-- La (2) se tiene por la propiedad reflexiva.

-- Demostraciones con Lean4
-- ========================

import Mathlib.Analysis.SpecialFunctions.Log.Basic
open Real
variable (a b c d f : ℝ)

-- 1ª demostración
example
  (h1 : a ≤ b)
  (h2 : c < d)
  : a + exp c + f < b + exp d + f :=
by
  have h3 : exp c < exp d :=
    exp_lt_exp.mpr h2
  have h4 : a + exp c < b + exp d :=
    add_lt_add_of_le_of_lt h1 h3
  show a + exp c + f < b + exp d + f
  exact add_lt_add_right h4 f

-- 2ª demostración
example
  (h1 : a ≤ b)
  (h2 : c < d)
  : a + exp c + f < b + exp d + f :=
by
  apply add_lt_add_of_lt_of_le
  { apply add_lt_add_of_le_of_lt
    { exact h1 }
    { apply exp_lt_exp.mpr
      exact h2 } }
  { apply le_refl }

-- 3ª demostración
example
  (h1 : a ≤ b)
  (h2 : c < d)
  : a + exp c + f < b + exp d + f :=
by
  apply add_lt_add_of_lt_of_le
  . apply add_lt_add_of_le_of_lt h1
    apply exp_lt_exp.mpr h2
  rfl

-- Lemas usados
-- ============

-- #check (add_lt_add_of_le_of_lt : a ≤ b → c < d → a + c < b + d)
-- #check (add_lt_add_of_lt_of_le : a < b → c ≤ d → a + c < b + d)
-- #check (add_lt_add_right : b < c → ∀ (a : ℝ), b + a < c + a)
-- #check (exp_lt_exp : exp a < exp b ↔ a < b)
-- #check (le_refl a : a ≤ a)
\end{verbatim}
Se puede interactuar con las pruebas anteriores en \href{https://lean.math.hhu.de/\#url=https://raw.githubusercontent.com/jaalonso/Calculemus2/main/src/Inecuaciones\_con\_exponenciales\_2.lean}{Lean 4 Web}

\section{En ℝ, si d ≤ f, entonces c + e\^{}(a + d) ≤ c + e\^{}(a + f)}
\label{sec:orgddb661d}
\begin{verbatim}
-- ---------------------------------------------------------------------
-- Demostrar que si a, c, d y f son números reales tales que
--    d ≤ f
-- entonces
--    c + exp (a + d) ≤ c + exp (a + f)
-- ----------------------------------------------------------------------

-- Demostraciones en lenguaje natural (LN)
-- =======================================

-- 1ª demostración en LN
-- =====================

-- De la hipótesis, por la monotonia de la suma, se tiene
--    a + d ≤ a + f
-- que, por la monotonía de la exponencial, da
--    exp (a + d) ≤ exp (a + f)
-- y, por la monotonía de la suma, se tiene
--    c + exp (a + d) ≤ c + exp (a + f)

-- 2ª demostración en LN
-- =====================

-- Tenemos que demostrar que
--    c + exp (a + d) ≤ c + exp (a + f)
-- Por la monotonía de la suma, se reduce a
--    exp (a + d) ≤ exp (a + f)
-- que, por la monotonía de la exponencial, se reduce a
--    a + d ≤ a + f
-- que, por la monotonía de la suma, se reduce a
--    d ≤ f
-- que es la hipótesis.

-- Demostraciones con Lean4
-- ========================

import Mathlib.Analysis.SpecialFunctions.Log.Basic
open Real
variable (a c d f : ℝ)

-- 1ª demostración
example
  (h : d ≤ f)
  : c + exp (a + d) ≤ c + exp (a + f) :=
by
  have h1 : a + d ≤ a + f :=
    add_le_add_left h a
  have h2 : exp (a + d) ≤ exp (a + f) :=
    exp_le_exp.mpr h1
  show c + exp (a + d) ≤ c + exp (a + f)
  exact add_le_add_left h2 c

-- 2ª demostración
example
  (h : d ≤ f)
  : c + exp (a + d) ≤ c + exp (a + f) :=
by
  apply add_le_add_left
  apply exp_le_exp.mpr
  apply add_le_add_left
  exact h

-- Lemas usados
-- ============

-- variable (b : ℝ)
-- #check (add_le_add_left : b ≤ c → ∀ (a : ℝ), a + b ≤ a + c)
-- #check (exp_le_exp : exp a ≤ exp b ↔ a ≤ b)
\end{verbatim}
Se puede interactuar con las pruebas anteriores en \href{https://lean.math.hhu.de/\#url=https://raw.githubusercontent.com/jaalonso/Calculemus2/main/src/Inecuaciones\_con\_exponenciales\_3.lean}{Lean 4 Web}

\section{En ℝ, si a ≤ b, entonces log(1+e\^{}a) ≤ log(1+e\^{}b)}
\label{sec:org8a29021}
\begin{verbatim}
-- ---------------------------------------------------------------------
-- Demostrar que si a y b son números reales tales que
--    a ≤ b
-- entonces
--    log(1+e^a) ≤ log(1+e^b)
-- ----------------------------------------------------------------------

-- Demostración en lenguaje natural
-- ================================

-- Por la monotonía del logaritmo, basta demostrar que
--    0 < 1 + e^a                 (1)
--    1 + e^a ≤ 1 + e^b           (2)
--
-- La (1), por la suma de positivos, se reduce a
--    0 < 1                       (1.1)
--    0 < e^a                     (1.2)
-- La (1.1) es una propiedad de los números naturales y la (1.2) de la
-- función exponencial.
--
-- La (2), por la monotonía de la suma, se reduce a
--    e^a ≤ e^b
-- que, por la monotonía de la exponencial, se reduce a
--    a ≤ b
-- que es la hipótesis.

-- Demostraciones con Lean4
-- ========================

import Mathlib.Analysis.SpecialFunctions.Log.Basic
open Real
variable (a b : ℝ)

-- 1ª demostración
example
  (h : a ≤ b)
  : log (1 + exp a) ≤ log (1 + exp b) :=
by
  have h1 : (0 : ℝ) < 1 :=
    zero_lt_one
  have h2 : 0 < exp a :=
    exp_pos a
  have h3 : 0 < 1 + exp a :=
    add_pos h1 h2
  have h4 : exp a ≤ exp b :=
    exp_le_exp.mpr h
  have h5 : 1 + exp a ≤ 1 + exp b :=
    add_le_add_left h4 1
  show log (1 + exp a) ≤ log (1 + exp b)
  exact log_le_log' h3 h5

-- 2ª demostraciṕn
example
  (h : a ≤ b)
  : log (1 + exp a) ≤ log (1 + exp b) :=
by
  apply log_le_log'
  { apply add_pos
    { exact zero_lt_one }
    { exact exp_pos a }}
  { apply add_le_add_left
    exact exp_le_exp.mpr h }

-- Lemas usados
-- ============

-- variable (c : ℝ)
-- #check (add_le_add_left : b ≤ c → ∀ (a : ℝ), a + b ≤ a + c)
-- #check (add_pos : 0 < a → 0 < b → 0 < a + b)
-- #check (exp_le_exp : exp a ≤ exp b ↔ a ≤ b)
-- #check (exp_pos a : 0 < exp a)
-- #check (log_le_log' : 0 < a → a ≤ b → log a ≤ log b)
-- #check (zero_lt_one : 0 < 1)
\end{verbatim}
Se puede interactuar con las pruebas anteriores en \href{https://lean.math.hhu.de/\#url=https://raw.githubusercontent.com/jaalonso/Calculemus2/main/src/Desigualdad\_logaritmica.lean}{Lean 4 Web}

\section{En ℝ, si a ≤ b, entonces c - e\^{}b ≤ c - e\^{}a}
\label{sec:orge4766b1}
\begin{verbatim}
-- ---------------------------------------------------------------------
-- Sean a, b y c números reales. Demostrar que si
--    a ≤ b
-- entonces
--    c - e^b ≤ c - e^a
-- ----------------------------------------------------------------------

-- Demostración en lenguaje natural
-- ================================

-- Aplicando la monotonía de la exponencial a la hipótesis, se tiene
--    e^a ≤ e^b
-- y, restando de c, se invierte la desigualdad
--    c - e^b ≤ c - e^a

-- Demostraciones con Lean4
-- ========================

import Mathlib.Analysis.SpecialFunctions.Log.Basic

open Real

variable (a b c : ℝ)

-- 1ª demostración
example
  (h : a ≤ b)
  : c - exp b ≤ c - exp a :=
by
   have h1 : exp a ≤ exp b :=
     exp_le_exp.mpr h
   show c - exp b ≤ c - exp a
   exact sub_le_sub_left h1 c

-- 2ª demostración
example
  (h : a ≤ b)
  : c - exp b ≤ c - exp a :=
by
   apply sub_le_sub_left _ c
   apply exp_le_exp.mpr h

-- 3ª demostración
example
  (h : a ≤ b)
  : c - exp b ≤ c - exp a :=
sub_le_sub_left (exp_le_exp.mpr h) c

-- 4ª demostración
example
  (h : a ≤ b)
  : c - exp b ≤ c - exp a :=
by linarith [exp_le_exp.mpr h]

-- Lemas usados
-- ============

-- #check (exp_le_exp : exp a ≤ exp b ↔ a ≤ b)
-- #check (sub_le_sub_left : a ≤ b → ∀ (c : ℝ), c - b ≤ c - a)
\end{verbatim}
Se puede interactuar con las pruebas anteriores en \href{https://lean.math.hhu.de/\#url=https://raw.githubusercontent.com/jaalonso/Calculemus2/main/src/Desigualdad\_logaritmica.lean}{Lean 4 Web}

\section{En ℝ, 2ab ≤ a² + b²}
\label{sec:org6cac4ba}
\begin{verbatim}
-- ---------------------------------------------------------------------
-- Sean a y b números reales. Demostrar que
--    2ab ≤ a² + b²
-- ----------------------------------------------------------------------

-- Demostración en lenguaje natural
-- ================================

-- Puesto que los cuadrados son positivos, se tiene
--    (a - b)² ≥ 0
-- Desarrollando el cuadrado, se obtiene
--    a² - 2ab + b² ≥ 0
-- Sumando 2ab a ambos lados, queda
--    a² + b² ≥ 2ab

-- Demostraciones con Lean4
-- ========================

import Mathlib.Data.Real.Basic

variable (a b : ℝ)

-- 1ª demostración
example : 2*a*b ≤ a^2 + b^2 :=
by
  have h1 : 0 ≤ (a - b)^2         := sq_nonneg (a - b)
  have h2 : 0 ≤ a^2 - 2*a*b + b^2 := by linarith only [h1]
  show 2*a*b ≤ a^2 + b^2
  linarith

-- 2ª demostración
example : 2*a*b ≤ a^2 + b^2 :=
by
  have h : 0 ≤ a^2 - 2*a*b + b^2
  { calc a^2 - 2*a*b + b^2
         = (a - b)^2                 := (sub_sq a b).symm
       _ ≥ 0                         := sq_nonneg (a - b) }
  calc 2*a*b
       = 2*a*b + 0                   := (add_zero (2*a*b)).symm
     _ ≤ 2*a*b + (a^2 - 2*a*b + b^2) := add_le_add (le_refl _) h
     _ = a^2 + b^2                   := by ring

-- 3ª demostración
example : 2*a*b ≤ a^2 + b^2 :=
by
  have h : 0 ≤ a^2 - 2*a*b + b^2
  { calc a^2 - 2*a*b + b^2
         = (a - b)^2       := (sub_sq a b).symm
       _ ≥ 0               := sq_nonneg (a - b) }
  linarith only [h]

-- 4ª demostración
example : 2*a*b ≤ a^2 + b^2 :=
-- by apply?
two_mul_le_add_sq a b

-- Lemas usados
-- ============

-- variable (c : ℝ)
-- #check (add_le_add : a ≤ b → c ≤ d → a + c ≤ b + d)
-- #check (add_zero a : a + 0 = a)
-- #check (sq_nonneg a : 0 ≤ a ^ 2)
-- #check (sub_sq a b : (a - b) ^ 2 = a ^ 2 - 2 * a * b + b ^ 2)
-- #check (two_mul_le_add_sq a b : 2 * a * b ≤ a ^ 2 + b ^ 2)
\end{verbatim}
Se puede interactuar con las pruebas anteriores en \href{https://lean.math.hhu.de/\#url=https://raw.githubusercontent.com/jaalonso/Calculemus2/main/src/Desigualdad\_logaritmica.lean}{Lean 4 Web}

\section{En ℝ, |ab| ≤ (a²+b²)/2]}
\label{sec:org1bbe06d}
\begin{verbatim}
-- ---------------------------------------------------------------------
-- Sean a y b números reales. Demostrar que
--    |a*b| ≤ (a^2 + b^2) / 2
-- ----------------------------------------------------------------------

-- Demostración en lenguaje natural
-- ================================

-- Para demostrar
--    |ab| ≤ (a² + b² / 2
-- basta demostrar estas dos desigualdades
--    ab ≤ (a² + b²) / 2                                              (1)
--    -(ab) ≤ (a² + b²) / 2                                           (2)
--
-- Para demostrar (1) basta demostrar que
--    2ab ≤ a² + b²
-- que se prueba como sigue. En primer lugar, como los cuadrados son no
-- negativos, se tiene
--   (a - b)² ≥ 0
-- Desarrollando el cuandrado,
--   a² - 2ab + b² ≥ 0
-- Sumando 2ab,
--   a² + b² ≥ 2ab
--
-- Para demostrar (2) basta demostrar que
--    -2ab ≤ a² + b²
-- que se prueba como sigue. En primer lugar, como los cuadrados son no
-- negativos, se tiene
--   (a + b)² ≥ 0
-- Desarrollando el cuandrado,
--   a² + 2ab + b² ≥ 0
-- Restando 2ab,
--   a² + b² ≥ -2ab

-- Demostraciones con Lean4
-- ========================

import Mathlib.Data.Real.Basic

variable (a b : ℝ)

-- Lemas auxiliares
-- ================

lemma aux1 : a * b * 2 ≤ a ^ 2 + b ^ 2 := by
  have h : 0 ≤ a ^ 2 - 2 * a * b + b ^ 2
  calc
    a ^ 2 - 2 * a * b + b ^ 2
      = (a - b) ^ 2            := by ring
    _ ≥ 0                      := pow_two_nonneg (a - b)
  linarith only [h]

lemma aux2 : -(a * b) * 2 ≤ a ^ 2 + b ^ 2 := by
  have h : 0 ≤ a ^ 2 + 2 * a * b + b ^ 2
  calc
    a ^ 2 + 2 * a * b + b ^ 2
      = (a + b) ^ 2            := by ring
    _ ≥ 0                      := pow_two_nonneg (a + b)
  linarith only [h]

-- 1ª demostración
-- ===============

example : |a * b| ≤ (a ^ 2 + b ^ 2) / 2 := by
  have h : (0 : ℝ) < 2 := by norm_num
  apply abs_le'.mpr
  constructor
  { have h1 : a * b * 2 ≤ a ^ 2 + b ^ 2 := aux1 a b
    show a * b ≤ (a ^ 2 + b ^ 2) / 2
    exact (le_div_iff h).mpr h1 }
  { have h2 : -(a * b) * 2 ≤ a ^ 2 + b ^ 2 := aux2 a b
    show -(a * b) ≤ (a ^ 2 + b ^ 2) / 2
    exact (le_div_iff h).mpr h2 }

-- 2ª demostración
-- ===============

example : |a * b| ≤ (a ^ 2 + b ^ 2) / 2 := by
  have h : (0 : ℝ) < 2 := by norm_num
  apply abs_le'.mpr
  constructor
  { exact (le_div_iff h).mpr (aux1 a b) }
  { exact (le_div_iff h).mpr (aux2 a b) }

-- 3ª demostración
-- ===============

example : |a * b| ≤ (a ^ 2 + b ^ 2) / 2 := by
  have h : (0 : ℝ) < 2 := by norm_num
  apply abs_le'.mpr
  constructor
  { rw [le_div_iff h]
    apply aux1 }
  { rw [le_div_iff h]
    apply aux2 }

-- Lemas usados
-- ============

-- variable (c : ℝ)
-- #check (abs_le' : |a| ≤ b ↔ a ≤ b ∧ -a ≤ b)
-- #check (le_div_iff : 0 < c → (a ≤ b / c ↔ a * c ≤ b))
-- #check (pow_two_nonneg a : 0 ≤ a ^ 2)
\end{verbatim}
Se puede interactuar con las pruebas anteriores en \href{https://lean.math.hhu.de/\#url=https://raw.githubusercontent.com/jaalonso/Calculemus2/main/src/Ejercicio\_desigualdades\_absolutas.lean}{Lean 4 Web}

\section{En ℝ, min(a,b) = min(b,a)}
\label{sec:org668c9c1}
\begin{verbatim}
-- ---------------------------------------------------------------------
-- Sean a y b números reales. Demostrar que
--    min a b = min b a
-- ----------------------------------------------------------------------

-- Demostración en lenguaje natural
-- ================================

-- Es consecuencia de la siguiente propiedad
--    min(a, b) ≤ min(b, a)                                          (1)
-- En efecto, intercambiando las variables en (1) se obtiene
--    min(b, a) ≤ min(a, b)                                          (2)
-- Finalmente de (1) y (2) se obtiene
--    min(b, a) = min(a, b)
--
-- Para demostrar (1), se observa que
--    min(a, b) ≤ b
--    min(a, b) ≤ a
-- y, por tanto,
--    min(a, b) = min(b, a)

-- Demostraciones con Lean4
-- ========================

import Mathlib.Data.Real.Basic

variable (a b : ℝ)

-- Lema auxiliar
-- =============

-- 1ª demostración del lema auxiliar
-- =================================

example : min a b ≤ min b a :=
by
  have h1 : min a b ≤ b := min_le_right a b
  have h2 : min a b ≤ a := min_le_left a b
  show min a b ≤ min b a
  exact le_min h1 h2

-- 2ª demostración del lema auxiliar
-- =================================

example : min a b ≤ min b a :=
by
  apply le_min
  { apply min_le_right }
  { apply min_le_left }

-- 3ª demostración del lema auxiliar
-- =================================

lemma aux : min a b ≤ min b a :=
by exact le_min (min_le_right a b) (min_le_left a b)

-- 1ª demostración
-- ===============

example : min a b = min b a :=
by
  apply le_antisymm
  { exact aux a b}
  { exact aux b a}

-- 2ª demostración
-- ===============

example : min a b = min b a :=
le_antisymm (aux a b) (aux b a)

-- 3ª demostración
-- ===============

example : min a b = min b a :=
min_comm a b

-- Lemas usados
-- ============

-- variable (c : ℝ)
-- #check (le_antisymm : a ≤ b → b ≤ a → a = b)
-- #check (le_min : c ≤ a → c ≤ b → c ≤ min a b)
-- #check (min_comm a b : min a b = min b a)
-- #check (min_le_left a b : min a b ≤ a)
-- #check (min_le_right a b : min a b ≤ b)
\end{verbatim}
Se puede interactuar con las pruebas anteriores en \href{https://lean.math.hhu.de/\#url=https://raw.githubusercontent.com/jaalonso/Calculemus2/main/src/Conmutatividad\_del\_minimo.lean}{Lean 4 Web}

\section{En ℝ, max(a,b) = max(b,a)}
\label{sec:orgfb1927b}
\begin{verbatim}
-- ---------------------------------------------------------------------
-- Sean a y b números reales. Demostrar que
--    max a b = max b a
-- ----------------------------------------------------------------------

-- Demostración en lenguaje natural
-- ================================

-- Es consecuencia de la siguiente propiedad
--    max(a, b) ≤ max(b, a)                                          (1)
-- En efecto, intercambiando las variables en (1) se obtiene
--    max(b, a) ≤ max(a, b)                                          (2)
-- Finalmente de (1) y (2) se obtiene
--    max(b, a) = max(a, b)
--
-- Para demostrar (1), se observa que
--    a ≤ max(b, a)
--    b ≤ max(b, a)
-- y, por tanto,
--    max(a, b) ≤ max(b, a)

-- Demostraciones con Lean4
-- ========================

import Mathlib.Data.Real.Basic

variable (a b : ℝ)

-- Lema auxiliar
-- =============

-- 1ª demostración del lema auxiliar
-- =================================

example : max a b ≤ max b a :=
by
  have h1 : a ≤ max b a := le_max_right b a
  have h2 : b ≤ max b a := le_max_left b a
  show max a b ≤ max b a
  exact max_le h1 h2

-- 2ª demostración del lema auxiliar
-- =================================

example : max a b ≤ max b a :=
by
  apply max_le
  { apply le_max_right }
  { apply le_max_left }

-- 3ª demostración del lema auxiliar
-- =================================

lemma aux : max a b ≤ max b a :=
by exact max_le (le_max_right b a) (le_max_left b a)

-- 1ª demostración
-- ===============

example : max a b = max b a :=
by
  apply le_antisymm
  { exact aux a b}
  { exact aux b a}

-- 2ª demostración
-- ===============

example : max a b = max b a :=
le_antisymm (aux a b) (aux b a)

-- 3ª demostración
-- ===============

example : max a b = max b a :=
max_comm a b

-- Lemas usados
-- ============

-- variable (c : ℝ)
-- #check (le_antisymm : a ≤ b → b ≤ a → a = b)
-- #check (le_max_left a b : a ≤ max a b)
-- #check (le_max_right a b : b ≤ max a b)
-- #check (max_comm a b : max a b = max b a)
-- #check (max_le : a ≤ c → b ≤ c → max a b ≤ c)
\end{verbatim}
Se puede interactuar con las pruebas anteriores en \href{https://lean.math.hhu.de/\#url=https://raw.githubusercontent.com/jaalonso/Calculemus2/main/src/Conmutatividad\_del\_maximo.lean}{Lean 4 Web}

\section{En ℝ, min(min(a,b),c) = min(a,min(b,c))}
\label{sec:orge68e25d}
\begin{verbatim}
-- ---------------------------------------------------------------------
-- Sean a, b y c números reales. Demostrar que
--    min (min a b) c = min a (min b c)
-- ----------------------------------------------------------------------

-- Demostración en lenguaje natural
-- ================================

-- Por la propiedad antisimétrica, la igualdad es consecuencia de las
-- siguientes desigualdades
--    min(min(a, b), c) ≤ min(a, min(b, c))                           (1)
--    min(a, min(b, c)) ≤ min(min(a, b), c)                           (2)
--
-- La (1) es consecuencia de las siguientes desigualdades
--    min(min(a, b), c) ≤ a                                          (1a)
--    min(min(a, b), c) ≤ b                                          (1b)
--    min(min(a, b), c) ≤ c                                          (1c)
-- En efecto, de (1b) y (1c) se obtiene
--    min(min(a, b), c) ≤ min(b,c)
-- que, junto con (1a) da (1).
--
-- La (2) es consecuencia de las siguientes desigualdades
--    min(a, min(b, c)) ≤ a                                          (2a)
--    min(a, min(b, c)) ≤ b                                          (2b)
--    min(a, min(b, c)) ≤ c                                          (2c)
-- En efecto, de (2a) y (2b) se obtiene
--    min(a, min(b, c)) ≤ min(a, b)
-- que, junto con (2c) da (2).
--
-- La demostración de (1a) es
--    min(min(a, b), c) ≤ min(a, b) ≤ a
-- La demostración de (1b) es
--    min(min(a, b), c) ≤ min(a, b) ≤ b
-- La demostración de (2b) es
--    min(a, min(b, c)) ≤ min(b, c) ≤ b
-- La demostración de (2c) es
--    min(a, min(b, c)) ≤ min(b, c) ≤ c
-- La (1c) y (2a) son inmediatas.

-- Demostraciones con Lean4
-- ========================

import Mathlib.Data.Real.Basic

variable {a b c : ℝ}

-- Lemas auxiliares
-- ================

lemma aux1a : min (min a b) c ≤ a :=
calc min (min a b) c
     ≤ min a b := by exact min_le_left (min a b) c
   _ ≤ a       := min_le_left a b

lemma aux1b : min (min a b) c ≤ b :=
calc min (min a b) c
     ≤ min a b := by exact min_le_left (min a b) c
   _ ≤ b       := min_le_right a b

lemma aux1c : min (min a b) c ≤ c :=
by exact min_le_right (min a b) c

-- 1ª demostración del lema aux1
lemma aux1 : min (min a b) c ≤ min a (min b c) :=
by
  apply le_min
  { show min (min a b) c ≤ a
    exact aux1a }
  { show min (min a b) c ≤ min b c
    apply le_min
    { show min (min a b) c ≤ b
      exact aux1b }
    { show min (min a b) c ≤ c
      exact aux1c }}

-- 2ª demostración del lema aux1
lemma aux1' : min (min a b) c ≤ min a (min b c) :=
le_min aux1a (le_min aux1b aux1c)

lemma aux2a : min a (min b c) ≤ a :=
by exact min_le_left a (min b c)

lemma aux2b : min a (min b c) ≤ b :=
calc min a (min b c)
     ≤ min b c        := by exact min_le_right a (min b c)
   _ ≤ b              := min_le_left b c

lemma aux2c : min a (min b c) ≤ c :=
calc min a (min b c)
     ≤ min b c        := by exact min_le_right a (min b c)
   _ ≤ c              := min_le_right b c

-- 1ª demostración del lema aux2
lemma aux2 : min a (min b c) ≤ min (min a b) c :=
by
  apply le_min
  { show min a (min b c) ≤ min a b
    apply le_min
    { show min a (min b c) ≤ a
      exact aux2a }
    { show min a (min b c) ≤ b
      exact aux2b }}
  { show min a (min b c) ≤ c
    exact aux2c }

-- 2ª demostración del lema aux2
lemma aux2' : min a (min b c) ≤ min (min a b) c :=
le_min (le_min aux2a aux2b) aux2c

-- 1ª demostración
-- ===============

example :
  min (min a b) c = min a (min b c) :=
by
  apply le_antisymm
  { show min (min a b) c ≤ min a (min b c)
    exact aux1 }
  { show min a (min b c) ≤ min (min a b) c
    exact aux2 }

-- 2ª demostración
-- ===============

example : min (min a b) c = min a (min b c) :=
by
  apply le_antisymm
  { exact aux1 }
  { exact aux2 }

-- 3ª demostración
-- ===============

example : min (min a b) c = min a (min b c) :=
le_antisymm aux1 aux2

-- 4ª demostración
-- ===============

example : min (min a b) c = min a (min b c) :=
min_assoc a b c

-- Lemas usados
-- ============

-- #check (le_antisymm : a ≤ b → b ≤ a → a = b)
-- #check (le_min : c ≤ a → c ≤ b → c ≤ min a b)
-- #check (min_assoc a b c : min (min a b) c = min a (min b c))
-- #check (min_le_left a b : min a b ≤ a)
-- #check (min_le_right a b : min a b ≤ b)
\end{verbatim}
Se puede interactuar con las pruebas anteriores en \href{https://lean.math.hhu.de/\#url=https://raw.githubusercontent.com/jaalonso/Calculemus2/main/src/Asociatividad\_del\_minimo.lean}{Lean 4 Web}

\section{En ℝ, min(a,b)+c = min(a+c,b+c)}
\label{sec:orge2af08d}
\begin{verbatim}
-- ---------------------------------------------------------------------
-- Sean a, b y c números reales. Demostrar que
--    min a b + c = min (a + c) (b + c)
-- ----------------------------------------------------------------------

-- Demostraciones en lenguaje natural (LN)
-- =======================================

-- 1ª demostración en LN
-- =====================

-- Aplicando la propiedad antisimétrica a las siguientes desigualdades
--    min(a, b) + c ≤ min(a + c, b + c)                               (1)
--    min(a + c, b + c) ≤ min(a, b) + c                               (2)
--
-- Para demostrar (1) basta demostrar que se verifican las siguientes
-- desigualdades
--    min(a, b) + c ≤ a + c                                          (1a)
--    min(a, b) + c ≤ b + c                                          (1b)
-- que se tienen porque se verifican las siguientes desigualdades
--    min(a, b) ≤ a
--    min(a, b) ≤ b
--
-- Para demostrar (2) basta demostrar que se verifica
--    min(a + c, b + c) - c ≤ min(a, b)
-- que se demuestra usando (1); en efecto,
--    min(a + c, b + c) - c ≤ min(a + c - c, b + c - c)    [por (1)]
--                          = min(a, b)

-- 2ª demostración en LN
-- =====================

-- Por casos según a ≤ b.
--
-- 1º caso: Supongamos que a ≤ b. Entonces,
--    min(a, b) + c = a + c
--                  = min(a + c, b + c)
--
-- 2º caso: Supongamos que a ≰ b. Entonces,
--    min(a, b) + c = b + c
--                  = min(a + c, b + c)

-- Demostraciones con Lean4
-- ========================

import Mathlib.Data.Real.Basic

variable {a b c : ℝ}

-- En las demostraciones se usarán los siguientes lemas auxiliares
--    aux1 : min a b + c ≤ min (a + c) (b + c)
--    aux2 : min (a + c) (b + c) ≤ min a b + c
-- cuyas demostraciones se exponen a continuación.

-- 1ª demostración de aux1
lemma aux1 :
  min a b + c ≤ min (a + c) (b + c) :=
by
  have h1 : min a b ≤ a :=
    min_le_left a b
  have h2 : min a b + c ≤ a + c :=
    add_le_add_right h1 c
  have h3 : min a b  ≤ b :=
    min_le_right a b
  have h4 : min a b + c ≤ b + c :=
    add_le_add_right h3 c
  show min a b + c ≤ min (a + c) (b + c)
  exact le_min h2 h4

-- 2ª demostración de aux1
example :
  min a b + c ≤ min (a + c) (b + c) :=
by
  apply le_min
  { apply add_le_add_right
    exact min_le_left a b }
  { apply add_le_add_right
    exact min_le_right a b }

-- 3ª demostración de aux1
example :
  min a b + c ≤ min (a + c) (b + c) :=
le_min (add_le_add_right (min_le_left a b) c)
       (add_le_add_right (min_le_right a b) c)

-- 1ª demostración de aux2
lemma aux2 :
  min (a + c) (b + c) ≤ min a b + c :=
by
  have h1 : min (a + c) (b + c) + -c ≤ min a b
  { calc min (a + c) (b + c) + -c
         ≤ min (a + c + -c) (b + c + -c) := aux1
       _ = min a b                       := by ring_nf }
  show min (a + c) (b + c) ≤ min a b + c
  exact add_neg_le_iff_le_add.mp h1

-- 1ª demostración del ejercicio
example :
  min a b + c = min (a + c) (b + c) :=
by
  have h1 : min a b + c ≤ min (a + c) (b + c) := aux1
  have h2 : min (a + c) (b + c) ≤ min a b + c := aux2
  show min a b + c = min (a + c) (b + c)
  exact le_antisymm h1 h2

-- 2ª demostración del ejercicio
example :
  min a b + c = min (a + c) (b + c) :=
by
  apply le_antisymm
  { show min a b + c ≤ min (a + c) (b + c)
    exact aux1 }
  { show min (a + c) (b + c) ≤ min a b + c
    exact aux2 }

-- 3ª demostración del ejercicio
example :
  min a b + c = min (a + c) (b + c) :=
by
  apply le_antisymm
  { exact aux1 }
  { exact aux2 }

-- 4ª demostración del ejercicio
example :
  min a b + c = min (a + c) (b + c) :=
le_antisymm aux1 aux2

-- 5ª demostración del ejercicio
example : min a b + c = min (a + c) (b + c) :=
by
  by_cases h : a ≤ b
  { have h1 : a + c ≤ b + c := add_le_add_right h c
    calc min a b + c = a + c               := by simp [min_eq_left h]
                   _ = min (a + c) (b + c) := by simp [min_eq_left h1]}
  { have h2: b ≤ a := le_of_not_le h
    have h3 : b + c ≤ a + c := add_le_add_right h2 c
    calc min a b + c = b + c               := by simp [min_eq_right h2]
                   _ = min (a + c) (b + c) := by simp [min_eq_right h3]}

-- 6ª demostración del ejercicio
example : min a b + c = min (a + c) (b + c) :=
(min_add_add_right a b c).symm

-- Lemas usados
-- ============

-- #check (add_le_add_right : b ≤ c → ∀ (a : ℝ),  b + a ≤ c + a)
-- #check (add_neg_le_iff_le_add : a - b ≤ c ↔ a ≤ c + b)
-- #check (le_antisymm : a ≤ b → b ≤ a → a = b)
-- #check (le_min : c ≤ a → c ≤ b → c ≤ min a b)
-- #check (min_add_add_right a b c : min (a + c) (b + c) = min a b + c)
-- #check (min_eq_left : a ≤ b → min a b = a)
-- #check (min_eq_right : b ≤ a → min a b = b)
-- #check (min_le_left a b : min a b ≤ a)
-- #check (min_le_right a b : min a b ≤ b)
\end{verbatim}
Se puede interactuar con las pruebas anteriores en \href{https://lean.math.hhu.de/\#url=https://raw.githubusercontent.com/jaalonso/Calculemus2/main/src/Minimo\_de\_suma.lean}{Lean 4 Web}

\section{En ℝ, |a| - |b| ≤ |a - b|}
\label{sec:org4e702f4}
\begin{verbatim}
-- ---------------------------------------------------------------------
-- Sean a y b números reales. Demostrar que
--    |a| - |b| ≤ |a - b|
-- ----------------------------------------------------------------------

-- Demostraciones en lenguaje natural (LN)
-- =======================================

-- 1ª demostración en LN
-- =====================

-- Por la siguiente cadena de desigualdades
--    |a| - |b| = |a - b + b| - |b|
--              ≤ (|a - b| + |b|) - |b|   [por la desigualdad triangular]
--              = |a - b|

-- 2ª demostración en LN
-- =====================

-- Por la desigualdad triangular
--    |a - b + b| ≤ |a - b| + |b|
-- simplificando en la izquierda
--    |a| ≤ |a - b| + |b|
-- y, pasando |b| a la izquierda
--    |a| - |b| ≤ |a - b|

-- Demostraciones con Lean4
-- ========================

import Mathlib.Data.Real.Basic

variable (a b : ℝ)

-- 1ª demostración (basada en la 1ª en LN)
example : |a| - |b| ≤ |a - b| :=
calc |a| - |b|
     = |a - b + b| - |b| :=
          congrArg (fun x => |x| - |b|) (sub_add_cancel a b).symm
   _ ≤ (|a - b| + |b|) - |b| :=
           sub_le_sub_right (abs_add (a - b) b) (|b|)
   _ = |a - b| :=
          add_sub_cancel (|a - b|) (|b|)

-- 2ª demostración (basada en la 1ª en LN)
example : |a| - |b| ≤ |a - b| :=
calc |a| - |b|
     = |a - b + b| - |b| := by
          rw [sub_add_cancel]
   _ ≤ (|a - b| + |b|) - |b| := by
          apply sub_le_sub_right
          apply abs_add
   _ = |a - b| := by
          rw [add_sub_cancel]

-- 3ª demostración (basada en la 2ª en LN)
example : |a| - |b| ≤ |a - b| :=
by
  have h1 : |a - b + b| ≤ |a - b| + |b| := abs_add (a - b) b
  rw [sub_add_cancel] at h1
  exact abs_sub_abs_le_abs_sub a b

-- 4ª demostración
example : |a| - |b| ≤ |a - b| :=
abs_sub_abs_le_abs_sub a b

-- Lemas usados
-- ============

-- #check (abs_add a b : |a + b| ≤ |a| + |b|)
-- #check (abs_sub_abs_le_abs_sub a b : |a| - |b| ≤ |a - b|)
-- #check (add_sub_cancel a b : a + b - b = a)
-- #check (sub_add_cancel a b : a - b + b = a)
-- #check (sub_le_sub_right : a ≤ b → ∀ (c : ℝ), a - c ≤ b - c)
\end{verbatim}
Se puede interactuar con las pruebas anteriores en \href{https://lean.math.hhu.de/\#url=https://raw.githubusercontent.com/jaalonso/Calculemus2/main/src/abs\_sub.lean}{Lean 4 Web}

\section{En ℝ, \{0 < ε, ε ≤ 1, |x| < ε, |y| < ε\} ⊢ |xy| < ε}
\label{sec:orgc4d7e04}
\begin{verbatim}
-- ---------------------------------------------------------------------
-- Demostrar que para todos los números reales x, y, ε si
--    0 < ε
--    ε ≤ 1
--    |x| < ε
--    |y| < ε
-- entonces
--    |x * y| < ε
-- ----------------------------------------------------------------------

-- Demostración en lenguaje natural
-- ================================

-- Se usarán los siguientes lemas
--    abs_mul          : |a * b| = |a| * |b|
--    zero_mul         : 0 * a = 0
--    abs_nonneg a     : 0 ≤ |a|
--    lt_of_le_of_ne   : a ≤ b → a ≠ b → a < b
--    ne_comm          : a ≠ b ↔ b ≠ a
--    mul_lt_mul_left  : 0 < a → (a * b < a * c ↔ b < c)
--    mul_lt_mul_right : 0 < a → (b * a < c * a ↔ b < c)
--    mul_le_mul_right : 0 < a → (b * a ≤ c * a ↔ b ≤ c)
--    one_mul          : 1 * a = a
--
-- Sean x y ε ∈ ℝ tales que
--    0 < ε                                                         (he1)
--    ε ≤ 1                                                         (he2)
--    |x| < ε                                                       (hx)
--    |y| < ε                                                       (hy)
-- y tenemos que demostrar que
--    |x * y| < ε
-- Lo haremos distinguiendo caso según |x| = 0.
--
-- 1º caso. Supongamos que
--    |x| = 0                                                        (1)
-- Entonces,
--    |x * y| = |x| * |y|    [por abs_mul]
--            = 0 * |y|      [por h1]
--            = 0            [por zero_mul]
--            < ε            [por he1]
--
-- 2º caso. Supongamos que
--    |x| ≠ 0                                                        (2)
-- Entonces, por lt_of_le_of_ne, abs_nonneg y ne_comm, se tiene
--    0 < x                                                          (3)
-- y, por tanto,
--    |x * y| = |x| * |y|    [por abs_mul]
--            < |x| * ε      [por mul_lt_mul_left, (3) y (hy)]
--            < ε * ε        [por mul_lt_mul_right, (he1) y (hx)]
--            ≤ 1 * ε        [por mul_le_mul_right, (he1) y (he2)]
--            = ε            [por one_mul]

-- Demostraciones con Lean4
-- ========================

import Mathlib.Data.Real.Basic

-- 1ª demostración
-- ===============

example :
  ∀ {x y ε : ℝ}, 0 < ε → ε ≤ 1 → |x| < ε → |y| < ε → |x * y| < ε :=
by
  intros x y ε he1 he2 hx hy
  by_cases h : (|x| = 0)
  . -- h : |x| = 0
    show |x * y| < ε
    calc
      |x * y|
         = |x| * |y| := abs_mul x y
      _  = 0 * |y|   := by rw [h]
      _  = 0         := zero_mul (abs y)
      _  < ε         := he1
  . -- h : ¬|x| = 0
    have h1 : 0 < |x| := by
      have h2 : 0 ≤ |x| := abs_nonneg x
      show 0 < |x|
      exact lt_of_le_of_ne h2 (ne_comm.mpr h)
    show |x * y| < ε
    calc |x * y|
         = |x| * |y| := abs_mul x y
       _ < |x| * ε   := (mul_lt_mul_left h1).mpr hy
       _ < ε * ε     := (mul_lt_mul_right he1).mpr hx
       _ ≤ 1 * ε     := (mul_le_mul_right he1).mpr he2
       _ = ε         := one_mul ε

-- 2ª demostración
-- ===============

example :
  ∀ {x y ε : ℝ}, 0 < ε → ε ≤ 1 → |x| < ε → |y| < ε → |x * y| < ε :=
by
  intros x y ε he1 he2 hx hy
  by_cases (|x| = 0)
  . -- h : |x| = 0
    show |x * y| < ε
    calc
      |x * y| = |x| * |y| := by apply abs_mul
            _ = 0 * |y|   := by rw [h]
            _ = 0         := by apply zero_mul
            _ < ε         := by apply he1
  . -- h : ¬|x| = 0
    have h1 : 0 < |x| := by
      have h2 : 0 ≤ |x| := by apply abs_nonneg
      exact lt_of_le_of_ne h2 (ne_comm.mpr h)
    show |x * y| < ε
    calc
      |x * y| = |x| * |y| := by rw [abs_mul]
            _ < |x| * ε   := by apply (mul_lt_mul_left h1).mpr hy
            _ < ε * ε     := by apply (mul_lt_mul_right he1).mpr hx
            _ ≤ 1 * ε     := by apply (mul_le_mul_right he1).mpr he2
            _ = ε         := by rw [one_mul]

-- 3ª demostración
-- ===============

example :
  ∀ {x y ε : ℝ}, 0 < ε → ε ≤ 1 → |x| < ε → |y| < ε → |x * y| < ε :=
by
  intros x y ε he1 he2 hx hy
  by_cases (|x| = 0)
  . -- h : |x| = 0
    show |x * y| < ε
    calc |x * y| = |x| * |y| := by simp only [abs_mul]
               _ = 0 * |y|   := by simp only [h]
               _ = 0         := by simp only [zero_mul]
               _ < ε         := by simp only [he1]
  . -- h : ¬|x| = 0
    have h1 : 0 < |x| := by
      have h2 : 0 ≤ |x| := by simp only [abs_nonneg]
      exact lt_of_le_of_ne h2 (ne_comm.mpr h)
    show |x * y| < ε
    calc
      |x * y| = |x| * |y| := by simp [abs_mul]
            _ < |x| * ε   := by simp only [mul_lt_mul_left, h1, hy]
            _ < ε * ε     := by simp only [mul_lt_mul_right, he1, hx]
            _ ≤ 1 * ε     := by simp only [mul_le_mul_right, he1, he2]
            _ = ε         := by simp only [one_mul]

-- Lemas usados
-- ============

-- variable (a b c : ℝ)
-- #check (abs_mul a b : |a * b| = |a| * |b|)
-- #check (abs_nonneg a : 0 ≤ |a|)
-- #check (lt_of_le_of_ne : a ≤ b → a ≠ b → a < b)
-- #check (mul_le_mul_right : 0 < a → (b * a ≤ c * a ↔ b ≤ c))
-- #check (mul_lt_mul_left : 0 < a → (a * b < a * c ↔ b < c))
-- #check (mul_lt_mul_right : 0 < a → (b * a < c * a ↔ b < c))
-- #check (ne_comm : a ≠ b ↔ b ≠ a)
-- #check (one_mul a : 1 * a = a)
-- #check (zero_mul a : 0 * a = 0)
\end{verbatim}
Se puede interactuar con las pruebas anteriores en \href{https://lean.math.hhu.de/\#url=https://raw.githubusercontent.com/jaalonso/Calculemus2/main/src/Acotacion\_del\_producto.lean}{Lean 4 Web}

\section{Hay algún número real entre 2 y 3}
\label{sec:orgd452d3f}
\begin{verbatim}
-- ---------------------------------------------------------------------
-- Demostrar que hay algún número real entre 2 y 3.
-- ----------------------------------------------------------------------

-- Demostración en lenguaje natural
-- ================================

-- Puesto que 2 < 5/2 < 3, basta elegir 5/2.

-- Demostracione con Lean4
-- =======================

import Mathlib.Data.Real.Basic

-- 1ª demostración
example : ∃ x : ℝ, 2 < x ∧ x < 3 :=
by
  have h : 2 < (5 : ℝ) / 2 ∧ (5 : ℝ) / 2 < 3 :=
    by norm_num
  show ∃ x : ℝ, 2 < x ∧ x < 3
  exact Exists.intro (5 / 2) h

-- 2ª demostración
example : ∃ x : ℝ, 2 < x ∧ x < 3 :=
by
  have h : 2 < (5 : ℝ) / 2 ∧ (5 : ℝ) / 2 < 3 :=
    by norm_num
  show ∃ x : ℝ, 2 < x ∧ x < 3
  exact ⟨5 / 2, h⟩

-- 3ª demostración
example : ∃ x : ℝ, 2 < x ∧ x < 3 :=
by
  use 5 / 2
  norm_num

-- 4ª demostración
example : ∃ x : ℝ, 2 < x ∧ x < 3 :=
⟨5 / 2, by norm_num⟩
\end{verbatim}
Se puede interactuar con las pruebas anteriores en \href{https://lean.math.hhu.de/\#url=https://raw.githubusercontent.com/jaalonso/Calculemus2/main/src/Existencia\_de\_valor\_intermedio.lean}{Lean 4 Web}

\chapter{Divisibilidad}
\label{sec:orgb7de814}

\section{Si x,y,z ∈ ℕ, entonces x ∣ yxz}
\label{sec:org767f5ac}
\begin{verbatim}
-- ---------------------------------------------------------------------
-- Demostrar que si x, y, z ∈ ℕ, entonces
--    x ∣ y * x * z
-- ----------------------------------------------------------------------

-- Demostración en lenguaje natural
-- ================================

-- Por la transitividad de la divisibilidad aplicada a las relaciones
--    x ∣ yx
--    yx ∣ yxz

-- Demostraciones con Lean4
-- ========================

import Mathlib.Data.Real.Basic
variable (x y z : ℕ)

-- 1ª demostración
-- ===============

example : x ∣ y * x * z :=
by
  have h1 : x ∣ y * x :=
    dvd_mul_left x y
  have h2 : (y * x) ∣ (y * x * z) :=
    dvd_mul_right (y * x) z
  show x ∣ y * x * z
  exact dvd_trans h1 h2

-- 2ª demostración
-- ===============

example : x ∣ y * x * z :=
dvd_trans (dvd_mul_left x y) (dvd_mul_right (y * x) z)

-- 3ª demostración
-- ===============

example : x ∣ y * x * z :=
by
  apply dvd_mul_of_dvd_left
  apply dvd_mul_left

-- Lemas usados
-- ============

-- #check (dvd_mul_left x y : x ∣ y * x)
-- #check (dvd_mul_right x y : x ∣ x * y)
-- #check (dvd_trans : x ∣ y → y ∣ z → x ∣ z)
\end{verbatim}
Se puede interactuar con las pruebas anteriores en \href{https://lean.math.hhu.de/\#url=https://raw.githubusercontent.com/jaalonso/Calculemus2/main/src/Divisibilidad\_de\_producto.lean}{Lean 4 Web}

\section{Si x divide a w, también divide a y(xz)+x²+w²}
\label{sec:orged336fb}
\begin{verbatim}
-- ---------------------------------------------------------------------
-- Demostrar que si
--    x ∣ w
-- entonces
--    x ∣ y * (x * z) + x^2 + w^2
-- ----------------------------------------------------------------------

-- Demostración en lenguaje natural
-- ================================

-- Por la divisibilidad de la suma basta probar que
--    x | yxz                                                         (1)
--    x | x²                                                          (2)
--    x | w²                                                          (3)
--
-- Para demostrar (1), por la divisibilidad del producto se tiene
--    x | xz
-- y, de nuevo por la divisibilidad del producto,
--    x | y(xz).
--
-- La propiedad (2) se tiene por la definición de cuadrado y la
-- divisibilidad del producto.
--
-- La propiedad (3) se tiene por la definición de cuadrado, la hipótesis
-- y la divisibilidad del producto.

-- Demostraciones con Lean4
-- ========================

import Mathlib.Data.Real.Basic
variable (w x y z : ℕ)

-- 1ª demostración
example
  (h : x ∣ w)
  : x ∣ y * (x * z) + x^2 + w^2 :=
by
  have h1 : x ∣ x * z :=
    dvd_mul_right x z
  have h2 : x ∣ y * (x * z) :=
    dvd_mul_of_dvd_right h1 y
  have h3 : x ∣ x^2 := by
    apply dvd_mul_left
  have h4 : x ∣ w * w :=
    dvd_mul_of_dvd_left h w
  have h5 : x ∣ w^2 := by
    rwa [← pow_two w] at h4
  have h6 : x ∣ y * (x * z) + x^2 :=
    dvd_add h2 h3
  show x ∣ y * (x * z) + x^2 + w^2
  exact dvd_add h6 h5

-- 2ª demostración
example
  (h : x ∣ w)
  : x ∣ y * (x * z) + x^2 + w^2 :=
by
  apply dvd_add
  { apply dvd_add
    { apply dvd_mul_of_dvd_right
      apply dvd_mul_right }
    { rw [pow_two]
      apply dvd_mul_right }}
  { rw [pow_two]
    apply dvd_mul_of_dvd_left h }

-- 3ª demostración
example
  (h : x ∣ w)
  : x ∣ y * (x * z) + x^2 + w^2 :=
by
  repeat' apply dvd_add
  { apply dvd_mul_of_dvd_right
    apply dvd_mul_right }
  { rw [pow_two]
    apply dvd_mul_right }
  { rw [pow_two]
    apply dvd_mul_of_dvd_left h }

-- Lemas usados
-- ============

-- #check (dvd_add : x ∣ y → x ∣ z → x ∣ y + z)
-- #check (dvd_mul_left x y : x ∣ y * x)
-- #check (dvd_mul_right x y : x ∣ x * y)
-- #check (dvd_mul_of_dvd_left : x ∣ y → ∀ (c : ℕ), x ∣ y * c)
-- #check (dvd_mul_of_dvd_right : x ∣ y → ∀ (c : ℕ), x ∣ c * y)
-- #check (pow_two x : x ^ 2 = x * x)
\end{verbatim}
Se puede interactuar con las pruebas anteriores en \href{https://lean.math.hhu.de/\#url=https://raw.githubusercontent.com/jaalonso/Calculemus2/main/src/Ejercicio\_de\_divisibilidad.lean}{Lean 4 Web}

\section{Conmutatividad del máximo común divisor}
\label{sec:org3eba5b1}
\begin{verbatim}
-- ---------------------------------------------------------------------
-- Demostrar que si m y n son números naturales, entonces
--    gcd m n = gcd n m
-- ----------------------------------------------------------------------

-- Demostración en lenguaje natural
-- ================================

-- Es consecuencia del siguiente lema auxiliar
--    (∀ x, y ∈ ℕ)[gcd(x,y) ∣ gcd(y,x)]                               (1)
-- En efecto, sustituyendo en (1) x por m e y por n, se tiene
--    gcd(m, n) ∣ gcd(n, m)                                           (2)
-- y sustituyendo en (1) x por n e y por m, se tiene
--    gcd(n, m) ∣ gcd(m, n)                                           (3)
-- Finalmente, aplicando la propiedad antisimétrica de la divisibilidad
-- a (2) y (3), se tiene
--    gcd(m, n) = gcd(n, m)
--
-- Para demostrar (1), por la definición del máximo común divisor, basta
-- demostrar las siguientes relaciones
--    gcd(m, n) ∣ n
--    gcd(m, n) ∣ m
-- y ambas se tienen por la definición del máximo común divisor.

-- Demostraciones con Lean4
-- ========================

import Mathlib.Data.Real.Basic
variable (k m n : ℕ)

open Nat

-- 1ª demostración del lema auxiliar
lemma aux : gcd m n ∣ gcd n m :=
by
  have h1 : gcd m n ∣ n :=
    gcd_dvd_right m n
  have h2 : gcd m n ∣ m :=
    gcd_dvd_left m n
  show gcd m n ∣ gcd n m
  exact dvd_gcd h1 h2

-- 2ª demostración del lema auxiliar
example : gcd m n ∣ gcd n m :=
dvd_gcd (gcd_dvd_right m n) (gcd_dvd_left m n)

-- 1ª demostración
example : gcd m n = gcd n m :=
by
  have h1 : gcd m n ∣ gcd n m := aux m n
  have h2 : gcd n m ∣ gcd m n := aux n m
  show gcd m n = gcd n m
  exact _root_.dvd_antisymm h1 h2

-- 2ª demostración
example : gcd m n = gcd n m :=
by
  apply _root_.dvd_antisymm
  { exact aux m n }
  { exact aux n m }

-- 3ª demostración
example : gcd m n = gcd n m :=
_root_.dvd_antisymm (aux m n) (aux n m)

-- 4ª demostración
example : gcd m n = gcd n m :=
-- by apply?
gcd_comm m n

-- Lemas usados
-- ============

-- #check (_root_.dvd_antisymm : m ∣ n → n ∣ m → m = n)
-- #check (dvd_gcd : k ∣ m → k ∣ n → k ∣ gcd m n)
-- #check (gcd_comm m n : gcd m n = gcd n m)
-- #check (gcd_dvd_left  m n: gcd m n ∣ m)
-- #check (gcd_dvd_right m n : gcd m n ∣ n)
\end{verbatim}
Se puede interactuar con las pruebas anteriores en \href{https://lean.math.hhu.de/\#url=https://raw.githubusercontent.com/jaalonso/Calculemus2/main/src/Conmutatividad\_del\_gcd.lean}{Lean 4 Web}

\chapter{Retículos}
\label{sec:orgda15c16}

\section{En los retículos, x ⊓ y = y ⊓ x}
\label{sec:org79de6b7}
\begin{verbatim}
-- ---------------------------------------------------------------------
-- Demostrar que en los retículos se verifica que
--     x ⊓ y = y ⊓ x
-- ----------------------------------------------------------------------

-- Demostración en lenguaje natural
-- ================================

-- Es consecuencia del siguiente lema auxiliar
--    (∀ a, b)[a ⊓ b ≤ b ⊓ a]                                         (1)
-- En efecto, sustituyendo en (1) a por x y b por y, se tiene
--    x ⊓ y ≤ y ⊓ x                                                   (2)
-- y sustituyendo en (1) a por y y b por x, se tiene
--    y ⊓ x ≤ x ⊓ y                                                   (3)
-- Finalmente, aplicando la propiedad antisimétrica de la divisibilidad
-- a (2) y (3), se tiene
--    x ⊓ y = y ⊓ x
--
-- Para demostrar (1), por la definición del ínfimo, basta demostrar
-- las siguientes relaciones
--    y ⊓ x ≤ x
--    y ⊓ x ≤ y
-- y ambas se tienen por la definición del ínfimo.

-- Demostraciones con Lean4
-- ========================

import Mathlib.Order.Lattice
variable {α : Type _} [Lattice α]
variable (x y z : α)

-- 1ª demostración del lema auxiliar
lemma aux : x ⊓ y ≤ y ⊓ x :=
by
  have h1 : x ⊓ y ≤ y :=
    inf_le_right
  have h2 : x ⊓ y ≤ x :=
    inf_le_left
  show x ⊓ y ≤ y ⊓ x
  exact le_inf h1 h2

-- 2ª demostración del lema auxiliar
example : x ⊓ y ≤ y ⊓ x :=
by
  apply le_inf
  { apply inf_le_right }
  { apply inf_le_left }

-- 3ª demostración del lema auxiliar
example : x ⊓ y ≤ y ⊓ x :=
le_inf inf_le_right inf_le_left

-- 1ª demostración
example : x ⊓ y = y ⊓ x :=
by
  have h1 : x ⊓ y ≤ y ⊓ x :=
    aux x y
  have h2 : y ⊓ x ≤ x ⊓ y :=
    aux y x
  show x ⊓ y = y ⊓ x
  exact le_antisymm h1 h2

-- 2ª demostración
example : x ⊓ y = y ⊓ x :=
by
  apply le_antisymm
  { apply aux }
  { apply aux }

-- 3ª demostración
example : x ⊓ y = y ⊓ x :=
le_antisymm (aux x y) (aux y x)

-- 4ª demostración
example : x ⊓ y = y ⊓ x :=
by apply le_antisymm; simp ; simp

-- 5ª demostración
example : x ⊓ y = y ⊓ x :=
-- by apply?
inf_comm

-- Lemas usados
-- ============

-- #check (inf_comm : x ⊓ y = y ⊓ x)
-- #check (inf_le_left : x ⊓ y ≤ x)
-- #check (inf_le_right : x ⊓ y ≤ y)
-- #check (le_antisymm : x ≤ y → y ≤ x → x = y)
-- #check (le_inf : z ≤ x → z ≤ y → z ≤ x ⊓ y)
\end{verbatim}
Se puede interactuar con las pruebas anteriores en \href{https://lean.math.hhu.de/\#url=https://raw.githubusercontent.com/jaalonso/Calculemus2/main/src/Conmutatividad\_del\_infimo.lean}{Lean 4 Web}

\section{En los retículos, x ⊔ y = y ⊔ x}
\label{sec:org4dab74d}
\begin{verbatim}
-- ---------------------------------------------------------------------
-- Demostrar que en los retículos se verifica que
--    x ⊔ y = y ⊔ x
-- para todo x e y en el retículo.
-- ----------------------------------------------------------------------

-- Demostración en lenguaje natural
-- ================================

-- Es consecuencia del siguiente lema auxiliar
--    (∀ a, b)[a ⊔ b ≤ b ⊔ a]                                         (1)
-- En efecto, sustituyendo en (1) a por x y b por y, se tiene
--    x ⊔ y ≤ y ⊔ x                                                   (2)
-- y sustituyendo en (1) a por y y b por x, se tiene
--    y ⊔ x ≤ x ⊔ y                                                   (3)
-- Finalmente, aplicando la propiedad antisimétrica de la divisibilidad
-- a (2) y (3), se tiene
--    x ⊔ y = y ⊔ x
--
-- Para demostrar (1), por la definición del supremo, basta demostrar
-- las siguientes relaciones
--    x ≤ y ⊔ x
--    y ≤ y ⊔ x
-- y ambas se tienen por la definición del supremo.

-- Demostraciones con Lean4
-- ========================

import Mathlib.Order.Lattice
variable {α : Type _} [Lattice α]
variable (x y z : α)

-- 1ª demostración del lema auxiliar
lemma aux : x ⊔ y ≤ y ⊔ x :=
by
  have h1 : x ≤ y ⊔ x :=
    le_sup_right
  have h2 : y ≤ y ⊔ x :=
    le_sup_left
  show x ⊔ y ≤ y ⊔ x
  exact sup_le h1 h2

-- 2ª demostración del lema auxiliar
example : x ⊔ y ≤ y ⊔ x :=
by
  apply sup_le
  { apply le_sup_right }
  { apply le_sup_left }

-- 3ª demostración del lema auxiliar
example : x ⊔ y ≤ y ⊔ x :=
sup_le le_sup_right le_sup_left

-- 1ª demostración
example : x ⊔ y = y ⊔ x :=
by
  have h1 : x ⊔ y ≤ y ⊔ x :=
    aux x y
  have h2 : y ⊔ x ≤ x ⊔ y :=
    aux y x
  show x ⊔ y = y ⊔ x
  exact le_antisymm h1 h2

-- 2ª demostración
example : x ⊔ y = y ⊔ x :=
by
  apply le_antisymm
  { apply aux }
  { apply aux }

-- 3ª demostración
example : x ⊔ y = y ⊔ x :=
le_antisymm (aux x y) (aux y x)

-- 4ª demostración
example : x ⊔ y = y ⊔ x :=
by apply le_antisymm; simp ; simp

-- 5ª demostración
example : x ⊔ y = y ⊔ x :=
-- by apply?
sup_comm

-- Lemas usados
-- ============

-- #check (le_antisymm : x ≤ y → y ≤ x → x = y)
-- #check (le_sup_left : x ≤ x ⊔ y)
-- #check (le_sup_right : y ≤ x ⊔ y)
-- #check (sup_comm : x ⊔ y = y ⊔ x)
-- #check (sup_le : x ≤ z → y ≤ z → x ⊔ y ≤ z)
\end{verbatim}
Se puede interactuar con las pruebas anteriores en \href{https://lean.math.hhu.de/\#url=https://raw.githubusercontent.com/jaalonso/Calculemus2/main/src/Conmutatividad\_del\_supremo.lean}{Lean 4 Web}

\section{En los retículos, (x ⊓ y) ⊓ z = x ⊓ (y ⊓ z)}
\label{sec:org56dbca5}
\begin{verbatim}
-- ---------------------------------------------------------------------
-- Demostrar que en los retículos se verifica que
--     (x ⊓ y) ⊓ z = x ⊓ (y ⊓ z)
-- ----------------------------------------------------------------------

-- Demostración en lenguaje natural
-- ================================

-- En la demostración se usarán los siguientes lemas
--    le_antisymm : x ≤ y → y ≤ x → x = y
--    le_inf : z ≤ x → z ≤ y → z ≤ x ⊓ y
--    inf_le_left : x ⊓ y ≤ x
--    inf_le_right : x ⊓ y ≤ y)
--
-- Por le_antisym, es suficiente demostrar las siguientes relaciones:
--    (x ⊓ y) ⊓ z ≤ x ⊓ (y ⊓ z)                                      (1)
--    x ⊓ (y ⊓ z) ≤ (x ⊓ y) ⊓ z                                      (2)
--
-- Para demostrar (1), por le_inf, basta probar que
--    (x ⊓ y) ⊓ z ≤ x                                               (1a)
--    (x ⊓ y) ⊓ z ≤ y ⊓ z                                           (1b)
--
-- La (1a) se demuestra por la siguiente cadena de desigualdades
--    (x ⊓ y) ⊓ z ≤ x ⊓ y   [por inf_le_left]
--                ≤ x       [por inf_le_left]
--
-- Para demostrar (1b), por le_inf, basta probar que
--    (x ⊓ y) ⊓ z ≤ y                                              (1b1)
--    (x ⊓ y) ⊓ z ≤ z                                              (1b2)
--
-- La (1b1) se demuestra por la siguiente cadena de desigualdades
--    (x ⊓ y) ⊓ z ≤ x ⊓ y   [por inf_le_left]
--                ≤ y       [por inf_le_right]
--
-- La (1b2) se tiene por inf_le_right.
--
-- Para demostrar (2), por le_inf, basta probar que
--    x ⊓ (y ⊓ z) ≤ x ⊓ y                                           (2a)
--    x ⊓ (y ⊓ z) ≤ z                                               (2b)
--
-- Para demostrar (2a), por le_inf, basta probar que
--    x ⊓ (y ⊓ z) ≤ x                                              (2a1)
--    x ⊓ (y ⊓ z) ≤ y                                              (2a2)
--
-- La (2a1) se tiene por inf_le_left.
--
-- La (2a2) se demuestra por la siguiente cadena de desigualdades
--    x ⊓ (y ⊓ z) ≤ y ⊓ z   [por inf_le_right]
--                ≤ y       [por inf_le_left]
--
-- La (2b) se demuestra por la siguiente cadena de desigualdades
--    x ⊓ (y ⊓ z) ≤ y ⊓ z   [por inf_le_right]
--                ≤ z       [por inf_le_right]

-- Demostraciones con Lean4
-- ========================

import Mathlib.Order.Lattice
variable {α : Type _} [Lattice α]
variable (x y z : α)

-- 1ª demostración
-- ===============

example : (x ⊓ y) ⊓ z = x ⊓ (y ⊓ z) :=
by
  have h1 : (x ⊓ y) ⊓ z ≤ x ⊓ (y ⊓ z) := by
  { have h1a : (x ⊓ y) ⊓ z ≤ x := calc
      (x ⊓ y) ⊓ z ≤ x ⊓ y := by exact inf_le_left
                _ ≤ x     := by exact inf_le_left
    have h1b : (x ⊓ y) ⊓ z ≤ y ⊓ z := by
    { have h1b1 : (x ⊓ y) ⊓ z ≤ y := calc
        (x ⊓ y) ⊓ z ≤ x ⊓ y := by exact inf_le_left
                  _ ≤ y     := by exact inf_le_right
      have h1b2 : (x ⊓ y) ⊓ z ≤ z :=
        inf_le_right
      show (x ⊓ y) ⊓ z ≤ y ⊓ z
      exact le_inf h1b1 h1b2 }
    show (x ⊓ y) ⊓ z ≤ x ⊓ (y ⊓ z)
    exact le_inf h1a h1b }
  have h2 : x ⊓ (y ⊓ z) ≤ (x ⊓ y) ⊓ z := by
  { have h2a : x ⊓ (y ⊓ z) ≤ x ⊓ y := by
    { have h2a1 : x ⊓ (y ⊓ z) ≤ x :=
        inf_le_left
      have h2a2 : x ⊓ (y ⊓ z) ≤ y := calc
        x ⊓ (y ⊓ z) ≤ y ⊓ z := by exact inf_le_right
                  _ ≤ y     := by exact inf_le_left
      show x ⊓ (y ⊓ z) ≤ x ⊓ y
      exact le_inf h2a1 h2a2 }
    have h2b : x ⊓ (y ⊓ z) ≤ z := by calc
      x ⊓ (y ⊓ z) ≤ y ⊓ z := by exact inf_le_right
                _ ≤ z     := by exact inf_le_right
    show x ⊓ (y ⊓ z) ≤ (x ⊓ y) ⊓ z
    exact le_inf h2a h2b }
  show (x ⊓ y) ⊓ z = x ⊓ (y ⊓ z)
  exact le_antisymm h1 h2

-- 2ª demostración
-- ===============

example : x ⊓ y ⊓ z = x ⊓ (y ⊓ z) := by
  apply le_antisymm
  · apply le_inf
    · apply le_trans
      apply inf_le_left
      apply inf_le_left
    . apply le_inf
      · apply le_trans
        apply inf_le_left
        apply inf_le_right
      . apply inf_le_right
  . apply le_inf
    · apply le_inf
      · apply inf_le_left
      . apply le_trans
        apply inf_le_right
        apply inf_le_left
    . apply le_trans
      apply inf_le_right
      apply inf_le_right

-- 3ª demostración
-- ===============

example : (x ⊓ y) ⊓ z = x ⊓ (y ⊓ z) :=
by
  apply le_antisymm
  . apply le_inf
    . apply inf_le_of_left_le inf_le_left
    . apply le_inf (inf_le_of_left_le inf_le_right) inf_le_right
  . apply le_inf
    . apply le_inf inf_le_left (inf_le_of_right_le inf_le_left)
    . apply inf_le_of_right_le inf_le_right

-- 4ª demostración
-- ===============

example : (x ⊓ y) ⊓ z = x ⊓ (y ⊓ z) :=
le_antisymm
  (le_inf
    (inf_le_of_left_le inf_le_left)
    (le_inf (inf_le_of_left_le inf_le_right) inf_le_right))
  (le_inf
    (le_inf inf_le_left (inf_le_of_right_le inf_le_left))
    (inf_le_of_right_le inf_le_right))

-- 5ª demostración
-- ===============

example : (x ⊓ y) ⊓ z = x ⊓ (y ⊓ z) :=
-- by apply?
inf_assoc

-- Lemas usados
-- ============

-- #check (inf_assoc : (x ⊓ y) ⊓ z = x ⊓ (y ⊓ z))
-- #check (inf_le_left : x ⊓ y ≤ x)
-- #check (inf_le_of_left_le : x ≤ z → x ⊓ y ≤ z)
-- #check (inf_le_of_right_le : y ≤ z → x ⊓ y ≤ z)
-- #check (inf_le_right : x ⊓ y ≤ y)
-- #check (le_antisymm : x ≤ y → y ≤ x → x = y)
-- #check (le_inf : z ≤ x → z ≤ y → z ≤ x ⊓ y)
-- #check (le_trans : x ≤ y → y ≤ z → x ≤ z)
\end{verbatim}
Se puede interactuar con las pruebas anteriores en \href{https://lean.math.hhu.de/\#url=https://raw.githubusercontent.com/jaalonso/Calculemus2/main/src/Asociatividad\_del\_infimo.lean}{Lean 4 Web}

\section{En los retículos, (x ⊔ y) ⊔ z = x ⊔ (y ⊔ z)}
\label{sec:org670f323}
\begin{verbatim}
-- ---------------------------------------------------------------------
-- Demostrar que en los retículos se verifica que
--    (x ⊔ y) ⊔ z = x ⊔ (y ⊔ z)
-- ---------------------------------------------------------------------

-- Demostración en lenguaje natural
-- ================================

-- En la demostración se usarán los siguientes lemas
--    le_antisymm  : x ≤ y → y ≤ x → x = y
--    le_sup_left  : x ≤ x ⊔ y
--    le_sup_right : y ≤ x ⊔ y
--    sup_le       : x ≤ z → y ≤ z → x ⊔ y ≤ z
--
-- Por le_antisymm, basta demostrar las siguientes relaciones:
--    (x ⊔ y) ⊔ z ≤ x ⊔ (y ⊔ z)                                      (1)
--    x ⊔ (y ⊔ z) ≤ (x ⊔ y) ⊔ z                                      (2)
--
-- Para demostrar (1), por sup_le, basta probar
--    x ⊔ y ≤ x ⊔ (y ⊔ z)                                           (1a)
--    z ≤ x ⊔ (y ⊔ z)                                               (1b)
--
-- Para demostrar (1a), por sup_le, basta probar
--    x ≤ x ⊔ (y ⊔ z)                                              (1a1)
--    y ≤ x ⊔ (y ⊔ z)                                              (1a2)
--
-- La (1a1) se tiene por le_sup_left.
--
-- La (1a2) se tiene por la siguiente cadena de desigualdades:
--    y ≤ y ⊔ z          [por le_sup_left]
--      ≤ x ⊔ (y ⊔ z)    [por le_sup_right]
--
-- La (1b) se tiene por la siguiente cadena de desigualdades
--    z ≤ y ⊔ z          [por le_sup_right]
--      ≤ x ⊔ (y ⊔ z)    [por le_sup_right]
--
-- Para demostrar (2), por sup_le, basta probar
--    x ≤ (x ⊔ y) ⊔ z                                               (2a)
--    y ⊔ z ≤ (x ⊔ y) ⊔ z                                           (2b)
--
-- La (2a) se demuestra por la siguiente cadena de desigualdades:
--    x ≤ x ⊔ y          [por le_sup_left]
--      ≤ (x ⊔ y) ⊔ z    [por le_sup_left]
--
-- Para demostrar (2b), por sup_le, basta probar
--    y ≤ (x ⊔ y) ⊔ z                                              (2b1)
--    z ≤ (x ⊔ y) ⊔ z                                              (2b2)
--
-- La (2b1) se demuestra por la siguiente cadena de desigualdades:
--    y ≤ x ⊔ y          [por le_sup_right]
--      ≤ (x ⊔ y) ⊔ z    [por le_sup_left]
--
-- La (2b2) se tiene por le_sup_right.

-- Demostraciones con Lean 4
-- =========================

import Mathlib.Order.Lattice

variable {α : Type _} [Lattice α]
variable (x y z : α)

-- 1ª demostración
-- ===============

example : (x ⊔ y) ⊔ z = x ⊔ (y ⊔ z) :=
by
  have h1 : (x ⊔ y) ⊔ z ≤ x ⊔ (y ⊔ z) := by
  { have h1a : x ⊔ y ≤ x ⊔ (y ⊔ z) := by
    { have h1a1 : x ≤ x ⊔ (y ⊔ z) := by exact le_sup_left
      have h1a2 : y ≤ x ⊔ (y ⊔ z) := calc
        y ≤ y ⊔ z       := by exact le_sup_left
        _ ≤ x ⊔ (y ⊔ z) := by exact le_sup_right
      show x ⊔ y ≤ x ⊔ (y ⊔ z)
      exact sup_le h1a1 h1a2 }
    have h1b : z ≤ x ⊔ (y ⊔ z) := calc
      z ≤ y ⊔ z       := by exact le_sup_right
      _ ≤ x ⊔ (y ⊔ z) := by exact le_sup_right
    show (x ⊔ y) ⊔ z ≤ x ⊔ (y ⊔ z)
    exact sup_le h1a h1b }
  have h2 : x ⊔ (y ⊔ z) ≤ (x ⊔ y) ⊔ z := by
  { have h2a : x ≤ (x ⊔ y) ⊔ z := calc
      x ≤ x ⊔ y       := by exact le_sup_left
      _ ≤ (x ⊔ y) ⊔ z := by exact le_sup_left
    have h2b : y ⊔ z ≤ (x ⊔ y) ⊔ z := by
    { have h2b1 : y ≤ (x ⊔ y) ⊔ z := calc
        y ≤ x ⊔ y       := by exact le_sup_right
        _ ≤ (x ⊔ y) ⊔ z := by exact le_sup_left
      have h2b2 : z ≤ (x ⊔ y) ⊔ z := by
        exact le_sup_right
      show  y ⊔ z ≤ (x ⊔ y) ⊔ z
      exact sup_le h2b1 h2b2 }
    show x ⊔ (y ⊔ z) ≤ (x ⊔ y) ⊔ z
    exact sup_le h2a h2b }
  show (x ⊔ y) ⊔ z = x ⊔ (y ⊔ z)
  exact le_antisymm h1 h2

-- 2ª demostración
-- ===============

example : x ⊔ y ⊔ z = x ⊔ (y ⊔ z) :=
by
  apply le_antisymm
  · -- (x ⊔ y) ⊔ z ≤ x ⊔ (y ⊔ z)
    apply sup_le
    · -- x ⊔ y ≤ x ⊔ (y ⊔ z)
      apply sup_le
      . -- x ≤ x ⊔ (y ⊔ z)
        apply le_sup_left
      · -- y ≤ x ⊔ (y ⊔ z)
        apply le_trans
        . -- y ≤ y ⊔ z
          apply @le_sup_left _ _ y z
        . -- y ⊔ z ≤ x ⊔ (y ⊔ z)
          apply le_sup_right
    . -- z ≤ x ⊔ (y ⊔ z)
      apply le_trans
      . -- z ≤ x ⊔ (y ⊔ z)
        apply @le_sup_right _ _ y z
      . -- y ⊔ z ≤ x ⊔ (y ⊔ z)
        apply le_sup_right
  . -- x ⊔ (y ⊔ z) ≤ (x ⊔ y) ⊔ z
    apply sup_le
    · -- x ≤ (x ⊔ y) ⊔ z
      apply le_trans
      . -- x ≤ x ⊔ y
        apply @le_sup_left _ _ x y
      . -- x ⊔ y ≤ (x ⊔ y) ⊔ z
        apply le_sup_left
    . -- y ⊔ z ≤ (x ⊔ y) ⊔ z
      apply sup_le
      · -- y ≤ (x ⊔ y) ⊔ z
        apply le_trans
        . -- y ≤ x ⊔ y
          apply @le_sup_right _ _ x y
        . -- x ⊔ y ≤ (x ⊔ y) ⊔ z
          apply le_sup_left
      . -- z ≤ (x ⊔ y) ⊔ z
        apply le_sup_right

-- 3ª demostración
-- ===============

example : x ⊔ y ⊔ z = x ⊔ (y ⊔ z) :=
by
  apply le_antisymm
  · apply sup_le
    · apply sup_le
      . apply le_sup_left
      · apply le_trans
        . apply @le_sup_left _ _ y z
        . apply le_sup_right
    . apply le_trans
      . apply @le_sup_right _ _ y z
      . apply le_sup_right
  . apply sup_le
    · apply le_trans
      . apply @le_sup_left _ _ x y
      . apply le_sup_left
    . apply sup_le
      · apply le_trans
        . apply @le_sup_right _ _ x y
        . apply le_sup_left
      . apply le_sup_right

-- 4ª demostración
-- ===============

example : (x ⊔ y) ⊔ z = x ⊔ (y ⊔ z) :=
by
  apply le_antisymm
  . -- (x ⊔ y) ⊔ z ≤ x ⊔ (y ⊔ z)
    apply sup_le
    . -- x ⊔ y ≤ x ⊔ (y ⊔ z)
      apply sup_le le_sup_left (le_sup_of_le_right le_sup_left)
    . -- z ≤ x ⊔ (y ⊔ z)
      apply le_sup_of_le_right le_sup_right
  . -- x ⊔ (y ⊔ z) ≤ (x ⊔ y) ⊔ z
    apply sup_le
    . -- x ≤ (x ⊔ y) ⊔ z
      apply le_sup_of_le_left le_sup_left
    . -- y ⊔ z ≤ (x ⊔ y) ⊔ z
      apply sup_le (le_sup_of_le_left le_sup_right) le_sup_right

-- 5ª demostración
-- ===============

example : (x ⊔ y) ⊔ z = x ⊔ (y ⊔ z) :=
by
  apply le_antisymm
  . apply sup_le
    . apply sup_le le_sup_left (le_sup_of_le_right le_sup_left)
    . apply le_sup_of_le_right le_sup_right
  . apply sup_le
    . apply le_sup_of_le_left le_sup_left
    . apply sup_le (le_sup_of_le_left le_sup_right) le_sup_right

-- 6ª demostración
-- ===============

example : (x ⊔ y) ⊔ z = x ⊔ (y ⊔ z) :=
le_antisymm
  (sup_le
    (sup_le le_sup_left (le_sup_of_le_right le_sup_left))
    (le_sup_of_le_right le_sup_right))
  (sup_le
    (le_sup_of_le_left le_sup_left)
    (sup_le (le_sup_of_le_left le_sup_right) le_sup_right))

-- 7ª demostración
-- ===============

example : (x ⊔ y) ⊔ z = x ⊔ (y ⊔ z) :=
-- by apply?
sup_assoc

-- Lemas usados
-- ============

-- #check (le_antisymm : x ≤ y → y ≤ x → x = y)
-- #check (le_sup_left : x ≤ x ⊔ y)
-- #check (le_sup_of_le_left : z ≤ x → z ≤ x ⊔ y)
-- #check (le_sup_of_le_right : z ≤ y → z ≤ x ⊔ y)
-- #check (le_sup_right : y ≤ x ⊔ y)
-- #check (le_trans : x ≤ y → y ≤ z → x ≤ z)
-- #check (sup_assoc : (x ⊔ y) ⊔ z = x ⊔ (y ⊔ z))
-- #check (sup_le : x ≤ z → y ≤ z → x ⊔ y ≤ z)
\end{verbatim}
Se puede interactuar con las pruebas anteriores en \href{https://lean.math.hhu.de/\#url=https://raw.githubusercontent.com/jaalonso/Calculemus2/main/src/Asociatividad\_del\_supremo.lean}{Lean 4 Web}

\section{En los retículos, x ⊓ (x ⊔ y) = x}
\label{sec:orgf9ef02f}
\begin{verbatim}
-- ---------------------------------------------------------------------
-- Demostrar que en los retículos se verifica que
--    x ⊓ (x ⊔ y) = x
-- ---------------------------------------------------------------------

-- Demostración en lenguaje natural
-- ================================

-- En la demostración se usarán los siguientes lemas
--    le_antisymm  : x ≤ y → y ≤ x → x = y
--    inf_le_left  : x ⊓ y ≤ x
--    le_inf       : z ≤ x → z ≤ y → z ≤ x ⊓ y
--    le_rfl       : x ≤ x
--    le_sup_left  : x ≤ x ⊔ y
--
-- Por le_antisymm, basta demostrar las siguientes relaciones:
--    x ⊓ (x ⊔ y) ≤ x                                                (1)
--    x ≤ x ⊓ (x ⊔ y)                                                (2)
--
-- La (1) se tiene por inf_le_left.
--
-- Para demostrar la (2), por le_inf, basta probar las relaciones:
--    x ≤ x                                                         (2a)
--    x ≤ x ⊔ y                                                     (2b)
--
-- La (2a) se tiene por le_rfl.
--
-- La (2b) se tiene por le_sup_left

-- Demostraciones con Lean4
-- ========================

import Mathlib.Order.Lattice
variable {α : Type _} [Lattice α]
variable (x y : α)

-- 1ª demostración
-- ===============

example : x ⊓ (x ⊔ y) = x :=
by
  have h1 : x ⊓ (x ⊔ y) ≤ x := inf_le_left
  have h2 : x ≤ x ⊓ (x ⊔ y)
  { have h2a : x ≤ x := le_rfl
    have h2b : x ≤ x ⊔ y := le_sup_left
    show x ≤ x ⊓ (x ⊔ y)
    exact le_inf h2a h2b }
  show x ⊓ (x ⊔ y) = x
  exact le_antisymm h1 h2

-- 2ª demostración
-- ===============

example : x ⊓ (x ⊔ y) = x :=
by
  have h1 : x ⊓ (x ⊔ y) ≤ x := by simp
  have h2 : x ≤ x ⊓ (x ⊔ y) := by simp
  show x ⊓ (x ⊔ y) = x
  exact le_antisymm h1 h2

-- 3ª demostración
-- ===============

example : x ⊓ (x ⊔ y) = x :=
by
  apply le_antisymm
  . -- x ⊓ (x ⊔ y) ≤ x
    apply inf_le_left
  . -- x ≤ x ⊓ (x ⊔ y)
    apply le_inf
    . -- x ≤ x
      apply le_rfl
    . -- x ≤ x ⊔ y
      apply le_sup_left

-- 4ª demostración
-- ===============

example : x ⊓ (x ⊔ y) = x :=
le_antisymm inf_le_left (le_inf le_rfl le_sup_left)

-- 5ª demostración
-- ===============

example : x ⊓ (x ⊔ y) = x :=
-- by apply?
inf_sup_self

-- 6ª demostración
-- ===============

example : x ⊓ (x ⊔ y) = x :=
by simp

-- Lemas usados
-- ============

-- variable (z : α)
-- #check (inf_le_left : x ⊓ y ≤ x)
-- #check (inf_sup_self : x ⊓ (x ⊔ y) = x)
-- #check (le_antisymm : x ≤ y → y ≤ x → x = y)
-- #check (le_inf : z ≤ x → z ≤ y → z ≤ x ⊓ y)
-- #check (le_rfl : x ≤ x)
-- #check (le_sup_left : x ≤ x ⊔ y)
\end{verbatim}
Se puede interactuar con las pruebas anteriores en \href{https://lean.math.hhu.de/\#url=https://raw.githubusercontent.com/jaalonso/Calculemus2/main/src/Leyes\_de\_absorcion\_1.lean}{Lean 4 Web}

\section{En los retículos, x ⊔ (x ⊓ y) = x}
\label{sec:org72b5c31}
\begin{verbatim}
-- ---------------------------------------------------------------------
-- Demostrar que en los retículos se verifica que
--    x ⊔ (x ⊓ y) = x
-- ---------------------------------------------------------------------

-- Demostración en lenguaje natural
-- ================================

-- En la demostración se usarán los siguientes lemas
--   le_antisymm  : x ≤ y → y ≤ x → x = y
--   inf_le_left  : x ⊓ y ≤ x
--   le_rfl       : x ≤ x
--   le_sup_left  : x ≤ x ⊔ y
--   sup_le       : x ≤ z → y ≤ z → x ⊔ y ≤ z
--
-- Por le_antisymm, basta demostrar las siguientes relaciones:
--    x ⊔ (x ⊓ y) ≤ x                                                (1)
--    x ≤ x ⊔ (x ⊓ y)    [que se tiene por le_sup_left]
--
-- Para demostrar (1), por sup_le, basta probar las relaciones:
--    x ≤ x              [que se tiene por le_rfl]
--    x ⊓ y ≤ x          [que se tiene por inf_le_left]

-- Demostraciones con Lean4
-- ========================

import Mathlib.Order.Lattice
variable {α : Type _} [Lattice α]--
variable (x y : α)

-- 1ª demostración
-- ===============

example : x ⊔ (x ⊓ y) = x :=
by
  have h1 : x ⊔ (x ⊓ y) ≤ x
  { have h1a : x ≤ x := le_rfl
    have h1b : x ⊓ y ≤ x := inf_le_left
    show x ⊔ (x ⊓ y) ≤ x
    exact sup_le h1a h1b }
  have h2 : x ≤ x ⊔ (x ⊓ y) := le_sup_left
  show x ⊔ (x ⊓ y) = x
  exact le_antisymm h1 h2

-- 2ª demostración
-- ===============

example : x ⊔ (x ⊓ y) = x :=
by
  have h1 : x ⊔ (x ⊓ y) ≤ x := by simp
  have h2 : x ≤ x ⊔ (x ⊓ y) := by simp
  show x ⊔ (x ⊓ y) = x
  exact le_antisymm h1 h2

-- 3ª demostración
-- ===============

example : x ⊔ (x ⊓ y) = x :=
by
  apply le_antisymm
  . -- x ⊔ (x ⊓ y) ≤ x
    apply sup_le
    . -- x ≤ x
      apply le_rfl
    . -- x ⊓ y ≤ x
      apply inf_le_left
  . -- x ≤ x ⊔ (x ⊓ y)
    apply le_sup_left

-- 4ª demostración
-- ===============

example : x ⊔ (x ⊓ y) = x :=
-- by apply?
sup_inf_self

-- 5ª demostración
-- ===============

example : x ⊔ (x ⊓ y) = x :=
by simp

-- Lemas usados
-- ============

-- variable (z : α)
-- #check (le_rfl : x ≤ x)
-- #check (inf_le_left : x ⊓ y ≤ x)
-- #check (sup_le : x ≤ z → y ≤ z → x ⊔ y ≤ z)
-- #check (le_sup_left : x ≤ x ⊔ y)
-- #check (le_antisymm : x ≤ y → y ≤ x → x = y)
-- #check (sup_inf_self : x ⊔ (x ⊓ y) = x)
\end{verbatim}
Se puede interactuar con las pruebas anteriores en \href{https://lean.math.hhu.de/\#url=https://raw.githubusercontent.com/jaalonso/Calculemus2/main/src/Leyes\_de\_absorcion\_2.lean}{Lean 4 Web}

\section{En los retículos, una distributiva del ínfimo implica la otra}
\label{sec:org971b79a}
\begin{verbatim}
-- ---------------------------------------------------------------------
-- Demostrar que si α es un retículo tal que
--    ∀ x y z : α, x ⊓ (y ⊔ z) = (x ⊓ y) ⊔ (x ⊓ z))
-- entonces
--    (a ⊔ b) ⊓ c = (a ⊓ c) ⊔ (b ⊓ c)
-- para todos los elementos de α.
-- ----------------------------------------------------------------------

-- Demostración en lenguaje natural
-- ================================

-- Se demuestra por la siguiente cadena de igualdades
--    (a ⊔ b) ⊓ c = c ⊓ (a ⊔ b)          [por conmutatividad de ⊓]
--                = (c ⊓ a) ⊔ (c ⊓ b)    [por la hipótesis]
--                = (a ⊓ c) ⊔ (c ⊓ b)    [por conmutatividad de ⊓]
--                = (a ⊓ c) ⊔ (b ⊓ c)    [por conmutatividad de ⊓]

-- Demostraciones con Lean4
-- ========================

import Mathlib.Order.Lattice
variable {α : Type _} [Lattice α]
variable (a b c : α)

-- 1ª demostración
example
  (h : ∀ x y z : α, x ⊓ (y ⊔ z) = (x ⊓ y) ⊔ (x ⊓ z))
  : (a ⊔ b) ⊓ c = (a ⊓ c) ⊔ (b ⊓ c) :=
calc
  (a ⊔ b) ⊓ c = c ⊓ (a ⊔ b)       := by rw [inf_comm]
            _ = (c ⊓ a) ⊔ (c ⊓ b) := by rw [h]
            _ = (a ⊓ c) ⊔ (c ⊓ b) := by rw [@inf_comm _ _ c a]
            _ = (a ⊓ c) ⊔ (b ⊓ c) := by rw [@inf_comm _ _ c b]

-- 2ª demostración
example
  (h : ∀ x y z : α, x ⊓ (y ⊔ z) = (x ⊓ y) ⊔ (x ⊓ z))
  : (a ⊔ b) ⊓ c = (a ⊓ c) ⊔ (b ⊓ c) :=
by simp [h, inf_comm]

-- Lemas usados
-- ============

-- #check (inf_comm : a ⊓ b = b ⊓ a)
\end{verbatim}
Se puede interactuar con las pruebas anteriores en \href{https://lean.math.hhu.de/\#url=https://raw.githubusercontent.com/jaalonso/Calculemus2/main/src/Propiedad\_distributiva\_1.lean}{Lean 4 Web}

\section{En los retículos, una distributiva del supremos implica la otra}
\label{sec:orgfe8342b}
\begin{verbatim}
-- ---------------------------------------------------------------------
-- Demostrar que si α es un retículo tal que
--    ∀ x y z : α, x ⊔ (y ⊓ z) = (x ⊔ y) ⊓ (x ⊔ z)
-- entonces
--    (a ⊓ b) ⊔ c = (a ⊔ c) ⊓ (b ⊔ c)
-- ----------------------------------------------------------------------

-- Demostración en lenguaje natural
-- ================================

-- Se demuestra por la siguiente cadena de igualdades
--    (a ⊓ b) ⊔ c = c ⊔ (a ⊓ b)          [por la conmutatividad de ⊔]
--                = (c ⊔ a) ⊓ (c ⊔ b)    [por la hipótesis]
--                = (a ⊔ c) ⊓ (c ⊔ b)    [por la conmutatividad de ⊔]
--                = (a ⊔ c) ⊓ (b ⊔ c)    [por la conmutatividad de ⊔]

-- Demostraciones con Lean4
-- ========================

import Mathlib.Order.Lattice
variable {α : Type _} [Lattice α]
variable (a b c : α)

-- 1ª demostración
example
  (h : ∀ x y z : α, x ⊔ (y ⊓ z) = (x ⊔ y) ⊓ (x ⊔ z))
  : (a ⊓ b) ⊔ c = (a ⊔ c) ⊓ (b ⊔ c) :=
calc
  (a ⊓ b) ⊔ c = c ⊔ (a ⊓ b)       := by rw [sup_comm]
            _ = (c ⊔ a) ⊓ (c ⊔ b) := by rw [h]
            _ = (a ⊔ c) ⊓ (c ⊔ b) := by rw [@sup_comm _ _ c a]
            _ = (a ⊔ c) ⊓ (b ⊔ c) := by rw [@sup_comm _ _ c b]

-- 2ª demostración
example
  (h : ∀ x y z : α, x ⊔ (y ⊓ z) = (x ⊔ y) ⊓ (x ⊔ z))
  : (a ⊓ b) ⊔ c = (a ⊔ c) ⊓ (b ⊔ c) :=
by simp [h, sup_comm]

-- Lemas usados
-- ============

-- #check (sup_comm : a ⊔ b = b ⊔ a)
\end{verbatim}
Se puede interactuar con las pruebas anteriores en \href{https://lean.math.hhu.de/\#url=https://raw.githubusercontent.com/jaalonso/Calculemus2/main/src/Propiedad\_distributiva\_2.lean}{Lean 4 Web}

\chapter{Anillos ordenados}
\label{sec:org8ceb10e}

\section{En los anillos ordenados, a ≤ b → 0 ≤ b - a}
\label{sec:org6695632}
\begin{verbatim}
-- ---------------------------------------------------------------------
-- Demostrar que en los anillos ordenados se verifica que
--    a ≤ b → 0 ≤ b - a
-- ----------------------------------------------------------------------

-- Demostración en lenguaje natural
-- ================================

-- Se usarán los siguientes lemas:
--    sub_self         : a - a = 0
--    sub_le_sub_right : a ≤ b → ∀ (c : R), a - c ≤ b - c
--
-- Supongamos que
--    a ≤ b                                                          (1)
-- La demostración se tiene por la siguiente cadena de desigualdades:
--    0 = a - a    [por sub_self]
--      ≤ b - a    [por (1) y sub_le_sub_right]

-- Demostraciones con Lean4
-- ========================

import Mathlib.Algebra.Order.Ring.Defs
variable {R : Type _} [StrictOrderedRing R]
variable (a b c : R)

-- 1ª demostración
example : a ≤ b → 0 ≤ b - a :=
by
  intro h
  calc
    0 = a - a := (sub_self a).symm
    _ ≤ b - a := sub_le_sub_right h a

-- 2ª demostración
example : a ≤ b → 0 ≤ b - a :=
sub_nonneg.mpr

-- 3ª demostración
example : a ≤ b → 0 ≤ b - a :=
by simp

-- Lemas usados
-- ============

-- #check (sub_le_sub_right : a ≤ b → ∀ (c : R), a - c ≤ b - c)
-- #check (sub_nonneg : 0 ≤ a - b ↔ b ≤ a)
-- #check (sub_self a : a - a = 0)
\end{verbatim}
Se puede interactuar con las pruebas anteriores en \href{https://lean.math.hhu.de/\#url=https://raw.githubusercontent.com/jaalonso/Calculemus2/main/src/Ejercicio\_sobre\_anillos\_ordenados\_1.lean}{Lean 4 Web}

\section{En los anillos ordenados, 0 ≤ b - a → a ≤ b}
\label{sec:org07647ae}
\begin{verbatim}
-- ---------------------------------------------------------------------
-- Demostrar que en los anillos ordenados
--    0 ≤ b - a → a ≤ b
-- ----------------------------------------------------------------------

-- Demostración en lenguaje natural
-- ================================

-- Se usarán los siguientes lemas:
--    zero_add a : 0 + a = a
--    add_le_add_right : b ≤ c → ∀ (a : R),  b + a ≤ c + a
--    sub_add_cancel a b : a - b + b = -a
-- Supongamos que
--    0 ≤ b - a                                                      (1)
-- La demostración se tiene por la siguiente cadena de desigualdades:
--    a = 0 + a          [por zero_add]
--      ≤ (b - a) + a    [por (1) y add_le_add_right]
--      = b              [por sub_add_cancel]

-- Demostraciones con Lean4
-- ========================

import Mathlib.Algebra.Order.Ring.Defs
variable {R : Type _} [StrictOrderedRing R]
variable (a b c : R)

-- 1ª demostración
-- ===============

example : 0 ≤ b - a → a ≤ b :=
by
  intro h
  calc
    a = 0 + a       := (zero_add a).symm
    _ ≤ (b - a) + a := add_le_add_right h a
    _ = b           := sub_add_cancel b a

-- 2ª demostración
-- ===============

example : 0 ≤ b - a → a ≤ b :=
-- by apply?
sub_nonneg.mp

-- 3ª demostración
-- ===============

example : 0 ≤ b - a → a ≤ b :=
by simp

-- Lemas usados
-- ============

-- #check (zero_add a : 0 + a = a)
-- #check (add_le_add_right : b ≤ c → ∀ (a : R),  b + a ≤ c + a)
-- #check (sub_add_cancel a b : a - b + b = a)
-- #check (sub_nonneg : 0 ≤ a - b ↔ b ≤ a)
\end{verbatim}
Se puede interactuar con las pruebas anteriores en \href{https://lean.math.hhu.de/\#url=https://raw.githubusercontent.com/jaalonso/Calculemus2/main/src/Ejercicio\_sobre\_anillos\_ordenados\_2.lean}{Lean 4 Web}

\section{En los anillos ordenados, \{a ≤ b, 0 ≤ c\} ⊢ ac ≤ bc}
\label{sec:org401fb94}
\begin{verbatim}
-- ---------------------------------------------------------------------
-- Demostrar que, en los anillos ordenados, si
--    a ≤ b
--    0 ≤ c
-- entonces
--    a * c ≤ b * c
-- ----------------------------------------------------------------------

-- Demostración en lenguaje natural
-- ================================

-- Se usarán los siguientes lemas:
--    sub_nonneg                 : 0 ≤ a - b ↔ b ≤ a)
--    mul_nonneg                 : 0 ≤ a → 0 ≤ b → 0 ≤ a * b)
--    sub_mul a b c              : (a - b) * c = a * c - b * c)
--
-- Supongamos que
--    a ≤ b                                                          (1)
--    0 ≤ c
-- De (1), por sub_nonneg, se tiene
--    0 ≤ b - a
-- y con (2), por mul_nonneg, se tiene
--    0 ≤ (b - a) * c
-- que, por sub_mul, da
--    0 ≤ b * c - a * c
-- y, aplicándole sub_nonneg, se tiene
--    a * c ≤ b * c

-- Demostraciones con Lean4
-- ========================

import Mathlib.Algebra.Order.Ring.Defs
variable {R : Type _} [StrictOrderedRing R]
variable (a b c : R)

-- 1ª demostración
-- ===============

example
  (h1 : a ≤ b)
  (h2 : 0 ≤ c)
  : a * c ≤ b * c :=
by
  have h3 : 0 ≤ b - a :=
    sub_nonneg.mpr h1
  have h4 : 0 ≤ b * c - a * c := calc
    0 ≤ (b - a) * c   := mul_nonneg h3 h2
    _ = b * c - a * c := sub_mul b a c
  show a * c ≤ b * c
  exact sub_nonneg.mp h4

-- 2ª demostración
-- ===============

example
  (h1 : a ≤ b)
  (h2 : 0 ≤ c)
  : a * c ≤ b * c :=
by
  have h3 : 0 ≤ b - a := sub_nonneg.mpr h1
  have h4 : 0 ≤ (b - a) * c := mul_nonneg h3 h2
  -- h4 : 0 ≤ b * c - a * c
  rw [sub_mul] at h4
  -- a * c ≤ b * c
  exact sub_nonneg.mp h4

-- 3ª demostración
-- ===============

example
  (h1 : a ≤ b)
  (h2 : 0 ≤ c)
  : a * c ≤ b * c :=
by
  -- 0 ≤ b * c - a * c
  apply sub_nonneg.mp
  -- 0 ≤ (b - a) * c
  rw [← sub_mul]
  apply mul_nonneg
  . -- 0 ≤ b - a
    exact sub_nonneg.mpr h1
  . -- 0 ≤ c
    exact h2

-- 4ª demostración
-- ===============

example
  (h1 : a ≤ b)
  (h2 : 0 ≤ c)
  : a * c ≤ b * c :=
by
  apply sub_nonneg.mp
  rw [← sub_mul]
  apply mul_nonneg (sub_nonneg.mpr h1) h2

-- 5ª demostración
example
  (h1 : a ≤ b)
  (h2 : 0 ≤ c)
  : a * c ≤ b * c :=
-- by apply?
mul_le_mul_of_nonneg_right h1 h2

-- Lemas usados
-- ============

-- #check (mul_le_mul_of_nonneg_right : a ≤ b → 0 ≤ c → a * c ≤ b * c)
-- #check (mul_nonneg : 0 ≤ a → 0 ≤ b → 0 ≤ a * b)
-- #check (sub_mul a b c : (a - b) * c = a * c - b * c)
-- #check (sub_nonneg : 0 ≤ a - b ↔ b ≤ a)
\end{verbatim}
Se puede interactuar con las pruebas anteriores en \href{https://lean.math.hhu.de/\#url=https://raw.githubusercontent.com/jaalonso/Calculemus2/main/src/Ejercicio\_sobre\_anillos\_ordenados\_3.lean}{Lean 4 Web}

\chapter{Espacios métricos}
\label{sec:org7a829d9}

\section{En los espacios métricos, dist(x,y) ≥ 0}
\label{sec:org7d5f6b0}
\begin{verbatim}
-- ---------------------------------------------------------------------
-- Ejercicio. Demostrar que en los espacios métricos
--    0 ≤ dist x y
-- ----------------------------------------------------------------------

-- Demostración en lenguaje natural
-- ================================

-- Se usarán los siguientes lemas:
--    dist_comm x y             : dist x y = dist y x
--    dist_self x               : dist x x = 0
--    dist_triangle x y z       : dist x z ≤ dist x y + dist y z
--    mul_two a                 : a * 2 = a + a
--    nonneg_of_mul_nonneg_left : 0 ≤ a * b → 0 < b → 0 ≤ a
--    zero_lt_two               : 0 < 2
--
-- Por nonneg_of_mul_nonneg_left es suficiente demostrar las siguientes
-- desigualdades:
--    0 ≤ dist x y * 2                                               (1)
--    0 < 2                                                          (2)
--
-- La (1) se demuestra por las siguiente cadena de desigualdades:
--    0 = dist x x               [por dist_self]
--      ≤ dist x y + dist y x    [por dist_triangle]
--      = dist x y + dist x y    [por dist_comm]
--      = dist x y * 2           [por mul_two]
--
-- La (2) se tiene por zero_lt_two.

-- Demostraciones con Lean4
-- ========================

import Mathlib.Topology.MetricSpace.Basic
variable {X : Type _} [MetricSpace X]
variable (x y : X)

-- 1ª demostración
example : 0 ≤ dist x y :=
by
  have h1 : 0 ≤ dist x y * 2 := calc
    0 = dist x x            := (dist_self x).symm
    _ ≤ dist x y + dist y x := dist_triangle x y x
    _ = dist x y + dist x y := by rw [dist_comm x y]
    _ = dist x y * 2        := (mul_two (dist x y)).symm
  show 0 ≤ dist x y
  exact nonneg_of_mul_nonneg_left h1 zero_lt_two

-- 2ª demostración
example : 0 ≤ dist x y :=
by
  apply nonneg_of_mul_nonneg_left
  . -- 0 ≤ dist x y * 2
    calc 0 = dist x x            := by simp only [dist_self]
         _ ≤ dist x y + dist y x := by simp only [dist_triangle]
         _ = dist x y + dist x y := by simp only [dist_comm]
         _ = dist x y * 2        := by simp only [mul_two]
  . -- 0 < 2
    exact zero_lt_two

-- 3ª demostración
example : 0 ≤ dist x y :=
by
  have : 0 ≤ dist x y + dist y x := by
    rw [← dist_self x]
    apply dist_triangle
  linarith [dist_comm x y]

-- 3ª demostración
example : 0 ≤ dist x y :=
-- by apply?
dist_nonneg

-- Lemas usados
-- ============

-- variable (a b : ℝ)
-- variable (z : X)
-- #check (dist_comm x y : dist x y = dist y x)
-- #check (dist_nonneg : 0 ≤ dist x y)
-- #check (dist_self x : dist x x = 0)
-- #check (dist_triangle x y z : dist x z ≤ dist x y + dist y z)
-- #check (mul_two a : a * 2 = a + a)
-- #check (nonneg_of_mul_nonneg_left : 0 ≤ a * b → 0 < b → 0 ≤ a)
-- #check (zero_lt_two : 0 < 2)
\end{verbatim}
Se puede interactuar con las pruebas anteriores en \href{https://lean.math.hhu.de/\#url=https://raw.githubusercontent.com/jaalonso/Calculemus2/main/src/Ejercicio\_en\_espacios\_metricos.lean}{Lean 4 Web}

\chapter{Funciones reales}
\label{sec:org82f8342}

\section{La suma de una cota superior de f y una cota superior de g es una cota superior de f+g}
\label{sec:orgc090bd3}
\begin{verbatim}
-- ---------------------------------------------------------------------
-- Demostrar que la suma de una cota superior de f y una cota superior
-- de g es una cota superior de f + g.
-- ----------------------------------------------------------------------

-- Demostración en lenguaje natural
-- ================================

-- Se usará el siguiente lema
--    add_le_add : a ≤ b → c ≤ d → a + c ≤ b + d
--
-- Por la definición de cota superior, hay que demostrar que
--    (∀ x ∈ ℝ) [f(x) + g(x) ≤ a + b]                                  (1)
-- Para ello, sea x ∈ R. Puesto que es a es una cota superior de f, se
-- tiene que
--    f(x) ≤ a                                                         (2)
-- y, puesto que b es una cota superior de g, se tiene que
--    g(x) ≤ b                                                         (3)
-- De (2) y (3), por add_le_add, se tiene que
--    f(x) + g(x) ≤ a + b
-- que es lo que había que demostrar.

-- Demostraciones con Lean4
-- ========================

import Mathlib.Data.Real.Basic

-- (CotaSuperior f a) se verifica si a es una cota superior de f.
def CotaSuperior (f : ℝ → ℝ) (a : ℝ) : Prop :=
  ∀ x, f x ≤ a

variable {f g : ℝ → ℝ}
variable {a b : ℝ}

-- 1ª demostración
-- ===============

example
  (hfa : CotaSuperior f a)
  (hgb : CotaSuperior g b)
  : CotaSuperior (f + g) (a + b) :=
by
  have h1 : ∀ x, (f + g) x  ≤ a + b := by
  { intro x
    have h2 : f x ≤ a := hfa x
    have h3 : g x ≤ b := hgb x
    show (f + g) x ≤ a + b
    exact add_le_add h2 h3 }
  show CotaSuperior (f + g) (a + b)
  exact h1

-- 2ª demostración
-- ===============

example
  (hfa : CotaSuperior f a)
  (hgb : CotaSuperior g b)
  : CotaSuperior (f + g) (a + b) :=
by
  have h1 : ∀ x, (f + g) x ≤ a + b := by
  { intro x
    show (f + g) x ≤ a + b
    exact add_le_add (hfa x) (hgb x) }
  show CotaSuperior (f + g) (a + b)
  exact h1

-- 3ª demostración
-- ===============

example
  (hfa : CotaSuperior f a)
  (hgb : CotaSuperior g b)
  : CotaSuperior (f + g) (a + b) :=
by
  intro x
  dsimp
  apply add_le_add
  . apply hfa
  . apply hgb

-- 4ª demostración
-- ===============

theorem sumaCotaSup
  (hfa : CotaSuperior f a)
  (hgb : CotaSuperior g b)
  : CotaSuperior (f + g) (a + b) :=
λ x ↦ add_le_add (hfa x) (hgb x)

-- Lemas usados
-- ============

-- variable (c d : ℝ)
-- #check (add_le_add : a ≤ b → c ≤ d → a + c ≤ b + d)
\end{verbatim}
Se puede interactuar con las pruebas anteriores en \href{https://lean.math.hhu.de/\#url=https://raw.githubusercontent.com/jaalonso/Calculemus2/main/src/Suma\_de\_cotas\_superiores.lean}{Lean 4 Web}

\section{La suma de una cota inferior de f y una cota inferior de g es una cota inferior de f+g}
\label{sec:org4f3e242}
\begin{verbatim}
-- ---------------------------------------------------------------------
-- Demostrar que la suma de una cota inferior de f y una  cota inferior
-- de g es una cota inferior de f + g.
-- ----------------------------------------------------------------------

-- Demostración en lenguaje natural
-- ================================

-- Se usará el siguiente lema
--    add_le_add : a ≤ b → c ≤ d → a + c ≤ b + d
--
-- Por la definición de cota inferior, hay que demostrar que
--    (∀ x ∈ ℝ) [a + b ≤ f(x) + g(x)]                                  (1)
-- Para ello, sea x ∈ R. Puesto que es a es una cota inferior de f, se
-- tiene que
--    a ≤ f(x)                                                         (2)
-- y, puesto que b es una cota inferior de g, se tiene que
--    b ≤ g(x)                                                         (3)
-- De (2) y (3), por add_le_add, se tiene que
--    a + b ≤ f(x) + g(x)
-- que es lo que había que demostrar.

-- Demostraciones con Lean4
-- ========================

import Mathlib.Data.Real.Basic

-- (CotaInferior f a) expresa que a es una cota inferior de f.
def CotaInferior (f : ℝ → ℝ) (a : ℝ) : Prop :=
  ∀ x, a ≤ f x

variable {f g : ℝ → ℝ}
variable {a b : ℝ}

-- 1ª demostración
example
  (hfa : CotaInferior f a)
  (hgb : CotaInferior g b)
  : CotaInferior (f + g) (a + b) :=
by
  have h1 : ∀ x, a + b ≤ f x + g x
  { intro x
    have h1a : a ≤ f x := hfa x
    have h1b : b ≤ g x := hgb x
    show a + b ≤ f x + g x
    exact add_le_add h1a h1b }
  show CotaInferior (f + g) (a + b)
  exact h1

-- 2ª demostración
example
  (hfa : CotaInferior f a)
  (hgb : CotaInferior g b)
  : CotaInferior (f + g) (a + b) :=
by
  have h1 : ∀ x, a + b ≤ f x + g x
  { intro x
    show a + b ≤ f x + g x
    exact add_le_add (hfa x) (hgb x) }
  show CotaInferior (f + g) (a + b)
  exact h1

-- 3ª demostración
example
  (hfa : CotaInferior f a)
  (hgb : CotaInferior g b)
  : CotaInferior (f + g) (a + b) :=
by
  intro x
  dsimp
  apply add_le_add
  . apply hfa
  . apply hgb

-- 4ª demostración
theorem sumaCotaInf
  (hfa : CotaInferior f a)
  (hgb : CotaInferior g b)
  : CotaInferior (f + g) (a + b) :=
λ x ↦ add_le_add (hfa x) (hgb x)

-- Lemas usados
-- ============

-- variable (c d : ℝ)
-- #check (add_le_add : a ≤ b → c ≤ d → a + c ≤ b + d)
\end{verbatim}
Se puede interactuar con las pruebas anteriores en \href{https://lean.math.hhu.de/\#url=https://raw.githubusercontent.com/jaalonso/Calculemus2/main/src/Suma\_de\_cotas\_inferiores.lean}{Lean 4 Web}

\section{El producto de funciones no negativas es no negativo}
\label{sec:org153d3ae}
\begin{verbatim}
-- ---------------------------------------------------------------------
-- Demostrar que el producto de dos funciones no negativas es no
-- negativa.
-- ----------------------------------------------------------------------

-- Demostración en lenguaje natural
-- ================================

-- Se usará el siguiente lema
--    mul_nonneg : 0 ≤ a → 0 ≤ b → 0 ≤ a * b
--
-- Hay que demostrar que
--    (∀ x ∈ ℝ) [0 ≤ f(x) * g(x)]                                      (1)
-- Para ello, sea x ∈ R. Puesto que f es no negatica, se tiene que
--    0 ≤ f(x)                                                         (2)
-- y, puesto que g es no negativa, se tiene que
--    0 ≤ g(x)                                                         (3)
-- De (2) y (3), por mul_nonneg, se tiene que
--    0 ≤ f(x) * g(x)
-- que es lo que había que demostrar.

-- Demostraciones con Lean4
-- ========================

import Mathlib.Data.Real.Basic

-- (CotaInferior f a) expresa que a es una cota inferior de f.
def CotaInferior (f : ℝ → ℝ) (a : ℝ) : Prop :=
  ∀ x, a ≤ f x

variable (f g : ℝ → ℝ)

-- 1ª demostración
example
  (nnf : CotaInferior f 0)
  (nng : CotaInferior g 0)
  : CotaInferior (f * g) 0 :=
by
  have h1 : ∀x, 0 ≤ f x * g x
  { intro x
    have h2: 0 ≤ f x := nnf x
    have h3: 0 ≤ g x := nng x
    show 0 ≤ f x * g x
    exact mul_nonneg h2 h3 }
  show CotaInferior (f * g) 0
  exact h1

-- 2ª demostración
example
  (nnf : CotaInferior f 0)
  (nng : CotaInferior g 0)
  : CotaInferior (f * g) 0 :=
by
  have h1 : ∀x, 0 ≤ f x * g x
  { intro x
    show 0 ≤ f x * g x
    exact mul_nonneg (nnf x) (nng x) }
  show CotaInferior (f * g) 0
  exact h1

-- 3ª demostración
example
  (nnf : CotaInferior f 0)
  (nng : CotaInferior g 0)
  : CotaInferior (f * g) 0 :=
by
  intro x
  dsimp
  apply mul_nonneg
  . apply nnf
  . apply nng

-- 4ª demostración
example
  (nnf : CotaInferior f 0)
  (nng : CotaInferior g 0)
  : CotaInferior (f * g) 0 :=
λ x ↦ mul_nonneg (nnf x) (nng x)

-- Lemas usados
-- ============

-- variable (a b : ℝ)
-- #check (mul_nonneg : 0 ≤ a → 0 ≤ b → 0 ≤ a * b)
\end{verbatim}
Se puede interactuar con las pruebas anteriores en \href{https://lean.math.hhu.de/\#url=https://raw.githubusercontent.com/jaalonso/Calculemus2/main/src/Producto\_de\_funciones\_no\_negativas.lean}{Lean 4 Web}

\section{Si a es una cota superior no negativa de f y b es es una cota superior de la función no negativa g, entonces ab es una cota superior de fg}
\label{sec:org442e735}
\begin{verbatim}
-- ---------------------------------------------------------------------

-- ---------------------------------------------------------------------
-- Demostrar que si a es una cota superior de f, b es una cota superior
-- de g, a es no negativa y g es no negativa, entonces ab es una cota
-- superior de fg.
-- ----------------------------------------------------------------------

-- Demostración en lenguaje natural
-- ================================

-- Se usará el siguiente lema
--    mul_le_mul : a ≤ b → c ≤ d → 0 ≤ c → 0 ≤ b → a * c ≤ b * d
--
-- Hay que demostrar que
--    (∀ x ∈ ℝ) [f x * g x ≤ a * b]                                    (1)
-- Para ello, sea x ∈ R. Puesto que a es una cota superior de f, se tiene que
--    f(x) ≤ a                                                         (2)
-- puesto que b es una cota superior de g, se tiene que
--    g(x) ≤ b                                                         (3)
-- puesto que g es no negativa, se tiene que
--    0 ≤ g(x)                                                         (4)
-- y, puesto que a es no negativa, se tiene que
--    0 ≤ a                                                            (5)
-- De (2), (3), (4) y (5), por mul_le_mul, se tiene que
--    f x * g x ≤ a * b
-- que es lo que había que demostrar.

-- Demostraciones con Lean4
-- ========================

import Mathlib.Data.Real.Basic

-- (CotaSuperior f a) se verifica si a es una cota superior de f.
def CotaSuperior (f : ℝ → ℝ) (a : ℝ) : Prop :=
  ∀ x, f x ≤ a

-- (CotaInferior f a) expresa que a es una cota inferior de f.
def CotaInferior (f : ℝ → ℝ) (a : ℝ) : Prop :=
  ∀ x, a ≤ f x

variable (f g : ℝ → ℝ)
variable (a b : ℝ)

-- 1ª demostración
example
  (hfa : CotaSuperior f a)
  (hgb : CotaSuperior g b)
  (nng : CotaInferior g 0)
  (nna : 0 ≤ a)
  : CotaSuperior (f * g) (a * b) :=
by
  have h1 : ∀ x, f x * g x ≤ a * b
  { intro x
    have h2 : f x ≤ a := hfa x
    have h3 : g x ≤ b := hgb x
    have h4 : 0 ≤ g x := nng x
    show f x * g x ≤ a * b
    exact mul_le_mul h2 h3 h4 nna }
  show CotaSuperior (f * g) (a * b)
  exact h1

-- 2ª demostración
example
  (hfa : CotaSuperior f a)
  (hgb : CotaSuperior g b)
  (nng : CotaInferior g 0)
  (nna : 0 ≤ a)
  : CotaSuperior (f * g) (a * b) :=
by
  intro x
  dsimp
  apply mul_le_mul
  . apply hfa
  . apply hgb
  . apply nng
  . apply nna

-- 3ª demostración
example
  (hfa : CotaSuperior f a)
  (hgb : CotaSuperior g b)
  (nng : CotaInferior g 0)
  (nna : 0 ≤ a)
  : CotaSuperior (f * g) (a * b) :=
by
  intro x
  have h1:= hfa x
  have h2:= hgb x
  have h3:= nng x
  exact mul_le_mul h1 h2 h3 nna

-- 4ª demostración
example
  (hfa : CotaSuperior f a)
  (hgb : CotaSuperior g b)
  (nng : CotaInferior g 0)
  (nna : 0 ≤ a)
  : CotaSuperior (f * g) (a * b) :=
by
  intro x
  specialize hfa x
  specialize hgb x
  specialize nng x
  exact mul_le_mul hfa hgb nng nna

-- 5ª demostración
example
  (hfa : CotaSuperior f a)
  (hgb : CotaSuperior g b)
  (nng : CotaInferior g 0)
  (nna : 0 ≤ a)
  : CotaSuperior (f * g) (a * b) :=
λ x ↦ mul_le_mul (hfa x) (hgb x) (nng x) nna

-- Lemas usados
-- ============

-- variable (c d : ℝ)
-- #check (mul_le_mul : a ≤ b → c ≤ d → 0 ≤ c → 0 ≤ b → a * c ≤ b * d)
\end{verbatim}
Se puede interactuar con las pruebas anteriores en \href{https://lean.math.hhu.de/\#url=https://raw.githubusercontent.com/jaalonso/Calculemus2/main/src/Cota\_superior\_del\_producto.lean}{Lean 4 Web}

\section{La suma de dos funciones acotadas superiormente también lo está}
\label{sec:org926473b}
\begin{verbatim}
-- ---------------------------------------------------------------------
-- Demostrar que la suma de dos funciones acotadas superiormente también
-- lo está.
-- ----------------------------------------------------------------------

-- Demostración en lenguaje natural
-- ================================

-- Del ejercicio "La suma de una cota superior de f y una cota superior
-- de g es una cota superior de f+g" (que se encuentra en
-- https://bit.ly/3QauluK ) usaremos la definición de cota superior
-- (CotaSuperior) y el lema sumaCotaSup.
--
-- Puesto que f está acotada superiormente, tiene una cota superior. Sea
-- a una de dichas cotas. Análogamentte, puesto que g está acotada
-- superiormente, tiene una cota superior. Sea b una de dichas
-- cotas. Por el lema sumaCotaSup, a+b es una cota superior de f+g. or
-- consiguiente, f+g está acotada superiormente.

-- Demostraciones con Lean4
-- ========================

import src.Suma_de_cotas_superiores

variable {f g : ℝ → ℝ}

-- (acotadaSup f) afirma que f tiene cota superior.
def acotadaSup (f : ℝ → ℝ) :=
  ∃ a, CotaSuperior f a

-- 1ª demostración
example
  (hf : acotadaSup f)
  (hg : acotadaSup g)
  : acotadaSup (f + g) :=
by
  cases' hf with a ha
  -- a : ℝ
  -- ha : CotaSuperior f a
  cases' hg with b hb
  -- b : ℝ
  -- hb : CotaSuperior g b
  have h1 : CotaSuperior (f + g) (a + b) :=
    sumaCotaSup ha hb
  have h2 : ∃ z, CotaSuperior (f+g) z :=
    Exists.intro (a + b) h1
  show acotadaSup (f + g)
  exact h2

-- 2ª demostración
example
  (hf : acotadaSup f)
  (hg : acotadaSup g)
  : acotadaSup (f + g) :=
by
  cases' hf with a ha
  -- a : ℝ
  -- ha : FnUb f a
  cases' hg with b hb
  -- b : ℝ
  -- hb : FnUb g b
  use a + b
  apply sumaCotaSup ha hb

-- 4ª demostración
example
  (hf : acotadaSup f)
  (hg : acotadaSup g)
  : acotadaSup (f + g) :=
by
  rcases hf with ⟨a, ha⟩
  rcases hg with ⟨b, hb⟩
  exact ⟨a + b, sumaCotaSup ha hb⟩

-- 5ª demostración
example :
  acotadaSup f → acotadaSup g → acotadaSup (f + g) :=
by
  rintro ⟨a, ha⟩ ⟨b, hb⟩
  exact ⟨a + b, sumaCotaSup ha hb⟩

-- 6ª demostración
example :
  acotadaSup f → acotadaSup g → acotadaSup (f + g) :=
fun ⟨a, ha⟩ ⟨b, hb⟩ ↦ ⟨a + b, sumaCotaSup ha hb⟩

-- Lemas usados
-- ============

-- #check (sumaCotaSup : CotaSuperior f a → CotaSuperior g b → CotaSuperior (f + g) (a + b))
\end{verbatim}
Se puede interactuar con las pruebas anteriores en \href{https://lean.math.hhu.de/\#url=https://raw.githubusercontent.com/jaalonso/Calculemus2/main/src//Suma\_de\_funciones\_acotadas\_superiormente.lean}{Lean 4 Web}

\section{La suma de dos funciones acotadas inferiormente también lo está}
\label{sec:orgb0ce69c}
\begin{verbatim}
-- ---------------------------------------------------------------------
-- Demostrar que la suma de dos funciones acotadas inferiormente también
-- lo está.
-- ----------------------------------------------------------------------

-- Demostración en lenguaje natural
-- ================================

-- Del ejercicio "La suma de una cota inferior de f y una cota inferior
-- de g es una cota inferior de f+g" usaremos la definición de cota
-- inferior (CotaInferior) y el lema sumaCotaInf.
--
-- Puesto que f está acotada inferiormente, tiene una cota inferior. Sea
-- a una de dichas cotas. Análogamentte, puesto que g está acotada
-- inferiormente, tiene una cota inferior. Sea b una de dichas
-- cotas. Por el lema FnLb_add, a+b es una cota inferior de f+g. Por
-- consiguiente, f+g está acotada inferiormente.

-- Demostraciones con Lean4
-- ========================

import src.Suma_de_cotas_inferiores
variable {f g : ℝ → ℝ}

-- (acotadaInf f) afirma que f tiene cota inferior.
def acotadaInf (f : ℝ → ℝ) :=
  ∃ a, CotaInferior f a

-- 1ª demostración
example
  (hf : acotadaInf f)
  (hg : acotadaInf g)
  : acotadaInf (f + g) :=
by
  cases' hf with a ha
  -- a : ℝ
  -- ha : CotaInferior f a
  cases' hg with b hb
  -- b : ℝ
  -- hb : CotaInferior g b
  have h1 : CotaInferior (f + g) (a + b) := sumaCotaInf ha hb
  have h2 : ∃ z, CotaInferior (f + g) z :=
    Exists.intro (a + b) h1
  show acotadaInf (f + g)
  exact h2

-- 2ª demostración
example
  (hf : acotadaInf f)
  (hg : acotadaInf g)
  : acotadaInf (f + g) :=
by
  cases' hf with a ha
  -- a : ℝ
  -- ha : FnLb f a
  cases' hg with b hgb
  -- b : ℝ
  -- hgb : FnLb g b
  use a + b
  -- ⊢ FnLb (f + g) (a + b)
  apply sumaCotaInf ha hgb

-- 3ª demostración
example
  (hf : acotadaInf f)
  (hg : acotadaInf g)
  : acotadaInf (f + g) :=
by
  rcases hf with ⟨a, ha⟩
  -- a : ℝ
  -- ha : FnLb f a
  rcases hg with ⟨b, hb⟩
  -- b : ℝ
  -- hb : FnLb g b
  exact ⟨a + b, sumaCotaInf ha hb⟩

-- 4ª demostración
example :
  acotadaInf f → acotadaInf g → acotadaInf (f + g) :=
by
  rintro ⟨a, ha⟩ ⟨b, hb⟩
  -- a : ℝ
  -- ha : FnLb f a
  -- b : ℝ
  -- hb : FnLb g b
  exact ⟨a + b, sumaCotaInf ha hb⟩

-- 5ª demostración
example :
  acotadaInf f → acotadaInf g → acotadaInf (f + g) :=
fun ⟨a, ha⟩ ⟨b, hb⟩ ↦ ⟨a + b, sumaCotaInf ha hb⟩

-- Lemas usados
-- ============

-- #check (sumaCotaInf : FnLb f a → FnLb g b → FnLb (f + g) (a + b))
\end{verbatim}
Se puede interactuar con las pruebas anteriores en \href{https://lean.math.hhu.de/\#url=https://raw.githubusercontent.com/jaalonso/Calculemus2/main/src/Suma\_de\_funciones\_acotadas\_inferiormente.lean}{Lean 4 Web}

\section{Si a es una cota superior de f y c ≥ 0, entonces ca es una cota superior de cf}
\label{sec:org0a15732}
\begin{verbatim}
-- ---------------------------------------------------------------------
-- Demostrar que si a es una cota superior de f y c ≥ 0,
-- entonces c * a es una cota superior de c * f.
-- ----------------------------------------------------------------------

-- Demostración en lenguaj natural
-- ===============================

-- Se usará el lema
--    {b ≤ c, 0 ≤ a} ⊢ ab ≤ ac                                      (L1)
--
-- Tenemos que demostrar que
--    (∀ y ∈ ℝ) cf(y) ≤ ca.
-- Sea y ∈ R. Puesto que a es una cota de f, se tiene que
--    f(y) ≤ a
-- que, junto con c ≥ 0, por el lema L1 nos da
--    cf(y) ≤ ca

-- Demostraciones con Lean4
-- ========================

import Mathlib.Data.Real.Basic

-- (CotaSuperior f a) se verifica si a es una cota superior de f.
def CotaSuperior (f : ℝ → ℝ) (a : ℝ) : Prop :=
  ∀ x, f x ≤ a

variable {f : ℝ → ℝ}
variable {c : ℝ}

-- Demostraciones con Lean4
-- ========================

-- 1ª demostración
example
  (hfa : CotaSuperior f a)
  (h : c ≥ 0)
  : CotaSuperior (fun x ↦ c * f x) (c * a) :=
by
  intro y
  -- y : ℝ
  -- ⊢ (fun x => c * f x) y ≤ c * a
  have ha : f y ≤ a := hfa y
  calc (fun x => c * f x) y
       = c * f y := by rfl
     _ ≤ c * a   := mul_le_mul_of_nonneg_left ha h

-- 2ª demostración
example
  (hfa : CotaSuperior f a)
  (h : c ≥ 0)
  : CotaSuperior (fun x ↦ c * f x) (c * a) :=
by
  intro y
  calc (fun x => c * f x) y
       = c * f y := by rfl
     _ ≤ c * a   := mul_le_mul_of_nonneg_left (hfa y) h

-- 3ª demostración
example
  (hfa : CotaSuperior f a)
  (h : c ≥ 0)
  : CotaSuperior (fun x ↦ c * f x) (c * a) :=
by
  intro y
  show (fun x => c * f x) y ≤ c * a
  exact mul_le_mul_of_nonneg_left (hfa y) h

-- 4ª demostración
lemma CotaSuperior_mul
  (hfa : CotaSuperior f a)
  (h : c ≥ 0)
  : CotaSuperior (fun x ↦ c * f x) (c * a) :=
fun y ↦ mul_le_mul_of_nonneg_left (hfa y) h

-- Lemas usados
-- ============

-- variable (c : ℝ)
-- #check (mul_le_mul_of_nonneg_left : b ≤ c → 0 ≤ a → a * b ≤ a * c)
\end{verbatim}
Se puede interactuar con las pruebas anteriores en \href{https://lean.math.hhu.de/\#url=https://raw.githubusercontent.com/jaalonso/Calculemus2/main/src/Cota\_superior\_de\_producto\_por\_escalar.lean}{Lean 4 Web}

\section{Si c ≥ 0 y f está acotada superiormente, entonces c·f también lo está}
\label{sec:org32c2933}
\begin{verbatim}
-- ---------------------------------------------------------------------
-- Demostrar que si c ≥ 0 y f está acotada superiormente, entonces c * f
-- también lo está.
-- ----------------------------------------------------------------------

-- Demostración en lenguaje natural
-- ================================

-- Usaremos el siguiente lema:
--    CotaSuperior_mul : CotaSuperior f a → c ≥ 0 → CotaSuperior (fun x ↦ c * f x) (c * a)
--
-- Puesto que f está acotada superiormente, tiene una cota superior. Sea
-- a una de dichas cotas. Entonces, por el lema CotaSuperior_mul, ca es una cota
-- superior de cf. Por consiguiente, cf está acotada superiormente.

-- Demostraciones con Lean4
-- ========================

import src.Cota_superior_de_producto_por_escalar

variable {f : ℝ → ℝ}
variable {c : ℝ}

-- (acotadaSup f) afirma que f tiene cota superior.
def acotadaSup (f : ℝ → ℝ) :=
  ∃ a, CotaSuperior f a

-- 1ª demostración
example
  (hf : acotadaSup f)
  (hc : c ≥ 0)
  : acotadaSup (fun x ↦ c * f x) :=
by
  cases' hf with a ha
  -- a : ℝ
  -- ha : CotaSuperior f a
  have h1 : CotaSuperior (fun x ↦ c * f x) (c * a) :=
    CotaSuperior_mul ha hc
  have h2 : ∃ z, ∀ x, (fun x ↦ c * f x) x ≤ z :=
    Exists.intro (c * a) h1
  show acotadaSup (fun x ↦ c * f x)
  exact h2

-- 2ª demostración
example
  (hf : acotadaSup f)
  (hc : c ≥ 0)
  : acotadaSup (fun x ↦ c * f x) :=
by
  cases' hf with a ha
  -- a : ℝ
  -- ha : CotaSuperior f a
  use c * a
  -- ⊢ CotaSuperior (fun x => c * f x) (c * a)
  apply CotaSuperior_mul ha hc

-- 3ª demostración
example
  (hf : acotadaSup f)
  (hc : c ≥ 0)
  : acotadaSup (fun x ↦ c * f x) :=
by
  rcases hf with ⟨a, ha⟩
  -- a : ℝ
  -- ha : CotaSuperior f a
  exact ⟨c * a, CotaSuperior_mul ha hc⟩

-- 4ª demostración
example
  (hc : c ≥ 0)
  : acotadaSup f → acotadaSup (fun x ↦ c * f x) :=
by
  rintro ⟨a, ha⟩
  -- a : ℝ
  -- ha : CotaSuperior f a
  exact ⟨c * a, CotaSuperior_mul ha hc⟩

-- 5ª demostración
example
  (hc : c ≥ 0)
  : acotadaSup f → acotadaSup (fun x ↦ c * f x) :=
fun ⟨a, ha⟩ ↦ ⟨c * a, CotaSuperior_mul ha hc⟩

-- Lemas usados
-- ============

-- #check (CotaSuperior_mul : CotaSuperior f a → c ≥ 0 → CotaSuperior (fun x ↦ c * f x) (c * a))
\end{verbatim}
Se puede interactuar con las pruebas anteriores en \href{https://lean.math.hhu.de/\#url=https://raw.githubusercontent.com/jaalonso/Calculemus2/main/src/Producto\_por\_escalar\_acotado\_superiormente.lean}{Lean 4 Web}

\section{Suma de funciones monótonas}
\label{sec:org609c171}
\begin{verbatim}
-- ---------------------------------------------------------------------
-- Demostrar que la suma de dos funciones monótonas es monótona.
-- ----------------------------------------------------------------------

-- Demostración en lenguaje natural
-- ================================

-- Se usará el siguiente lema:
--    add_le_add : a ≤ b → c ≤ d → a + c ≤ b + d
--
-- Supongamos que f y g son monótonas y teneno que demostrar que f+g
-- también lo es; que
--    ∀ a b, a ≤ b → (f + g)(a) ≤ (f + g)(b)
-- Sean a, b ∈ ℝ tales que
--    a ≤ b                                                          (1)
-- Entonces, por ser f y g monótonas se tiene
--    f(a) ≤ f(b)                                                    (2)
--    g(a) ≤ g(b)                                                    (3)
-- Entonces,
--    (f + g)(a) = f(a) + g(a)
--               ≤ f(b) + g(b)    [por add_le_add, (2) y (3)]
--               = (f + g)(b)

-- Demostraciones con Lean4
-- ========================

import Mathlib.Data.Real.Basic

variable (f g : ℝ → ℝ)

-- 1ª demostración
example
  (mf : Monotone f)
  (mg : Monotone g)
  : Monotone (f + g) :=
by
  have h1 : ∀ a b, a ≤ b → (f + g) a ≤ (f + g) b
  { intros a b hab
    have h2 : f a ≤ f b := mf hab
    have h3 : g a ≤ g b := mg hab
    calc (f + g) a
         = f a + g a := rfl
       _ ≤ f b + g b := add_le_add h2 h3
       _ = (f + g) b := rfl }
  show Monotone (f + g)
  exact h1

-- 2ª demostración
example
  (mf : Monotone f)
  (mg : Monotone g)
  : Monotone (f + g) :=
by
  have h1 : ∀ a b, a ≤ b → (f + g) a ≤ (f + g) b
  { intros a b hab
    calc (f + g) a
         = f a + g a := rfl
       _ ≤ f b + g b := add_le_add (mf hab) (mg hab)
       _ = (f + g) b := rfl }
  show Monotone (f + g)
  exact h1

-- 3ª demostración
example
  (mf : Monotone f)
  (mg : Monotone g)
  : Monotone (f + g) :=
by
  have h1 : ∀ a b, a ≤ b → (f + g) a ≤ (f + g) b
  { intros a b hab
    show (f + g) a ≤ (f + g) b
    exact add_le_add (mf hab) (mg hab) }
  show Monotone (f + g)
  exact h1

-- 4ª demostración
example
  (mf : Monotone f)
  (mg : Monotone g)
  : Monotone (f + g) :=
by
  -- a b : ℝ
  -- hab : a ≤ b
  intros a b hab
  apply add_le_add
  . -- f a ≤ f b
    apply mf hab
  . --  g a ≤ g b
    apply mg hab

-- 5ª demostración
example
  (mf : Monotone f)
  (mg : Monotone g)
  : Monotone (f + g) :=
λ _ _ hab ↦ add_le_add (mf hab) (mg hab)

-- Lemas usados
-- ============

-- variable (a b c d : ℝ)
-- #check (add_le_add : a ≤ b → c ≤ d → a + c ≤ b + d)
\end{verbatim}
Se puede interactuar con las pruebas anteriores en \href{https://lean.math.hhu.de/\#url=https://raw.githubusercontent.com/jaalonso/Calculemus2/main/src/Suma\_de\_funciones\_monotonas.lean}{Lean 4 Web}

\section{Si c es no negativo y f es monótona, entonces cf es monótona.}
\label{sec:orgcb94858}
\begin{verbatim}
-- ---------------------------------------------------------------------
-- Demostrar que si c es no negativo y f es monótona, entonces cf es
-- monótona.
-- ----------------------------------------------------------------------

-- Demostración en lenguaje natural
-- ================================

-- Se usará el Lema
--    mul_le_mul_of_nonneg_left : b ≤ c → 0 ≤ a → a * b ≤ a * c
--
-- Tenemos que demostrar que
--    (∀ a, b ∈ ℝ) [a ≤ b → (cf)(a) ≤ (cf)(b)]
-- Sean a, b ∈ ℝ tales que a ≤ b. Puesto que f es monótona, se tiene
--    f(a) ≤ f(b).
-- y, junto con la hipótesis de que c es no negativo, usando el lema
-- mul_le_mul_of_nonneg_left, se tiene que
--    cf(a) ≤ cf(b)
-- que es lo que había que demostrar.

-- Demostraciones con Lean4
-- ========================

import Mathlib.Data.Real.Basic

variable (f : ℝ → ℝ)
variable {c : ℝ}

-- 1ª demostración
example
  (mf : Monotone f)
  (nnc : 0 ≤ c)
  : Monotone (fun x ↦ c * f x) :=
by
  have h1 : ∀ a b, a ≤ b → (fun x ↦ c * f x) a ≤ (fun x ↦ c * f x) b
  { intros a b hab
    have h2 : f a ≤ f b := mf hab
    show (fun x ↦ c * f x) a ≤ (fun x ↦ c * f x) b
    exact mul_le_mul_of_nonneg_left h2 nnc }
  show Monotone (fun x ↦ c * f x)
  exact h1

-- 2ª demostración
example
  (mf : Monotone f)
  (nnc : 0 ≤ c)
  : Monotone (fun x ↦ c * f x) :=
by
  -- a b : ℝ
  -- hab : a ≤ b
  intros a b hab
  -- (fun x => c * f x) a ≤ (fun x => c * f x) b
  apply mul_le_mul_of_nonneg_left
  . -- f a ≤ f b
    apply mf hab
  . -- 0 ≤ c
    apply nnc

-- 3ª demostración
example (mf : Monotone f) (nnc : 0 ≤ c) :
  Monotone (fun x ↦ c * f x) :=
λ _ _ hab ↦ mul_le_mul_of_nonneg_left (mf hab) nnc

-- Lemas usados
-- ============

-- variable (a b : ℝ)
-- #check (mul_le_mul_of_nonneg_left : b ≤ c → 0 ≤ a → a * b ≤ a * c)
\end{verbatim}
Se puede interactuar con las pruebas anteriores en \href{https://lean.math.hhu.de/\#url=https://raw.githubusercontent.com/jaalonso/Calculemus2/main/src/Producto\_de\_un\_positivo\_por\_una\_funcion\_monotona.lean}{Lean 4 Web}

\section{La composición de dos funciones monótonas es monótona}
\label{sec:orgf0f7579}
\begin{verbatim}
-- ---------------------------------------------------------------------
-- Demostrar que la composición de dos funciones monótonas es monótona.
-- ----------------------------------------------------------------------

-- Demostración en lenguaje natural
-- ================================

-- Sean f y g dos funciones monótonas de ℝ en ℝ. Tenemos que demostrar
-- que f ∘ g es monótona; es decir, que
--    (∀ a, b ∈ ℝ) [a ≤ b → (f ∘ g)(a) ≤ (f ∘ g)(b)]
-- Sean a, b ∈ ℝ tales que a ≤ b. Por ser g monótona, se tiene
--    g(a) ≤ g(b)
-- y, por ser f monótona, se tiene
--    f(g(a)) ≤ f(g(b))
-- Finalmente, por la definición de composición,
--    (f ∘ g)(a) ≤ (f ∘ g)(b)
-- que es lo que había que demostrar.

-- Demostraciones con Lean4
-- ========================

import Mathlib.Data.Real.Basic

variable (f g : ℝ → ℝ)

-- 1ª demostración
example
  (mf : Monotone f)
  (mg : Monotone g)
  : Monotone (f ∘ g) :=
by
  have h1 : ∀ a b, a ≤ b → (f ∘ g) a ≤ (f ∘ g) b
  { intros a b hab
    have h1 : g a ≤ g b := mg hab
    show (f ∘ g) a ≤ (f ∘ g) b
    exact mf h1 }
  show Monotone (f ∘ g)
  exact h1

-- 2ª demostración
example
  (mf : Monotone f)
  (mg : Monotone g)
  : Monotone (f ∘ g) :=
by
  have h1 : ∀ a b, a ≤ b → (f ∘ g) a ≤ (f ∘ g) b
  { intros a b hab
    show (f ∘ g) a ≤ (f ∘ g) b
    exact mf (mg hab) }
  show Monotone (f ∘ g)
  exact h1

-- 3ª demostración
example
  (mf : Monotone f)
  (mg : Monotone g)
  : Monotone (f ∘ g) :=
by
  -- a b : ℝ
  -- hab : a ≤ b
  intros a b hab
  -- (f ∘ g) a ≤ (f ∘ g) b
  apply mf
  -- g a ≤ g b
  apply mg
  -- a ≤ b
  apply hab

-- 4ª demostración
example (mf : Monotone f) (mg : Monotone g) :
  Monotone (f ∘ g) :=
λ _ _ hab ↦ mf (mg hab)
\end{verbatim}
Se puede interactuar con las pruebas anteriores en \href{https://lean.math.hhu.de/\#url=https://raw.githubusercontent.com/jaalonso/Calculemus2/main/src/Composicion\_de\_funciones\_monotonas.lean}{Lean 4 Web}

\section{La suma de dos funciones pares es par}
\label{sec:org12bc0ed}
\begin{verbatim}
-- ---------------------------------------------------------------------
-- Demostrar que la suma de dos funciones pares es par.
-- ----------------------------------------------------------------------

-- Demostración en lenguaje natural
-- ================================

-- Supongamos que f y g son funciones pares. Tenemos que demostrar que
-- f+g es par; es decir, que
--    (∀ x ∈ ℝ) (f + g)(x) = (f + g)(-x)
-- Sea x ∈ ℝ. Entonces,
--    (f + g) x = f x + g x
--              = f (-x) + g x    [porque f es par]
--              = f (-x) + g (-x) [porque g es par]
--              = (f + g) (-x)

-- Demostraciones con Lean4
-- ========================

import Mathlib.Data.Real.Basic

variable (f g : ℝ → ℝ)

-- (esPar f) expresa que f es par.
def esPar (f : ℝ → ℝ) : Prop :=
  ∀ x, f x = f (-x)

-- 1ª demostración
-- ===============

example
  (h1 : esPar f)
  (h2 : esPar g)
  : esPar (f + g) :=
by
  intro x
  have h1 : f x = f (-x) := h1 x
  have h2 : g x = g (-x) := h2 x
  calc (f + g) x
       = f x + g x       := rfl
     _ = f (-x) + g x    := congrArg (. + g x) h1
     _ = f (-x) + g (-x) := congrArg (f (-x) + .) h2
     _ = (f + g) (-x)    := rfl

-- 2ª demostración
-- ===============

example
  (h1 : esPar f)
  (h2 : esPar g)
  : esPar (f + g) :=
by
  intro x
  calc (f + g) x
       = f x + g x       := rfl
     _ = f (-x) + g x    := congrArg (. + g x) (h1 x)
     _ = f (-x) + g (-x) := congrArg (f (-x) + .) (h2 x)
     _ = (f + g) (-x)    := rfl

-- 3ª demostración
-- ===============

example
  (h1 : esPar f)
  (h2 : esPar g)
  : esPar (f + g) :=
by
  intro x
  calc (f + g) x
       = f x + g x       := rfl
     _ = f (-x) + g (-x) := by rw [h1, h2]
     _ = (f + g) (-x)    := rfl
\end{verbatim}
Se puede interactuar con las pruebas anteriores en \href{https://lean.math.hhu.de/\#url=https://raw.githubusercontent.com/jaalonso/Calculemus2/main/src/Suma\_funciones\_pares.lean}{Lean 4 Web}

\section{El producto de dos funciones impares es par}
\label{sec:orgb5f25ed}
\begin{verbatim}
-- ---------------------------------------------------------------------
-- Demostrar que el producto de dos funciones impares es par.
-- ----------------------------------------------------------------------

-- Demostración en lenguaje natural
-- ================================

-- Supongamos que f y g son funciones impares. Tenemos que demostrar que
-- f·g es par; es decir, que
--    (∀ x ∈ ℝ) (f·g)(x) = (f·g)(-x)
-- Sea x ∈ ℝ. Entonces,
--    (f·g) x = f(x)g(x)
--            = (-f(-x))g(x)      [porque f es impar]
--            = (-f(-x)(-g(-x))   [porque g es impar]
--            = f(-x)g(-x))
--            = (f·g)(-x)

-- Demostraciones con Lean4
-- ========================

import Mathlib.Data.Real.Basic

variable (f g : ℝ → ℝ)

-- (esPar f) expresa que f es par.
def esPar (f : ℝ → ℝ) : Prop :=
  ∀ x, f x = f (-x)

-- (esImpar f) expresa que f es impar.
def esImpar  (f : ℝ → ℝ) : Prop :=
  ∀ x, f x = - f (-x)

-- 1ª demostración
example
  (h1 : esImpar f)
  (h2 : esImpar g)
  : esPar (f * g) :=
by
  intro x
  have h1 : f x = -f (-x) := h1 x
  have h2 : g x = -g (-x) := h2 x
  calc (f * g) x
       = f x * g x             := rfl
     _ = (-f (-x)) * g x       := congrArg (. * g x) h1
     _ = (-f (-x)) * (-g (-x)) := congrArg ((-f (-x)) * .) h2
     _ = f (-x) * g (-x)       := neg_mul_neg (f (-x)) (g (-x))
     _ = (f * g) (-x)          := rfl

-- 2ª demostración
example
  (h1 : esImpar f)
  (h2 : esImpar g)
  : esPar (f * g) :=
by
  intro x
  calc (f * g) x
       = f x * g x             := rfl
     _ = (-f (-x)) * g x       := congrArg (. * g x) (h1 x)
     _ = (-f (-x)) * (-g (-x)) := congrArg ((-f (-x)) * .) (h2 x)
     _ = f (-x) * g (-x)       := neg_mul_neg (f (-x)) (g (-x))
     _ = (f * g) (-x)          := rfl

-- 3ª demostración
example
  (h1 : esImpar f)
  (h2 : esImpar g)
  : esPar (f * g) :=
by
  intro x
  calc (f * g) x
       = f x * g x         := rfl
     _ = -f (-x) * -g (-x) := by rw [h1, h2]
     _ = f (-x) * g (-x)   := by rw [neg_mul_neg]
     _ = (f * g) (-x)      := rfl

-- 4ª demostración
example
  (h1 : esImpar f)
  (h2 : esImpar g)
  : esPar (f * g) :=
by
  intro x
  calc (f * g) x
       = f x * g x       := rfl
     _ = f (-x) * g (-x) := by rw [h1, h2, neg_mul_neg]
     _ = (f * g) (-x)    := rfl

-- Lemas usados
-- ============

-- variable (a b : ℝ)
-- #check (neg_mul_neg a b : -a * -b = a * b)
\end{verbatim}
Se puede interactuar con las pruebas anteriores en \href{https://lean.math.hhu.de/\#url=https://raw.githubusercontent.com/jaalonso/Calculemus2/main/src/Producto\_de\_funciones\_impares.lean}{Lean 4 Web}

\section{El producto de una función par por una impar es impar}
\label{sec:orgc714d67}
\begin{verbatim}
-- ---------------------------------------------------------------------
-- Demostrar que el producto de una función par por una impar es impar.
-- ----------------------------------------------------------------------

-- Demostración en lenguaje natural
-- ================================

-- Supongamos que f es una función par y g lo es impar. Tenemos que
-- demostrar que f·g es imppar; es decir, que
--    (∀ x ∈ ℝ) (f·g)(x) = -(f·g)(-x)
-- Sea x ∈ ℝ. Entonces,
--    (f·g) x = f(x)g(x)
--            = f(-x)g(x)       [porque f es par]
--            = f(-x)(-g(-x))   [porque g es impar]
--            = -f(-x)g(-x))
--            = -(f·g)(-x)

-- Demostraciones con Lean4
-- ========================

import Mathlib.Data.Real.Basic

variable (f g : ℝ → ℝ)

-- (esPar f) expresa que f es par.
def esPar (f : ℝ → ℝ) : Prop :=
  ∀ x, f x = f (-x)

-- (esImpar f) expresa que f es impar.
def esImpar  (f : ℝ → ℝ) : Prop :=
  ∀ x, f x = - f (-x)

-- 1ª demostración
example
  (h1 : esPar f)
  (h2 : esImpar g)
  : esImpar (f * g) :=
by
  intro x
  have h1 : f x = f (-x) := h1 x
  have h2 : g x = -g (-x) := h2 x
  calc (f * g) x
       = f x * g x            := rfl
     _ = (f (-x)) * g x       := congrArg (. * g x) h1
     _ = (f (-x)) * (-g (-x)) := congrArg (f (-x) * .) h2
     _ = -(f (-x) * g (-x))   := mul_neg (f (-x)) (g (-x))
     _ = -(f * g) (-x)        := rfl

-- 2ª demostración
example
  (h1 : esPar f)
  (h2 : esImpar g)
  : esImpar (f * g) :=
by
  intro x
  calc (f * g) x
       = f x * g x          := rfl
    _  = f (-x) * -g (-x)   := by rw [h1, h2]
    _  = -(f (-x) * g (-x)) := by rw [mul_neg]
    _  = -(f * g) (-x)      := rfl

-- 3ª demostración
example
  (h1 : esPar f)
  (h2 : esImpar g)
  : esImpar (f * g) :=
by
  intro x
  calc (f * g) x
       = f x * g x          := rfl
     _ = -(f (-x) * g (-x)) := by rw [h1, h2, mul_neg]
     _ = -((f * g) (-x))    := rfl

-- Lemas usados
-- ===========

-- variable (a b : ℝ)
-- #check (mul_neg a b : a * -b = -(a * b))
\end{verbatim}
Se puede interactuar con las pruebas anteriores en \href{https://lean.math.hhu.de/\#url=https://raw.githubusercontent.com/jaalonso/Calculemus2/main/src/Producto\_funcion\_par\_e\_impar.lean}{Lean 4 Web}

\section{Si f es par y g es impar, entonces (f ∘ g) es par}
\label{sec:org60dda90}
\begin{verbatim}
-- ---------------------------------------------------------------------
-- Demostrar que si f es par y g es impar, entonces f ∘ g es par.
-- ----------------------------------------------------------------------

-- Demostración en lenguaje natural
-- ================================

-- Supongamos que f es una función par y g lo es impar. Tenemos que
-- demostrar que (f ∘ g) es par; es decir, que
--    (∀ x ∈ ℝ) (f ∘ g)(x) = (f ∘ g)(-x)
-- Sea x ∈ ℝ. Entonces,
--    (f ∘ g)(x) = f(g(x))
--               = f(-g(-x))    [porque g es impar]
--               = f(g(-x))     [porque f es par]
--               = (f ∘ g)(-x)

-- Demostraciones con Lean4
-- ========================

import Mathlib.Data.Real.Basic

variable (f g : ℝ → ℝ)

-- (esPar f) expresa que f es par.
def esPar (f : ℝ → ℝ) : Prop :=
  ∀ x, f x = f (-x)

-- (esImpar f) expresa que f es impar.
def esImpar  (f : ℝ → ℝ) : Prop :=
  ∀ x, f x = - f (-x)

-- 1ª demostración
example
  (h1 : esPar f)
  (h2 : esImpar g)
  : esPar (f ∘ g) :=
by
  intro x
  calc (f ∘ g) x
       = f (g x)      := rfl
    _  = f (-g (-x))  := congr_arg f (h2 x)
    _  = f (g (-x))   := (h1 (g (-x))).symm
    _  = (f ∘ g) (-x) := rfl

-- 2ª demostración
example
  (h1 : esPar f)
  (h2 : esImpar g)
  : esPar (f ∘ g) :=
by
  intro x
  calc (f ∘ g) x
       = f (g x)      := rfl
     _ = f (-g (-x))  := by rw [h2]
     _ = f (g (-x))   := by rw [← h1]
     _ = (f ∘ g) (-x) := rfl

-- 3ª demostración
example
  (h1 : esPar f)
  (h2 : esImpar g)
  : esPar (f ∘ g) :=
by
  intro x
  calc (f ∘ g) x
       = f (g x)      := rfl
     _ = f (g (-x))   := by rw [h2, ← h1]
\end{verbatim}
Se puede interactuar con las pruebas anteriores en \href{https://lean.math.hhu.de/\#url=https://raw.githubusercontent.com/jaalonso/Calculemus2/main/src/Composicion\_de\_par\_e\_impar.lean}{Lean 4 Web}

\section{Para cualquier conjunto s, s ⊆ s}
\label{sec:orga9fadc4}
\begin{verbatim}
-- ---------------------------------------------------------------------
-- Demostrar que para cualquier conjunto s, s ⊆ s.
-- ----------------------------------------------------------------------

-- Demostración en lenguaje natural
-- ================================

-- Tenemos que demostrar que
--    (∀ x) [x ∈ s → × ∈ s]
-- Sea x tal que
--    x ∈ s                                                          (1)
-- Entonces, por (1), se tiene que
--    x ∈ s
-- que es lo que teníamos que demostrar.

-- Demostraciones con Lean 4
-- =========================

import Mathlib.Tactic

variable {α : Type _}
variable (s : Set α)

-- 1ª demostración
example : s ⊆ s :=
by
  intro x xs
  exact xs

-- 2ª demostración
example : s ⊆ s :=
  fun (x : α) (xs : x ∈ s) ↦ xs

-- 3ª demostración
example : s ⊆ s :=
  fun _ xs ↦ xs

-- 4ª demostración
example : s ⊆ s :=
  -- by exact?
  rfl.subset

-- 5ª demostración
example : s ⊆ s :=
by rfl
\end{verbatim}
Se puede interactuar con las pruebas anteriores en \href{https://lean.math.hhu.de/\#url=https://raw.githubusercontent.com/jaalonso/Calculemus2/main/src/Propiedad\_reflexiva\_del\_subconjunto.lean}{Lean 4 Web}

\chapter{Teoría de conjuntos}
\label{sec:org98be9a3}

\section{Si r ⊆ s y s ⊆ t, entonces r ⊆ t}
\label{sec:orga7cc0ef}
\begin{verbatim}
-- ---------------------------------------------------------------------
-- Demostrar que si r ⊆ s y s ⊆ t, entonces r ⊆ t.
-- ----------------------------------------------------------------------

-- Demostración en lenguaje natural (LN)
-- =====================================

-- 1ª demostración en LN
-- ---------------------

-- Tenemos que demostrar que
--    (∀ x) [x ∈ r → x ∈ t]
-- Sea x tal que
--    x ∈ r.
-- Puesto que r ⊆ s, se tiene que
--    x ∈ s
-- y, puesto que s ⊆ t, se tiene que
--    x ∈ t
-- que es lo que teníamos que demostrar.

-- 2ª demostración en LN
-- ---------------------

-- Tenemos que demostrar que
--    (∀ x) [x ∈ r → x ∈ t]
-- Sea x tal que
--    x ∈ r
-- Tenemos que demostrar que
--    x ∈ t
-- que, puesto que s ⊆ t, se reduce a
--    x ∈ s
-- que, puesto que r ⊆ s, se redece a
--    x ∈ r
-- que es lo que hemos supuesto.

-- Demostraciones con Lean4
-- ========================

import Mathlib.Tactic

open Set

variable {α : Type _}
variable (r s t : Set α)

-- 1ª demostración
example
  (rs : r ⊆ s)
  (st : s ⊆ t)
  : r ⊆ t :=
by
  intros x xr
  -- xr : x ∈ r
  have xs : x ∈ s := rs xr
  show x ∈ t
  exact st xs

-- 2ª demostración
example
  (rs : r ⊆ s)
  (st : s ⊆ t)
  : r ⊆ t :=
by
  intros x xr
  -- x : α
  -- xr : x ∈ r
  apply st
  -- ⊢ x ∈ s
  apply rs
  -- ⊢ x ∈ r
  exact xr

-- 3ª demostración
example
  (rs : r ⊆ s)
  (st : s ⊆ t)
  : r ⊆ t :=
fun _ xr ↦ st (rs xr)

-- 4ª demostración
example
  (rs : r ⊆ s)
  (st : s ⊆ t)
  : r ⊆ t :=
-- by exact?
Subset.trans rs st

-- 5ª demostración
example
  (rs : r ⊆ s)
  (st : s ⊆ t)
  : r ⊆ t :=
by tauto

-- Lemas usados
-- ============

-- #check (Subset.trans : r ⊆ s → s ⊆ t → r ⊆ t)
\end{verbatim}
Se puede interactuar con las pruebas anteriores en \href{https://lean.math.hhu.de/\#url=https://raw.githubusercontent.com/jaalonso/Calculemus2/main/src/Propiedad\_transitiva\_del\_subconjunto.lean}{Lean 4 Web}

\section{Si a es una cota superior de s y a ≤ b, entonces b es una cota superior de s}
\label{sec:org375bb35}
\begin{verbatim}
-- ---------------------------------------------------------------------
-- Demostrar que si a es una cota superior de s y a ≤ b, entonces b es
-- una cota superior de s.
-- ----------------------------------------------------------------------

import Mathlib.Tactic

variable {α : Type _} [PartialOrder α]
variable (s : Set α)
variable (a b : α)

-- (CotaSupConj s a) afirma que a es una cota superior del conjunto s.
def CotaSupConj (s : Set α) (a : α) :=
  ∀ {x}, x ∈ s → x ≤ a

-- Demostración en lenguaje natural
-- ================================

-- Tenemos que demostrar que
--    (∀ x) [x ∈ s → x ≤ b]
-- Sea x tal que x ∈ s. Entonces,
--    x ≤ a   [porque a es una cota superior de s]
--      ≤ b
-- Por tanto, x ≤ b.

-- 1ª demostración
example
  (h1 : CotaSupConj s a)
  (h2 : a ≤ b)
  : CotaSupConj s b :=
by
  intro x (xs : x ∈ s)
  have h3 : x ≤ a := h1 xs
  show x ≤ b
  exact le_trans h3 h2

-- 2ª demostración
example
  (h1 : CotaSupConj s a)
  (h2 : a ≤ b)
  : CotaSupConj s b :=
by
  intro x (xs : x ∈ s)
  calc x ≤ a := h1 xs
       _ ≤ b := h2
-
-- Lemas usados
-- ============

-- variable (c : α)
-- #check (le_trans : a ≤ b → b ≤ c → a ≤ c)
\end{verbatim}
Se puede interactuar con las pruebas anteriores en \href{https://lean.math.hhu.de/\#url=https://raw.githubusercontent.com/jaalonso/Calculemus2/main/src/Cotas\_superiores\_de\_conjuntos.lean}{Lean 4 Web}

\section{La función (x ↦ x + c) es inyectiva}
\label{sec:org7b84daa}
\begin{verbatim}
-- ---------------------------------------------------------------------
-- Demostrar que, para todo c la función
--    f(x) = x + c
-- es inyectiva
-- ----------------------------------------------------------------------

-- Demostración en lenguaje natural
-- ================================

-- Se usará el lema
--    (∀ a, b, c) [a + b = c + b → a = c]                            (L1)
-- Hay que demostrar que
--    (∀ x₁ x₂) [f(x₁) = f(x₂) → x₁ = x₂]
-- Sean x₁, x₂ tales que f(x₁) = f(x₂). Entonces,
--    x₁ + c = x₂ + c
-- y, por L1, x₁ = x₂.

-- Demostraciones con Lean4
-- ========================

import Mathlib.Data.Real.Basic
open Function

variable {c : ℝ}

-- 1ª demostración
example : Injective ((. + c)) :=
by
  intro (x1 : ℝ) (x2 : ℝ) (h1 : x1 + c = x2 + c)
  show x1 = x2
  exact add_right_cancel h1

-- 2ª demostración
example : Injective ((. + c)) :=
by
  intro x1 x2 h1
  show x1 = x2
  exact add_right_cancel h1

-- 3ª demostración
example : Injective ((. + c)) :=
  fun _ _ h ↦ add_right_cancel h

-- Lemas usados
-- ============

-- variable {a b : ℝ}
-- #check (add_right_cancel : a + b = c + b → a = c)
\end{verbatim}
Se puede interactuar con las pruebas anteriores en \href{https://lean.math.hhu.de/\#url=https://raw.githubusercontent.com/jaalonso/Calculemus2/main/src/Suma\_constante\_es\_inyectiva.lean}{Lean 4 Web}

\section{Si c ≠ 0, entonces la función (x ↦ cx) es inyectiva}
\label{sec:org9738809}
\begin{verbatim}
-- ---------------------------------------------------------------------
-- Ejercicio 3. Demostrar que para todo c distinto de cero la función
--    f(x) = c * x
-- es inyectiva
-- ----------------------------------------------------------------------

-- Demostración en lenguaje natural
-- ================================

-- Se usará el lema
--    (∀ a, b, c) [a ≠ 0 → (a * b = a * c ↔ b = c))]             (L1)
-- Hay que demostrar que
--    (∀ x₁, x₂) [f(x₁) = f(x₂) → x₁ = x₂]
-- Sean x₁, x₂ tales que f(x₁) = f(x₂). Entonces,
--    cx₁ = cx₂
-- y, por L1 y puesto que c ≠ 0, se tiene que
--    x₁ = x₂.

-- Demostraciones con Lean4
-- ========================

import Mathlib.Data.Real.Basic
open Function
variable {c : ℝ}

-- 1ª demostración
example
  (h : c ≠ 0)
  : Injective ((c * .)) :=
by
  intro (x1 : ℝ) (x2 : ℝ) (h1 : c * x1 = c * x2)
  show x1 = x2
  exact (mul_right_inj' h).mp h1

-- 2ª demostración
example
  (h : c ≠ 0)
  : Injective ((c * .)) :=
fun _ _ h1 ↦ mul_left_cancel₀ h h1

-- Lemas usados
-- ============

-- variable (a b : ℝ)
-- #check (mul_right_inj' : a ≠ 0 → (a * b = a * c ↔ b = c))
-- #check (mul_left_cancel₀ : a ≠ 0 → a * b = a * c → b = c)
\end{verbatim}
Se puede interactuar con las pruebas anteriores en \href{https://lean.math.hhu.de/\#url=https://raw.githubusercontent.com/jaalonso/Calculemus2/main/src/Producto\_constante\_no\_nula\_es\_inyectiva.lean}{Lean 4 Web}

\section{La composición de funciones inyectivas es inyectiva}
\label{sec:org08e06d9}
\begin{verbatim}
-- Demostraciones en lenguaje natural (LN)
-- =======================================

-- 1ª demostración en LN
-- =====================

-- Tenemos que demostrar que
--    (∀ x, y) [(g ∘ f)(x) = (g ∘ f)(y) → x = y]
-- Sean x, y tales que
--    (g ∘ f)(x) = (g ∘ f)(y)
-- Entonces, por la definición de la composición,
--    g(f(x)) = g(f(y))
-- y, ser g inyectiva,
--    f(x) = f(y)
-- y, ser f inyectiva,
--    x = y

-- 2ª demostración en LN
-- =====================

-- Tenemos que demostrar que
--    (∀ x, y) [(g ∘ f)(x) = (g ∘ f)(y) → x = y]
-- Sean x, y tales que
--    (g ∘ f)(x) = (g ∘ f)(y)                                        (1)
-- y tenemos que demostrar que
--    x = y                                                          (2)
-- El objetivo (2), usando que f es inyectiva, se reduce a
--    f(x) = f(y)
-- que, usando que g es inyectiva, se reduce a
--    g(f(x)) = g(f(y))
-- que, por la definición de la composición, coincide con (1).

-- Demostraciones con Lean4
-- ========================

import Mathlib.Tactic

open Function

variable {α : Type _} {β : Type _} {γ : Type _}
variable {f : α → β} {g : β → γ}

-- 1ª demostración (basada en la 1ª en LN)
example
  (hg : Injective g)
  (hf : Injective f) :
  Injective (g ∘ f) :=
by
  intro (x : α) (y : α) (h1: (g ∘ f) x = (g ∘ f) y)
  have h2: g (f x) = g (f y) := h1
  have h3: f x = f y := hg h2
  show x = y
  exact hf h3

-- 2ª demostración
example
  (hg : Injective g)
  (hf : Injective f) :
  Injective (g ∘ f) :=
by
  intro (x : α) (y : α) (h1: (g ∘ f) x = (g ∘ f) y)
  have h2: f x = f y := hg h1
  show x = y
  exact hf h2

-- 3ª demostración
example
  (hg : Injective g)
  (hf : Injective f) :
  Injective (g ∘ f) :=
by
  intro x y h
  exact hf (hg h)

-- 4ª demostración
example
  (hg : Injective g)
  (hf : Injective f) :
  Injective (g ∘ f) :=
fun _ _ h ↦ hf (hg h)

-- 5ª demostración (basada en la 2ª en LN)
example
  (hg : Injective g)
  (hf : Injective f) :
  Injective (g ∘ f) :=
by
  intros x y h
  -- x y : α
  -- h : (g ∘ f) x = (g ∘ f) y
  apply hf
  -- ⊢ f x = f y
  apply hg
  -- ⊢ g (f x) = g (f y)
  apply h

-- 6ª demostración
example
  (hg : Injective g)
  (hf : Injective f) :
  Injective (g ∘ f) :=
-- by exact?
Injective.comp hg hf

-- 7ª demostración
example
  (hg : Injective g)
  (hf : Injective f) :
  Injective (g ∘ f) :=
by tauto

-- Lemas usados
-- ============

-- #check (Injective.comp : Injective g → Injective f → Injective (g ∘ f))
\end{verbatim}
Se puede interactuar con las pruebas anteriores en \href{https://lean.math.hhu.de/\#url=https://raw.githubusercontent.com/jaalonso/Calculemus2/main/src/Composicion\_de\_funciones\_inyectivas.lean}{Lean 4 Web}

\input Calculemus2_bibliografia

\chapter*{Lemas usados}
\addcontentsline{toc}{chapter}{Lemas usados}
\begin{verbatim}
import Mathlib.Algebra.Group.Basic
import Mathlib.Algebra.Order.Ring.Defs        -- 1
import Mathlib.Algebra.Ring.Defs
import Mathlib.Analysis.SpecialFunctions.Log.Basic
import Mathlib.Data.Real.Basic
import Mathlib.Order.Lattice
import Mathlib.Topology.MetricSpace.Basic

-- Números naturales
-- =================

section naturales
variable (x y z k m n : ℕ)
#check (_root_.dvd_antisymm : m ∣ n → n ∣ m → m = n)
#check (dvd_add : x ∣ y → x ∣ z → x ∣ y + z)
#check (dvd_gcd : k ∣ m → k ∣ n → k ∣ gcd m n)
#check (dvd_mul_left x y : x ∣ y * x)
#check (dvd_mul_of_dvd_left : x ∣ y → ∀ (c : ℕ), x ∣ y * c)
#check (dvd_mul_of_dvd_right : x ∣ y → ∀ (c : ℕ), x ∣ c * y)
#check (dvd_mul_right x y : x ∣ x * y)
#check (dvd_trans : x ∣ y → y ∣ z → x ∣ z)
#check (Dvd.intro k : m * k = n → m ∣ n)
#check (gcd_comm m n : gcd m n = gcd n m)
#check (gcd_dvd_left  m n: gcd m n ∣ m)
#check (gcd_dvd_right m n : gcd m n ∣ n)
end naturales

-- Números reales
-- ==============

section reales
open Real
variable (a b c d : ℝ)
#check (abs_add a b : |a + b| ≤ |a| + |b|)
#check (abs_le' : |a| ≤ b ↔ a ≤ b ∧ -a ≤ b)
#check (abs_mul a b : |a * b| = |a| * |b|)
#check (abs_nonneg a : 0 ≤ |a|)
#check (abs_sub_abs_le_abs_sub a b : |a| - |b| ≤ |a - b|)
#check (add_le_add : a ≤ b → c ≤ d → a + c ≤ b + d)
#check (add_le_add_left : b ≤ c → ∀ (a : ℝ), a + b ≤ a + c)
#check (add_le_add_right : b ≤ c → ∀ (a : ℝ),  b + a ≤ c + a)
#check (add_lt_add_of_le_of_lt : a ≤ b → c < d → a + c < b + d)
#check (add_lt_add_of_lt_of_le : a < b → c ≤ d → a + c < b + d)
#check (add_lt_add_right : b < c → ∀ (a : ℝ), b + a < c + a)
#check (add_neg_le_iff_le_add : a - b ≤ c ↔ a ≤ c + b)
#check (add_pos : 0 < a → 0 < b → 0 < a + b)
#check (add_sub_cancel a b : a + b - b = a)
#check (div_mul_cancel a : b ≠ 0 → (a / b) * b = a)
#check (exp_le_exp : exp a ≤ exp b ↔ a ≤ b)
#check (exp_lt_exp : exp a < exp b ↔ a < b)
#check (exp_pos a : 0 < exp a)
#check (le_antisymm : a ≤ b → b ≤ a → a = b)
#check (le_div_iff : 0 < c → (a ≤ b / c ↔ a * c ≤ b))
#check (le_max_left a b : a ≤ max a b)
#check (le_max_right a b : b ≤ max a b)
#check (le_min : c ≤ a → c ≤ b → c ≤ min a b)
#check (le_refl a : a ≤ a)
#check (log_le_log' : 0 < a → a ≤ b → log a ≤ log b)
#check (lt_of_lt_of_le : a < b → b ≤ c → a < c)
#check (lt_of_le_of_lt : a ≤ b → b < c → a < c)
#check (lt_of_le_of_ne : a ≤ b → a ≠ b → a < b)
#check (lt_trans : a < b → b < c → a < c)
#check (max_comm a b : max a b = max b a)
#check (max_le : a ≤ c → b ≤ c → max a b ≤ c)
#check (min_add_add_right a b c : min (a + c) (b + c) = min a b + c)
#check (min_assoc a b c : min (min a b) c = min a (min b c))
#check (min_comm a b : min a b = min b a)
#check (min_eq_left : a ≤ b → min a b = a)
#check (min_eq_right : b ≤ a → min a b = b)
#check (min_le_left a b : min a b ≤ a)
#check (min_le_right a b : min a b ≤ b)
#check (mul_comm a b : a * b = b * a)
#check (mul_div_cancel' a : b ≠ 0 → b * (a / b) = a)
#check (mul_le_mul : a ≤ b → c ≤ d → 0 ≤ c → 0 ≤ b → a * c ≤ b * d)
#check (mul_le_mul_right : 0 < a → (b * a ≤ c * a ↔ b ≤ c))
#check (mul_left_cancel₀ : a ≠ 0 → a * b = a * c → b = c)
#check (mul_lt_mul_left : 0 < a → (a * b < a * c ↔ b < c))
#check (mul_lt_mul_right : 0 < a → (b * a < c * a ↔ b < c))
#check (mul_neg a b : a * -b = -(a * b))
#check (mul_right_inj' :  a ≠ 0 → (a * b = a * c ↔ b = c))
#check (mul_sub a b c : a * (b - c) = a * b - a * c)
#check (mul_two a : a * 2 = a + a)
#check (ne_comm : a ≠ b ↔ b ≠ a)
#check (neg_add_self a : -a + a = 0)
#check (neg_mul_neg a b : -a * -b = a * b)
#check (nonneg_of_mul_nonneg_left : 0 ≤ a * b → 0 < b → 0 ≤ a)
#check (pow_two a : a ^ 2 = a * a)
#check (pow_two_nonneg a : 0 ≤ a ^ 2)
#check (sq_nonneg a : 0 ≤ a ^ 2)
#check (sub_add_cancel a b : a - b + b = a)
#check (sub_le_sub_left : a ≤ b → ∀ (c : ℝ), c - b ≤ c - a)
#check (sub_le_sub_right : a ≤ b → ∀ (c : ℝ), a - c ≤ b - c)
#check (sub_sq a b : (a - b) ^ 2 = a ^ 2 - 2 * a * b + b ^ 2)
#check (two_mul a : 2 * a = a + a)
#check (two_mul_le_add_sq a b : 2 * a * b ≤ a ^ 2 + b ^ 2)
#check (zero_lt_one : 0 < 1)
#check (zero_lt_two : 0 < 2)
end reales

-- Anillos
-- =======

section anillos
variable {R : Type _} [Ring R]
variable (a b c : R)
#check (add_assoc a b c : (a + b) + c = a + (b + c))
#check (add_comm a b : a + b = b + a)
#check (add_eq_zero_iff_eq_neg : a + b = 0 ↔ a = -b)
#check (add_left_cancel : a + b = a + c → b = c)
#check (add_left_neg a : -a + a = 0)
#check (add_mul a b c : (a + b) * c = a * c + b * c)
#check (add_neg_cancel_right a b : (a + b) + -b = a)
#check (add_neg_self a : a + -a = 0)
#check (add_right_cancel : a + b = c + b → a = c)
#check (add_right_neg a : a + -a = 0)
#check (add_zero a : a + 0 = a)
#check (mul_add a b c : a * (b + c) = a * b + a * c)
#check (mul_zero a : a * 0 = 0)
#check (neg_add_cancel_left a b : -a + (a + b) = b)
#check (neg_eq_iff_add_eq_zero : -a = b ↔ a + b = 0)
#check (neg_eq_of_add_eq_zero_left : a + b = 0 → -b = a)
#check (neg_eq_of_add_eq_zero_right : a + b = 0 → -a = b)
#check (neg_neg a : -(-a) = a)
#check (neg_zero : -0 = 0)
#check (one_add_one_eq_two : (1 : R) + 1 = 2)
#check (sub_add_cancel a b : a - b + b = a)
#check (sub_eq_add_neg a b : a - b = a + -b)
#check (sub_mul a b c : (a - b) * c = a * c - b * c)
#check (sub_self a : a - a = 0)
#check (two_mul a : 2 * a = a + a)
#check (zero_add a : 0 + a = a)
#check (zero_mul a : 0 * a = 0)
end anillos

-- Grupos
-- ======

section grupos
variable {G : Type _} [Group G]
variable (a b c : G)
#check (inv_eq_of_mul_eq_one_right : a * b = 1 → a⁻¹ = b)
#check (mul_assoc a b c : (a * b) * c = a * (b * c))
#check (mul_inv_self a : a * a⁻¹ = 1)
#check (mul_inv_rev a b : (a * b)⁻¹ = b⁻¹ * a⁻¹)
#check (mul_left_inv a : a⁻¹  * a = 1)
#check (mul_one a : a * 1 = a)
#check (mul_right_inv a : a * a⁻¹ = 1)
#check (one_mul a : 1 * a = a)
end grupos

-- Retículos
-- =========

section reticulos
variable {α : Type _} [Lattice α]
variable (x y z : α)
#check (inf_assoc : (x ⊓ y) ⊓ z = x ⊓ (y ⊓ z))
#check (inf_comm : x ⊓ y = y ⊓ x)
#check (inf_le_left : x ⊓ y ≤ x)
#check (inf_le_of_left_le : x ≤ z → x ⊓ y ≤ z)
#check (inf_le_of_right_le : y ≤ z → x ⊓ y ≤ z)
#check (inf_le_right : x ⊓ y ≤ y)
#check (inf_sup_self : x ⊓ (x ⊔ y) = x)
#check (le_antisymm : x ≤ y → y ≤ x → x = y)
#check (le_inf : z ≤ x → z ≤ y → z ≤ x ⊓ y)
#check (le_rfl : x ≤ x)
#check (le_sup_left : x ≤ x ⊔ y)
#check (le_sup_of_le_left : z ≤ x → z ≤ x ⊔ y)
#check (le_sup_of_le_right : z ≤ y → z ≤ x ⊔ y)
#check (le_sup_right : y ≤ x ⊔ y)
#check (le_trans : x ≤ y → y ≤ z → x ≤ z)
#check (sup_assoc : (x ⊔ y) ⊔ z = x ⊔ (y ⊔ z))
#check (sup_comm : x ⊔ y = y ⊔ x)
#check (sup_inf_self : x ⊔ (x ⊓ y) = x)
#check (sup_le : x ≤ z → y ≤ z → x ⊔ y ≤ z)
end reticulos

-- AnillosOrdenados
-- ================

section AnillosOrdenados
variable {R : Type _} [StrictOrderedRing R]
variable (a b c : R)
#check (add_le_add_right : b ≤ c → ∀ (a : R),  b + a ≤ c + a)
#check (mul_le_mul_of_nonneg_left : b ≤ c → 0 ≤ a → a * b ≤ a * c)
#check (mul_le_mul_of_nonneg_right : a ≤ b → 0 ≤ c → a * c ≤ b * c)
#check (mul_nonneg : 0 ≤ a → 0 ≤ b → 0 ≤ a * b)
#check (sub_le_sub_right : a ≤ b → ∀ (c : R), a - c ≤ b - c)
#check (sub_nonneg : 0 ≤ a - b ↔ b ≤ a)
end AnillosOrdenados

-- Espacios métricos
-- =================

section EspacioMetrico
variable {X : Type _} [MetricSpace X]
variable (x y z : X)
#check (dist_comm x y : dist x y = dist y x)
#check (dist_nonneg : 0 ≤ dist x y)
#check (dist_self x : dist x x = 0)
#check (dist_triangle x y z : dist x z ≤ dist x y + dist y z)
end EspacioMetrico

-- Conjuntos
-- =========

section Conjuntos
open Set
variable {α : Type _}
variable (r s t : Set α)
#check (Subset.trans : r ⊆ s → s ⊆ t → r ⊆ t)
end Conjuntos

-- Órdenes parciales
-- =================

section OrdenParcial
variable {α : Type _} [PartialOrder α]
variable (a b c : α)
#check (le_trans : a ≤ b → b ≤ c → a ≤ c)
end OrdenParcial

-- Funciones
-- =========

section Funciones
open Function
variable {α : Type _} {β : Type _} {γ : Type _}
variable {f : α → β} {g : β → γ}
variable (c : ℝ)
#check (Injective.comp : Injective g → Injective f → Injective (g ∘ f))
#check (Surjective.comp : Surjective g → Surjective f → Surjective (g ∘ f))
#check (add_right_surjective c : Surjective (fun x ↦ x + c))
#check (mul_left_surjective₀ : c ≠ 0 → Surjective (fun x ↦ c * x))
end Funciones
\end{verbatim}

\end{document}
